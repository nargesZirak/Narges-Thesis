
\chapter{Introduction}  \label{Introduction}
%\section{Introduction}  \label{Introduction}

Understanding the similarities and differences of a set of source code fragments is a potentially complex problem that has various actual or potential applications in program analysis. For example: detecting code clones \cite{bulychev2009evaluation}, automating source code reuse \cite{cottrell2008semi}, recommending replacements for APIs between various versions of a software library \cite{cossette2014using}, collating application programming interface (API) usage patterns, and automating the merge operation of various branches in a version control system. As a specific application, the focus of this study is on characterizing where logging is used in the source code by understanding the structural correspondences and differences of a set of source code fragments enclosing logging calls within a software system or from different software systems.

Logging is a conventional programming practice to record an application's state and/or actions during the program's execution \cite{gupta2005pro}, and log system analysis assist developers in diagnosing the presence or absence of a particular event,  understanding the state of an application, and following a program's execution flow. The importance of logging has been identified by its various applications in software development and maintenance tasks such as problem diagnosis \cite{lou2010mining}, system behavioral understanding \cite{fu2013contextual}, quick debugging \cite{gupta2005pro}, performance diagnosis \cite{nagaraj2012structured}, easy software maintenance \cite{gupta2005pro}, and troubleshooting \cite{fu2009execution}.

%Given the importance of logging, its quality can affect the quality of an application directly.
Developers can perform logging in various ways, as they make different decisions about where and what to log. For example, they can apply logging to record the occurrence of every event of an application and use logging calls at the start and end of the body of every method in the source code \cite{clarke1999dimension,clarke1999subject}. However, three main problems are associated with excessive logging. First, it can generate a lot of redundant information that might be confusing and misleading for developers to perform system log analysis, masking significant information. Second, excessive logging is costly. It requires extra time and effort to write, debug, and maintain the logging code. Third, it can cause system resource overhead and thus the application performance will be negatively affected. On the other hand, insufficient logging may result in losing some necessary run-time information for software analysis.
Therefore, logging should be done in an appropriate manner to be effective.

Despite the importance of logging for software development and maintenance, few studies have been conducted on understanding logging usage in real-world applications since logging has been considered as a trivial task \cite{clarke1999dimension,clarke1999subject}. However, the availability of several complex frameworks (e.g., \name{log4j},\name{SLF4J}) that assists developers to log suggests that effective logging is not a straightforward task to perform in practice. In addition, \citet{yuan2012characterizing} showed that developers spend a great effort to modify logging practices as after-thoughts, which indicates that it is not that simple for developers to perform logging practices efficiently in their first attempt.

In this research, I would like to understand where developers log in practice in a detailed way. The location of logging calls has a great impact on the quality of logging as it helps developers to trace the code execution path to identify the root causes of an error in log system analysis. However, no previous work has focused on characterizing where logging calls are used in real-world applications.
In this study, I address this gap by developing an automated approach to detect the detailed structural correspondences and differences in the usage of logging within a system and between systems.

% that would assist developers to make informed decisions about where and what to log
%logging has been considered as a trivial task (AOP/AOSD literature)

%\item \NZ{I am a little confused about how to find commonalities and differences of logging usage between systems? do you mean that I should compare the results taken from clustering of all Java classes in one system to another one?} \RW{The process of locating commonalities and differences can be applied within a single version of one system, across multiple versions of one system, across single versions of multiple systems, or across multiple versions of multiple systems.  It would be useful to understand the differences and similarities between different systems.  Is there some reason that this would be harder to achieve?} \NZ{I should think about it more. I can answer to this question after per-system analysis with more confidence}


\section{Programmatic support for logging} \label{background Logging}

A typical logging call takes parameters including a log text message and verbosity level. A log text message consists of static text to describe the logged event and some optional variables related to the event. The verbosity level is intended to classify the severity of a logged event such as a debugging note, a minor issue, or a fatal error. Figure~\ref{fig:log-call-examples} provides examples of logging calls from the \name{log4j} framework in descending order of severity. The \name{fatal} level designates a very severe error event that will likely lead the application to terminate. The \name{error} level indicates that a non-fatal but clearly erroneous situation has occurred. The \name{warn} level indicates that the application has encountered a potentially harmful situation. The \name{info} level designates important information that might be helpful in detecting root cases of an error or to understand the application behaviour. The \name{debug} level designates useful information to debug an application and is usually used by developers only during the development phase. In general, verbosity level is used for classification to avoid the overhead of creating large log files in high performance code.
%he first three cases are used to indicate problems in an application.


\begin{figure}[H]
\vspace*{1em}
\begin{center}
\begin{minipage}{4in}
\begin{lstlisting}[frame=single,numbers=none]
log.fatal("Fatal Message %s", variable);
log.error("Error Message %s", variable);
log.warn("Warn Message %s", variable);
log.info("Info Message %s", variable);
log.debug("Debug Message %s", variable);
\end{lstlisting}
\end{minipage}
\caption{Logging call examples from the \name{log4j} framework.\label{fig:chap1_logCode}\label{fig:log-call-examples}}
\end{center}
\end{figure}

%\RW{This isn't useful because it is too short and has already been covered by your introductory comments.}
%\section{Overview of related work} \label{intro-rw}
%
%Research on the problem of characterizing logging practices can be divided into two main topics: context and location of logging calls. The context refers to the log text messages and the location refers to where logging calls are used in the source code.
%Few studies have been conducted on characterizing log message modifications \cite{yuan2012characterizing} and developing tools to automatically enhance existing log messages \cite{yuan2012improving, yuan2010sherlog}. However, no research has been conducted on studying the location of logging calls in real-world software systems.
%
%
%Understanding the commonalities and differences between source code fragments has been used for various applications (e.g., \cite{cottrell2007determining, cottrell2008semi, cottrell2009compare, cossette2014using, bulychev2009evaluation}). However, my study makes the first attempt to characterize the usage of logging calls by automatically detecting the detailed structural correspondences and differences of a set of  source code fragments enclosing them.



\section{Broad thesis overview} \label{intro-overview}

In this study, I aim to provide a concise description of where logging is used in the source code by constructing generalizations that represents the detailed structural similarities and differences between entities that make use logging calls. My study investigates the location of logging calls from the point of view of logged Java methods (LJM) which are the Java methods enclosing logging calls in source code. To investigate how to construct generalizations using the syntax and semantics of the Java programming language, I looked to previous research conducted by \citet{cottrell2008semi} that determines the detailed structural correspondences between two Java source code fragments through the application of approximated anti-unification such that one fragment can be integrated to the other one for small scale code reuse. However, my problem context is different as I need to generalize a set of source code fragments with special attention to logging calls. Therefore, my approach must take into account the logging calls when performing the generalization task via the determination of structural correspondences.

%My approach to construct structural generalizations applies a hierarchical clustering algorithm to classify LJMs into groups using a measure of similarity.

\RW{Good.}
My approach employs a hierarchical clustering algorithm to create a generalization hierarchy from a set of LJMs using a measure of similarity. It uses an approximated antiunification algorithm to construct a structural generalization representing the similarities and differences between a pair of LJMs. My antiunification approach proceeds in three steps. First, it uses the Jigsaw framework \cite{cottrell2008semi} to determine all potential correspondences between the two LJMs using a measure of similarity that relies on structural correspondences along with a simple knowledge of semantic equivalences in the Java language specifications. Second, it develops a greedy selection algorithm to approximate the best anti-unifier to my problem by determining the most similar correspondence for each substructure in my structures, applying some constraints in determining correspondences to prevent the anti-unification of logging calls with anything else. Third, it constructs an anti-unifier through the anti-unification of two structures and develops a measure of structural similarity between them.

\RW{I'm not sure if it is necessary to overview the empirical studies here.  You mention them below.  To be revisited later.}
%most similar  or best correspondence?
%constraints wording?
%my emprical study?
%My tool has been applied on the source code of these systems and extracts all logged Java classes from these systems to construct the structural generalizations.
%My evaluation shows ...

\section{Thesis Statement} \label{intro-stmt}
The thesis of this work is to determine the detailed structural similarities and differences between entities of the source code that make use of logging calls to provide a concise description of where logging calls do occur in real-world software systems.

\section{Thesis Organization} \label{intro-org}
The remainder of the thesis is organized as follows. Chapter~\ref{ch2} motivates the problem of understanding where to use logging calls in the source code through a scenario in which a developer attempts to perform a logging task. This scenario outlines the potential problems she may encounter and illustrates that the current logging practice is insufficiently supported.

Chapter~\ref{background} provides background information on important concepts and past research that are key to my work.  I described abstract syntax trees (ASTs), which are the basic structure I will use for describing software source code. I provide a definition of anti-unification and explain why its basic forms do not adequately address my problem. Then I define higher-order anti-unification modulo theories (HOAUMT), an extension to anti-unification that can be applied on an extended form of AST to solve my problem. Afterwards I discuss the Jigsaw framework, an existing tool that could assist us in the determination of potential structural correspondences by applying HOAUMT on extended structures
\RW{This may need some modification to introduce the fact that you do some studies to validate Jigsaw, etc.}
%for a subset of HOAUMT, which I extend
% how ASTs are realized in the Eclipse integrated development environment, the industrial tool I will build atop

Chapter~\ref{methodology} describes my proposed approach as a set of novel algorithms, and its implementation as a plug-in to the Eclipse integrated development environment (IDE).

Chapter~\ref{eval} presents two empirical studies. The first study is conducted to evaluate the accuracy of my approach and its implemented tool by conducting an experiment on 10 sample logged Java methods. The second study is conducted to characterize the location of logging usage in three open-source software systems.
%characterize the location of logging usage

Chapter~\ref{diss} discusses the results and findings of my work, threats to its validity, and the remaining issues. Chapter~\ref{rw} describes related work to my research problem and how it does not adequately address the problem. Chapter~\ref{conc} concludes the dissertation and presents the contributions of this study and future work. Additional materials appended to the end of this dissertation is provided in Appendix A.


