\chapter{Evaluation}  \label{eval}
\begin{itemize} [leftmargin=.01in]
\item Two empirical studies were conducted
\section{Experiment 1}  \label{expr1}
\item First experiment is conducted to evaluate the accuracy of our approach and tool
\item It addresses the following research questions:
\begin{itemize} [leftmargin=.3in]
\item \textsc{RQ1: }can our tool determine the structural similarities and differences between logged Java classes correctly?
\item \textsc{RQ2: }can our tool compute the similarity between logged Java classes correctly?
\end{itemize}
\item To do so, 10 logged Java classes were selected randomly from jEdit v4.2 pre 15 (2004), as our test set
\item We apply our tool on the test set to create generalizations  and to compute the similarity value between logged Java classes in a pairwise manner
\item To address the first research question we compute the following measurements for each test case:
\begin{itemize} [leftmargin=.3in]
\item the number of correspondences that our tool detects correctly
\item the total number of correspondences
\end{itemize}
\item To determine the correct correspondences we performed a manual investigation
\item To address the second research question we compute:
\begin{itemize} [leftmargin=.3in]
\item the number of similarity values between logged Java classes that are computed correctly by our tool
\item the total number of comparisons
\end{itemize}
\item The correct similarity value for each comparison is calculated manually by
%\begin{itemize} [leftmargin=.3in]
%\item counting common simple values based on the selection of corresponding AUAST nodes
%\item counting total number of simple values of AUAST nodes
%\end{itemize}
\item The results taken form our tool are compared with the results taken by manual investigation using the JUnit testing framework

\section{Experiment 2}  \label{expr2}
\item Second experiment is conducted to address the following research questions:
\begin{itemize} [leftmargin=.3in]
\item \textsc{RQ3: }what structural similarities and differences do logged Java classes have?
\item \textsc{RQ4: }Is it possible to find common patterns in where logging calls do occur?
\end{itemize}
\item To do so, we applied our tool on the source code of three open-source full systems that make use of logging to determine the patterns on a per-system class-granularity basis analysis
\item These systems are different from the system that the test set is selected from
\section{Results}  \label{results2}
\item I will describe the results taken form the second experiment
\section{Lessons learned}  \label{lessons}
\item I will describe our findings
\end{itemize}