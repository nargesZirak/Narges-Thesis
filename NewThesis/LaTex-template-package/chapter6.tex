\chapter{Discussion}  \label{diss}

In this chapter, I discuss the validity of my evaluation and the characterization study (Section~\ref{threats}), and a number of remaining issues including: the limitations and pitfalls of my approach and the tool support (Section~\ref{limitations}); the usage of anti-unification theory for other applications (Section~\ref{auTheory}); and the definition of an intermediate form of structural variable constraints (Section~\ref{intC}).

\section{Threats to validity}  \label{threats}

\subsection{Threats to internal validity}  \label{internal_threats}
\begin{itemize} [leftmargin=.5in]
\item \textit{Researcher bias:} The results of my manual analysis might be biased, as I determined the correct correspondences for my test suite myself. To limit the bias, third-parties could be involved to double-check the accuracy of my manual inspection in future work.
\item \textit{Limited sample size:} Another potential threat is that the rate of detecting correct best correspondences by my tool might have happened coincidentally for my test suite, as the test suite size is limited in my experiments. To reduce this doubt, I manually examined the cases where my tool fails to detect correct correspondences, and I found that the failures are due to the fundamental limitations and complexities in the construction of structural generalizations through the use of structural correspondence. That is, my tool creates structural generalization successfully with regard to what my algorithm should generate.
\end{itemize}

\subsection{Threats to external validity}  \label{external_threats}
\begin{itemize} [leftmargin=.5in]
\item \textit{Representativeness of the studied systems:} A potential threat to the validity of the exploratory study is the degree to which my selected set of software systems is representative of all real-world logging practices. To address this issue, I selected various open-source software projects in terms of application domain. The studied software systems have been widely used by many developers for a long period of time. However, the fact that I only studied Apache log4j statements might limit the generalizability of the findings. To improve the generalizability of this study, I have chosen one system not from the Apache Software Foundation. However, my findings might reflect the characteristics of logging usage in systems that are closed source, from other domains, or written in programming languages other than Java\@. In addition, the experiments have examined one set of logged methods (LMs) from a real-world software system; different sets and different systems may generate different results. Despite the fact that I cannot claim that the set of tested LMs is representative of all LMs in real-world software systems, the experimental results are still promising, as the location of logging statements in the tested methods differ markedly from each other with respect to their location and details. Hence, these experiments have sufficed to indicate the effectiveness of my approach in constructing structural generalizations.
\item \textit{The number of software systems under study:}	 My study was conducted on four large open-source software systems written in Java\@. While the studied systems are limited in number, I believe that my findings of the logging practices in these systems could be generalizable to many other software systems. This threat could be reduced by more experiments on wider range of software systems in future work.
\item \textit{The number of researchers who performed the manual examination:} Due to the limitations during the course of my research, the manual examination was conducted only by the first author. Future work can reduce this thread by involving other people to verify the results.
\end{itemize}


\subsection{Threats to construct validity}  \label{construct_threats}
As there is no existing benchmark dataset to assess the quality attributes of the anti-unifiers in terms of accuracy, reliability, and performance, I have applied the \func{Determine-Locations} algorithm (Algorithm~\ref{alg-determine-locations}) to measure the performance of ELUS in terms of exactness (precision) and completeness (recall) for the studied systems. However, I acknowledge that these metrics may not perfectly measure all the quality attributes of the anti-unifiers of logging code. Another potential construct threat is the degree to which my similarity metric perfectly represents how similar logged methods are. To reduce this threat, I conducted an experimental study in which I manually estimated the similarity between the usages of logging statements in different methods of a sample test suite and compared the similarity values with the results extracted by the automated approach.



\section{The pitfalls of ELUS}  \label{limitations}
%title?? choose a name for our tool????
There are some issues that the approximation approach and my tool support are not able to handle perfectly, including inaccurate node ordering, and the resolution of conflicts that happened in constructing the anti-unifiers.


\subsection{Inaccurate node ordering}  \label{mismatch}
My anti-unification algorithm does not guarantee to maintain the correct sequence of statements in the body of methods in the case of anti-unifying two method declaration nodes, as the order of statement nodes is not considered in determining the best correspondences. For example, consider that we have two corresponding methods $\id{method_1}$ and $\id{method_2}$ involved in two sequences of statements $\{\id{a_1}$, $\id{a_2}\}$ and $\{\id{b_1}$, $\id{b_2}\}$. If the $\id{b_1}$ and $\id{b_2}$ nodes are found to be the best correspondences of the $\id{a_2}$ and $\id{a_1}$ nodes, respectively, $\id{a_1}$ will be anti-unified with $\id{b_2}$  and $\id{a_2}$ will be anti-unified with $\id{b_1}$ to construct the structural generalization. Therefore, the anti-unification algorithm does not preserve the correct ordering of nodes in the original structures.
%????
%% IIf the $\id{b_1}$ and $\id{b_2}$ nodes are found to be the best correspondences of the $\id{a_2}$ and $\id{a_1}$ nodes, respectively, $\id{a_1}$ will be anti-unified with $\id{b_2}$  and $\id{a_2}$  will be anti-unified with $\id{b_1}$ to construct the structural generalization.??

%%%??????
%f methods in case of anti-unifying two method declaration node??

%\subsection{handling conflicts in constructing anti-unifiers}  \label{conflicts}
\subsection{Conflict resolution}  \label{conflicts}
The decisions that I made, to resolve these conflicts, occurred in constructing structural generalizations that might affect the accuracy of our results.
For example, in situations where I have two correspondences with the same similarity value in the ordered list of correspondence connections, my approach picks the one which involves two subtrees with the higher number of nodes, though this might not be the best choice for all cases.
In addition, I consider AST hierarchies to perform anti-unification. That is, my algorithm does not anti-unify two nodes if their parent nodes are not found to also correspond. As a result, situations can occur where in fact two nodes should be anti-unified with each other, while they are not anti-unified by the tool. Though these decisions led to results that represent suboptimal anti-unifiers, they helped to limit the complexity of my approach, allowing the implementation of it to be a practical solution.
%%%????




%%%WHY PRECISON AND RECALL IS NOT 100%???


 % add semantic to structural information to detect correspondences??
%limited typing information to determine correspondences by %Jigsaw
%\item Our tool does not guarantee the correctness of determining the best correspondences due to
%\begin{itemize} [leftmargin=.3in]
%\item the various conflicts that happen
%\item limited typing information to determine correspondences by %Jigsaw
%\end{itemize}
%\item Structural generalizations constructed by our tool are not in %the form of executable code
%[leftmargin=.3in]\end{itemize}


%\section{Other applications}  \label{other_applications}
%Any applications that are involved in the inference of structural patterns in source code even infrequently-used patterns might benefit from our tool's underlying framework.
%% EXAMPLE???
%Furthermore, understanding the commonalities and differences between source code fragments has application in several areas of software engineering, such as code clone detection, API usage pattern collation, recommending replacements for API migration, and merging different branches of version control systems. Our tool`s functionality to construct a detailed view of structural generalizations of a set of source code fragments via structural correspondence could be used to improve the results of these studies as well.
%edit!!!
\section{Applications of anti-unification}  \label{auTheory}
My study demonstrates the application of an extended from of anti-unification (higher-order anti-unification modulo theories) to infer usage patterns of log statements in source code via the creation of structural generalizations. Anti-unification and its extensions have previously been applied to solve several theoretical and practical problems, such as analogy making \cite{guhe2010mathematical}, determining lemma generation in equational inductive proofs \cite{2005:aij:burghardt}, and detecting the construction laws for a sequence of structures \cite{2005:aij:burghardt}.
%CITATION???


Higher-order anti-unification modulo theories (HOAUMT) can be used to create generalizations in different contexts, and therefore the set of equational theories must be developed specifically for the higher-order structure found in each problem context. That is, the utility of these theories are highly dependent on how well they allow the incorporation of semantic knowledge of structures. In addition, useful theories should ensure that only a finite number of instances exist for each structure. The practical experiments I have conducted through the application of my tool on a test suite demonstrate that an approximation of HOAUMT can be successfully used to construct structural generalizations required to solve a problem.
%finite?
%%%Taking all these considerations into account enables HOAUMT to anti-unify sets of structures in a particular context.????



\section{Defining intermediate constrained variables}  \label{intC}
As explained in Section~\ref{evaluation}, I conducted an empirical experiment to evaluate the quality of the anti-unifiers for both constrained and unconstrained variables. As shown in Figures~\ref{fig:precision} and~\ref{fig:recall}, the average precision and recall values for the unconstrained experiment is lower than and higher than, respectively, the average precision and recall values for the constrained experiment. However, to keep the balance between the exactness (precision) and completeness (recall) of the anti-unifiers, an intermediate form of constraints can be defined for structural variables. That is, instead of constraining to only the specific substitutions, intermediate constraints could be defined such that only specific types of nodes (e.g., \textit{infixExpression}) could be used to substitute a structural variable. A future experiment can be run to see whether considering variables with intermediate constraints would yield better overall results than the fully constrained and unconstrained forms.
% the precision (exactness) and recall (completeness)


\section{Summary}  \label{diss-summary}
I discussed the potential threats to validity of my evaluation and characterization study. To limit the bias of my experiments, I selected the test cases form a real system with various levels of similarity in the usage of log statements. Furthermore, I examined the failed test cases to ensure that my tool works when it should work with regard to the proposed algorithm. I will also make my test suite available for public examination to further check the accuracy of my manual inspection. For the characterization study, I selected various software systems in terms of functionality. I also discussed the remaining issues with the tool support, including inaccurate node ordering and handling the conflicts happened in the construction of anti-unifiers.
% I conducted to evaluate the effectiveness of my approach and the tool support

This work aims to provide a detailed view of structural generalizations constructed from a set of source code fragments that use log statements via the application of anti-unification and clustering. However, I argued how higher-order anti-unification modulo theories can be effectively approximated for various applications by means of developing an appropriate set of equational theories particularly for the higher-order structure used in each problem context. I also explained how the definition of a form of intermediate constraints on structural variables may lead to better experimental results.
%I also explained how the definition an intermediate form of constraints for structural variables of anti-unifiers may yield to better experimental results.?

%in future as the extensions of my work.

% I discussed that any other applications involved in the inference of structural usage patterns of a particular statement or understanding the commonalities and differences between a set of source code fragments could benefit from our tool's underlying framework.
