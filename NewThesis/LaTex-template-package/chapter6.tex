\chapter{Discussion}  \label{diss}
\begin{itemize} [leftmargin=.01in]
\section{Threats to validity}  \label{threads}
\item Our goal is to recognize the limitations and pitfalls of our approach and its developed tool support
\item The first potential thread to validity of our characterization study is the degree to which our sample set of software systems is a good representation of all real-world logging practices. To address this issue we selected software systems that:
\begin{itemize} [leftmargin=.3in]
\item are different in terms of application
\item are among the most popular applications in their own product category
\item has long history in software development
\end{itemize}
\item Secondly, our manual investigation to find the correct correspondences might be biased due to human errors. To limit the bias
\begin{itemize} [leftmargin=.3in]
\item other people can be involved to double check the accuracy of manual work in the future work
\end{itemize}
\item However, these results are still promising

\section{Our tool output}  \label{output}
\item We investigated the cases where our tool fails and we found that the failures are due to:
\begin{itemize} [leftmargin=.3in]
\item the assumptions taken in developing the algorithms
\item the fundamental limitations and complexities in determining the detailed structural similarities and differences
\end{itemize}

\item The are some issues that our tool is not able to handle perfectly during generalization:
\begin{itemize} [leftmargin=.3in]
\item maintaining the correct ordering of statements inside the method bodies
\item resolving all the conflicts that happen in determining the best correspondences
\item producing executable generalizations
\end{itemize}
%\item Our tool does not guarantee the correctness of determining the best correspondences due to
%\begin{itemize} [leftmargin=.3in]
%\item the various conflicts that happen
%\item limited typing information to determine correspondences by %Jigsaw
%\end{itemize}
%\item Structural generalizations constructed by our tool are not in %the form of executable code
%[leftmargin=.3in]\end{itemize}
\section{Theoretical foundation}  \label{theory}
\item anti-unification and its extensions has several theoretical and practical applications:
\begin{itemize} [leftmargin=.3in]
\item analogy making [Schmidt, 2010]
\item determining lemma generation in equational inductive proofs [Burghardt, 2005]
\item detecting the construction laws for a sequence of structures [Burghardt, 2005]
\end{itemize}
\item Using higher-order anti-unification modulo theories in our application, which is undecidable in general, leads us to take approximations suitable to our context
\item The set of equational theories should be developed particularly for the structure used in each problem context
\end{itemize} 