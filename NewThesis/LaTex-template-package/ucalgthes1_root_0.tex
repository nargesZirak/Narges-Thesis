\documentclass{ucalgthes1}
\usepackage{times}
\usepackage[letterpaper,top=1in, bottom= 1in, left= 1in, right= 1in]{geometry}
\usepackage{fancyhdr}
\usepackage{graphicx}
\usepackage{subfig}
\usepackage{listings}
\usepackage[dvipsnames]{xcolor}
\usepackage[bookmarksopen=false,colorlinks=false,pdfborder={0 0 0},plainpages=false,pdfpagelabels]{hyperref}
\usepackage[chapter]{algorithm}

\usepackage{amssymb}% http://ctan.org/pkg/amssymb
\usepackage{pifont}% http://ctan.org/pkg/pifont
\newcommand{\cmark}{\ding{51}}%
\newcommand{\xmark}{\ding{55}}%
\usepackage{listings}
\usepackage{makecell}
%\usepackage{lstlinebgrd}
\usepackage{algpseudocode}
\usepackage{booktabs}
\usepackage{caption}
\usepackage{amsmath}
\usepackage{multirow}
\usepackage{array}
\usepackage{float}
\usepackage{enumitem}
\usepackage{textcomp}
\usepackage{color}
\usepackage{tabularx}
\usepackage{longtable}
\usepackage{amsthm}
\theoremstyle{plain}
\newtheorem{theorem}{Theorem}
\usepackage{natbib}
%\newtheorem{definition}{Definition}[section]
\theoremstyle{definition}
\newtheorem{defn}{Definition} [section]


\usepackage{rotating}
\usepackage{pgf}
\usepackage{tikz}
\usepackage{tikz-cd}
\usetikzlibrary{arrows, matrix}

\let\cite\citep
\newcommand{\tool}{\relax}
\newcommand{\id}{\mathit}
\newcommand{\vars}{\textit}
\newcommand{\func}{\textsc}
\newcommand{\cons}{\textrm}
%\newcommand{\bold}{\textbf}
\newcommand{\name}[1]{#1}%{\textsf{\small #1}}
\let\oldReturn\Return
\renewcommand{\Return}{\State\oldReturn}
\algnewcommand\And{\textbf{and}}
\algnewcommand\Instanceof{\textbf{ instanceof }}
\algnewcommand\ori{\textbf{ or }}

\definecolor{yellowGreen}{rgb}{0.7,0.85,0.0}
\algnewcommand{\ComputeSimilarity}{\Statex \textbf{\func{Compute-Similarity}($\id{nodeA}$,$\id{nodeB}$)}}


\algnewcommand{\ApplyConstraints}{\Statex \textbf{\func{Apply-Constraints}($\id{list}$)}}

\algnewcommand{\DeterminePotentialCorrespondences}{\Statex \textbf{\func{Create-Correspondence-Connection}($\id{A}$,$\id{B}$, $\id{matches}$)}}

\algnewcommand{\DetermineBest}{\Statex \textbf{\func{Determine-Best-Correspondences}($\id{A}$)}}
\algnewcommand{\Determine}{\Statex \textbf{\func{Determine-Correspondences}($\id{AuastA}$, $\id{AuastB}$)}}

\algnewcommand{\DetermineLocations}{\Statex \textbf{\func{Determine-Locations}($\id{antiUnifier}$,$\id{methods}$)}}



\algnewcommand{\AntiUnify}{\Statex \textbf{\func{Antiunify}($\id{A}$, $\id{B}$)}}

\algnewcommand{\CreateAntiunifier}{\Statex \textbf{\func{Construct-Antiunifier}($\id{A}$, $\id{B}$)}}

\algnewcommand{\ComputeBestMatches}{\Statex \textbf{\func{Compute-Best-Matches}($\id{A}$,$\id{B}$)}}
\algnewcommand{\ComputeMatches}{\Statex \textbf{\func{Determine-Matches}($\id{A}$,$\id{B}$)}}

\algnewcommand{\AntiUnifyProperty}{\Statex \textbf{\func{Antiunify-Properties}($\id{propA}$, $\id{propB}$)}}

\algnewcommand{\RemoveOtherCEs}{\Statex \textbf{\func{Remove-Other-Correspondences}($\id{corr}$, $\id{list}$)}}

\let\cite\citep
\newcolumntype{P}[1]{>{\centering\arraybackslash}p{#1}}

\lstloadlanguages{Java}

\lstset{language=Java,
escapeinside={{(*@}{@*)}},
columns=flexible,
basicstyle=\small\sffamily,
numberstyle=\tiny,
numbers=left,
stepnumber=1,
showspaces=false,
showstringspaces=false,
numbersep=0.1cm,
breaklines=true,
xleftmargin=2em,
xrightmargin=0em}

\long\def\RW#1{ \begin{bfseries}[RW: #1]\end{bfseries} }
\long\def\NZ#1{ \begin{scshape}[NZ: #1]\end{scshape} }

\newcommand{\nothing}{{``nothing''}}
\newtheorem{principle}{Principle}
\newtheorem{constraint}{Constraint}
\newcommand{\NIL}[1]{\textsf{NIL}}
\newcommand{\code}{\lstinline}
\renewcommand{\algorithmicrequire}{\textbf{Input:}}
\renewcommand{\algorithmicensure}{\textbf{Output:}}

\newcommand{\thestitle}{Automatically Characterizing Logging Usage: An Application of Anti-unification}

\usepackage{hyperref}
\urlstyle{sf}
\title{Automatically Characterizing Logging Usage:\\An Application of Anti-unification}
%
%            Insert the correct information between the {}
%
\author{Narges Zirakchianzadeh}
\thesisyear{2016}
\thesis{thesis}    % the word dissertation can be inserted between {}
\newcommand{\thesistitle}{\title}
\monthname{November}
\dept{COMPUTER SCIENCE}
\degree{MASTER OF SCIENCE}
%
%                    End of supplied information
%
\begin{document}
\makethesistitle
\pagenumbering{roman}     % resets page counter to one
\setcounter{page}{1}


%\chapter*{UNIVERSITY OF CALGARY \\ FACULTY OF GRADUATE STUDIES}
%\thispagestyle{empty}
%The undersigned certify that they have read, and recommend
%to the Faculty of Graduate Studies for acceptance, a \Thesis\ entitled
%``\thestitle'' submitted by \Author\
%in partial fulfillment of the requirements for the degree of
%\Degree.\\
%
%%
%%                 Substitute  List of Examiners
%%
%
%\begin{signing}{Department of Computer Science}
%
%% \newsigncolumn         use this command to start a new column if necessary
%\newsigncolumn
%
%\signline
%Dr.~Robert~J. Walker \\
%Supervisor \\
%Department of Computer Science  \\
%
%
%\signline
%Dr.~J{\"o}rg Denzinger \\
%Examiner \\
%Department of  Computer Science \\
%
%\signline
%Dr.~G\"{u}nther Ruhe\\
%Examiner \\
%Department of  Computer Science \\
%\end{signing}

\newpage
\phantomsection
\altchapter{{Abstract}}
Logging has been a common practice to record the runtime behaviour of a software system, typically performed by inserting log statements in its source code. While several frameworks have been specifically created to help developers perform logging tasks, these do not provide guidance on where the log statements should be located in the source code. Thus, developers usually rely on their common sense to decide where to log. If logging is done properly, it can provide valuable information for software development and maintenance; if it is done poorly, system performance can degrade and maintenance can be made more difficult. Few studies have been conducted to characterize logging usage in real-world applications. This work tries to address the problem of where to log by proposing an automated approach that characterizes the location of log statements through the approximation of an anti-unification approach (specifically, higher-order anti-unification modulo theories) and a hierarchical clustering technique to construct a set of anti-unifiers, each describing the commonalities and differences between source code fragments that embody log statements. This approach has been reified in a prototype tool, called \tool{ELUS}, that greedily identifies the best structural correspondences with respect to the highest similarity and some constraints. An empirical study was conducted by applying the tool on the source code of four open source systems and manually examining the generated anti-unifiers. The analysis resulted in five main categories of anti-unifiers in the logging usage. Two empirical evaluations were conducted in this work: (1)~an experiment was conducted to evaluate the effectiveness of the proposed approach through the application of its supporting tool on a test suite; and (2)~an experiment was performed to evaluate the quality of the anti-unifiers in describing the location of log statements in source code.
%, each describes?
% automated or semi-automated?
\newpage
\phantomsection
\altchapter{Acknowledgments}

I would like to express my warmest gratitude to my supervisor, Dr.~Robert~J. Walker.  I would like to thank him for his great support, guidance, and insightful discussions during my study. He always encouraged me to challenge myself by asking new research questions and to come up with new ways to solve a problem. I deeply appreciate his belief in me when I doubted myself, and his constant commitment and precious comments throughout the course of my research and thesis writing. I was always motivated by his support, understanding, and the freedom he gave me.

I am truly grateful to Dr.~J{\"o}rg Denzinger, Dr.~G\"{u}nther Ruhe, and Dr.~Christian Jacob for providing support and useful feedback for my work. Their technical knowledge and valuable feedback helped me a lot throughout the course of my research.
%I would also like to thank the members of my oral defense committee Dr.~G\"{u}nther Ruhe and ... for their time. ????	

Many special thanks and love to my parents and my brothers for being such a great family. I am forever indebted to them for their support, patience, and encouragement throughout this work.


I would like thank my fellow labmates, Hamid, Elham, Soha, Mostafa, May, and Hao for their academic and friendship support. Their precious comments helped me a lot over the course of my research.


\begin{singlespace}
\newpage
\phantomsection
\tableofcontents
\pagestyle{plain}
\newpage
\phantomsection
\listoftables
\pagestyle{plain}
\newpage
\phantomsection
\listoffigures
\pagestyle{plain}
\clearpage
\clearpage          % otherwise tables will be numbered wrong
\end{singlespace}
\newpage
\phantomsection
%\chapter*{\bf{List of Symbols, Abbreviations and Nomenclature}\hfill}
\chapter*{\bf{List of Abbreviations}\hfill}
 \addcontentsline{toc}{chapter}{List of Abbreviations}
\addtocontents{toc}{\protect\addvspace{10pt}}
\listofsymbols
\pagestyle{plain}
\clearpage



\pagenumbering{arabic}
\setcounter{page}{1}
\addtocontents{toc}{\protect\addvspace{10pt}}
\chapter{Introduction}  \label{Introduction}
%\section{Introduction}  \label{Introduction}

Understanding the similarities and differences between a set of source code fragments is a potentially complex problem that has many actual or potential applications in various software engineering research areas, such as: code clones detection \cite{2009:iwsc:bulychev}; automating source code reuse \cite{2008:fse:cottrell}; recommending application programming interface (API) replacements amongst various versions of a software library \cite{2014:uofc:cossette}; collating API usage patterns; and automating the merge operation in a version control system. As a specific application, the focus of this study is on characterizing where log statements are used in source code via the determination of structural correspondences between a set of source code fragments enclosing them.

%Clarke???
Logging is a conventional programming practice that has usually been used by developers to diagnose the presence or absence of a particular event in a system, to understand the state of an application, and to follow a program's execution flow to find the root causes of an error. The importance of logging is notable in its various applications during software development, such as problem diagnosis \cite{lou2010mining}, system behavioural understanding \cite{fu2013contextual}, quick debugging \cite{gupta2005pro}, performance diagnosis \cite{nagaraj2012structured}, easy software maintenance \cite{gupta2005pro}, and troubleshooting \cite{fu2009execution}. Despite the significance of logging for software development and maintenance, few studies have been conducted on understanding its usage in real-world applications, as it has been considered to be a trivial task \cite{clarke1999dimension,clarke1999subject}. However, the availability of several complex frameworks (e.g., \name{Apache log4j}, \name{SLF4J}) that assist developers in logging suggests that in practice effective logging is not a straightforward task. In addition, a study by \citet{yuan2012characterizing} showed that developers expend great effort in modifying their logging practices as an afterthought. This indicates that it is not that simple for developers to perform logging effectively on their first attempt.

%\cite{clarke1999dimension,clarke1999subject}?
%showed?
The challenges associated with high quality logging arises form the fact that developers are usually left with the burden of deciding where and what to log manually, thus log statements can be inserted in various locations of source code. For example, a developer may decide to insert log statements at the start and end of every \name{method} to record the occurrence of every event of an application. However, three main problems are associated with excessive logging. First, it can produce a lot of redundant information that makes the system log analysis confusing and misleading. Second, excessive logging is costly. It requires extra time and effort to write, debug, and maintain the logging code. Third, it can generate system resource overhead and thus the application performance will be negatively affected. On the other hand, insufficient usage of log statements may result in the loss of run-time information necessary for software analysis. Therefore, logging should be done in an appropriate manner to be effective.


Research on the problem of understanding logging practices can be divided into two main topics: the context and the location of log statements. The context refers to the log text messages, while the location refers to where logging statements are used in source code. The context of log statements is important to perform high quality logging, as it provides necessary information needed for system analysis. The location of log statements also has a great impact on the quality of logging, as it helps developers to trace the code execution path to identify the root causes of an error within a system. A few studies have been conducted on characterizing log text message modifications \cite{yuan2012characterizing} and developing tools to automatically enhance the context of existing logging statements \cite{yuan2012improving, yuan2010sherlog}. \citet{yuan2012conservative} proposed \tool{Errlog} to automatically insert additional log statements into a software system to log all the generic exceptions in order to enhance failure diagnosis. \citet{zhu2015learning} applied machine learning techniques to determine the important factors impacting the location of the log statements in source code. In this study, I address the problem of understanding where to log by developing an automated approach that investigates the feasibility of finding patterns of where log statements occur in source code through the construction of a detailed view of structural generalizations representing the commonalities and differences between source code fragments that contain logging statements.

%\citet{zhu2015learning}??

%patterns of?
%automated approach
%\citet{yuan2012conservative} proposed \tool{Errlog} to automatically insert additional log statements into a software system to log all the generic exceptions in order to enhance failure diagnosis.?
\section{Programmatic support for logging} \label{background Logging}

A typical log statement takes parameters including a log text message and a verbosity level. A log text message consists of static text that describes the logged event and some optional variables related to the event. The verbosity level is intended to classify the severity of a logged event such as a debugging note, a minor issue, or a fatal error. Figure~\ref{fig:log-call-examples} provides examples of log statements from the \name{Apache log4j} framework in descending order of severity. The \name{fatal} level designates a very severe error event that will likely lead the application to terminate. The \name{error} level indicates that a non-fatal but clearly erroneous situation has occurred. The \name{warn} level indicates that the application has encountered a potentially harmful situation. The \name{info} level designates important information that might be helpful in detecting root causes of an error or in understanding the application behaviour. The \name{debug} level provides useful information for debugging an application, and it is usually used by developers only during the development phase. In general, verbosity level is used for classification, in order to avoid the overhead of creating large log files in high performance code.

\begin{figure}[H]
\vspace*{1em}
\begin{center}
\begin{minipage}{3.5in}
\begin{lstlisting}[frame=single,numbers=none]
 log.fatal("Fatal Message %s", variable);
 log.error("Error Message %s", variable);
 log.warn("Warn Message %s", variable);
 log.info("Info Message %s", variable);
 log.debug("Debug Message %s", variable);
\end{lstlisting}
\end{minipage}
\caption{Log statement examples from the \protect\name{Apache log4j} framework.\label{fig:chap1_logCode}\label{fig:log-call-examples}}
\end{center}
\end{figure}

%\RW{This isn't useful because it is too short and has already been covered by your introductory comments.}
%\section{Overview of related work} \label{intro-rw}
%
%Research on the problem of characterizing logging practices can be divided into two main topics: context and location of logging calls. The context refers to the log text messages and the location refers to where logging calls are used in the source code.
%A few studies have been conducted on characterizing log message modifications \cite{yuan2012characterizing} and developing tools to automatically enhance existing log messages \cite{yuan2012improving, yuan2010sherlog}. However, no research has been conducted on studying the location of logging calls in real-world software systems.
%
%
%Various applications have used an understanding of the commonalities and differences between source code fragments
%Understanding the commonalities and differences between source code fragments has been used for various applications (e.g., \cite{2007:esec_fse:cottrell, 2008:fse:cottrell, 2009:vissoft:cottrell, 2014:uofc:cossette, 2009:iwsc:bulychev}). However, my study makes the first attempt to characterize the usage of logging calls by automatically detecting the detailed structural correspondences and differences of a set of  source code fragments enclosing them.



\section{Broad thesis overview} \label{intro-overview}
I aim to create an approach that provides a description of where logging statements are used in source code by constructing generalizations that represent the structural similarities and differences between \name{methods} that make use of log statements, which I call \emph{logged methods} (LMs). In order to evaluate this idea, I implemented the approach to operate on programs written in the Java programming language. To determine how to construct generalizations using the syntax and semantics of the Java programming language, I looked to previous research conducted by \citet{2008:fse:cottrell} that determined the structural correspondences between two Java source code fragments through the application of approximated anti-unification, such that one fragment can be integrated with the other one for small-scale code reuse. However, my problem context is different, as I need to generalize a set of source code fragments with special attention to log statements. Therefore, my approach must take the logs into account when I perform the generalization task via the determination of structural correspondences.

%my study investigates the location of logging calls from the point of view of LMs.
%My approach employs a hierarchical clustering algorithm to create a generalization hierarchy from a set of LMs using a measure of similarity. It uses an approximated anti-unification algorithm to construct a structural generalization representing the similarities and differences between a pair of LMs. My anti-unification approach proceeds in three steps. First, it uses the Jigsaw framework \cite{2008:fse:cottrell} to determine all potential correspondences between the two LMs using a measure of similarity that relies on structural correspondences along with a simple knowledge of semantic equivalences in the Java language specifications. Second, it develops a greedy selection algorithm to approximate the best anti-unifier for my problem by determining the most similar correspondence for each substructure in my structures, applying some constraints in determining correspondences to prevent the anti-unification of logging calls with anything else. Third, it constructs an anti-unifier through the anti-unification of two structures and develops a measure of structural similarity between them.

%\RW{Your research is not about implementing a tool. Your research is about developing an approach, a concept, which you then implement in order to perform experiments.  The concept does not depend on Java, or the JDT, or Jigsaw: these are implementation-level choices.  If your implementation contains bugs, this does not immediately reflect on the concept.  If the concept has bugs, this will be reflected in the implementation.}
My approach to characterizing logging usage proceeds in four steps (as shown in Figure~\ref{fig:system_overview}). First, potential structural correspondences are determined between the abstract syntax trees (ASTs) of LMs in a pairwise manner, and stored in a novel structure: the \emph{anti-unifier AST} (AUAST), which allows the application of anti-unification on AST structures. Second, I use an approximated anti-unification algorithm to construct a structural generalization (an anti-unifier) representing the commonalities and differences between AUAST pairs, which employs a greedy selection algorithm to approximate the best anti-unifier for the problem by determining the most similar correspondence for each node. The anti-unification algorithm also applies some constraints prior to determining the best correspondences, in order to prevent the anti-unification of log statements with any other types of nodes in the tree structure. The anti-unifier is constructed through the anti-unification of each AUAST node with its best correspondence and then a measure of structural similarity is developed between the two AUASTs.
%The first two steps are described in Chapter~\ref{methodology}.
In the third step, I employ a hierarchical clustering algorithm to group the AUASTs into a number of clusters using the structural similarity measure and I then create a structural generalization from each cluster. The last step involves creating a detailed view of each structural generalization, which I called \emph{logging usage schema} (LUS), that represent the structural commonalities and differences between the set of LMs within each cluster. I manually went through the LUSs to characterize the location of logging statements in source code.
%examined manually

To evaluate the approach, I implemented it in a tool called \tool{ELUS}, written in the Java programming language. I used the Eclipse JDT framework to extract the AST of LMs from a Java program, and employed the \tool{Jigsaw} framework developed by \citet{2008:fse:cottrell} to find potential structural correspondences. My anti-unifier building tool (built atop Jigsaw) is applied to construct the structural generalizations, and my clustering tool is developed atop of it to perform the clustering algorithm .

To characterize logging usage using my approach, I applied \tool{ELUS} to the source code of four open-source software systems: \name{Tomcat}, \name{Hibernate}, \name{Camel}, and \name{Solr}. My analysis has resulted in five main categories of anti-unifiers in the logging usage. To evaluate the usefulness of my findings, I have conducted an empirical study to asses the performance of \tool{ELUS}. This experiment shows that \tool{ELUS} has an average precision of 84\% and an average recall of 80\%, and thus can be used to automatically construct the anti-unifiers of logging usage in source code.


%tool?

. %This step is realized by my clustering tool (Chapter~\ref{clustering}).


\begin{figure} [t]
  \centering\includegraphics [width = \textwidth]{Drawing4/SystemOverview.pdf}
  \caption{Overview of the approach. %\protect\RW{I don't want to see mention of tools here.  This should be focused on the concepts.}
  }
  \label{fig:system_overview}
\end{figure}

%most similar  or best correspondence?
%constraints wording?
%my empirical study?
%My tool has been applied on the source code of these systems and extracts all logged Java methods from these systems to construct the structural generalizations.
%My evaluation shows ...

\section{Thesis statement} \label{intro-stmt}
%The thesis of this work is that the detailed structural similarities and differences between source code fragments containing log statements can be determined via higher-order anti-unification modulo theories, providing a concise and accurate description of where logging calls do occur in real-world software systems.

The thesis of this work is to characterize where log statements occur in source code by constructing structural generalizations that describe the commonalities and differences between source code fragments containing log statements, thus providing the developers with some guidelines on where to use them effectively in source code.
%???

\section{Thesis organization} \label{intro-org}
The remainder of the thesis is organized as follows.

Chapter~\ref{ch2} motivates the problem of understanding where to use log statements in source code through a scenario in which a developer attempts to perform a logging task. This scenario outlines the potential problems she may encounter and illustrates that the current logging practice is not sufficiently supported.

Chapter~\ref{background} provides background information that I build atop: abstract syntax trees (ASTs), which are the basic structure I will use for describing software source code; the Eclipse JDT, an industrial framework for producing and manipulating ASTs for source code written in the Java programming language; anti-unification, which is a theoretical approach for constructing structural generalizations; and on Jigsaw, a research tool based on the Eclipse JDT for performing anti-unification.
%clustering added?


Chapters~\ref{background2},~\ref{methodology}, and~\ref{clustering} present the first three steps of my approach. Determining structural correspondences between AUASTs; constructing structural generalizations from an AUAST pair; and classifying a set of AUASTs into separate clusters, respectively. In each chapter, I discuss the implementation of my approach as an Eclipse plug-in, and conduct an experimental study to assess the effectiveness of my approach by applying its tool support on a sample test suite extracted from a real software system.



Chapter~\ref{eval} presents an empirical study I conducted to characterize the location of log statements of four open-source software systems. Chapter~\ref{diss} discusses the results and findings of my work, threats to its validity, and remaining issues. Chapter~\ref{rw} describes work related to my research problem and how it does not adequately address the problem. Chapter~\ref{conc} concludes the dissertation and presents the contributions of this study and future work. %Additional materials appended this dissertation are provided in Appendix A.




\addtocontents{toc}{\protect\addvspace{10pt}}
\chapter{Motivational Scenario}  \label{ch2}

Printing messages to the console or to a log file is an integral part of software development and can be used to test, debug, and understand what is happening inside an application. In Java programming language, \name{print} statements
are commonly used to print something on console. However, the availability of tools, frameworks, and APIs for logging that offers more powerful and advanced Java logging features, flexibility, and improvement in logging quality suggests that using \name{print} statements is not sufficient for real-world applications.


The logging framework offers many more features that are not possible using \name{print} statements. In most logging frameworks (e.g., \name{log4j}, \name{SLF4j}, \name{java.util.logging}), different verbosity levels of logging are available for use. That is, by logging at a particular log level, messages will get logged for that level and all levels above it and not for the levels below. As an example, \name{debug} log level messages can be used in a test environment, while \name{error} log level messages can be used in production environments. This feature not only produces fewer log messages, but also improves the performance of an application. In addition, most logging frameworks allow the production of formatted log messages, which makes it easier for a developer to monitor the behaviour of a system. Furthermore, when one is working on a server side application, the only way to know what is going on inside the server is by monitoring the log file. Although logging is a valuable practice for software development and maintenance, it imposes extra time and energy on developers to write, test, and run the code, while affecting the application performance. Since latency and speed are major concerns for most software systems, it is necessary  for a developer to understand and learn logging in great detail in order to perform it in an efficient manner.

To illustrate the inherent challenges of effectively performing logging practices in software systems, one may consider a scenario in which a developer is asked to log an event-based mechanism of a text editor tool written in the Java programming language. In this scenario, the developer is trying to log a Java class of this system (Figure~\ref{ch2-ex}) using the \name{log4j} logging framework. She knows that components of this application register with the \code{EditBus} class to receive messages that reflect changes in the application's state, and that the \code{EditBus} class maintains a list of components that have requested to receive messages. That is, when a message is sent using this class, all registered components receive it in turn. Furthermore, any classes that subscribe to the \code{EditBus} and implement the \code{EBComponent} interface define the method \code{EBComponent.handleMessage(EBMessage)} to handle a message sent on the \code{EditBus}. To perform this logging task, the developer might ask herself several fundamental questions, mostly related to where and what to log.

\begin{figure}[p]
\def\baselinestretch{1}
\begin{lstlisting}
public class EditBus {
    private static ArrayList components = new ArrayList();
    private static EBComponent[] copyComponents;
  
    private EditBus() {
    }
  
    public static void addToBus(EBComponent comp) {
        synchronized(components) {
            components.add(comp);
            copyComponents = null;
        }
    }

    public static void removeFromBus(EBComponent comp) {
        synchronized(components) {
            components.remove(comp);
            copyComponents = null;
        }
    }

    public static EBComponent[] getComponents() {
        synchronized(components) {
            if(copyComponents == null) {
                EBComponent[] arr = new EBComponent[components.size()];
                copyComponents = 
                    (EBComponent[])components.toArray(arr);
            }
        }
        return copyComponents;
    }
  
    public static void send(EBMessage message) {
        EBComponent[] comps = getComponents();
        for(int i = 0; i < comps.length; i++) {
            EBComponent comp = comps[i];
            long start = System.currentTimeMillis();
            comp.handleMessage(message);
            long time = (System.currentTimeMillis() - start);
        }
    }
}
\end{lstlisting}
\caption{The \code{EditBus} class.\label{ch2-ex}}
\end{figure}
%Developer's logging task context on

Her first solution might be to simply log at the start and end of every method. However, she believes that logging at the start and end of the \code{addToBus(EBComponent)}, \code{remove}\-\code{From}\-\code{Bus(EBComponent)}, and \code{getComponents()} methods are useless, and will produce redundant information. She assumes that the more she logs, the more she performs file I/O, which slows down the application. Therefore, she decides to log only important information necessary to debug or troubleshoot potential problems. She proceeds to identify the information needed to be logged and then decides on where to use logging calls. She thinks that it is important to log the information related to a message sent to a registered component, including the message content and the transmission time, to troubleshoot potential problems in sending messages. She simply wants to begin by using a logging call at the start of the \code{send()} method (line~2 of Figure~\ref{ch2-ex-logged-m1}) to log the information. However, she realizes that this logging call does not allow her to log the information she wants, as the \code{time} variable is not initialized at the beginning of this method. Therefore, she proceeds to examine the body of the \code{send()} method line-by-line and uses another logging call after the \code{time} variable is initialized inside an \code{if} statement that checks that the value of the variable \code{time} is not invalid (shown in lines~9--11 of Figure~\ref{ch2-ex-logged-m2}).

\begin{figure}[p]
\def\baselinestretch{1}
\begin{lstlisting}
public static void send(EBMessage message){
   //logging call
   EBComponent[] comps = getComponents();
   for (int i = 0; i < comps.length; i++) {
       EBComponent comp = comps[i];
       long start = System.currentTimeMillis();
       comp.handleMessage(message);
       long time = (System.currentTimeMillis() - start);
   }
}
\end{lstlisting}
\caption{The developer's initial determination of the usage of logging calls for the \code{send(EBMessage)} method.\label{ch2-ex-logged-m1}}
\end{figure}

\begin{figure}[p]
\def\baselinestretch{1}
\begin{lstlisting}
public static void send(EBMessage message) {
    // logging call
    EBComponent[] comps = getComponents();
    for(int i = 0; i < comps.length; i++) {
        EBComponent comp = comps[i];
        long start = System.currentTimeMillis();
        comp.handleMessage(message);
        long time = (System.currentTimeMillis() - start);
        if(time != 0){
            // logging call
        }
    }
}
\end{lstlisting}
\caption{The developer's second determination of the usage of logging calls for the \code{send(EBMessage)} method.\label{ch2-ex-logged-m2}}
\end{figure}

\begin{figure}[p]
\def\baselinestretch{1}
\begin{lstlisting}
public static void send(EBMessage message){
    try {
        // logging call
        EBComponent[] comps = getComponents();
        for(int i = 0; i < comps.length; i++) {
            EBComponent comp = comps[i];
            long start = System.currentTimeMillis();
            comp.handleMessage(message);
            long time = (System.currentTimeMillis() - start);
            if(time != 0){
                // logging call
            }
        }
    } catch(Throwable t) {
       // logging call
    }
}
\end{lstlisting}
\caption{The developer's third determination of the usage of logging calls for the \code{send(EBMessage)} method.\label{ch2-ex-logged-m3}}
\end{figure}

She also believes that it is important to log an error if any problems occur in sending messages to the components. She decides to use a \code{try}/\code{catch} statement, as it is a common way to handle exceptions in the Java programming language. She creates a \code{try}/\code{catch} block to capture the potential failure in sending messages, and uses a logging call inside the \code{catch} block to log the exception (shown in lines~2--16 of Figure~\ref{ch2-ex-logged-m3}). However, she realizes that using this logging call will not allow her to reach the desired functionality, as it does not reveal to which component the problem is related. Thus, she decides to relocate the \code{try}/\code{catch} block inside the \code{for} statement to log an error in case of a problem in sending messages to any components (shown in lines~5--15 of Figure~\ref{ch2-ex-logged-m4}).


\begin{figure}[p]
\def\baselinestretch{1}
\begin{lstlisting}
public static void send(EBMessage message) {
    // logging call
    EBComponent[] comps = getComponents();
    for (int i = 0; i < comps.length; i++) {
        try {
            EBComponent comp = comps[i];
            long start = System.currentTimeMillis();
            comp.handleMessage(message);
            long time = (System.currentTimeMillis() - start);
            if(time != 0) {
                // logging call
            }
        } catch(Throwable t) {
            // logging call
        }
    }
}
\end{lstlisting}
\caption{The developer's fourth determination of the usage of logging calls for the \code{send(EBMessage)} method.\label{ch2-ex-logged-m4}}
\end{figure}

\begin{figure}[p]
\def\baselinestretch{0.94}
\begin{lstlisting}
public class EditBus {
    private static ArrayList components = new ArrayList();
    private static EBComponent[] copyComponents;

    private EditBus() {
    }

    public static void addToBus(EBComponent comp) {
        synchronized(components) {
            components.add(comp);
            copyComponents = null;
        }
    }
  
    public static void removeFromBus(EBComponent comp) {
        synchronized(components) {
            components.remove(comp);
            copyComponents = null;
        }
    }

    public static EBComponent[] getComponents() {
        synchronized(components) {
            if(copyComponents == null) {
                EBComponent[] arr = new EBComponent[components.size()];
                copyComponents = (EBComponent[])components.toArray(arr);
            }
        }
        return copyComponents;
    }
  
    public static void send(EBMessage message) {
        // logging call
        EBComponent[] comps = getComponents();
        for(int i = 0; i < comps.length; i++) {
            try {
                EBComponent comp = comps[i];
                long start = System.currentTimeMillis();
                comp.handleMessage(message);
                long time = (System.currentTimeMillis() - start);
                if(time != 0) {
                    // logging call
                }
            } catch(Throwable t) {
                // logging call
            }
        }
    }
}
\end{lstlisting}
\caption{The developer's final determination of the usage of logging calls for the \code{EditBus} class.\label{ch2-ex-logged}}
\end{figure}

Figure~\ref{ch2-ex-logged} shows the developer's final determination to use logging calls to perform the logging task of the \code{EditBus} class. By making appropriate decisions about where to use logging calls, the developer is in a good position to proceed to write the logging messages by examining the remaining conceptually complex questions. Which information should I log? How should I choose the log message format? Which information goes to which level of logging? If the developer had reached this point more easily and quickly, she would have had more time and energy to make decisions about the remaining issues and could have completed the logging practice in a timely and appropriate manner.

%\RW{This is followed by a summary, so it is rather redundant.}
%To sum up, this scenario required the developer to make numerous decisions and then act on them. Her attention was split between detailed and the high-level decisions. In addition, time constraints and the performance of tedious tasks can cause the developer to make bad decisions. However, having a vision of where developers usually use logging calls in similar situations could guide her to make informed decisions about where to use logging calls, and she could then perform the logging task in a faster and less error-prone manner.


\section{Summary}  \label{ch2-summary}

This motivational scenario highlights the problems a developer may encounter in performing a logging task.
The core problem she faces in this scenario is the difficulty in understanding where to use logging calls that enable her to log the desired information. However, having an understanding of how developers usually log in similar situations might assist her to make informed decisions about where to use logging calls more quickly, and so she could pay more attention to the remaining, conceptually complex issues to complete the logging task.





\addtocontents{toc}{\protect\addvspace{10pt}}
%\chapter{Background}  \label{background}
\chapter{Background}  \label{background}
%The syntax of a programming language describes its structure, and t
The structure of a program can be described using its syntax and a Java source code can be represented as an instance of an Abstract Syntax Tree. To construct structural generalizations describing the correspondences and differences between logged Java classes, first we should understand what AST is and how specific information about each Java element is held in AST structure. Then we should investigate the application of anti-unification and its extensions on this structure to produce structural generalizations. We should also figure out how the Jigsaw framework could assist us in determining potential candidate structural correspondences.

Anti-unification is summarized in section~\ref{AU} starting with an introduction to unification and its dual anti-unification and followed by a discourse regarding to limitations of anti-unification to address our problem. 
Sections~\ref{AST} and~\ref{AUAST} of this chapter establish a brief description of the Abstract Syntax Tree (AST) structure and its extended form, necessary to understand the requirements that guided the development of an anti-unification algorithm for our application.
Section~\ref{HOAUMT} is dedicated to explain higher-order anti-unification modulo theories, an extension to anti-unification, in which a set of equivalence theories are defined and applied on higher-order extended structures to incorporate background knowledge. 
Afterwards we discuss the Jigsaw framework and its application in determining potential candidate structural correspondences in Section~\ref{Jigsaw}.

\section{Anti-unification}   \label{AU}

To describe unification theory and its dual anti-unification theory, we first introduce a formal definition of term, the application of a substitution on a term, and the definition of instance and anti-instance of a term, as the requirements needed to understand the theories. 

\begin{defn}[Term]\label{def:term} 
A term is a set of function symbols, variables, and constants, such that function symbols can come up with unlimited number of arguments. 
\end{defn}

In the definition of a term, function symbols are represented by identifiers starting with a lowercase letter (e.g., \vars{f(a,b)}), variables are represented by identifiers starting with an uppercase letter (e.g., \vars{X}, \vars{Y}), and constants are function symbols with no arguments (e.g., \vars{a}, \vars{b}). the followings are examples of term:
\begin{itemize} [leftmargin=0.7in]
\item \vars{Y}
\item \vars{a}
\item \vars{f(X, c)}
\item \vars{f(g(X, b),Y, g(a, Z))}
\end{itemize}
%A term is called grounded if it does not contain any variables (e.g., \vars{f(a,b)}), and 
\begin{defn}[Applying substitutions]\label{def:substitution} 
A substitution is a mapping from variables to terms, and the application of a substitution to a term would result in replacing all occurrences of each variable in the term by a proper subterm.
\end{defn}

As an example, an application of a substitution $\ominus$ =
\begin{tikzcd}[column sep=small] \vars{X} \arrow[r,->,shift right,""] & \vars{a} \end{tikzcd}
 on a term \vars{f(X,b)} is defined by replacing all occurrences of the variable \vars{X} by the term \vars{a} and thus\begin{tikzcd}[column sep=small] \vars{f(X,b)} \arrow[r,->,shift right,"\ominus"] & \vars{f(a,b)} \end{tikzcd}.


\begin{defn}[instance \& anti-instance]\label{def:instance} 
 \vars{a} is an instance of a term \vars{X} and \vars{X} is an anti-instance of \vars{a}, if there is a substitution $\ominus$ such that the application of $\ominus$ on \vars{X} results in \vars{a} (\begin{tikzcd}[column sep=small]\vars{X} \arrow[r,->,shift right,"\ominus"] & \vars{a}\end{tikzcd}).
\end{defn} 

\begin{defn}[Unifier]\label{def:unifier} 
An unifier is a common instance of two given terms.
\end{defn} 

Unification usually aims to create the Most General Unifier (MGU), that is, \vars{U} is MGU of two terms such that for all unifiers \vars{U}${\prime}$ there exist a substitution $\ominus$ such that \begin{tikzcd}[column sep=small] \vars{U} \arrow[r,->,shift right,"\ominus"] & \vars{U$\prime$} \end{tikzcd}. Unification has been used for various applications, however, it is not helpful to solve our problem as we need to construct generalizations based on the following description:

\begin{defn}[Generalization]\label{def:generalization} 
\vars{X} is a generalization for \vars{a} and \vars{b}, where \vars{X} is an anti-instance for  \vars{a} and \vars{b} under substitutions $\ominus_1$ and $\ominus_2$, respectively (\begin{tikzcd}[column sep=small] \vars{X} \arrow[r,->,shift right,"\ominus_1"] & \vars{a} \end{tikzcd} and\begin{tikzcd}[column sep=small] \vars{Y} \arrow[r,->,shift right,"\ominus_2"] & \vars{b} \end{tikzcd}).
\end{defn}
To create a generalization of two given terms, we should use the inverse of unification, which is called anti-unification, where two original terms are instances of new anti-unified term. 

\begin{defn}[Anti-unifier]\label{def:anti-unifier} 
An anti-unifier is a common generalization of two given terms.
\end{defn} 

An anti-unifier contains common pieces of the original terms, while the differences are abstracted away using variables. An anti-unifier for a pair of terms always exists since we can anti-unify any two terms by creating a variable \vars{X}. However, anti-unification usually aims to find the Most Specific Anti-unifier (MSA), that is , \vars{A} is MSA of two structures where there exist no anti-unifier \vars{A}${\prime}$ such that\begin{tikzcd}[column sep=small] \vars{A} \arrow[r,->,shift right,"\ominus"] & \vars{A$\prime$} \end{tikzcd}.

As an example, the anti-unifier of two given terms \vars{f(X,b)} and \vars{f(a,Y)} is the new term \vars{f(X,Y)}, containing common pieces of two original terms. The variable \vars{Y} in the anti-unifier \vars{f(X,Y)} can be substituted by the term \vars{b} to re-create \vars{f(X,b)} (with $\ominus_1$ =\begin{tikzcd}[column sep=small] \vars{Y} \arrow[r,->,shift right,"\ominus"] & \vars{b} \end{tikzcd}) and the variable \vars{X} in the anti-unifier can be substituted by the term \vars{a} to re-create \vars{f(a,Y)} 
(with $\ominus_2$ =\begin{tikzcd}[column sep=small] \vars{X} \arrow[r,->,shift right,"\ominus"] & \vars{a} \end{tikzcd}),  as depicted in Figure~\ref{fig:uni-anti-uni}. 
In addition, the unifier \vars{f(a,b)} of the two terms can be instantiated by applying the substitutions $\ominus_1'$ =\begin{tikzcd}[column sep=small] \vars{X} \arrow[r,->,shift right,"\ominus"] & \vars{a} \end{tikzcd} and $\ominus_2'$ =\begin{tikzcd}[column sep=small] \vars{Y} \arrow[r,->,shift right,"\ominus"] & \vars{b} \end{tikzcd} on the terms \vars{f(X,b)} and \vars{f(a,Y)}, respectively. 

\begin{figure} [H]
  \[
\begin{tikzcd}[column sep=small] 
&  
  {\makecell[l]{\hspace{0.5cm}f(X,Y)\\anti-unifier}}
  \arrow{dr}{\ominus_2 = X \rightarrow a}
  \arrow[->,swap]{dl}{\ominus_1 = Y \rightarrow b} % <-- reflect the direction of the hook
\\
f(X,b)
 \arrow[->,swap]{dr}{\ominus_1' = X \rightarrow a}
 %\arrow{dr}  
&&
f(a, Y)
  \arrow{dl}{\ominus_2' = Y \rightarrow b}
  \\
&
{\makecell[l]{f(a,b)\\unifier}} 
\end{tikzcd}
\]
  \caption{The unification and anti-unification of the terms \vars{f(X,b)} and \vars{f(a,Y)}.}
  \label{fig:uni-anti-uni}
\end{figure}

MSA should preserve as much of common pieces of both original terms as possible, however, anti-unification fails to capture complex commonalities as it restricts substitutions to replace only first-order variables by terms. That is, when two terms differ in function symbols, anti-unification fails to capture common details of them. For example, the anti-unifier of the terms \vars{f(a,b)} and \vars{g(a,b)} is \vars{X} using anti-unification as depicted in Figure~\ref{fig:first-anti-uni}.

\begin{figure} [H]
\[
\begin{tikzcd}[column sep=small] 
&  
  X
  \arrow{dr}{\ominus_2 = X \rightarrow g(a,b)}
  \arrow[->,swap]{dl}{\ominus_1 = X \rightarrow f(a,b)} % <-- reflect the direction of the hook
\\
f(a,b)
&&
g(a,b)  
\end{tikzcd}
\]
  \caption{The anti-unification of the terms \vars{f(X,b)} and \vars{f(a,Y)}.}
  \label{fig:first-anti-uni}
\end{figure}

An extended form of anti-unification, which is called higher-order anti-unification, would allow us to create MSA by extending the set of possible substitutions such that variables can be replaced by not only constants but also functional symbols to retain the detailed commonalities. For example, the anti-unifier of the terms \vars{f(a,b)} and \vars{g(a,b)} is \vars{X(a,b)} using higher-order anti-unification as depicted in Figure~\ref{fig:higher-anti-uni}.
\begin{figure} [H]
\[
\begin{tikzcd}[column sep=small] 
&  
  X(a,b)
  \arrow{dr}{\ominus_2 = X \rightarrow g}
  \arrow[->,swap]{dl}{\ominus_1 = X \rightarrow f} % <-- reflect the direction of the hook
\\
f(a,b)
&&
g(a,b)  
\end{tikzcd}
\]
  \caption{The higher-order anti-unification of the terms \vars{f(X,b)} and \vars{f(a,Y)}.}
  \label{fig:higher-anti-uni}
\end{figure}

In the following sections, we will provide a brief description of AST structures and the application of anti-unification on its extended form to construct structural generalizations.

\section{Abstract Syntax Tree}   \label{AST}
% JDT
The Eclipse Java Development Tools (JDT) framework provides APIs to access and manipulate Java source code via Abstract Syntax Tree (AST). AST maps Java source code in a tree structure form and thus every Java source code can be represented as tree of AST nodes, where each represents an element of the Java Programming Language. AST helps developers to modify and analyze the Java program in a more convenient way than text-bases source code by providing a language parser of the Java source code, determining the bindings between name and type references, and providing specific information of each Java element. For example, the simple AST structure of two sample logged Java classes in Figures~\ref{ch3-ex1} an~\ref{ch3-ex2} is shown in Figure~\ref{fig:ast}

%We introduce an example to illustrate the anti-unification process over two ASTs, and so 

\begin{figure}[H]
\def\baselinestretch{1}
\begin{lstlisting}
public abstract class EBPlugin extends EditPlugin implements EBComponent {
  private Boolean seenWarning;
  protected EBPlugin(){
  }
  public void handleMessage(  EBMessage message){
    if (seenWarning)     return;
    seenWarning=true;
    Log.log(Log.WARNING,this,getClassName() + " should extend" + " EditPlugin not EBPlugin since it has an empty"+ " handleMessage()");
  }
}
\end{lstlisting}
\caption{A Java class that uses a logging call. This will be referred to as Example 1.\label{ch3-ex1}}
\end{figure}


\begin{figure}[H]
\def\baselinestretch{1}
\begin{lstlisting}
public static class Wrapper implements ActionListener {
  private ActionContext context;
  private String actionName;
  public Wrapper(  ActionContext context,  String actionName){
    this.context=context;
    this.actionName=actionName;
  }
  public void actionPerformed(  ActionEvent evt){
    EditAction action=context.getAction(actionName);
    if (action == null) {
      Log.log(Log.ERROR,this,"Unknown action: " + actionName);
    }
    else     context.invokeAction(evt,action);
  }
}
\end{lstlisting}
\caption{A Java class that uses a logging call. This will be referred to as Example 2.\label{ch3-ex2}}
\end{figure}

\begin{figure} [H]
  \centering\includegraphics [width = \textwidth]{Drawing4/AST.pdf}
  \caption{Simple AST structure of examples in Figures~\ref{ch3-ex1} and~\ref{ch3-ex2}.}
  \label{fig:ast}
\end{figure}

In the JDT framework, structural properties of each AST node can be used to obtain specific information of the Java element it represents. These properties are stored in a map data structure that associates each property to its value and are divided into three types:
\begin{itemize} [leftmargin=0.7in]
\item \textit{Simple structural properties:} that contain a simple value which has a primitive or simple type or a basic AST constant (e.g., identifier property of a name node whose value is a String)
\item \textit{Child structural properties:} where the value is a single AST node (e.g., name property of a method declaration node)
\item \textit{Child list structural properties}: where the value is a list of child nodes (e.g., body declarations property of a class declaration node whose value is a list of body declaration nodes, including method declaration and field declaration nodes.)
\end{itemize}
%AST is made up of AST nodes as subtrees and simple values as leaves.
An instance of an AST structure can be represented in an abstract form that can be mapped to the definition of a term described in Section~\ref{AU}. As an example, the abstract form of ASTs of logging calls of Java classes in Figures~\ref{ch3-ex1} and~\ref{ch3-ex2} can be represented as:
\begin{itemize} [leftmargin=0.7in]
\item \textit{expression(expression(Log),name(log),arguments(leftoperand(message),+,\\rightoperand(" is empty"),qualifier(Log),name(WARNING)))}
\item \textit{expression(expression(Log),name(log),arguments(leftoperand(actionName),\\+,rightoperand("is an unknown action"),qualifier(Log),name(WARNING)))}
\end{itemize}

Where ASTNodes (e.g., \textit{expression, name,, qualifier}) might be viewed as function symbols and simple values (e.g., \textit{log, WARNING}) might be viewed as constants in the term definition. As described in Section~\ref{AU}, anti-unification utilizes variables that must be substituted with proper structures to re-create original structures. However, the AST structure does not contain any variables and so we need to construct an extended form of AST, which will be described in the following section.

% In order to map an instance of an AST to that form of structure, first we should understand how an AST structure holds its subtrees and leaves
%The goal of this phase is to construct an extension of the AST structure that would allow the creation of an anti-unified structure. 

\section{Constructing the AUAST} \label{AUAST}  
AUAST (Anti-unified AST) is an extended form of AST that allows the insertion of variables in place of any node in the tree structure, including both subtrees and leaves, to indicate variations between original structures. The AUAST addresses the limitations of AST to construct an anti-unifier by adding the following structural properties:
\begin{itemize} [leftmargin=.4in]
\item \textit{Simple Variable Property}: an extension of simple property referring to two simple values to allow the insertion of variables in place of leaves.
\end{itemize}
\begin{itemize} [leftmargin=.4in]
\item \textit{Child Variable Property}: an extension of child property referring to two child AST nodes to allow the insertion of variables in place of subtrees.
\end{itemize}
The anti-unification of AUASTs of logging calls in Figures~\ref{ch3-ex1} and~\ref{ch3-ex2} is depicted in Figure~\ref{fig:logging-anti}. The new variables \vars{X} and \vars{Y} are created to abstract away the structural variations. 
%\item it can be mapped to our recursive definition of a term, where AST nodes and simple values may be viewed as function-symbols and constants, respectively


\begin{figure} [H]
\[
\begin{tikzcd}[column sep=small] 
&  
{\makecell[l]{ expression(expression(Log),\\name(log),arguments(\\leftoperand(X),+,\\rightoperand(Y),\\qualifier(Log),\\name(WARNING)))}}
  \arrow{dr}{\ominus_2 = (X \rightarrow actionName, Y \rightarrow "is an unknown action")}
  \arrow[->,swap]{dl}{\ominus_1 = (X \rightarrow message, Y \rightarrow " is empty")} % <-- reflect the direction of the hook
\\
{\makecell[l]{expression(expression(Log),\\name(log),arguments(\\leftoperand(message),+,\\rightoperand(" is empty"),\\qualifier(Log),\\name(WARNING)))}}
&&
{\makecell[l]{
expression(expression(Log),\\name(log),arguments(\\leftoperand(actionName),\\+,rightoperand("is an unknown action"),\\qualifier(Log),\\name(WARNING))) }}
\end{tikzcd}
\]
  \caption{ The anti-unification of AUASTs of logging calls in the Examples 1 and 2.}
  \label{fig:logging-anti}
\end{figure}
%The AUASTs of log Method Invocation nodes from the Java classes in Figure~\ref{ch3-ex1} and Figure~\ref{ch3-ex2}.

Applying higher-order anti-unification on AUAST structures could help us in constructing a structural generalization by maintaining the common pieces and abstracting the differences away using variables. However, it is not comprehensive enough to solve our problem as it does not consider background knowledge about AST structures, such as syntactically different but semantically relevant structures, missing structures, and different ordering of arguments. In the following section, we will look at an extension of anti-unification, higher-order anti-unification modulo theories, and how it can sufficiently address the limitations of anti-unification in our context.


\section{Higher-order anti-unification modulo theories}   \label{HOAUMT}
%Anti-unification cannot incorporate any background knowledge such as sematic knowledge required to solve our problem, and we should apply an extended form of anti-unification, called higher-order anti-unification modulo theories, where a set of equivalence equations is defined to incorporate semantic knowledge of structural equivalences supported by the Java language specification. An equivalence equation $=_E$ determines which terms are considered equal, and the set of equivalence equations must be applied on higher-order extended structures to allow the anti-unification of AST structures that are not identical but are semantically equivalent.
In higher-order anti-unification modulo theories, a set of equivalence equations is defined to incorporate background knowledge. Each equivalence equation $=_E$ determines which terms are considered equal and a set of these equations can be applied on higher-order extended structures to determine structural equivalences. For example, we have introduced an equivalence equation $=_E$, such that F(X,Y) $=_E$ F(Y,X) to indicate that the ordering of arguments does not matter in our context.

% nil restriction!!!!
We have also introduced a theory, called NIL- theory, that adds the concept of NIL structure, which is defined to create a structure out of nothing, and defines an equivalence equation $=_E$ for it. The NIL structure can be used to anti-unify two structures when a substructure exists in one but is missing from the other. However, some requirements should be taken to avoid the overuse of NIL structures such that the original structures must have common substructures but vary in the size for dissimilar substructures. For example, we can anti-unify the two structures \vars{b} and \vars{f(a,b)} through the application of NIL-theory by creating the term \vars{nil(nil,b)} which is $=_E$ to \vars{f(b)} and anti-unifying \vars{nil(nil,b)} with \vars{f(a,b)} as depicted in Figure~\ref{fig:anti-nil}.
% to introduce an equivalence equation =E for the NIL structure
% it should be modified
% should it be different?
\begin{figure} [H]
\[
\begin{tikzcd}[column sep=small] 
&  
  X(Y,b)
  \arrow{dr}{\ominus_2 = ( X \rightarrow nil, Y \rightarrow nil) }
  \arrow[->,swap]{dl}{\ominus_1 = ( X \rightarrow f, Y \rightarrow a)} % <-- reflect the direction of the hook
\\
f(a,b)
&&
nil(nil,b) =_E  b 
\end{tikzcd}
\]
  \caption{ The anti-unification of the terms f(a, b) and nil(nil,b).}
  \label{fig:anti-nil}
\end{figure}


We have also defined a set of equivalence equations to incorporate semantic knowledge of structural equivalences supported by the Java language specification as it provides various ways to define the same language specifications. These theories should be applies on higher-order extended structures to anti-unify AST structures that are not identical but are semantically equivalent. For example, consider for- and while- statements that are two types of looping structure in Java programming language that have different syntax but semantically cover the same concept. Let us look at the \vars{$for(i=0;i<10;i++)$} and \vars{$while(i<10)$} code snippets, whose AST structures can be represented as \vars{$for(initializer(i,=,0),expression(i,<,10), updaters(i,++))$} and \vars{$while(expression(i,<,10))$}, respectively. We could define an equivalence equation $=_E$ that allows the anti-unification of for- and while- statements which are semantically similar structures. We also need to utilize the NIL-theory to handle varying number of arguments as the for- loop has three arguments whereas the while- loop has one. Using the NIL-theory we can create the structure \vars{$while(nil(nil,nil,nil),expression(i,<,10), nil(nil,nil))$} that is $=_E$ to \vars{$while(expression(i,<,10))$} and construct the anti-unifier, \vars{$V_0(V_1(V_2,V_3,V_4),expression(i,<,10), V_5(V_2,V_6))$} as depicted in Figure~\ref{fig:for-while}.

\begin{figure} [H]
\[
\begin{tikzcd}[column sep=small] 
&  
  \makecell[l]{V_0(V_1(V_2,V_3,V_4),\\expression(i,<,10),\\ V_5(V_2,V_6)) }
  \arrow{dr}{\ominus_2 = \makecell[l]{ V_0 \rightarrow while, V_1 \rightarrow nil,\\ V_2 \rightarrow nil, V_3 \rightarrow nil,\\ V_4 \rightarrow nil, V_5 \rightarrow nil, V_6 \rightarrow nil} }
  \arrow[->,swap]{dl}{\ominus_1 = \makecell[l]{V_0 \rightarrow for, V_1 \rightarrow initializer, \\V_2 \rightarrow i, V_3 \rightarrow =, V_4 \rightarrow 0, \\V_5 \rightarrow updaters, V_6 \rightarrow ++ }} % <-- reflect the direction of the hook
\\
  \makecell[l]{for(initializer(i,=,0),\\expression(i,<,10),\\ updaters(i,++))} 
&&
  \makecell[l]{while(nil(nil,nil,nil),\\expression(i,<,10),\\ nil(nil,nil)) } 
\end{tikzcd}
\]
  \  \caption{ The anti-unification of the structures \vars{$for(initializer(i,=,0),expression(i,<,10), updater(i,++))$} and \vars{$while(nil(nil,nil,nil),expression(i,<,10), nil(nil,nil))$}.}
  \label{fig:for-while}
\end{figure}
% higher-order extension and the equational theories
% provide a better example
However, defining complex substitutions in higher-order anti-unification modulo theories results in losing the uniqueness of MSA. For example, consider the terms \vars{$f(g(a,e))$} and \vars{$f(g(a,b),g(d,e))$}. As described in Figure~\ref{fig:multipleMSA}, two MSAs exist for these terms: we can anti-unify \vars{$g(a,e)$} and \vars{$g(a,b)$} to create the anti-unifier \vars{$g(a,X_0)$} and anti-unify \vars{$g(d,e)$} with the NIL structure to create the anti-unifier \vars{$Y(Z,X_1)$}; or we can anti-unify \vars{$g(a,e)$} and \vars{$g(d,e)$} to create the anti-unifier \vars{$g(X_0,e)$} and anti-unify \vars{$g(a,b)$} with the NIL structure to create the anti-unifier \vars{$Y(Z,X_1)$}.

\begin{figure} [H]
\[
\begin{tikzcd}[column sep=small] 
&  
  f(g(X_0,e),Y(Z,X_1))
  \arrow{dr}{\ominus_2 = ( X_0 \rightarrow a, Y \rightarrow nil, Z \rightarrow nil, X_1 \rightarrow nil) }
  \arrow[->,swap]{dl}{\ominus_1 = ( X_0 \rightarrow d, Y \rightarrow g, Z \rightarrow a, X_1 \rightarrow b)} % <-- reflect the direction of the hook
\\
f(g(a,b), g(d,e))
&&
f(g(a,e)) 
\end{tikzcd}
\]	

\[
\begin{tikzcd}[column sep=small] 
&  
  f(g(a,X_0),Y(Z,X_1))
  \arrow{dr}{\ominus_2 = ( X_0 \rightarrow e, Y \rightarrow nil, Z \rightarrow nil, X_1 \rightarrow nil) }
  \arrow[->,swap]{dl}{\ominus_1 = ( X_0 \rightarrow b, Y \rightarrow g, Z \rightarrow d, X_1 \rightarrow e)} % <-- reflect the direction of the hook
\\
f(g(a,b), g(d,e))
&&
f(g(a,e)) 
\end{tikzcd}
\]
  \caption{The anti-unification of the terms \vars{f(g(a,b), g(d,e))}
and \vars{f(g(a,e))} that creates multiple MSAs.}
  \label{fig:multipleMSA}
\end{figure}


Despite having multiple potential MSAs, we need to determine one single MSA that is the most appropriate in our context. However, the complexity of finding an optimal MSA is undecidable in general [Cottrell et al., 2008] since an infinite number of possible substitutions can be applied on every variable. Therefore, we need to use an approximation technique to construct one of the best MSAs that can sufficiently solve our problem.
% reason since?
%Our goal is to find an MSA that is an approximation of the best fit to our application


\section{The Jigsaw framework}  \label{Jigsaw}
The Jigsaw tool is developed by Cottrell et al. [2008] to determine the structural correspondences between two Java source code fragments through the application of higher-order anti-unification modulo theories such that one fragment can be integrated to the other one for small scale code reuse. Jigsaw could help us to determine potential candidate structural correspondences between AST nodes of logged Java classes by producing an augmented form of AST, called CAST (Correspondence AST), where each node holds a list of candidate correspondence connections between the two structures, each representing an anti-unifier. It also develops a measure of similarity to indicate how similar the nodes involved in each correspondence connection are. The Jigsaw similarity function relies on structural correspondence along with a simple knowledge of semantic equivalences supported by the Java language specification, and it returns a value between 0 and 1 that indicates zero and total structural matching, respectively. In addition, several semantical heuristics are used to improve the accuracy of similarity measurement by allowing the comparison of AST nodes that are not syntactically identical but are semantically related to each other.

As an example, the similarity between names of AST nodes is measured using a normalized computation based on the length of longest common substring. Another example is the comparison of \vars{int} and \vars{long} variable types, where an arbitrary value of 0.5 is defined as the similarity value as they are not syntactically identical but are not semantically unrelated. In addition, the Jigsaw framework also detects the structural correspondence between  for-, enhanced-for-, while-, and do- loop statements; and if- and switch- conditional statements. As an example, Figure~\ref{fig:meth-ast-1} shows the structural correspondence connections created by Jigsaw between the AST nodes of Examples 1 and 2 along with the similarity value for each correspondence connection.

\begin{figure} [H]
  \centering\includegraphics [width = \textwidth]{Drawing4/FirstCorr.pdf}
  \caption{Simple CAST structure of examples in Figures~\ref{ch3-ex1} and~\ref{ch3-ex2}. The links between AST nodes indicate structural correspondence connections created by the Jigsaw framework along with the similarity value.}
  \label{fig:meth-ast-1}
\end{figure}

However, the Jigsaw tool does not suffice to construct an anti-unifier that is the best fit to our application. In addition, the Jigsaw similarity function does not measure the similarity of two logged Java classes with a focus on logging calls, which is needed in our context. To address these issues, we should develop a greedy selection algorithm to approximate the best anti-unifier by determining the best correspondence for each node. In the following chapter, we will discuss our approach to construct structural generalizations and our implementation by means of the higher-order anti-unification modulo theories and the Jigsaw framework.

 
\section{Summary}  \label{summary}
In this chapter, we described anti-unification as a technique to construct a common generalization of two given terms. We have also introduced an extended form of anti-unification, which is called higher-order anti-unification modulo theories, where a set of equivalence equations can be applied on higher-order extended structures to incorporate background knowledge. In addition,
we provided a brief description of AST that maps Java source code in a tree structure form, and why an extended form of it, named AUAST, is required to create higher-order structures specific to our problem context. Finally, we discuss the Jigsaw framework and how it could assist us in determining the potential structural correspondences. 

\addtocontents{toc}{\protect\addvspace{10pt}}
\chapter{Background: Eclipse JDT and Jigsaw}\label{background2}

In order to construct structural generalizations describing the commonalities and differences between logged Java methods (LJMs), we need a concrete framework for constructing and manipulating abstract syntax trees (ASTs).  The Eclipse integrated development environment provides such a framework in its Java Development Tools (JDT) component.  The details of our implementation will depend on certain details of Eclipse JDT, so we describe those in Section~\ref{JDT}.

A framework exists for determining structural correspondences between AST nodes and measuring similarity between them, called Jigsaw \cite{2008:fse:cottrell}, which is described in Section~\ref{Jigsaw}. We build atop that work in order to create our anti-unifiers. Furthermore, an empirical study was conducted to assess how Jigsaw could effectively be used to address these issues.

\section{Eclipse JDT}\label{JDT}

The Eclipse Java Development Tools (JDT) framework provides APIs to access and manipulate Java source code via ASTs. An AST represents Java source code in a tree form, where the typed nodes represent instances of certain syntactic structures from the Java programming language.  Each node type (in general) takes a set of child nodes, also typed and with certain constraints on their properties.  Groups of children are named on the basis of the conceptual purpose of those groups; optional groups can be empty, which we can represent with the \NIL{} element. Thus, any Java source code can be represented as a tree of AST nodes. For example, the simple AST structure of two sample LJMs in Figures~\ref{ch3-ex1} an~\ref{ch3-ex2} is shown in Figure~\ref{fig:ast}

\begin{figure}[p]
\def\baselinestretch{1}
\begin{lstlisting}
public abstract class EBPlugin extends EditPlugin implements EBComponent {
    private Boolean seenWarning;

    protected EBPlugin() {
    }

    public void handleMessage(EBMessage message) {
        if(seenWarning) return;
        seenWarning = true;
        Log.log(Log.WARNING, this, getClassName() + " should extend EditPlugin not EBPlugin since it has an empty " + handleMessage());
    }
}
\end{lstlisting}
\caption{A Java class that uses a logging call. This will be referred to as Example 1.\label{ch3-ex1}}
\end{figure}

\begin{figure}[p]
\def\baselinestretch{1}
\begin{lstlisting}
public static class Wrapper implements ActionListener {
    private ActionContext context;
    private String actionName;

    public Wrapper(ActionContext context,  String actionName) {
        this.context = context;
        this.actionName = actionName;
    }

    public void actionPerformed(ActionEvent evt) {
        EditAction action = context.getAction(actionName);
        if(action == null) {
            Log.log(Log.ERROR, this, "Unknown action: " + actionName);
        }
        else
            context.invokeAction(evt, action);
    }
}
\end{lstlisting}
\caption{A Java class that uses a logging call. This will be referred to as Example 2.\label{ch3-ex2}}
\end{figure}

\begin{figure} [p]
  \centering\includegraphics[width = \textwidth]{Drawing4/AST.pdf}
  \caption{Simple AST structure of examples in Figures~\ref{ch3-ex1} and~\ref{ch3-ex2}.}
  \label{fig:ast}
\end{figure}

In the JDT framework, structural properties of each AST node can be used to obtain specific information of the Java element that it represents. These properties are stored in a map data structure that associates each property to its value; this data is divided into three types:
\begin{itemize} [leftmargin=0.7in]
\item \textit{Simple structural properties:} These contain a simple value which has a primitive or simple type or a basic AST constant (e.g., identifier property of a name node whose value is a String).  For example, all the \textit{Identifier} nodes in Figure~\ref{fig:java-example-ast} fall in this case; each references an instance of \code{String} representing the string that constitutes the identifier.
\item \textit{Child structural properties:} These involve situations where the value is a single AST node (e.g., name property of a method declaration node).  For example, the \textit{ClassDeclaration} node in Figure~\ref{fig:java-example-ast} has a single child that represents its name as an \textit{Identifier} node; this would be a child structural property.
\item \textit{Child list structural properties}: These involve situations where the value is a list of child nodes.  For example, the \textit{ClassDeclaration} node in Figure~\ref{fig:java-example-ast} can possess multiple \textit{Modifier}s; these are recorded in the \textit{ClassDeclaration} as a child list structural property.
\end{itemize}

As an example, the ASTs of the logging calls at line~10 of Figure~\ref{ch3-ex1} and line~13 of Figure~\ref{ch3-ex2} can be represented respectively as:
\begin{itemize} [leftmargin=0.7in]
%\RW{These ASTs were messed up.  I fixed them according to what the code says.}
%\item \textit{expression(expression(Log), name(log), arguments(leftoperand(message), +, rightoperand(" is empty"), qualifier(Log), name(WARNING)))}
%\item \textit{expression(expression(Log), name(log), arguments(leftoperand(actionName),\\+, rightoperand("is an unknown action"), qualifier(Log), name(WARNING)))}
\item \textit{MethodCall}(\\
\hspace*{1em}\textit{QualifiedName}(\code{Log}, \textit{Identifier}(\code{log})),\\
\hspace*{1em}\textit{Arguments}(\\
\hspace*{2em}\textit{QualifiedName}(\code{Log}, \mbox{\textit{Identifier}(\code{WARNING})}),\\
\hspace*{2em}\textit{ThisExpression}(),\\
\hspace*{2em}\textit{AdditionExpression}(\\
\hspace*{3em}\textit{MethodInvocation}(\textit{Identifier}(\code{getClassName}), \textit{Arguments}()),\\
\hspace*{3em}\textit{StringLiteral}(\code{" should extend EditPlugin not EBPlugin since it has an empty "}),\\
\hspace*{3em}\textit{MethodInvocation}(\textit{Identifier}(\code{handleMessage}), \textit{Arguments}()))))\\
\item \textit{MethodInvocation}(\\
\hspace*{1em}\textit{QualifiedName}(\code{Log}, \textit{Identifier}(\code{log})),\\
\hspace*{1em}\textit{Arguments}(\\
\hspace*{2em}\textit{QualifiedName}(\code{Log}, \mbox{\textit{Identifier}(\code{ERROR})}),\\
\hspace*{2em}\textit{ThisExpression}(),\\
\hspace*{2em}\textit{AdditionExpression}(\\
\hspace*{3em}\textit{StringLiteral}(\code{"Unknown action: "}),\\
\hspace*{3em}\textit{Identifier}(\code{actionName}))))\\
\end{itemize}

\section{The Jigsaw framework}\label{Jigsaw}
The Jigsaw tool was developed by \citet{2008:fse:cottrell} to determine the structural correspondences between two Java source code fragments through the application of higher-order anti-unification modulo equational theories such that one fragment can be integrated to the other one for small-scale code reuse. Jigsaw could help determine potential candidate structural correspondences between AST nodes of LJMs by producing an augmented form of AST, called a \emph{correspondence AST} (CAST), where each node holds a list of candidate correspondence connections between the two structures, each implicitly representing an anti-unifier. Jigsaw also provides a measure of structural similarity to indicate how similar the nodes involved in each correspondence connection are. The Jigsaw similarity function relies on structural correspondence along with simple knowledge of semantic equivalences supported by the Java language specification. It returns a value in $[0, 1]$ where zero indicates complete lack of similarity and one indicates perfect similarity. In addition, several semantical heuristics are used to improve the accuracy of similarity measurement by allowing the comparison of AST nodes that are not syntactically identical but are semantically related to each other.

For example, the similarity between names of AST nodes is measured using a normalized computation based on the length of the longest common substring. The comparison of \code{int} and \code{long} types is another example, where an arbitrary value of 0.5 is defined as the similarity value as they are not syntactically identical but are semantically related. In addition, the Jigsaw framework also detects the structural correspondence between  \code{for}-, enhanced-\code{for}-, \code{while}-, and \code{do}-loop statements; and \code{if} and \code{switch} conditional statements. As an example, Figure~\ref{fig:meth-ast-1} shows the structural correspondence connections created by Jigsaw between the AST nodes of Examples 1 and 2 along with the similarity value for each correspondence connection.

\begin{figure} [H]
  \centering\includegraphics [width = \textwidth]{Drawing4/FirstCorr.pdf}
  \caption{Simple CAST structure of examples in Figures~\ref{ch3-ex1} and~\ref{ch3-ex2}. The links between AST nodes indicate structural correspondence connections created by the Jigsaw framework along with the similarity value.}
  \label{fig:meth-ast-1}
\end{figure}

However, the Jigsaw tool does not suffice to construct an anti-unifier that is the best fit to our application. In addition, the Jigsaw similarity function does not measure the similarity of two LJMs with a focus on logging calls, which is needed in our context. To address these issues, we should develop a greedy selection algorithm to approximate the best anti-unifier by determining the best correspondence for each node. In the following chapter, we will discuss our approach to construct structural generalizations and our implementation by means of the higher-order anti-unification modulo theories and the Jigsaw framework.


\section{An Assessment of Jigsaw}\label{jigsaw-assessment}

\RW{Describe here the procedure you used to select the examples, etc., how you tested Jigsaw, and what your findings were. At some point, you complained that Cottrell had not done something right ... do you have any evidence to demonstrate it?  How does this affect your work?  Such points can go in a discussion section towards the end of this chapter if they don't fit otherwise.  Full details of examples can go in an appendix; here, just describe enough so people can get the point.} 

\NZ{I think this experiment is not much about evaluating Jigsaw (The evaluation was conducted by Cottrell), but it is more about understanding what Jigsaw does and how to use it for my application . What I have mentioned was about detecting relevance links which is not related to my work. During my development, I added some statements to Jigsaw for the cases that was not covered in his work completely (e.g, Jigsaw did not detect the correspondence between inner type declarations of two nested type declarations when comparing the two upper type declarations } 

%chapter{Experimental Studies}  \label{studies}
%To evaluate our approach, we have implemented a tool, and conducted three empirical studies on a set of LJMs extracted from a real-world software system. In this section, we describe our experimental setup, present our studies, and discuss the results. 

We have conducted an experiment on a set of LJMs extracted from a real-world software system to assess how Jigsaw could effectively help us determine potential correspondences between AST nodes and measure similarity between them.
We implemented a plug-in to the Eclipse integrated development environment (IDE), which uses the \name{JDT} framework to extract ASTs of a pair of LJMs and applies the \name{Jigsaw} framework to generate correspondence connections between AST nodes. 

%the correspondence tool as
%\subsubsection{Experimental Setup}  \label{study1_setup}
\subsubsection{Setup}  \label{study1_setup}
%Our tool is a plug-in to the Eclipse integrated development environment (IDE) that implements our algorithm. The tool consists of three main components: a correspondence tool, an antiunifier-building tool, and a clustering tool. The correspondence tool inputs a pair of LJMs, uses the Jigsaw framework to determine potential correspondences between their AST nodes, and outputs the generated CASTs and the Jigsaw similarity between them. The antiunifier-building tool inputs a pair of LJMs, applies our anti-unification algorithm to construct an anti-unifier with a special attention to logging calls, and outputs the detailed view of anti-unifier and similarity measure (as described in Section~\ref{meth-antiUnifier}). The clustering tool inputs a set of LJMs, applies a hierarchical clustering algorithm to classify them based on the similarity measurement, and outputs the detailed view of the generated anti-unifier for each cluster (as described in Section~\ref{meth-clustering}). 
% figure of architecture?

As a subject for our study, we used \name{jEdit}, a programmer’s text editor tool written in Java programming language. We chose this subject because it is a real program that has been used constantly by many developers, and it employs real usage of logging calls. Our tool extracts all LJMs within the source code of this program. However, a subset of them was selected containing 9 LJMs that showed varying levels of similarity on manual examination, and it has been used as a test suite throughout this study (see Table~\ref{table:ljms}). The org.gjt.sp.jedit.EditBus.send(...) method contains two logging calls. To handle this case we split it into two cases: case 3 contained the first logging call while the second one was removed; case 4 contained only the second logging call. We will describe our approach for LJMs containing multiple logging calls in details in Section~\ref{meth-multipleLogs}. The last three LJMs were manually modified by adding some statements for the sake of dealing with important cases that we otherwise would have missed testing. Case 8 simulates the addition of an \code{if- }statement that formed a nested \code{if- }statement enclosing a logging call. Cases 9 and 10 simulate the addition of statements to improve the test coverage. 


\begin{figure} [H]
  \centering
  \begin{tabular}{|c|l|c|}
    \hline
    Case & Logged Java methods & Size(LOC)\\
    \hline
    1& org.gjt.sp.jedit.PluginJAR.generateCache() &104\\   
   \hline
    2& org.gjt.sp.jedit.MiscUtilities.isSupportedEncoding(...) &9\\   
   \hline
    3& org.gjt.sp.jedit.EditBus.send(...) &14\\   
   \hline
    4& org.gjt.sp.jedit.EditBus.send(...)* &14\\   
   \hline
    5& org.gjt.sp.jedit.EditAction.Wrapper.actionPerformed(...) &5\\   
   \hline
    6& org.gjt.sp.jedit.EBPlugin.handleMessage(...) &6\\   
   \hline
    7& org.gjt.sp.jedit.BufferHistory.RecentHandler.doctypeDecl(...) &3\\   
   \hline
    8& org.gjt.sp.jedit.JARClassLoader.loadClass(...) &32\\   
   \hline
    9& org.gjt.sp.jedit.io.VFS.DirectoryEntry.RootsEntry.rootEntry(...) &36\\   
   \hline
    10& org.gjt.sp.jedit.ServiceManager.loadServices(...) &20\\   
    \hline
  \end{tabular}
  \caption{Logged Java methods used as our test suite; all are contained in the \name{org.gjt.sp.jedit} package.}
  \label{table:ljms}
\end{figure}


%\subsubsection{Setup}  \label{study1-setup}
Our tool was used to compare LJMs of our test suite in a pairwise manner (55 test cases in total, including self-comparisons) and to produce the CASTs of each pair. The Jigsaw similarity was also measured for each of these test cases.
We examined the generated CASTs of these test cases and selected a subset of 4 cases with various levels of correspondences as depicted in the Table~\ref{jigsaw_4_test_cases}. Case 1 contains the comparison of a Java element with itself. Case 2 contains the comparison of two Java elements that are both syntactically and semantically dissimilar.  Case 3 contains the correspondence between two Java elements that are syntactically dissimilar but are semantically relevant. Case 4 contains the comparison of a logging call with another Java element that is not logging call but is syntactically relevant.

\subsubsection{Results}  \label{study1-results}
The results of the pairwise comparison between LJMs of the test suite is visualized in Figure~\ref{fig:jigsaw_graph}. As it is shown, the Jigsaw similarity for all self-comparisons is 1, while  the level of Jigsaw similarity is different for pairs containing distinct LJMs as our manual examination.

\begin{figure} [H]
  \centering\includegraphics [width = \textwidth]{graphviz/jigsaw.pdf}
  \caption{A similarity graph representing pairwise Jigsaw similarities between LJMs shown in Table~\ref{table:ljms}.}
  \label{fig:jigsaw_graph}
\end{figure}


The analysis of the 4 test cases is shown in Table~\ref{jigsaw_4_test_cases}. Test Case 1 shows that a Java element that is compared with itself has a Jigsaw similarity of 1. Test Case 2 indicates that no correspondence connection is created when a Java element is compared with another Java element that is utterly dissimilar. Test Case 3 indicates that the similarity between a \code{for-} statement and a \code{while-} statement is non-zero, and that Jigsaw is able to detect semantic correspondences between Java elements. Test Case 4 shows that a logging call has non-zero similarity with another Java element that is not a logging call but is syntactically relevant. This case will be handled via the removal of this kind of correspondence connection, as will be described in Section~\ref{meth-constraints}. 

\begin{figure}
  \centering
  \begin{tabular}{|c|l|c|}
    \hline
    \shortstack{Test\\case} & Java source code fragment & \shortstack{Jigsaw\\similarity}\\
    \hline
    
    \multirow{2}{*}{{1}}&Log.log(Log.WARNING,this,"Unknown action: " + actionName);& \multirow{2}{*}{1}\\
    \cline{2-2}
                         &Log.log(Log.WARNING,this,"Unknown action: " + actionName);\\
    \hline
    
       \multirow{2}{*}{2}&return entry& \multirow{2}{*}{\shortstack{No\\correspondence \\connection}}\\
    \cline{2-2}
       &int i=0;\\
    \hline
  
    
 \multirow{2}{*}{3}&
 for (int i=0; i < comps.length; i++) {...} \\
  

    \cline{2-2}
      & 
while (entries.hasMoreElements())  { ...}    
      \\
    \hline    
    
    \multirow{2}{*}{4}&Log.log(Log.WARNING,this,"Unknown action: " + actionName);& \multirow{2}{*}{0.33}\\
    \cline{2-2}
      &EditBus.removeFromBus(this);\\
    \hline
    
  \end{tabular}
  \caption{Results from examining the Jigsaw similarity for 4 sample Java source code fragment pairs.}
  \label{jigsaw_4_test_cases}
\end{figure}
 


\section{Summary}  \label{summary}
We described Eclipse JDT as a concrete framework that can be used  to manipulate ASTs of a source code written in the Java programming language. We also introduced Jigsaw, an existing framework for determining structural correspondences between AST nodes and measuring similarity between them. Furthermore, we assess the Jigsaw functionality to address our problem through an empirical study on a set of LJMs selected from a real- world software system.

%In this chapter, we described anti-unification as a technique to construct a common generalization of two given terms. We have also introduced an extended form of anti-unification, which is called higher-order anti-unification modulo theories, where a set of equivalence equations can be applied on higher-order extended structures to incorporate background knowledge. In addition, we provided a brief description of AST that maps Java source code in a tree structure form, and why an extended form of it, named AUAST, is required to create higher-order structures specific to our problem context. Finally, we discuss the Jigsaw framework and how it could assist us in determining the potential structural correspondences.
\addtocontents{toc}{\protect\addvspace{10pt}}
\chapter{Constructing Structural Generalizations} \label{ch4} \label{methodology}

%In Chapter~\ref{background}, I provided background information on higher-order anti-unification modulo theories, a theoretical framework that can be used to construct a generalization from two given ASTs. In Chapter~\ref{background2}, I presented and tested the Jigsaw framework proposed by \citet{2008:fse:cottrell} to construct the CAST structure, where each node holds a list of candidate structural correspondences. I next extended the CAST structure to an AUAST that would allow the creation of an anti-unifier.
I now consider how Jigsaw and its CAST structures could help me (1)~to construct an approximation of the best anti-unifier to my problem from the AUASTs of two logged methods with special attention to log statements and (2)~to develop a similarity measure between the two structures, which can provide us with useful information for clustering a set of logged methods in a later phase. The constructed anti-unifier can be viewed as a generalization that represents the structural commonalities and differences between the two logged methods.


%I also described how Jigsaw applies this framework on ASTs of a pair of \name{Java} methods to determine candidate structural correspondences.

%To this end, first we should create an extended form of AST, called AUAST (Anti-unifier AST) that allows the insertion of variables in place of any node in the tree, which is a requirement for HOAUMT (see Section~\ref{AUAST}).
%1) generates all possible candidate correspondence connections between the two AUASTs using the Jigsaw framework (line~1) (see Section~\ref{meth-CAST});

To approximate the best anti-unifier for my problem, I develop a greedy selection algorithm that determines the best correspondence for each node. My approach contains a sequence of 2 actions to find the best correspondences between two AUASTs: (1) it applies two constraints to prevent the anti-unification of log statement nodes with any other nodes; and (2) it determines the best correspondence for each AUAST node with the highest similarity and then removes the other correspondence connections involving those nodes (Section~\ref{best-corr}). However, to construct an anti-unifier, a further step is taken: the anti-unification of each AUAST node with its best correspondence (Section~\ref{meth-antiUnifier}). Furthermore, I developed an algorithm to measure similarity between the usage of log statements in methods based on the selected correspondences (Section~\ref{meth-similarity}). An overview of the proposed approach is shown in Figure~\ref{fig:meth_overview}.





%computed the ratio of the number of common elements over the total number of elements of the anti-unifier to measure similarity between two AUASTs (see Section~\ref{meth-similarity}).
%I computed the ratio of the number of common elements over the total number of elements of the anti-unifier to measure similarity between two AUASTs (see Section~\ref{meth-similarity}).
%(Section ~\ref{meth-constraints})
%through anti-unifying their structural properties



\begin{figure}[t]
  \centering\includegraphics [width = \textwidth]{Drawing4/auOverview.pdf}
  \caption{Overview of the generalization process.}
  \label{fig:meth_overview}
\end{figure}


To evaluate my approach, I developed the anti-unifier-building tool atop Jigsaw, and conducted an experimental study on a test suite. In Section~\ref{antiunifierTool}, I discuss the algorithms and decisions I made for implementing the anti-unifier-building tool and constructing a detailed view of the structural generalization. I also describe my experimental setup, present the study, and discuss the results in Section~\ref{anti-unifier-assessment}.



%abstracts structural correspondences of ASTs
%A generalization mechanism could be used to abstract commonalities in several structures containing logging calls into a more general structure;
%The higher-order anti-unification modulo theories framework could assist us in constructing a structural generalization from a pair of LCSs. Also, A modified version of a hierarchical clustering suited for our application could assist us to construct a generalization from a set of LMs.

%To construct a structural generalization from a pair of LMs, we developed a prototype tool that applies the Jigsaw framework to find candidate correspondences between two ASTs and uses higher-order anti-unification modulo theories to generalize the structures. It takes the source code of two logged \name{Java} methods as input and performs a sequence of six actions on them, outlined by the algorithm \func{Anti-unification}: (1) we input into the algorithm the ASTs of two given logged \name{Java} methods, constructed via the Eclipse Java Development Tools (JDT) framework; (2) we generate an augmented form of AST (called a CAST) using the Jigsaw framework, where each node holds a list of candidate correspondence connections between the two ASTs (line~1) (see Section~\ref{meth-CAST}); (3) we extend the CAST structure to a higher-order extended structure (called an AUAST) and apply some constraints to prevent the anti-unification of logging calls with anything else (lines~2--3) (see Sections~\ref{meth-constructAUAST} and~\ref{meth-constraints}); (4) we determine the best correspondence for each node of the AUASTs with the highest similarity and we remove the other correspondence connections involving the nodes (line~4) (see Section~\ref{meth-correspondence}); (5) we anti-unify structural properties with their best correspondence to construct an approximation of the best anti-unifier to our problem with special attention to logging calls (line~5) (see Section~\ref{meth-antiUnifier}); and (6) we measure similarity (line~6) (see Section~\ref{meth-similarity}).
%\begin{algorithm}
%\caption{Input into \func{Anti-unification}(\id{auastA},\id{auastB}) are AUAST nodes of two \name{Java} classes containing logging calls; this algorithm construct an anti-unified AUAST node through anti-unification of input node's structural properties} %and compute similarity between them with a focus on logging calls}
%\label{overview}
%\begin{algorithmic}[1]
%\antiunification
%    \State  $\func{Jigsaw-Correspondence}(\id{auastA},\id{auastB})$
%   \State $\id{auastA}, \id{auastB} \gets \func{Apply-Constrains}(\id{auastA},\id{auastB})$
%   \State $$list}  \gets \func{Determine-Best-Correspondences}(\id{auastA})$
%     \State $$compute-similarity} \gets \func{Compute-Similarity}(\id{auastA},\id{auastB})$
%    \State $$anti-unifier} \gets \func{Antiunify}(\id{auastA},\id{auastB})$
%\end{algorithmic}
%\end{algorithm}


%\section{Anti-unification of a pair of ASTs} \label{anti-unification pairs}
%\subsection{Anti-unification ASTs}\label{AUAST}
%\RW{Describe testing/evaluation.}


%\section{Anti-unification ASTs} \label{anti-unification pairs}






\section{Determining the best correspondences}  \label{best-corr}
%As described in Section~\ref{Jigsaw}, through the application of Jigsaw, each AUAST node will hold a list of candidate correspondence connections, each implicitly representing an anti-unifier. However, despite having multiple potential anti-unifiers, we need to determine one single anti-unifier that is helpful to solve our problem using the greedy selection algorithm discussed in Section~\ref{}.

As discussed in Section~\ref{HOAUMT}, in general, higher-order anti-unification modulo
theories has been proven to be undecidable. Therefore, to find one single anti-unifier that is an approximation of the optimal fit to my problem, I developed the \func{Determine-Best-Correspondences} algorithm that greedily selects the most similar correspondence as the best fit for each AUAST node. Hence, each AUAST node can either be anti-unified with its best correspondence in the other AUAST or with \nothing. This algorithm takes one of the AUASTs, visiting the AUAST nodes therein to store all candidate correspondence connections between the two AUAST nodes in a list, which is sorted in descending order based on the similarity measure (lines~1--9). However, to prevent the anti-unification of log statement nodes with any other nodes, I apply some constraints on the selection of correspondence connections prior to determining the best ones via the \func{Apply-Constraints} algorithm (line~8). Then the correspondence connection with the highest similarity value is determined as the best fit for the two nodes involved, and all other correspondence connections involving these nodes are removed using \func{Remove-Other-Correspondences} algorithm (lines~10--14). This process terminates when no more correspondence connections are left in the list.

%entire --> one single

\begin{algorithm}
\caption{\func{Determine-Best-Correspondences}($\id{A}$) takes in the one of the AUASTs and determines the best correspondence connection with the highest similarity for each node.}
\label{alg-determine}
\begin{algorithmic}[1]
\DetermineBest
    \State $\id{list} \gets \func{()}$
    \State $\id{nodes} \gets \func{visitor}(\id{A})$
	  \For {$\id{node} \in \id{A}$}
		\For {$\id{corr} \in  \id{corrs}[\id{node}]$}		
				 	\State{$\func{Append}(\id{corr},\id{list})$ }
			 	\EndFor  	
	   \EndFor	
	    \State{$ \func{Apply-Constraints}(\id{list}) $}	
	   \State{$\func{sort}(\id{list})$}
	   \For{$\id{corr} \in \id{list}$}
	   		\State $\id{bestCorr}[\id{nodeA}] \gets \id{corr}$
	   		\State $\id{bestCorr}[\id{nodeB}] \gets \id{corr}$
	   		\State{$\func{Remove-Other-Correspondences}(\id{corr},\id{list})$ }
	   \EndFor
 %\Return $\id{list} $  	
  \end{algorithmic}
\end{algorithm}

To construct an anti-unifier of the AUASTs of logged methods with a focus on log statements, some constraints should be applied in determining correspondences. The first constraint (as described below) should be applied to prevent the anti-unification of log statement nodes with any other types of nodes.
\begin{constraint}
A log statement should either be anti-unified with another log statement or should be anti-unified with \nothing.
\end{constraint}	
	
This constraint creates a further constraint:

\begin{constraint}
A structure containing a log statement should be anti-unified with a corresponding structure containing another log statement or should be anti-unified with \nothing.
\end{constraint}

As an illustration, consider the CASTs of the two examples in Figure~\ref{fig:meth-ast-1}. There is a candidate correspondence connection between the two log method invocation nodes and another between the two \code{if} statements. The second \code{if} statement contains a logging call, while there is no corresponding logging call in the first one. According to the first constraint, two log method invocation nodes should be anti-unified together. On the other hand, a correspondence connection is created between the two \code{if} statements; however, anti-unification of these statements includes anti-unifying their children nodes as well. Thus, statements inside the body of the \code{if} statements must be anti-unified with each other, indicating that log method invocation inside the body of \code{if} statement in the second example should be anti-unified with \nothing, which is contrary to our first assumption. In order to comply with the first constraint, the correspondence connection between two \code{if} statements should be deleted, leading us to apply the second constraint. My approach applies these constraints by taking the following steps prior to determining the best correspondences:
\begin{enumerate} [leftmargin=.4in]
\item	Augment the AUAST node with a property to mark log statement nodes and structures enclosing them as ``logged''.
\item	Remove correspondence connections where one node is marked as ``logged'' and the corresponding node is not via the \func{Apply-Constraints} algorithm.
\end{enumerate}

The \func{Apply-Constraints} algorithm takes the list of correspondence connections, and removes the ones involving two nodes where one is ``logged'' and the corresponding node is not (lines~1--9).
%, using the \func{Remove-Other-Correspondences} algorithm

\begin{algorithm}
  \caption{\func{Apply-Constraints}($\id{list}$) applies the constraints on the list of correspondence connections.}
  \label{applyConstraints}
  \begin{algorithmic}[1]
  \ApplyConstraints
      \For {$\id{corr} \in \id{list}$}
      \State $\id{nodeA} \gets \id{nodeA}[\id{corr}]$
	  \State $\id{nodeB} \gets \id{nodeB}[\id{corr}]$
 			\If {(($\id{nodeA} \Instanceof \cons{Logged}) \And (\id{nodeB} \Instanceof \cons{Non-Logged})) \ori $\\ $ ((\id{nodeA} \Instanceof \cons{Non-Logged}) \And (\id{nodeB} \Instanceof \cons{Logged}))$}	
	   	\State{$\func{Remove}(\id{corr},\id{list})$ }
	   	\State{$\func{Remove}(\id{corr},\id{corrs}[\id{nodeA}])$ }
	   	\State{$\func{Remove}(\id{corr},\id{corrs}[\id{nodeB}])$ }
 	  \EndIf 		
 \EndFor 	
	
  \end{algorithmic}
\end{algorithm}




The \func{Remove-Other-Correspondences} algorithm removes correspondence connections that are not selected as the best fit from three lists: the list of all correspondence connections (line~5 and line~12);
the list of correspondence connections of the first node involved in the connection (line~6 and line~13); and the list of correspondence connections of the second node involved in the connection (line~7 and line~14). As an example, Figure~\ref{fig:AUASTs} shows the correspondence connections between AUAST nodes after applying the \func{Determine-Best-Correspondence} algorithm on the list of potential correspondence connections.
%in Figure~\ref{fig:meth-ast-1}.

\begin{algorithm}
\caption{\func{Remove-Other-Correspondences}($\id{corr}$,$\id{list}$) removes all other correspondences involving the nodes of a particular correspondence connection ($\id{corr}$) from the lists of correspondences.}
\label{removeOtherCEs}
  \begin{algorithmic}[1]
  \RemoveOtherCEs
       \State $\id{listA} \gets \id{corrs}[\id{nodeA}[\id{corr}]]$
	   \State $\id{listB} \gets \id{corrs}[\id{nodeB}[\id{corr}]]$
	   \For {$\id{corrA} \in \id{listA}$}
			\If{$\id{corrA} \neq \id{corr}$}	
\State{$\func{Remove}(\id{corrA},\id{listA})$ } 			
\State{$\func{Remove}(\id{corrA},\id{corrs}[\id{nodeA}[\id{corrA}]])$ }
\State{$\func{Remove}(\id{corrA},\id{corrs}[\id{nodeB}[\id{corrA}]])$ }    		
	   		 \EndIf
	   \EndFor		
       \For {$\id{corrB} \in \id{listB}$}
			\If{$\id{corrB} \neq \id{corr}$}	   		 	 	
	 	 	\State{$\func{Remove}(\id{corrB},\id{listB})$ } 		 \State{$\func{Remove}(\id{corrB},\id{corrs}[\id{nodeA}[\id{corrB}]])$ }
\State{$\func{Remove}(\id{corrB},\id{corrs}[\id{nodeB}[\id{corrB}]])$ }
	   		 \EndIf
	   \EndFor	  	
  \end{algorithmic}
\end{algorithm}




\begin{sidewaysfigure}[p]
  \centering\includegraphics [width = \textwidth, height=0.5\textheight]{Drawing4/FirstCorr.pdf}
  \caption{Simple AUAST structure of examples in Figures~\ref{ch3-ex1} and~\ref{ch3-ex2}. The links between AUAST nodes indicate structural correspondences selected as the best fit using the \func{Determine-Best-Correspondences} algorithm.}
  \label{fig:AUASTs}
\end{sidewaysfigure}
%In general, higher order anti-unification modulo theories is undecidable [Cottrell et al., 2008]. That is, the complexity of determining the most optimal MSA is undecidable, but our desire is to create one of the best MSAs to approximate the optimal one that can sufficiently solve our problem, thus the anti-unification process should construct an anti-unifier that is the best approximate fit for our application. To this end, a greedy selection algorithm has been used, which is an approximation technique to determine the best correspondence for each node in the AUAST so constructing the anti-unifier that is approximately the best fit to our problem.

\section{Computing similarity between AUASTs} \label{meth-similarity}
Similarity computation is particularly important for the clustering phase that relies on accurate estimation of similarity between logged methods (see Chapter~\ref{clustering}). The notion of similarity can differ depending on the given context. That is, similarity between certain features could be highly important for a particular application, while it is not for another. The utility of a similarity function can be determined based on how well it enables us to produce accurate results for a particular task. In this study, a similarity measure is needed to classify logged methods based on the structural similarity in the usage of log statements. The similarity for my application can be defined as the number of common elements over the total number of elements of the anti-unifier constructed based on the selected correspondences from the previous step (described by the Equation~\ref{eq-sim}).
%To compute similarity between AST nodes, I re-used a function developed by \citet{2008:fse:cottrell}, which first computes the number of common elements between the two nodes and then divides it by the size of the largest node, as described by the Equation~\ref{eq-jigsaw-sim}.


\begin{equation}\label{eq-sim}
\id{similarity(\id{A},\id{B})} = \frac{\id{matches}(\id{A},\id{B})}{|\id{A}|+ |\id{B}|}
\end{equation}

%computed the ratio of the number of common elements over the total number of elements of the anti-unifier to measure similarity between two AUASTs


% Equation???
% heuristic???

The similarity between two AUAST nodes is computed by dividing the number of matched elements among the two nodes by the size of the largest node (Equation~\ref{eq-jigsaw-sim}). The number of matches between the two AUASTs $\id{A}$ and $\id{B}$ is computed via the \func{Compute-Best-Matches} algorithm. If the two AUASTs are leaf nodes, the number of matches is computed (lines~2--3) using the heuristics proposed by \citet{2008:fse:cottrell}, previously described in Section~\ref{Jigsaw}. Otherwise, the best correspondences between the two subtrees are selected using the \func{Determine-Best-Correspondences} algorithm, and the matches between each children of the subtree $\id{A}$ and its best corresponding node in the subtree $\id{B}$ is computed and propagated to the parent node (lines~4--12).


\begin{algorithm}
 \caption{\func{Compute-Best-Matches}($\id{A}$,$\id{B}$) computes the matches between the two ASTs based on the best correspondences.}
  \label{alg:cbm}
  \begin{algorithmic}[1]
  \ComputeBestMatches
  \State $\id{matches} \gets 0$
\If {$A\Instanceof \cons{Leaf-Node}$ }	
  \State $\id{matches} \gets   \func{matches}(\id{A},\id{B})$
  \ElsIf {$A \Instanceof \cons{Non-Leaf-Node}$ }	
    	\State \func{Determine-Best-Correspondences}($\id{A}$, $\id{B}$)
\For {$\id{childA} \in \id{children}[\id{A}]$}
\If {$ \id{bestCorr}[\id{childA}] \neq \cons{NULL}$}	 		
	%\State $\id{childB}\gets \id{bestCorr}[\id{childA}]$ 		
 		\State $\id{matches} \gets \id{matches}+\func{Compute-Best-Matches}(\id{childA},\id{nodeB}[\id{bestCorr}[\id{childA}]])$		
 \EndIf
 \EndFor

 \EndIf
 \Return $matches$
\end{algorithmic}
\end{algorithm}



%NIL->0????


%The number of identical structural properties between $\id{auastA}$ and $\id{auastB}$ is computed via the \textsc{Compute-Matches} algorithm through a recursive traversal of $\id{antiunifier}$ nodes' structural properties. For simple structural properties, the number of matches is added by one (lines~3-4). For child structural properties, the number of matches is added by one and the number of matches computed recursively for the child node (lines~5-8). For child list structural properties, the number of matches is computed for each child node recursively and is propagated to the parent node (lines~9-13). All matches are summed up to compute total number of matches between the two AUASTs. Then the following equation is used to compute structural similarity between $\id{auastA}$ and $\id{auastB}$:

\section{Constructing the anti-unifier}\label{meth-antiUnifier}\label{meth-antiunifier}
Once the best correspondences have been determined between two AUASTs, I construct a new anti-unified AUAST by traversing the original AUAST structures recursively and anti-unifying each node with its best correspondence. The new anti-unified structure is a generalization of two original structures where common structural nodes are represented by copies and differences are represented by structural variables. The variables may be inserted in place of any node in the AUAST, including both subtrees and leaves, and can be substituted with proper original substructures to regain the original structures.


% anti-unify the same type nodes????????

Anti-unification of two AUASTs is performed via the \func{Antiunify} algorithm. If the two AUASTs are leaf nodes, the anti-unifier will be created through the anti-unification of the two nodes (lines~2--3). Otherwise, the anti-unified subtree is created by anti-unifying each child of one subtree with its best corresponding node in the other subtree. If there is no correspondence for a node, the anti-unified node will be created through the anti-unification of the node with the \NIL{} structure. All these anti-unified nodes are appended to the list of children of the anti-unifier (lines~4--21). The details of constructing a detailed view of the anti-unifier for my application will be discussed in Section~\ref{meth-detailed-view}.


\begin{algorithm}
 \caption{\func{Antiunify}($\id{A}$,$\id{B}$) creates the anti-unifier of two AUASTs through the anti-unification of each node with its best correspondence.}
  \label{AntiUnify}
  \begin{algorithmic}[1]
\AntiUnify
\State{$ \id{antiunifier} \gets \cons{NULL}$}
\If {$A \Instanceof \cons{Leaf-Node}$ }	
  \State $\id{antiunifier} \gets   \func{construct-antiunifier}(\id{A},\id{B})$
  \ElsIf {$A \Instanceof \cons{Non-Leaf-Node}$ }	
 % \State \func{Determine-Best-Correspondences}($\id{a}$, $\id{b}$)
        \For {$\id{childA} \in \id{children}[\id{A}]$}
\If {$ bestCorr[childA]= \cons{NULL}$}	
  \State $\id{child} \gets   \func{Antiunify}(\id{childA},\cons{NIL})$
\Else	
 \State $\id{child} \gets   \func{Antiunify}(\id{childA},\id{nodeB}[bestCorr[childA]])$
\EndIf
\State $\func{Append}(\id{child},\id{children[antiunifier]})$
\EndFor
%	\EndIf	
  %\If {$\id{b} \Instanceof \cons{Non-Leaf-Node}$ }	
\EndIf
  \If {$B \Instanceof \cons{Non-Leaf-Node}$ }	

  \For {$\id{childB} \in \id{children}[\id{B}]$}
\If {$ bestCorr[childB]= \cons{NULL}$}	
  \State $\id{child} \gets   \func{Antiunify}(\id{childB},\cons{NIL})$
\EndIf
\State $\func{Append}(\id{child},\id{children[antiunifier]})$
\EndFor
 \EndIf
\Return $\id{antiunifier}$
\end{algorithmic}
\end{algorithm}

%$\id{root}[\id{B}] ?????

\section{The anti-unifier building tool} \label{antiunifierTool}
The anti-unifier building tool is a proof-of-concept implementation of my anti-unification approach, which is developed atop the Jigsaw framework to construct a detailed view of structural generalization. To create the AUAST structure using Eclipse JDT framework that addresses the limitations of CAST to constructing an anti-unifier, I added the following structural properties:
%namely?
%to address the limitations of CAST to constructing an anti-unifier
%It also creates an extended form of CAST, called AUAST,
\begin{itemize} [leftmargin=.5in]
\item \textit{Simple structural variable properties}: an extension of simple structural properties referring to two simple values to allow the insertion of variables in place of leaves.
\end{itemize}
\begin{itemize} [leftmargin=.5in]
\item \textit{Child structural variable properties}: an extension of child structural properties referring to two child AST nodes to allow the insertion of variables in place of subtrees.
\end{itemize}

I also needed to make additional algorithms and decisions to construct a detailed generalization view suited to my application, which will be described in the following sections.

%%change???


\subsection{Constructing the detailed generalization view} \label{meth-detailed-view}

To figure out the structural commonalities and differences amongst LMs, I developed an algorithm to construct a generalization view of the anti-unifier. Figure~\ref{fig:meth-anti-unifier} shows the detailed view of the anti-unifier constructed from the AUASTs of LMs of Examples 1 and 2, generated by \tool{ELUS}.
% where ``$\id{a}$-or-$\id{b}$'' represents a structural variable that must be substituted with either the $a$ or $b$ substructures to recover each original structure, and ``$\id{a}$'' represents that the $a$ substructure is common between the two AUASTs.
%??????


% modify the figure???



\begin{figure}[p]
  \centering\includegraphics[trim={0 10cm 18cm 0},clip,
  width=\textwidth]{Drawing4/SampleAntiUnifier.png}
  \centering\includegraphics[trim={3.5cm 0 8cm 33cm},clip,
  width=\textwidth]{Drawing4/SampleAntiUnifier.png}
  \caption{The detailed view of the anti-unifier generated by \tool{ELUS} from the AUASTs of Examples 1 and 2.}
  \label{fig:meth-anti-unifier}
\end{figure}


% the LMs in
% to create an anti-unifier description. For example, we supply the \func{Antiunify} algorithm with the AUASTs of LMs in Examples 1 and 2
%explain in more details???

To create the detailed view of the structural generalization from two given AUASTs, I applied the \func{Antiunify} algorithm (Section~\ref{meth-antiUnifier}) on the two AUAST nodes $\id{a}$ and $\id{b}$ through the anti-unification of their structural properties, as the Eclipse JDT utilizes these properties to record structural information of each \name{Java} element (Section~\ref{JDT}). That is, for each structural property of the two nodes, if the property is common between them, a copy of it will be created and added to the structural properties of the anti-unifier; otherwise, a variable structural property is constructed referring to two property values and added to the anti-unifier's structural properties.
%property values???


% (lines~8--9); if structural property is a child property, a child variable structure is constructed (lines~10--12). All these structural properties will be added to the structural property of the anti-unifier(lines~8--9).


%if structural property is a child list property, for each child of $\id{propA}$ and $\id{propB}$, where there is no correspondence in the other AUAST, an anti-unified node is created through the anti-unification of the child node with the \NIL{} structure via the \func{Antiunify} algorithm and added to the value of the anti-unified child list property; otherwise, the child node is anti-unified with its best correspondence (lines~6-21).


\subsection{Methods containing multiple log statements} \label{meth-multipleLogs}
There might be some cases in which our approach is not able to anti-unify log statements in two input seeds, when there is more than one logging call in an LM. For example, consider the LMs in Figures~\ref{multiple1} and~\ref{multiple2}. Figure~\ref{mast_1} shows the simple AUASTs for these examples and potential correspondence connections between the AUAST nodes. Figure~\ref{m_ast2} shows the correspondence connections selected as the best match using our greedy algorithm. To anti-unify \name{If Statement 1} with \name{If Statement 3}, we should anti-unify their structural properties. Thus, \name{Log Statement 1} should be anti-unified with \name{Log Statement 3}, and \name{Log Statement 4} should be anti-unified with \NIL{} since there is no corresponding log statement in the body of \name{If Statement 1}, while there is a corresponding node for \name{Log Statement 4} in the body of \name{If Statement 2} (\name{Log Statement 2}).

%\code{if statement 1}???
%In figures: If statement: if statement 1???

\begin{figure}[t]
\def\baselinestretch{1}
\begin{lstlisting}
public void method1(){
	...
	if(condition1){
		Log.log();
	}
	...
	if(condition2){
		Log.log();
	}
	...
}
\end{lstlisting}
\caption{A Java method that utilizes multiple log statements.\label{multiple1}}
\end{figure}



\begin{figure}[H]
\def\baselinestretch{1}
\begin{lstlisting}
public void method2(){
	...
	if(condition3){
		Log.log();
		Log.log();
	}
	...
}
\end{lstlisting}
\caption{Another Java method that utilizes multiple log statements.\label{multiple2}}
\end{figure}

\begin{figure} [H]
  \centering\includegraphics [width = \textwidth]{Drawing4/multipleLogging.pdf}
  \caption{Simple AUAST structure of examples in Figures~\ref{multiple1} and~\ref{multiple2}. Links between AUAST nodes indicate candidate structural correspondences detected by the Jigsaw framework.}
  \label{mast_1}
\end{figure}


\begin{figure} [H]
  \centering\includegraphics [width = \textwidth]{Drawing4/multipleLogging2.pdf}
  \caption{Simple AUAST structure of examples in Figures~\ref{multiple1} and~\ref{multiple2}. Links between AUAST nodes indicate structural correspondences selected as the best match using the greedy algorithm.}
  \label{m_ast2}
\end{figure}

To handle these cases, we can split them into more than one case, where each LM contains only one log statement. To do so, we need to create a copy of the LM for each log statement by maintaining that logging call and removing the other ones. For example, we need to create two copies for each logged \name{Java} method of examples in Figures~\ref{multiple1} and~\ref{multiple2} as depicted in Figures~\ref{multiple1-one} and~\ref{multiple2-one}, respectively.
% thus constructing four possible anti-unifier for  possible combination and compute the similarity for each combination.
%We can split this case into more than one case, each with one logging statement in every seed. That is, for each case all the other logging statements should be deleted from seeds.  For example, imagine we have AST 1 and AST 2. AST 1 contains three logging calls and AST 2 contains two logging calls. As explained, we split AST 1 into AST 1a, AST 1b, and AST 1c. Also, we split AST 2 into AST 2a and AST 2b. We can split this case into 6 possible cases and create an anti-unifier for each possible combination and then compute a measure of similarity for each case. The best match for each log statement can be selected based on anti-unifier with the highest similarity amongst the other options.


\begin{figure}[H]
\def\baselinestretch{1}
\begin{lstlisting}
public void method1(){
	...
	if(condition1){
		Log.log();
	}
	...
	if(condition2){
		//removed
	}
	...
}
\end{lstlisting}
\begin{lstlisting}
public void method1(){
	...
	if(condition1){
		//removed
	}
	...
	if(condition2){
		Log.log();
	}
	...
}

\end{lstlisting}
\caption{Creating multiple copies of the LM in Figure~\ref{multiple1} for each log statement.\label{multiple1-one}}
\end{figure}



\begin{figure}[H]
\def\baselinestretch{1}
\begin{lstlisting}
public void method2(){
	...
	if(condition3){
		//removed
		Log.log();
	}
	...
}
\end{lstlisting}
\begin{lstlisting}
public void method2(){
	...
	if(condition3){
		Log.log();
		//removed
	}
	...
}

\end{lstlisting}
\caption{Creating multiple copies of the LM in Figure~\ref{multiple2} for each log statement.\label{multiple2-one}}
\end{figure}

%\section{An assessment of the anti-unifier-building tool}\label{anti-unifier-assessment}

\section{Evaluation} \label{anti-unifier-assessment}
%\subsection{Study 2: Detailed anti-unifier view}  \label{study2}
To assess the effectiveness of my anti-unification algorithm and the supporting tool, I conducted an experiment on the test suite described in Section~\ref{study2-setup}. %The anti-unifier-building tool is developed atop the correspondence tool to construct an anti-unifier from AUASTs of each LM pair in our test suite.


\subsection{Setup}  \label{study2-setup}
In this study, I manually attempted to create the detailed anti-unifier view for each pair of LMs in the test suite (55 test cases in total). I first identified corresponding and non-corresponding \name{Java} elements for each LM pair with a focus on preventing the correspondence of log statements with anything else and then represented the anti-unifier in the detailed view (i.e., formatted as in Figure~\ref{fig:meth-anti-unifier}). I also computed the ratio of common \name{Java} elements in the detailed anti-unifier view to total number of \name{Java} elements of the two LMs to measure the similarity.
I also ran the anti-unifier-building tool on each pair of LMs to construct the detailed anti-unifier view for each pair with special attention to log statements and to measure the similarity between the LM pairs. Furthermore, I used \name{EclEmma}, which is a \name{Java} code coverage tool for \name{Eclipse}, to measure the test coverage. Test coverage is defined as a measure of the completeness of the set of test cases.


\begin{table}
  \centering
  \begin{tabular}{cl}
    \toprule
    Test case & Logged methods \\
    \midrule

    \multirow{2}{*}{{1}}&{PluginJAR.generateCache()}\\
                         &{PluginJAR.generateCache()}\\
    \midrule

    \multirow{2}{*}{2}&{PluginJAR.generateCache()}\\
                         &{EditBus.send(..)}*\\
    \midrule

    \multirow{2}{*}{3}&{MiscUtilities.isSupportedEncoding(..)}\\
                         &{EditBus.send(..)}\\
    \midrule

    \multirow{2}{*}{4}&{EditBus.send(..)}\\
                         &{EditBus.send(..)}*\\
    \midrule
    \multirow{2}{*}{5}&{EditBus.send(..)}*\\
                         &{EditAction.Wrapper.actionPerformed(..)}\\
    \midrule

    \multirow{2}{*}{6}&{EditBus.send(..)}*\\
                         &{BufferHistory.RecentHandler.doctypeDecl(..)}\\
    \midrule

    \multirow{2}{*}{7}&{EditAction.Wrapper.actionPerformed(..)}\\
                         &{JARClassLoader.loadClass(..)}\\
    \midrule

    \multirow{2}{*}{8}&{EditAction.Wrapper.actionPerformed(..)}\\
                         &{VFS.DirectoryEntry.RootsEntry.rootEntry(..)}\\
    \midrule

    \multirow{2}{*}{9}&{PluginJAR.generateCache()}\\
                         &{BufferHistory.RecentHandler.doctypeDecl(..)}\\
    \midrule

    \multirow{2}{*}{10}&{VFS.DirectoryEntry.RootsEntry.rootEntry(..)}\\
                         &{ServiceManager.loadServices(..)}\\
    \bottomrule

  \end{tabular}
  \caption[10 sample logged Java method pairs used as test cases.]{10 sample logged Java method pairs used as test cases; all are contained in the \protect\code{org.gjt.sp.jedit} package with the exception of cases 8 and 10 that are in the \protect\code{org.gjt.sp.jedit.io} package.}
  \label{study2_test_cases}
\end{table}





%The view would be in the form depicted in ..

\subsection{Results}  \label{study2-results}
I present the results of my analysis for a subset of 10 test cases (see Table~\ref{study2_test_cases}) in Table~\ref{study2_test_cases_results}. The analysis of the output has been divided into two categories: correspondence and similarity. ``Correspondence'' refers to the number of corresponding lines-of-code (LOC) detected by my tool that were found to be corresponded by my manual examination as well, and the number of LOC were not detected as corresponded by my tool but were found to be corresponded in manual inspection. I also present the percentage of the correct corresponding LOC to the total number of LOC of the two LMs. ``Similarity'' refers to the similarity that is computed based on the detected correspondences. It is calculated using both my tool and manual experiment.


In Case 8, the \code{rootEntry(..)} method contains a nested \code{if}-statement enclosing a log statement and the \code{actionPerformed(..)} method contains an \code{if}-statement enclosing another log statement. The analysis showed that a correct correspondence was detected between the inner \code{if}-statement inside the nested \code{if}- and the single \code{if}-statement. Cases~3 and~10 contain statements that are not found to correspond by my tool even though correspondences exist. For example, in Case 3, the \code{isSupportedEncoding(..)} method contains an assignment statement enclosed by an \code{if}-statement that does not have any correspondences and the \code{send(..)} method contains another assignment statement inside a \code{for}-statement without any correspondences. However, no correspondence was detected between the two assignment statements since their parent nodes do not correspond.

\begin{table}
  \centering
  \begin{tabular}{cccccc}
    \toprule
    \multirow{2}{*}{Test case}&\multicolumn{2}{c}{Correspondence}&\multicolumn{2}{c}{Similarity}\\
    \cmidrule(lr){2-3}
    \cmidrule(lr){4-5}
    &Correct (\%)&Incorrect&human&tool\\
    \midrule
    1&104 (100)&0& 1.0 & 1.0\\
    \midrule
    2&8 (100)&0& 0.13& 0.13\\
    \midrule
    3&6 (85)&1&0.19& 0.16\\
    \midrule
    4&4 (100)&0&0.29 &0.29\\
    \midrule
    5&5 (100)&0&0.21 &0.21\\
    \midrule
    6&3 (100)&0&0.2 &0.2\\
    \midrule
    7&5 (100)&0&0.11 &0.11\\
    \midrule
    8&7 (100)&0& 0.1&0.1\\
    \midrule
    9&3 (100)&0&0.03&0.03 \\
    \midrule
    10&14 (87)&2&0.27 &0.22\\
    \bottomrule
  \end{tabular}
  \caption{Results of constructing anti-unifiers with a focus on log statements for the test cases presented in  Table~\ref{study2_test_cases}}
  \label{study2_test_cases_results}
\end{table}

The results of the pairwise comparison between LMs of the test suite is visualized in Figure~\ref{fig:au_graph}. My anti-unifier-building tool succeeded in detecting correspondences with special attention to anti-unifying log statements and calculating pairwise similarities in 48 out of 55 test cases. In addition, the instruction coverage of our test cases is 82\% as measured with \name{EclEmma}.

\begin{figure} [t]
  \centering\includegraphics [width = \textwidth]{graphviz/au.pdf}
  \caption{A similarity graph representing pairwise similarities calculated by my tool between LMs shown in Table~\ref{study2_test_cases}.\label{fig:au_graph}}
\end{figure}




\section{Summary} \label{meth1-summary}
I have presented an approach for constructing a generalization from AUASTs of two LMs with special attention to logging statements via structural correspondence. This approach proceeds in three steps. First, several constraints have been applied on the selection of correspondences to prevent anti-unifying log method invocation nodes with any other types of nodes. Second, an approximation technique is employed to find the best correspondence for each AUAST node. Third, the anti-unification of two AUASTs is performed through the application of higher-order modulo theories over the AUAST structures. Furthermore, a measure of similarity has been developed that would provide us with useful information for the clustering phase.

An experimental study was conducted to evaluate the effectiveness of my anti-unification algorithm and the tool support in constructing an anti-unifier from AUASTs of each LM pair of my test suite with a focus on log statements and measuring similarity between them.



%This approach is implemented as an Eclipse plug-in which given two logged Java methods utilizes the Eclipse JDT framework to extract their ASTs. In order to be able to apply HOAUMT, we extended the AST structure to a higher-order structure, called AUAST, that would allow the insertion of variables in place of any nodes. We then applied the Jigsaw framework to identify potential correspondences between the two AUASTs and greedily determines the best correspondence for each node with the highest Jigsaw similarity.  Moreover





\addtocontents{toc}{\protect\addvspace{10pt}}
\chapter{Clustering}  \label{clustering}
In Chapter~\ref{ch4}, I described my anti-unification algorithm to construct an anti-unifier from the AUASTs of a pair of LMs, paying special attention to logging calls. Recall that the general point of this study is to provide a concise description of where logging calls happen in source code by constructing structural generalizations that represent the detailed commonalities and differences between AUASTs of LMs. To this end, we should develop an algorithm that:
\begin{itemize} [leftmargin=.5in]
\item clusters Java methods showing different usages of logging calls into separate clusters; and
% using a measure of similarity.
%classifies AUASTs of LMs into clusters using a measure of similarity such that AUASTs in each group has maximum similarity with each other and minimum similarity to other ones.

%such that entities
\item abstracts AUASTs of LMs of each group into a structural generalization representing the commonalities and differences between them.
%abstracts structural correspondences of ASTs
\end{itemize}

Clustering is the classification of a collection of unlabelled data items into meaningful groups \cite{jain1999data} \RW{Add to bib}, where category labels are obtained from the similarities between data items. Therefore, before the use a clustering algorithm to classify a set of AUASTs, I need to first define a similarity metric. To develop a measure of similarity between AUASTs, I used the similarity function described in Section~\ref{meth-similarity}. To perform clustering on a set of AUASTs of LMs, I developed a modified version of an agglomerative hierarchical clustering algorithm suited to my application (Section~\ref{clustering-alg}). The clustering algorithm is a bottom-up approach that starts with singleton clusters, each contains one AUAST, and then it repeatedly merges the closest clusters that are the ones with maximum similarity between their AUASTs. To evaluate my approach, I have implemented it as an Eclipse plugin \RW{Note: you don't describe this tool in any detail.  I am assuming it is straightforward.} %(Section~\ref{clusteringTool})
and conducted an experimental study through the application of it on the test suite introduced previously (see Section~\ref{study1_setup}). I describe my experimental study and discuss the results in Section~\ref{clustering-assessment}.



%However, to cluster Java methods that use logging calls, a further step should be taken that is the application of a clustering algorithm on the set of AUASTs of LMs extracted from the source code. Therefore, I have developed a modified version of an agglomerative hierarchical clustering algorithm suited to my application.
% when it is needed to merge two clusters.
%Figure~\ref{fig:meth_overview} shows an overview of the general process of our anti-unification technique, as will be described in the following sections.



\section{The hierarchical clustering algorithm} \label{clustering-alg}
%\subsection{Anti-unifying a set of AUASTs} \label{meth-clustering}
To anti-unify a set of AUASTs, I have developed a modified version of an agglomerative hierarchical clustering algorithm as described below:
%agglomerative hierarchical clustering(AHC)???

\begin{enumerate} [leftmargin=.5in]
\item Start with singleton clusters, each containing one AUAST.
\item Create a similarity matrix by computing pairwise similarities between clusters.
\item Find the closest clusters (a cluster pair with maximum similarity).
\item Merge the closest cluster pair and replace them with a new cluster containing the anti-unifier of AUASTs of the two clusters.
\item Update the similarity matrix by computing the similarity between the new cluster and all remaining clusters.
\end{enumerate}
\begin{itemize} [leftmargin=.5in]
\item Repeat steps 3, 4, and 5 until the similarity between closest clusters falls below a predetermined threshold value.
\end{itemize}


The hierarchical algorithm employs an $n \times n$ similarity matrix for a set of $n$ AUASTs, where the element in row $i$ and column $j$ represents the similarity between the $i$th and the $j$th clusters. In this algorithm, the similarity between a pair of clusters is defined as the similarity between their AUASTs, which is computed through the algorithm described in Section~\ref{meth-similarity}. However, to prevent the combination of a cluster pair when the usage of logging is different, I adjusted the similarity between them to zero. That is, if the anti-unification of AUASTs of a cluster pair does not allow the anti-unification of log statements with each other (as the structures enclosing them are not corresponded), they are defined to fall into separate clusters. I also used the anti-unification algorithm described in Section~\ref{meth-antiUnifier} to construct structural generalizations.

%threshold????
Figure~\ref{fig:overview2} illustrates the clustering process for a sample set of 5 AUASTs using the initial similarity matrix depicted in Figure~\ref{matrix}. In the first and second iterations, clusters 1 and 2, and then clusters 4 and 5, are selected as the closest clusters, merged, and replaced by clusters 6 and 7, respectively. If the threshold value were set to %determined as 
%Threshold $A = 
0.20%$
, the process will terminate as the similarity between the closest clusters (clusters 3 and 6) is below this threshold; otherwise, these clusters will be merged and replaced by cluster 8. However, the similarity between AUASTs of clusters 7 and 8 is zero, and thus they should not be merged with each other. In this study, the similarity threshold is determined through informal experimentation.


\begin{figure} [!h]
\begin{displaymath}
    similarity = \left[
        \begin{matrix}
        1.00 &  &  &  &   \\
0.28 & 1.00 &  &  &  \\
0.12 & 0.17 & 1.0 &  &  \\
0.00 & 0.00 & 0.00 & 1.0 &  \\
0.00 & 0.00 & 0.00 & 0.21 & 1.00
        \end{matrix}   \right]
\end{displaymath}
 \caption{The similarity matrix for a sample set of 5 AUASTs.}
  \label{matrix}
\end{figure}




\begin{sidewaysfigure} [p]
  \centering\includegraphics [width = \textwidth]{Drawing4/overview2.pdf}
  \caption{The agglomerative hierarchical clustering process on a sample set of  5 AUASTs (using the initial similarity matrix shown in Figure~\ref{matrix}). Two alternative threshold values are exemplified: at a threshold of 0.10, cluster 8 is formed; at a threshold of 0.2, clusters 6 and 7 are formed individually but are not merged, and so cluster 8 is not formed.}
  \label{fig:overview2}
\end{sidewaysfigure}


\section{Evaluation} \label{clustering-assessment}
To evaluate the clustering approach, I implemented the agglomerative hierarchical clustering process as a tool, and conducted an experiment on the set of AUASTs of LMs described in Table~\ref{table:ljms}. The clustering tool is an Eclipse plug-in built atop the anti-unifier building tool that: takes as input a set of AUASTs of LMs extracted from the source code; applies the clustering algorithm on them; and outputs a structural generalization for each cluster.

%\section{An assessment of the Clustering tool} \label{clustering-assessment}
%To assess the effectiveness of my clustering algorithm and the tool support, I conducted an experiment on the set of AUASTs of LMs described in Table~\ref{table:ljms}.
%The tool is developed atop the anti-unifier-building tool.

\subsection{Setup}  \label{study3-setup}
I manually attempted to perform the hierarchical clustering on the set AUASTs of LMs in the test suite and constructed the detailed anti-unifier view for each cluster \RW{Why did you do it manually?  You are assuming that your manual approach represents the ground truth, which may not be true; this is a point for discussion in threats to validity.}. I discarded anti-unifiers when the anti-unification of LMs did not allow the anti-unification of logging calls with one another, as the Java elements enclosing them were not found to be corresponding. I also measured the level of similarity between AUASTs in each cluster by computing the ratio of common Java elements in the detailed anti-unifier view to the total number of Java elements of all AUASTs in that cluster. I also ran the clustering tool on the set of AUASTs to classify them using the similarity measurement.

\subsection{Results}  \label{study3-results}
I present the results of my analysis in Table~\ref{results_clustering}. The analysis of the output has been divided into three categories: correspondence, similarity, and separateness. The analysis of correspondence and similarity was described in Section~\ref{study2-results}. ``Separateness'' refers to my tool's ability to cluster Java methods with different usages of logging calls into separate groups, and the ones with similar usages of logging calls into the same clusters. \RW{You are representing this by a binary measurement.  Is that really appropriate?} By ``similar usage of logging calls'', I mean the anti-unification of AUASTs of Java methods in each cluster allows the anti-unification of their log statement nodes with one another, as well.

%the usage of logging in Java methods of each cluster is similar that they should be grouped in the same cluster, and dissimilar to the other Java methods in the other clusters that should be scattered in different clusters.
% "Relevancy" refers to the number of LMs in each cluster that are detected to berelevant as their logging calls can be anti-unified with one another and cannot be anti-unified with logging calls of LMs in the other clusters.
%AUASTs of all LMs in each cluster

\begin{figure} [H]
  \centering
  \begin{tabular}{ccccccc}
    \toprule

    \multirow{2}{*}{Cluster}&\multicolumn{2}{c}{Correspondence}&\multicolumn{2}{c}{Similarity}&\multirow{2}{*}{Separateness}\\
    \cmidrule(lr){2-3}
    \cmidrule(lr){4-5}
    &Correct (\%)&Incorrect&human&tool&\\
    \midrule
    1&28 (100)&0&0.09&0.09  &\cmark \\
    \midrule
    3&24 (92)&2&0.19&0.2& \cmark\\
    \midrule
      2&9 (100)&0&0.25&0.25& \cmark\\
 	\bottomrule
  \end{tabular}
  \caption{Results from applying the clustering tool to the test suite described in Table~\ref{table:ljms}. \protect\RW{Why are clusters 2 and 3 not in order in this table?}}
  \label{results_clustering}
\end{figure}

The clustering tool succeeded in detecting the separateness amongst AUASTs of test cases correctly. Clusters 1, 2, and 3 contain logged Java methods of cases (1, 3, 5, 8), (4, 6, 7), and (2, 9, 10), respectively, as detected by my manual inspection. It also successfully calculated the similarity between LMs of 2 clusters out of 3. In Cluster 2, the error in detecting correspondences originated from the previous study and propagated to the clustering study. However, it is trivial (0.01 \RW{Not clear what this number means}) and would have a low impact on our final results.
%between the LMs of our test suite.

\section{Summary} \label{meth2-summary}
I have presented a modified version of the agglomerative hierarchical clustering algorithm to classify Java methods with different usages of logging calls into separate groups. This algorithm is implemented as an Eclipse plug-in that takes a set of AUASTs of LMs, clusters them via the application of the anti-unifier building tool to measure pairwise similarities between the AUASTs of cluster pairs, and generates a structural generalization for each cluster. Furthermore, an experimental study was conducted to validate the effectiveness of my clustering algorithm and the tool support on a test suite.

\addtocontents{toc}{\protect\addvspace{10pt}}
\chapter{Characterization Study}\label{discover}\label{eval}

To characterize the location of log statements in source code, I conducted an experimental study that addresses the following research questions:

\begin{itemize} [leftmargin=.5in]
\item \textsc{RQ1: }\emph{``Is it possible to find patterns of where log statements occur in source code?''} I aim to investigate whether there are clusters containing a large number of LMs. This suggests that there might be common ways of locating log statements in source code.

\item \textsc{RQ2: }\emph{``What common structural characteristics do logged methods have?''} I conducted a manual analysis on the logging usage schemas (LUSs) produced by \tool{ELUS} to identify the common structural characteristics of LMs in each cluster.
\end{itemize}


\section{Experiment}  \label{setup-characterization}
%\subsection{Setup}  \label{setup}
In this experiment, I will analyze logging usage of four popular open-source software systems: \name{Apache Tomcat}, \name{Hibernate ORM}, \name{Apache Camel}, and \name{Apache Solr}. Each system is written in the Java programming language and they all utilize the same logging framework, \name{Apache log4j}. I decided to study the usage of \name{log4j} statements in these systems, as \name{Apache log4j} is ranked as the most commonly used logging package for Java\footnote{\url{https://en.wikipedia.org/wiki/Java_logging_framework}}. The studied systems are from different application domains: \name{Apache Tomcat} is a Java Servlet; \name{Hibernate ORM} is an object relational-mapping framework; \name{Apache Camel} is a rule-based routing and mediation engine; and \name{Apache Solr} is an enterprise search platform. I chose these systems as my study subject due to their popularity in their area of application (7000+ commits to the \name{GitHub} repository) and their long history of development (9 to 13 years). Table~\ref{table:CSts} represents the details about these software systems. I also decided to exclude the \name{log4j} statements at the \name{trace} and \name{debug} verbosity levels, as they are usually used by developers only during the software development phase. I believe that studying these systems could give us an insight about logging usage in real-world applications.


%I only examined the log statements from the \name{Apache log4j} framework, and


\begin{figure} [H]
  \centering
  \begin{tabular}{llcccc}
    \toprule
    \textbf{Software system}  & \textbf{Description}   & \textbf{Version} & \textbf{Start time} & \textbf{LOC} & \textbf{Log statements} \\ \hline
    {Tomcat} & Server  & 9.0.11& 2003 &306,704 &  3,117 \\ \hline
{Hibernate ORM} & Framework & 4.2.23 & 2004 & 509,734 & 1,939 \\ \hline
    {Camel} &  Middleware & 2.18.0 &  2007 &120,528 & 2,177 \\
    \hline
    {Solr} &  Platform  & 6.2.1 &  2007 & 128,824 & 2,319 \\
%{OpenMeetings} & Web Conferencing & & 2.0 &38K &3506 \\ \hline
 %   {QuickFIX/J} & Engine & 1.6 & 48K & 2958 \\
   % \bottomrule
    \toprule
  \end{tabular}
   %\caption{Details of the four open-source software systems that make use of the {Apache log4j} logging framework.}
    \caption{Summary of the four software systems used in the characterization study.}
\label{table:CSts}
\end{figure}


My proof-of-concept implementation takes the source code of these systems as inputs, extracts the ASTs of their LMs, applies the proposed algorithm to construct AUASTs, categorizes the AUASTs into clusters, and outputs the structural generalization view for each cluster.
%, called LUS.


\subsection{Results}  \label{results-characterization}
The experimental results for each software system are presented in Table~\ref{tab_results_1}. This table describes the total number of detected \name{log4j} statements (debug- and trace-level log statements are excluded), the number of logged methods (LMs); the number of generated clusters; the number of generalized clusters containing more than one LM; the number of singleton clusters that only contain one LM; and the reduction percentage calculated by the Equation~\ref{reduction_eq}. In addition, Figure~\ref{fig:histograms} shows the histograms of the number of LMs per cluster for each system.


\begin{equation}\label{reduction_eq}
\id{reduction} = \frac{|\id{Primitive~clusters}| - |\id{Total~clusters}|}{|\id{Primitive~clusters}|}
\end{equation}


\begin{table}[h]
\vspace*{1em}
\let\A\relax
\newlength{\A} 
\settowidth{\A}{1098}
\let\B\relax
\newlength{\B}
\settowidth{\B}{128}
\let\C\relax
\newlength{\C}
\settowidth{\C}{632}
\let\D\relax
\newlength{\D}
\settowidth{\D}{1471}
\let\Pwa\relax
\newlength{\Pwa}
\settowidth{\Pwa}{\%}
\centering\begin{tabular}{lcccc@{\hspace{\Pwa}}}
  \toprule
   & \multicolumn{1}{c}{Tomcat}  & \multicolumn{1}{c}{Hibernate} & \multicolumn{1}{c}{Camel}  & \multicolumn{1}{c}{Solr} \\
  \midrule
  
  log4j statements               & \makebox[\A][r]{1098} & \makebox[\B][r]{128} & \makebox[\C][r]{632} & \makebox[\D][r]{1471}   \\
  
  LMs                            & \makebox[\A][r]{658}  & \makebox[\B][r]{81}  & \makebox[\C][r]{490} & \makebox[\D][r]{818}    \\\midrule

  Primitive clusters at start    & \makebox[\A][r]{1098} & \makebox[\B][r]{128} & \makebox[\C][r]{632} & \makebox[\D][r]{1471}   \\

  Non-singleton clusters resulting & \makebox[\A][r]{14}   & \makebox[\B][r]{4}   & \makebox[\C][r]{9}  & \makebox[\D][r]{14} \\

  Singleton clusters resulting   & \makebox[\A][r]{15}   & \makebox[\B][r]{3}   & \makebox[\C][r]{13}  & \makebox[\D][r]{24} \\

  Total clusters resulting       & \makebox[\A][r]{29}   & \makebox[\B][r]{7}   & \makebox[\C][r]{24}  & \makebox[\D][r]{38}\\\midrule

  Reduction                      & \makebox[\A][r]{97\%\hspace*{-\Pwa}} & \makebox[\B][r]{94\%\hspace*{-\Pwa}} & \makebox[\C][r]{96\%\hspace*{-\Pwa}}& \makebox[\D][r]{97\%\hspace*{-\Pwa}} \\


  \toprule
\end{tabular}
%\caption{Within-version experiment.}
\caption{The experimental results.}
%\caption{Experimental results for the software systems}
\label{tab_results_1} \vspace*{1em}
\end{table}
%[width = 1\textwidth, height = 0.4\textheight]

\begin{sidewaysfigure} [p]
    \centering
  \centering\includegraphics [width = 1\textwidth, height = 0.5\textheight] {Charts/Histograms.png}
  \caption{Histograms of the number of LMs per cluster.}
  \label{fig:histograms}
\end{sidewaysfigure}


\subsection{Analysis}  \label{analysis}
The first research question is: \emph{"Is it possible to find patterns of where log statements occur in source code?"} As shown in Table~\ref{tab_results_1}, the number of clusters has been reduced by more than 90\% in all the studied systems, indicating that developers follow some patterns for locating the log statements in source code. Furthermore, histograms depicted in Figure~\ref{fig:histograms} show that in all the studied systems, a few clusters contain a large number of LMs; however, the other clusters contain a very small number of LMs. This indicates that in these cases, developers follow a more complex or rare way of locating log statements. These exceptions might also happen due to the poor usage of logging statements in source code, which impacts the quality of the entire system negatively.

The second research question is : \emph{``What common structural characteristics do logged methods have?''}
To address this question, I manually went through the LUSs to identify the common structural characteristics of locating log statements in source code.

%a few rare exception
%ADD-> THE USAGE WITH AN EXCEPTION

\subsubsection{\emph{Categorizing logging usage}} \label{categories}
In this section, I will describe the anti-unifiers of logging usage by examining the LUSs produced by \tool{ELUS}. In general, there are five main categories of anti-unifiers in the logging usage. Each category represents one cluster of each software system that contains a large number of LMs, that is, the cluster anti-unifier represents a common way of locating log statements in source code. In the following sections, I will describe the common structural characteristics of each category represented by the anti-unifier. In addition, Figure~\ref{fig:categories} presents the number of LMs in each category and its percentage of the total number of LMs for each of the software systems. As shown in this figure, the distribution of logging categories vary from application to application, which implies that logging guidelines should be provided at application-specific level in order to establish effective logging practices.
% that their anti-unifiers are corresponded, as they have common structural characteristics. Also, these clusters
\begin{sidewaysfigure} [p]
   \centering\includegraphics [width = 1\textwidth, height = 0.7\textheight]{Charts/Categories.png}
  \caption{The distribution of the categories of anti-unifiers in the logging usage.}
  \label{fig:categories}
\end{sidewaysfigure}


% how to compare?

\subsubsection{\emph{A. Exception Catch-block Logging}}  \label{Exception catch-block logging}
The main common structural characteristics of the anti-unifiers of this category are the \code{try} statements, where the log statements are located inside the body of a \code{catch} clause. As shown in Figure~\ref{fig:categories}, 14\% to 52\% of the total LMs are described by the anti-unifiers of this category, and it is the most commonly used logging usage category in the \name{Tomcat} and \name{Hibernate} software systems. The popularity of this category among all the studied systems is due to the fact that exception handling using the \code{try}/\code{catch} blocks is a common technique in the Java programming language.

%a large portion of LMs are described by the anti-unifiers of this category.

\subsubsection{\emph{B. Conditional Logging}}  \label{conditional logging}
In this category, log statements are enclosed by \code{if}-statements with their test expressions mostly among \textit{infixExpression}, \textit{methodInvocation}, or \textit{binaryExpression} nodes. The \textit{infixExpression}s mostly either check the equality of an expression to the null literal or tests the validity of the value of a variable; the \code{if}-statements testing \textit{methodInvocation}s mostly check if the return value of an invoked method is an indicator of a potential problem within a system; and the \code{if}-statements testing \textit{binaryExpression}s mostly check if a Boolean literal is incorrect. As shown in Figure~\ref{fig:categories}, 16\% to 37\% of the total LMs are described by the anti-unifiers of this category, and it is the most commonly used logging category in the \name{Solr} system.

\subsubsection{\emph{C. Outer Method Logging}}  \label{method logging}
In this category, the log statements are located inside the body of \textit{methodDeclaration} nodes but outside of other structures nested therein. A common structural characteristic of the anti-unifiers in this category is that they mostly use the \code{throw} statement to throw an exception if an error occurs. The percentage of LMs that are described by the anti-unifiers of this category ranges from 3\% to 51\%, and it is the most common logging usage category in the \name{Camel} software system. This suggests that developers use logging to record important method granularity information about the state of a software system. This information might be used later to detect the root causes of an application problem.
% %?
% throw exception



\subsubsection{\emph{D. Control Flow Logging}}  \label{Control flow logging}
In this category, the log statements are located inside the body of either \code{switch}- or \code{if}-\code{else} statements. These log statements can be used to reveal necessary information to track the location of root causes of a potential problem in a software system. According to the Figure~\ref{fig:categories}, 0\% to 5\% of the total LMs are described by the anti-unifiers of this category.


\subsubsection{\emph{E. Exception Try-Block Logging}}  \label{Exception try-block logging}
In this category, the log statements are located inside the body of the \code{try} clause of \code{try}/\code{catch} statements. These log statements can be used to record important information about the code that may throw an exception. As shown in Figure~\ref{fig:categories}, 0\% to 7\% of the total LMs of the studied systems are described by the anti-unifiers of this category.

%COMPARE SYSTEMS???
%CATEGORIES FIGURE CHANGE???

\section{Evaluation}  \label{evaluation}
%\RW{For constrained variables, the precision should be 100\%.  The fact that it is not means that there are some bugs.  There is no point in pretending otherwise; you are better off pointing this out.  It's OK: all software contains bugs.  Later, in the Discussion, you should point out that intermediate forms of constraint are possible: instead of constraining to only the exactly desired substitutions, you can constrain to the legal types of nodes (like methodInvocation) that can be used to substitute a variable.  The (still open) question is if that would yield better results than unconstrained variables.}
%\NZ{I added it to the discussion chapter}

An empirical study is conducted to evaluate the quality of the anti-unifiers generated by \name{ELUS} in describing the location of log statements in source code. Section~\ref{precision} describes the process of evaluating the precision and recall of \tool{ELUS}.


\subsection{{Calculating the precision and recall}}  \label{precision}
To find the locations in source code that are described by an anti-unifier using \tool{ELUS}, I applied the \func{Determine-Locations} algorithm, which takes the anti-unifier and a list of all methods in source code and outputs a list of methods that their AUAST matches the anti-unifier AUAST. This algorithm anti-unifies each method in the list with the anti-unifier using the \func{Antiunify} algorithm described in Section~\ref{meth-antiUnifier} (lines~2--3). If the result equals the anti-unifier, that method will be added to the list of locations matching the anti-unifier (lines~4--5). \func{Equals} is a procedure that takes two AUAST nodes and checks whether they are equal or not. To evaluate the generalizability of the anti-unifiers, I have implemented this procedure in two ways: (1) when variables are considered to be \emph{constrained}, it tests that the non-variable nodes are identical in the two AUASTs and checks if the constraints of variable are identical or not; (2) when variables are considered to be \emph{unconstrained}, it tests that the non-variable nodes are identical in the two AUASTs, but permits unconstrained variables to differ. I ran my tool on the source code of the four studied systems and applied this algorithm to find the locations in the code that matches the structure of the generated anti-unifiers. Then, the precision and recall metrics are calculated using Equations~\ref{precision_eq} and~\ref{recall_eq}, respectively.


\begin{algorithm}
\caption{\func{Determine-Locations}($\id{antiUnifier}$,$\id{methods}$) finds the locations in source code that matches an anti-unifier.}
\label{alg-determine-locations}
\begin{algorithmic}[1]
\DetermineLocations
    \State $\id{locations} \gets \{\}$
    \For {$\id{method} \in \id{methods}$}
    \State $\id{result} \gets  \func{AntiUnify}(\id{antiUnfier}, \id{method})$
		\If {$\func{Equals}(\id{result}, \id{antiUnifier})$ }	
				 	\State{$\func{Append}(\id{method}, \id{locations})$ }
		\EndIf 		
		\EndFor
 \Return $\id{locations} $  	
  \end{algorithmic}
\end{algorithm}

\begin{equation}\label{precision_eq}
\id{precision} = \frac{\id{TP}}{{\id{TP}+\id{FP}}}
\end{equation}


\begin{equation}\label{recall_eq}
\id{recall} = \frac{\id{TP}}{{\id{TP}+\id{FN}}}
\end{equation}


Where $\id{TP}$ is the number of correct locations obtained, $\id{FP}$ is the number of incorrect locations retrieved, and $\id{FN}$ is the number of correct locations that were not retrieved. Figures~\ref{fig:precision} and \ref{fig:recall} show the precision and recall results for each software system where the experiment was run once with constrained variables and once with unconstrained variables.


\begin{figure} [H]
  \centering\includegraphics [width = 0.7\textwidth, height = 0.3\textheight]{Charts/Precision.png}
  \caption{The precision of \tool{ELUS}.}
  \label{fig:precision}
\end{figure}

\begin{figure} [H]
  \centering\includegraphics [width = 0.7\textwidth, height = 0.3\textheight]{Charts/Recall.png}
  \caption{The recall of \tool{ELUS}.}
  \label{fig:recall}
\end{figure}

\subsection{{Precision results}}  \label{precision-results}
The green and yellow bars in Figure~\ref{fig:precision} show the precision results when the experiment was run with constrained and unconstrained variables, respectively. I have also calculated the overall average precision of \tool{ELUS}, by averaging the precision values between the four software systems. The average precision for \tool{ELUS} is 84\% and 32\% for constrained and unconstrained variable experiments, respectively. In general, the precision for constrained variables is fairly high. The main reason behind the high precision of constrained variables is that in these cases, the variables can only be substituted with some particular nodes, which makes the anti-unifier very specific. However, there are two main reasons for the fact that precision of constrained experiment is not 100\%:
\begin{enumerate} [leftmargin=.5in]

\item \emph{Split cases}: To handle the cases containing multiple log statements, I split them into more than one case, where each contains only one logging statement (see Section~\ref{meth-multipleLogs}). However, to find the locations in source code that are described by anti-unifiers using the \func{Determine-Locations} algorithm, I compared them with all the methods in source code without splitting them into multiple cases, which results in retrieving a number of incorrect locations.

\item \emph{Software bugs}: The fact that precision results are not ideal indicates that \tool{ELUS} has some bugs. In the further work, I aim to improve these results by fixing the software bugs.
\end{enumerate}

According to the Figure~\ref{fig:precision}, the precision is fairly low for unconstrained variables. The main reason of the low precision for these cases is the fact that the unconstrained variables can be substituted by any nodes, which makes the anti-unifiers too general. As a result, the tool finds many incorrect locations the matches the anti-unifiers.





%\RW{Rewrite this according to the discussion we had in the meeting}

\subsection{{Recall results}}  \label{recall-results}
The green and yellow bars in Figure~\ref{fig:recall}  show the recall results when the experiment was run with constrained and unconstrained variables, respectively. I have also calculated the overall average recall of \tool{ELUS}, by averaging the recall values between the studied systems. The average recall for \tool{ELUS} is 80\% and 97\% for the constrained and unconstrained variable experiments, respectively. In general, when variables are constrained, \tool{ELUS} can detect many correct locations, as the recalls for all the studied systems are fairly high. Also, \tool{ELUS} can detect most of the correct locations in source code when no constraints are taken on variable nodes.

The main reason behind \tool{ELUS}'s failure to detect the correct locations is the potential complexities in constructing anti-unifiers from a large set of source code fragments. As in some cases, the anti-unifier might not maintain the correct locations of nodes in the AST hierarchy, and thus \tool{ELUS} would not be able to successfully construct the anti-unifiers of logging usage in source code.
%detect correct locations of log statements in the source code.
%?
%\RW{Ah, I see that you have a note in here about the intermediate constraints.}

%ADDDDD
% assess the generalizability of the anti-unifiers --> in between
%In general,  the tool retrieved some locations matching the anti-unifiers, while in fact they do not match.
%the tool failed to retrieve the correct locations, and in other cases



\section{Usage}  \label{usageELUS}
The insightful findings of my characterization study regarding the logging usage in several real-world software systems can be used to enhance the quality of existing logging practices by providing some logging guidelines for developers. For example, Figure~\ref{inapproprate-ex1} shows a logged method that belongs to a singleton cluster in my experiment. This \name{Java} method is an example of a poor usage of a log statement in code, as the list \code{liveNodes} can be \code{null}, and thus a \code{NullPointerException} can be thrown causing a system crash. To enhance the quality of the logging usage in this code snippet, a developer my look at our findings to be informed of how usually other developers locate log statements in similar situations. As noted in Section~\ref{categories}, to avoid the \code{NullPointerException}, developers usually insert the logging call into the body of an \code{if} statement to check if the value of the variable needed to be logged is not \code{null}. Hence, she can improve the quality of the logging usage in this example by inserting the logging call inside an \code{if} statement and log the needful information if the value of the list \code{liveNodes} is not \code{null} (lines~6--8 of Figure~\ref{approprate-ex1}). This example demonstrates how these findings can be used in practice to improve the quality of logging practices.
% in real-world application.


\begin{figure}[p]
\def\baselinestretch{1}
\begin{lstlisting}[escapechar=!]
public void setUp() throws Exception {
    SolrZkClient zkClient=new SolrZkClient(zkServer.getZkAddress(),AbstractZkTestCase.TIMEOUT);
    for (int i=0; i < 30; i++) {
       List<String> liveNodes=zkClient.getChildren("/live_nodes",null,true);
       Thread.sleep(1000);
       !\colorbox{yellowGreen}{log.info("Waiting for more nodes to come up, now: " + liveNodes.size());}!
    }
}
\end{lstlisting}
\caption[An example of an inappropriate usage of a log statement in a Java method.]{An example of an inappropriate usage of a log statement in a Java method.\label{inapproprate-ex1}}
\end{figure}



\begin{figure}[p]
\def\baselinestretch{1}
\begin{lstlisting}[escapechar=*]
public void setUp() throws Exception {
    SolrZkClient zkClient=new SolrZkClient(zkServer.getZkAddress(),AbstractZkTestCase.TIMEOUT);
    for (int i=0; i < 30; i++) {
       List<String> liveNodes=zkClient.getChildren("/live_nodes",null,true);
       Thread.sleep(1000);
       *\colorbox{yellowGreen}{if (liveNodes != null) }*
         *\colorbox{yellowGreen}{log.info("Waiting for more nodes to come up, now: " + liveNodes.size()); }*
    }
}
\end{lstlisting}
\caption[Modified Java method of Figure~\protect\ref{inapproprate-ex1} for the purpose of enhancing the logging usage.]{Modified Java method of Figure~\ref{inapproprate-ex1} for the purpose of enhancing the logging usage.\label{approprate-ex1}}
\end{figure}
   % !\colorbox{yellowGreen}{}}!
  %!\colorbox{yellowGreen}{if (liveNodes  null) { }!

\section{Summary}
I conducted an experimental study to characterize the location of log statements by applying my tool on the source code of four full software systems that make use of the \name{Apache log4j} logging framework. My tool inputs the source code of these systems, extracts ASTs of LMs, applies the proposed anti-unification and clustering algorithms, and outputs the anti-unifier for each cluster. I also conducted an experimental study to evaluate the precision and the recall of \tool{ELUS} in constructing the anti-unifiers that describe the location of log statements in source code. This experiment shows that \tool{ELUS} has achieved promising results in terms of precision and recall. Furthermore, the results taken from the characterization experiment shows that there are common ways of locating log statements. I manually examined the detailed view of structural generalizations and categorized the anti-unifiers of logging usage. In the last section of this chapter, I provided an example to demonstrate the usage of the findings of my characterization study in practice.


% In summary, this experiment shows that \tool{ELUS} has achieved promising results in terms of precision and recall metrics.

% My analysis has resulted in ... different anti-unifiers in the logging usage.

% figure out the common structural characteristics of LMs in each cluster.
%I found out that most log statements are embedded inside ....

\addtocontents{toc}{\protect\addvspace{10pt}}
\chapter{Discussion}  \label{diss}

In this chapter, I discuss the validity of my evaluation and the characterization study (Section~\ref{threads}), and a number of remaining issues, including the limitations and pitfalls of my approach and the tool support (Section~\ref{limitations}), the usage of anti-unification theory for other applications (Section~\ref{auTheory}), and the definition of an intermediate form of structural variable constraints (Section~\ref{intC}).

\section{Threats to validity}  \label{threads}
Prior to applying my tool for characterizing logging usage in real-world software systems, I have conducted three experiments to investigate the effectiveness of the proposed approach. However, there are several potential threats regarding the validity of these experiments. First, the results of my manual examination might be biased, as I determined the correct correspondences between the AUASTs of my test suite based on a similarity measurement. To limit the bias, other people can be involved to double check the accuracy of my manual inspection in a future work. Secondly, the experiments have examined one test suite containing a set of LMs from a real-world software system, though different test suites may generate different results. In spite of the fact that I cannot claim that the set of tested LMs is a good representative of all LMs in real-world software systems, the experimental results are still promising, as the locations of log statements in the tested methods are different. Hence, these experiments have sufficed to indicate the effectiveness of my approach in constructing structural generalizations. Another potential thread is that the successful rate of detecting correspondences by my tool might happened accidentally only for my test suite. To resolve this doubt, I examined the cases where my tool fails to detect correct correspondences, and I found that the failures are due to the fundamental limitations and complexities in the construction of structural generalizations through the use of structural correspondence. That is, my tool creates structural generalizations successfully with regard to what my algorithm should generate.

%these experiments have sufficed to indicate???

%and the correct way of classifying the set of AUASTs in our
%, and the assumptions taken in developing the algorithms.
%CITATION POPULARITY!!!
%There are several threats to the validity of our characterization study.

A potential thread to the validity of the characterization study is the degree to which my selected set of software systems is a good representation of all real-world logging practices. To address this issue, I selected various open-source software projects in terms of application. The studied software systems are widely used by many developers for a long period of time.  However, the fact that I only studied \name{Apache Log4j} statements might limit the generalizability of findings. To improve the generalizability of this study, I have chosen one system not from Apache Software Foundation. However, my findings might not be able to reflect the characteristics of logging usage in other types of systems such as industry systems, or software written in other programming languages.
%In addition, these systems are widely used by many developers for a long period of time. 


%application??
%citation??? 10 years
%application of softwares???
%other threads?

\section{The pitfalls of my tool}  \label{limitations}
%title?? choose a name for our tool????
There are some issues that the approximation approach and my tool support is not able to handle perfectly, including inaccurate node ordering, and the resolution of conflicts happened in constructing the anti-unifiers.


\subsubsection{Inaccurate node ordering}  \label{mismatch}
My anti-unification algorithm does not guarantee to maintain the correct sequence of statements in the body of methods in case of anti-unifying two method declaration nodes, as the order of statement nodes is not considered in determining the best correspondences. For example, consider we have two corresponding methods $\id{method_1}$ and $\id{method_2}$ embodying {$\id{a_1}$, $\id{a_2}$} and {$\id{b_1}$, $\id{b_2}$} sequences of statements, respectively. If the $\id{b_1}$ and $\id{b_2}$ nodes are found to be the best correspondences of the $\id{a_2}$ and $\id{a_1}$ nodes, respectively, $\id{a_1}$ will be anti-unified with $\id{b_2}$  and $\id{a_2}$ will be anti-unified with $\id{b_1}$ to construct the structural generalization. Therefore, the anti-unification algorithm does not preserve the correct ordering of nodes in the original structures.
%????
%% IIf the $\id{b_1}$ and $\id{b_2}$ nodes are found to be the best correspondences of the $\id{a_2}$ and $\id{a_1}$ nodes, respectively, $\id{a_1}$ will be anti-unified with $\id{b_2}$  and $\id{a_2}$  will be anti-unified with $\id{b_1}$ to construct the structural generalization.??

%%%??????
%f methods in case of anti-unifying two method declaration node??

%\subsection{handling conflicts in constructing anti-unifiers}  \label{conflicts}
\subsubsection{Conflict resolution}  \label{conflicts}
The decisions I have made to resolve the conflicts occurred in constructing structural generalizations might affect the accuracy of our results.
For example, in situations where I have two correspondences with the same similarity value in the ordered list of correspondence connections, my approach picks the one which involves two subtrees with higher number of nodes, though it might be not the best choice for all cases.
In addition, I consider AST hierarchies to perform anti-unification. That is, my algorithm does not anti-unify two nodes if their parent nodes are not found to be corresponded. As a result, situations can occur where in fact two nodes should be anti-unified with each other, while they are not anti-unified by the tool. Though these decisions led me to get approximate results, they helped to limit the complexity of my approach, allowing the implementation of it as a practical solution.
%%%????




%%%WHY PRECISON AND RECALL IS NOT 100%???


 % add semantic to structural information to detect correspondences??
%limited typing information to determine correspondences by %Jigsaw
%\item Our tool does not guarantee the correctness of determining the best correspondences due to
%\begin{itemize} [leftmargin=.3in]
%\item the various conflicts that happen
%\item limited typing information to determine correspondences by %Jigsaw
%\end{itemize}
%\item Structural generalizations constructed by our tool are not in %the form of executable code
%[leftmargin=.3in]\end{itemize}


%\section{Other applications}  \label{other_applications}
%Any applications that are involved in the inference of structural patterns in source code even infrequently-used patterns might benefit from our tool’s underlying framework.
%% EXAMPLE???
%Furthermore, understanding the commonalities and differences between source code fragments has application in several areas of software engineering, such as code clone detection, API usage pattern collation, recommending replacements for API migration, and merging different branches of version control systems. Our tool`s functionality to construct a detailed view of structural generalizations of a set of source code fragments via structural correspondence could be used to improve the results of these studies as well.
%edit!!!
\section{Applications of anti-unification}  \label{auTheory}
My study demonstrates the application of an extended from of anti-unification (HOAUMT) to infer usage patters of log statements in source code via the creation of structural generalizations. Anti-unification and its extensions have been already applied to solve several theoretical and practical problems, such as analogy making \cite{2010:bsc:schmidt}, determining lemma generation in equational inductive proofs \cite{2005:aij:burghardt}, and detecting the construction laws for a sequence of structures \cite{2005:aij:burghardt}.
%CITATION???


Higher-order anti-unification modulo theories can be used to create generalizations in different contexts, and therefore the set of equational theories should be developed particularly for the higher-order structure used in each problem context. That is, the utility of these theories are highly dependent on how well they allow the incorporation of semantic knowledge of structures. In addition, these theories should ensure that only a finite number of anti-instances exist for each structure. The practical experiments I have conducted through the application of my tool on a test suite demonstrate that an approximation of HOAUMT can be successfully used to construct structural generalizations required to solve a problem.
%finite?
%%%Taking all these considerations into account enables HOAUMT to anti-unify sets of structures in a particular context.????



\section{intermediate constrainted variables}  \label{intC}
%The fact that precision for constrained variables is not 100\% indicates that the \tool{ELUS} has some bugs. In the further work, I aim to improve these results by fixing the software bugs. 

As the anti-unifiers with constrained variables are too specific and the anti-unifiers with constrained variables are too general, an intermediate form of constraints can be defined for the structural variables of anti-unifiers. That is, instead of constraining to only the specific substitutions, the intermediate constraint could be defined as the types of nodes (e.g., \code{Infix} \code{Expression}) that can be used to substitute a variable. A future experiment can be run to see whether considering intermediate constrained variables would yield to more reasonable results than the unconstrained variables. 


\section{Summary}  \label{diss-summary}
I discussed the potential threads to validity of my evaluation and characterization study. To limit the bias of my experiments, I selected the test cases form a real system with various levels of similarity in the usage of log statements. Furthermore, I examined the failed test cases to assure that my tool works when it should work with regard to the proposed algorithm. I will also make my test suite available for public examination to further check the accuracy of my manual inspection. For the characterization study, I selected various software systems in terms of functionality. I also discussed the remaining issues with the tool support, including inaccurate node ordering and handling the conflicts happened in the construction of anti-unifiers.
% I conducted to evaluate the effectiveness of my approach and the tool support

This work aims to provide a detailed view of structural generalizations constructed from a set of source code fragments that use log statements via the application of anti-unification and clustering. However, I argued how higher-order anti-unification modulo theories can be effectively approximated for various applications by means of developing an appropriate set of equational theories particularly for the higher-order structure used in each problem context. I also explained how the definition an intermediate form of variable constraints may yield to better experimental results.
%I also explained how the definition an intermediate form of constraints for structural variables of anti-unifiers may yield to better experimental results.?

%in future as the extensions of my work.

% I discussed that any other applications involved in the inference of structural usage patterns of a particular statement or understanding the commonalities and differences between a set of source code fragments could benefit from our tool’s underlying framework.

\addtocontents{toc}{\protect\addvspace{10pt}}
\chapter{Related Work}  \label{rw}
In this chapter, we review related work to the topics of our study including: the application of logging in real-world software systems (Section~\ref{logging}), determining correspondences in the source code (Section~\ref{ch7-corr}), data mining approaches to extract API usage patterns (Section~\ref{ch7-usage-patterns}), anti-unification and its application to detect strcutural correspondences and construct generalizations (Section~\ref{ch7-au}), and clustering (Section~\ref{ch7-clustering}).
% constructing the structural generalizations (Section~\ref{ch7-generalization}
\section{Usage of logging}  \label{logging}
Logging is a conventional programming practice to record a software system's runtime information that can be used in post-modern analysis to trace the root causes of systems' activities. Log analysis is most often performed for failure diagnosis, system behavioral understanding, system security monitoring and performance diagnostics purposes as described below:
\begin{itemize} [leftmargin=0.7in]
\item \textbf{Log analysis for failure diagnosis: }\citet{xu2009detecting} use statistical techniques to learn a decision tree based signature from the console logs and then utilize the signature to diagnose anomalies. SherLog \cite{yuan2010sherlog} uses failure log messages to infer the source code paths that might have been executed during a failure.
\item \textbf{Log analysis for system behavior understanding: }\citet{fu2013contextual} present an approach for understanding system behavior through contextual analysis of logs. They first extracted execution patterns reflected by a sequence of system logs and then utilized the patterns to find contextual factors from logs that causes a specific system behavior. The Linux Trace Toolkit \cite{yaghmour2000measuringandcharacter} was created to record and analyze system behavior by providing an efficient kernel-level event logging infrastructure. A more flexible approach is taken by DTrace \cite{cantrill2004dynamic} which allows dynamic modification of kernel code.
\item \textbf{Log analysis for system security monitoring: }\citet{bishop1989model} proposes a formal model of system's security monitoring using logging and auditing. \citet{peisert2007toward} have developed a model that demonstrates a mechanism for extracting logging information to detect how an intrusion occurs in software systems.% Jiang et al. [2009b] present an approach to automatically detect problems of load tests by mining the execution logs of an application. Many software systems must be load tested for their functional and performance problems diagnosis.
\item \textbf{Log analysis for performance diagnosis: }\citet{nagaraj2012structured} developed an automated tool to assist developers in diagnosis and correction of performance issues in distributed systems by analyzing system behaviors extracted from the log data.
\end{itemize}

\citet{jiang2009understanding} study the effectiveness of logging in problem diagnosis. Their study shows that customer problems in software systems with logging resolve faster than those without logging by investigating the correlations between failure root causes and diagnosis time. Despite the importance of logging for software development and maintenance, few studies have been conducted in pursuit of understanding logging usage in real-world software. \citet{yuan2012characterizing} provides a quantitative characteristic study to investigate log message modifications on four open-source software systems by mining their revision history. Their study shows that developers spend a great effort to modify logging calls as after-thoughts, which indicates that they are not satisfied with the log quality in their first attempt. They also characterize where developers spend most of their time in modifying the log messages.

\citet{yuan2012conservative} studies the problem of lack of log messages for error diagnosis and suggests to log when generic error conditions happens. LogEnhancer \cite{yuan2012improving} automatically enhances existing log message by detecting important variable values and inserting them into the log messages. However, these studies only consider code snippets containing bugs that are needed to be logged and do not consider other code snippets containing no bugs but still need to be logged. Moreover, these studies mainly research log message modifications and potential enhancements of them, however, the focus of this study is on understanding where logging calls are used in the source code.
% where to log
% Sadi

\section{Correspondence}  \label{ch7-corr}

Several studies have been conducted to find similarities and differences between the source code fragments. \citet{baxter1998clone} develop an algorithm to detect code clones in source code that uses hash functions to partition subtrees of ASTs of a program source code and then find common subtrees in the same partition through a tree comparison algorithm. \citet{apiwattanapong2004differencing} present a top-down approach to detect differences and correspondences between two versions of a Java program, through comparison of the control flow graphs created from the source code. \citet{holmes2005strathcona} recommends relevant code snippet examples from a source code repository for the sake of helping developers to find examples of how to use an API by heuristically matching the structure of the code under development with the source code in the repository. Coogle \cite{sager2006detecting} is developed to detect similar Java classes through converting ASTs to a normalized format and then comparing them through tree similarity algorithms. However, none of these approaches determines the detailed structural correspondences needed in our context.

Umami \cite{2014:uofc:cossette} presents a new approach, called Matching via Structural generalization (MSG), to recommend replacements for API migration. He used the Jigsaw tool to find structural correspondences, however, their proposed algorithm does not suffice to our context since it does not construct a generalization to represent structural similarities and differences. It also does not take the required constraints in determining correspondences needed to solve our problem.

\section{API usages patterns}  \label{ch7-usage-patterns}

Various data mining approaches has been used to extract API usages patterns out of the source code such as unordered pattern mining and sequential pattern mining \cite{robillard2013automated}. Unordered pattern mining, such as association rule mining and itemset mining, extracts a set of API usage rules without considering their order \cite{agrawal1994fast}. CodeWeb \cite{michail2000data} uses data mining association rules to identify reuse patterns between a source code under development and a specific library. PR-Miner \cite{li2005pr} uses frequent itemset mining to extract implicit programming rules from source code and detect violations. The sequential pattern mining technique is different from the unordered one in the way that it considers the order of API usage. As an example, MAPO \cite{xie2006mapo} combines frequent subsequence mining with clustering to extract API usage patterns from the source code. The other technique for extracting API usage patterns is through statistical source code analysis. For example, PopCon \cite{holmes2008newbie} is a tool developed to help developers understanding how to use APIs in their source code through calculating popularity statistics for each API of a library. \citet{acharya2007mining} present a framework to extract API usage scenarios as partial orders. Specifications were extracted from frequent partial orders. They adapted a compile time model checker to generate control-flow-sensitive static traces of APIs, from which API usage scenarios were extracted. However, none of these approaches suffice to determine the detailed structural correspondences.

\section{Anti-unification}  \label{ch7-au}
Anti-unification is the problem of finding the most specific generalization of two terms. First-order syntactical anti-unification was introduced by \citet{plotkin1970note} and \citet{reynolds1970transformational} independently. \citet{burghardt1996implementing} extend the notion of anti-unification to E-anti-unification to incorporate background knowledge to syntactical anti-unification, which is required for some applications. anti-unification has been applied in various studies for program analysis. \citet{bulychev2009evaluation} suggest an anti-unification algorithm to detect clones in ASTs. Their approach consists of three stages: first, identifying similar statements through anti-unification and classifying them into clusters; second, determining similar sequences of statements with the same Cluster identifier; third, refining candidate statement sequences using an anti-unification based similarity measurement to generate final clones. However, their approach does not construct a generalization by determining the structural correspondences. \citet{2007:esec_fse:cottrell} propose Breakaway to automatically determine structural correspondences between a pair of abstract syntax trees (ASTs) to create a generalized correspondence view. However, their approach does not allow us to detect the best structural correspondence for each node suited to our problem. \citet{2008:fse:cottrell} develop Jigsaw to help developers integrate small-scale reused source code into their own code by determining structural correspondences through the application of higher-order anti-unification modulo theories. However, considering the limitations of our study in determining correspondences, their approach does not suffice to construct a structural generalization needed in our context.

%\item Sadi [2011] proposed an anti-unification algorithm to characterize the location of logging usages in the source code, however,
%\begin{itemize} [leftmargin=.3in]
%\item he has not applied anti-unification appropriately!!! \RW{You would need to explain what that means}
% Higher-order anti-unification modulo theories is formally undecidable [Burghardt, 2005].

\section{Clustering}  \label{ch7-clustering}

%optimal clusters
Clustering is an unsupervised machine mining technique that aims to organize a collection of data into clusters, such that intra-cluster similarity is maximized and the inter-cluster similarity is minimized \cite{karypis1999chameleon,grira2004unsupervised}. We divided existing clustering approaches into two major categories: partitional clustering and hierarchical clustering. Partitional clustering try to classify a data set into $k$ clusters such that the partition optimizes a pre-determined criterion \cite{karypis1999chameleon}. The most popular partitional clustering algorithm is k-means, which repeatedly assigns each data point to a cluster with the nearest centroid and computes the new cluster centroids accordingly until a pre-determined number of clusters is obtained \cite{bouguettaya2015efficient}. However, k-means clustering algorithm is not a good fit to our problem since it requires to predefine the number of clusters we want to come up with, which is not reasonable in our context. %?

Hierarchical clustering algorithms produce a nested grouping of clusters, with single point clusters at the bottom and an all-inclusive cluster at the top \cite{karypis1999chameleon}. Agglomerative hierarchical clustering is one of the main stream clustering methods \cite{day1984efficient} and has applications in document retrieval \cite{voorhees1986implementing} and information retrieval from a search engine query log \cite{beeferman2000agglomerative}. It starts with singleton clusters, where each contains one data point. Then it repeatedly merges the two most similar clusters to form a bigger one until a pre-determined number of clusters is obtained or the similarity between the closest clusters is below a pre-determined threshold value. Hierarchical clustering algorithms work implicitly or explicitly with the $n \times n$ similarity matrix such that an element in row $i$ and column $j$ represents the similarity between the $i^{\text{th}}$ and the $j^{\text{th}}$ clusters \cite{karypis1999chameleon}.

There are various versions of agglomerative hierarchical algorithms that mainly differ in how they update the similarity between clusters. There are various methods to measure the similarity between clusters, such as single linkage, complete linkage, average linkage, and centroids [Rasmussen, 1992]. In the single linkage method, the similarity is measured by the similarity of the closest pair of data points of the two clusters. In the complete linkage method, the similarity is computed by the similarity of the farthest pair of data points of the two clusters. In the average linkage method, the similarity is measured by the average similarity of all pairwise similarities of data points of the two clusters. In the centroids methods, each cluster is represented by a centroid of all data points in the cluster, and the similarity between two clusters is measured by the similarity of the clusters' centroids.
However, in our application, each cluster is composed of one AUAST, and the similarity between two clusters is measured by the similarity between the clusters' AUASTs, which is computed via anti-unification.
%optimal clusters- read dr. denzinger email
%similarity and distance two sides of a coin
% k-means,the predetermined criterion is met
% Hierarchical clustering approaches produce clusters of higher quality, however, these approaches suffer from high time cost[Bouguettay]

\section{Summary}  \label{back-summary}
%machine learning, other generalization approaches

Despite the great importance of logging and its various applications in software development and maintenance, few studies have focused on understanding logging usage in the source code.
Some work has been done on characterizing log messages modifications made by developers and to help them enhance the content of log messages. However, to the best of our knowledge, no study has been conducted on characterizing where logging is used in the source code through determining structural correspondences. Several data mining and statistical source code analysis techniques have been used to extract API usage patterns, however, none of them enable us to determine the detailed structural correspondences between source code fragments. On the other hand, using higher-order anti-unification modulo theories and an agglomerative hierarchical clustering algorithm allow us to construct structural generalizations that describe the similarities and differences between logged Java classes and classifying logged Java classes into groups based on the structural correspondences, respectively.


\addtocontents{toc}{\protect\addvspace{10pt}}

\chapter{Conclusion}  \label{conc}
Determining the detailed structural similarities and differences between a set of source code fragments is a complex task, and it can be applied to solve several source code analysis problems. As a specific application, the focus of this study is on detecting usage patterns of logging calls in source code via structural generalization and clustering.

Logging is a pervasive practice and has various applications in software development and maintenance. However, it is a challenging task for developers to understand how to use logging calls in source code. We have presented an approach to characterize where logging calls happen in source code by means of structural generalization and clustering. 
I have developed a prototype tool implementing my proposed approach that proceeds in three steps. First, it extracts the ASTs of logged Java methods using the Eclipse JDT framework and determines potential structural correspondences between the AST nodes via the Jigsaw framework. Second, it constructs an anti-unifier form ASTs of two given LJMs with a focus on logging calls through the implementation of higher-order anti-unification modulo theories. Due to the problem of undecidability of HOAUMT, it employs an approximation technique which greedily determines the best correspondence for each node with the highest similarity. It applies several constraints prior to determining the best correspondences to prevent the anti-unification of logging calls with anything else. It also develops a measure of structural similarity that determines how similar is the usage of logging calls in these Java methods.  Third, it classifies a set of logged Java methods via a hierarchical clustering algorithm suited to our application.

% uses several constraints to remove the correspondences that are not suited to our application
%it is a challenging task for developers to decide where, when, and what to log and their decisions can mainly affect the quality of logging.

We have conducted three experiments to evaluate the effectiveness of our approach in constructing structural generalizations and classifying Java methods that use logging calls.
I found that my tool was successful in determining correct correspondences for my application in \% of test cases. It was also successful in classifying logged Java methods into separate clusters using a similarity measure that indicates how similar logged Java methods are with a focus on the usage of logging calls. Furthermore, An study was conducted to describe the commonalities and differences between the usage of logging calls in the source code of three software systems via our tool that describes logging usage patterns on a per system and between systems method-granularity basis. Our characterization study shows …


In summary, our study makes the following contributions:
\begin{itemize} [leftmargin=.4in]
\item An approach to construct a structural generalization from AST structures of two logged Java methods with special attention to logging usage by determining structural correspondences between the ASTs via the Jigsaw framework and an approximated higher-order anti-unification modulo theories algorithm. 
\item An approach to develop a similarity measure that determines how similar two logged Java methods are with a focus on logging usage. 
\item An approach for classifying a set of ASTs via a hierarchical clustering algorithm. 
\item An approach for detecting usage patterns of logging calls in source code via structural generalization and clustering.
\end{itemize}


\section{Future Work}  \label{fw}
Future extensions could be applied to resolve the remaining problems of this study:
\begin{itemize} [leftmargin=.4in]
\item Data flow analysis techniques: to resolve the problem of inaccurate node ordering. 
\item Further analysis: to detect and resolve all the conflicts happen in deciding the best correspondences. 
\end{itemize}

Characterizing logging usage could be a huge step towards improving logging practices through the provision of some guidelines that might help developers in making decisions about where to log. We believe that further studies could be conducted to investigate the feasibility of predicting the location of logging calls based on the detected usage patterns. Future work can also be done to develop recommendation tool supports that not only save developers’ time and effort for making decisions about where to log, but also improve the quality of logging practices. 

To further validate the findings of our characterization study, the source code analysis can be performed on more software systems. In addition, a survey can be conducted to ask developers on the factors they consider to decide on where to log. It might also be helpful to recognize important structural and semantic information that should be taken into account for characterizing logging usage.

%Future studies can reduce our doubts about the accuracy of our characterization study by source code analysis of more software systems or by conducting a survey to ask developers about how they decide on where to log. 
% We believe that further studies to extract contextual characteristics of logged code snippets
%??



\bibliographystyle{plainnat}
\global\def\newblock{\hskip .11em plus .33em minus -.07em} %Avoids a latex complaint

% When you are working on the document, comment out Rob's line and uncomment out Narges's line, but don't delete either...

% NZ:
%\bibliography{../../../../bibtex/full,../../../../bibtex/references,../../../../bibtex/lsmr,mybiblio}
% RW:
%\bibliography{../../../../workspace/bibtex/full,../../../../workspace/bibtex/references,../../../../workspace/bibtex/lsmr,mybiblio}
\bibliography{../../../../workspace/bibtex/full,../../../../workspace/bibtex/references,../../../../workspace/bibtex/lsmr,mybiblio}



\appendix
%\fancyhead[RO,LE]{\thepage}
%\fancyfoot{}

\chapter{Logged Methods in Each Cluster of the Clustering Evaluation}\label{append}

In the clustering evaluation of Section~\ref{clustering-evaluation}, the following clusters were created, in order of descending size.

\begin{center}
\captionof{figure}{LMs in the cluster $C_1$}
\begin{tabular}{ll}\toprule
\multicolumn{1}{c}{Class}&\multicolumn{1}{c}{Method}\\\midrule
\lstinline/PluginJAR/&\raisebox{0pt}{\lstinline/generateCache()/}\\
\lstinline/EditBus/&\raisebox{0pt}{\lstinline/send(EBMessage)/}\\
\lstinline/Wrapper/&\raisebox{0pt}{\lstinline/actionPerformed(ActionEvent)/}\\
\lstinline/JARClassLoader/&\raisebox{0pt}{\lstinline/loadClass(String,boolean)/}\\
\bottomrule
\end{tabular}
\end{center}

\begin{center}
\captionof{figure}{LMs in the cluster $C_2$}
\begin{tabular}{ll}\toprule
\multicolumn{1}{c}{Class}&\multicolumn{1}{c}{Method}\\\midrule
\lstinline/MiscUtilities/&\raisebox{0pt}{\lstinline/isSupportedEncoding(EBMessage)/}\\
\lstinline/RootsEntry/&\raisebox{0pt}{\lstinline/rootEntry(File)/}\\
\lstinline/ServiceManager/&\raisebox{0pt}{\lstinline/loadServices(PluginJAR,URL,PluginCacheEntry)/}\\
\bottomrule
\end{tabular}
\end{center}

\begin{center}
\captionof{figure}{LMs in the cluster $C_3$}
\begin{tabular}{ll}\toprule
\multicolumn{1}{c}{Class}&\multicolumn{1}{c}{Method}\\\midrule
\lstinline/EditBus/&\raisebox{0pt}{\lstinline/send(EBMessage)* /}\\
\lstinline/EBPlugin/&\raisebox{0pt}{\lstinline/handleMessage(EBMessage )/}\\
\lstinline/RecentHandler/&\raisebox{0pt}{\lstinline/doctypeDecl(String,String,String)/}\\
\bottomrule
\end{tabular}
\end{center}

%\fancyhead[RO,LE]{\thepage}
%\fancyfoot{}

\chapter{Categorization of and Logged Methods in Each Cluster of the Characterization Study}\label{appendB}

In the characterization study of Chapter~\ref{eval}, four systems were studied: Tomcat (Section~\ref{tomcat}), Hibernate (Section~\ref{hibernate}), Camel (Section~\ref{camel}), and Solr (Section~\ref{solr}). In each section below, the clusters that were created for the given system are presented relative to their categorization, as described in Section~\ref{analysis}, in order of descending size.

\section{Tomcat}\label{tomcat}

\subsection{Cluster classified as Exception Catch-Block Logging}

\begin{center}
\captionof{figure}{LMs in the cluster $\id{T}_{\id{CB},1}$}
\begin{tabular}{ll}\toprule
\multicolumn{1}{c}{Class}&\multicolumn{1}{c}{Method}\\\midrule
\lstinline/Registry/&\raisebox{0pt}{\lstinline/ loadDescriptors(String,ClassLoader)/}\\ 
\lstinline/Registry/&\raisebox{0pt}{\lstinline/ loadDescriptors(String,ClassLoader)/}\\ 
\lstinline/Digester/&\raisebox{0pt}{\lstinline/ SAXExceptioncreateSAXException(String,Exception)/}\\ 
\lstinline/BasicDataSource/&\raisebox{0pt}{\lstinline/ ObjectNamepreRegister(MBeanServer,ObjectName)/}\\ 
\lstinline/BasicDataSource/&\raisebox{0pt}{\lstinline/ ObjectNamepreRegister(MBeanServer,ObjectName)/}\\ 
\lstinline/CoyoteAdapter/&\raisebox{0pt}{\lstinline/ convertURI(MessageBytes)/}\\ 
\lstinline/HostConfig/&\raisebox{0pt}{\lstinline/ checkUndeploy()/}\\ 
\lstinline/IntrospectionUtils/&\raisebox{0pt}{\lstinline/ ObjectgetProperty(Object,String)/}\\ 
\lstinline/IntrospectionUtils/&\raisebox{0pt}{\lstinline/ ObjectgetProperty(Object,String)/}\\ 
\lstinline/IntrospectionUtils/&\raisebox{0pt}{\lstinline/ ObjectgetProperty(Object,String)/}\\ 
\lstinline/IntrospectionUtils/&\raisebox{0pt}{\lstinline/ ObjectgetProperty(Object,String)/}\\ 
\lstinline/CGIRunner/&\raisebox{0pt}{\lstinline/ sendToLog(BufferedReaderrdr)/}\\ 
\lstinline/OpenSSLContext/&\raisebox{0pt}{\lstinline/ init(KeyManager[])/}\\ 
\lstinline/Poller/&\raisebox{0pt}{\lstinline/ timeout(int,boolean)/}\\ 
\lstinline/Poller/&\raisebox{0pt}{\lstinline/ timeout(int,boolean)/}\\ 
\lstinline/HostConfig/&\raisebox{0pt}{\lstinline/ checkUndeploy()/}\\ 
\lstinline/FarmWarDeployer/&\raisebox{0pt}{\lstinline/ copy(File,File)/}\\ 
\lstinline/RewriteValveextendsValveBase/&\raisebox{0pt}{\lstinline/ parse(BufferedReader)/}\\ 
\lstinline/SocketProcessorextendsSocketProcessorBase<Nio2Channel>/&\raisebox{0pt}{\lstinline/ doRun()/}\\ 
\lstinline/SocketProcessorextendsSocketProcessorBase<Nio2Channel>/&\raisebox{0pt}{\lstinline/ doRun()/}\\ 
\lstinline/WebappClassLoaderBaseextendsURLClassLoader/&\raisebox{0pt}{\lstinline/ findClassInternal(String)/}\\ 
\lstinline/WebappClassLoaderBaseextendsURLClassLoader/&\raisebox{0pt}{\lstinline/ findClassInternal(String)/}\\ 
\lstinline/SocketProcessor<NioChannel>/&\raisebox{0pt}{\lstinline/ doRun()/}\\ 
\lstinline/SocketProcessor<NioChannel>/&\raisebox{0pt}{\lstinline/ doRun()/}\\ 
\lstinline/JAASRealm/&\raisebox{0pt}{\lstinline/ authenticate(Stringu)/}\\ 
\lstinline/NioReceiver/&\raisebox{0pt}{\lstinline/ run()/}\\ 
\lstinline/NioReceiver/&\raisebox{0pt}{\lstinline/ run()/}\\ 
\lstinline/BlockPoller/&\raisebox{0pt}{\lstinline/ run()/}\\ 
\lstinline/Poller/&\raisebox{0pt}{\lstinline/ timeout(int,boolean)/}\\ 
\lstinline/Poller/&\raisebox{0pt}{\lstinline/ timeout(int,boolean)/}\\ 
\lstinline/AbstractReplicatedMap/&\raisebox{0pt}{\lstinline/ Vput()/}\\ 
\lstinline/NamingContextListener/&\raisebox{0pt}{\lstinline/ removeResourceLink(String)/}\\ 
\lstinline/NamingContextListener/&\raisebox{0pt}{\lstinline/ removeResourceLink(String)/}\\ 
\lstinline/IntrospectionUtils/&\raisebox{0pt}{\lstinline/ ObjectgetProperty(Object,String)/}\\ 
\lstinline/IntrospectionUtils/&\raisebox{0pt}{\lstinline/ ObjectgetProperty(Object,String)/}\\ 
\lstinline/IntrospectionUtils/&\raisebox{0pt}{\lstinline/ ObjectgetProperty(Object,String)/}\\ 
\lstinline/IntrospectionUtils/&\raisebox{0pt}{\lstinline/ ObjectgetProperty(Object,String)/}\\ 
\lstinline/SecurityClassLoad/&\raisebox{0pt}{\lstinline/ securityClassLoad(ClassLoader)/}\\ 
\lstinline/WebappClassLoader/&\raisebox{0pt}{\lstinline/publishEntryInfo(Object)/}\\ 
\lstinline/LazyReplicatedMap/&\raisebox{0pt}{\lstinline/publishEntryInfo(Object)/}\\ 
\lstinline/Poller/&\raisebox{0pt}{\lstinline/ timeout(int,boolean)/}\\ 
\lstinline/Poller/&\raisebox{0pt}{\lstinline/ timeout(int,boolean)/}\\ 
\lstinline/SocketProcessor<Nio2Channel>/&\raisebox{0pt}{\lstinline/ doRun()/}\\ 
\lstinline/Acceptor/&\raisebox{0pt}{\lstinline/ run()/}\\ 
\lstinline/Acceptor/&\raisebox{0pt}{\lstinline/ run()/}\\ 
\lstinline/WebappClassLoaderBase/&\raisebox{0pt}{\lstinline/ findClassInternal(String)/}\\ 
\lstinline/SimpleTcpCluster/&\raisebox{0pt}{\lstinline/ memberDisappeared(Member)/}\\ 
\lstinline/ContextConfig/&\raisebox{0pt}{\lstinline/ processAnnotationsFile(File)/}\\ 
\lstinline/Acceptor/&\raisebox{0pt}{\lstinline/ run()/}\\ 
\lstinline/Acceptor/&\raisebox{0pt}{\lstinline/ run()/}\\ 
\lstinline/Sendfile/&\raisebox{0pt}{\lstinline/ SendfileStateadd(SendfileData)/}\\ 
\lstinline/Sendfile/&\raisebox{0pt}{\lstinline/ SendfileStateadd(SendfileData)/}\\ 
\lstinline/CoyoteAdapter/&\raisebox{0pt}{\lstinline/ convertURI(MessageBytes)/}\\ 
\lstinline/ContainerBackgroundProcessor/&\raisebox{0pt}{\lstinline/ processChildren(Container)/}\\ 
\lstinline/ContainerBackgroundProcessor/&\raisebox{0pt}{\lstinline/ processChildren(Container)/}\\ 
\lstinline/FarmWarDeployer/&\raisebox{0pt}{\lstinline/ copy(File,File)/}\\ 
\lstinline/AbstractReplicatedMap</&\raisebox{0pt}{\lstinline/ Vput()/}\\ 
\lstinline/AbstractReplicatedMap</&\raisebox{0pt}{\lstinline/ Vput()/}\\ 
\lstinline/AbstractReplicatedMap</&\raisebox{0pt}{\lstinline/ Vput()/}\\ 
\lstinline/TcpFailureDetector/&\raisebox{0pt}{\lstinline/ memberAlive(Member)/}\\ 
\lstinline/FarmWarDeployer/&\raisebox{0pt}{\lstinline/ copy(File,File)/}\\ 
\lstinline/PersistentManagerBase/&\raisebox{0pt}{\lstinline/ startInternal()/}\\ 
\lstinline/Poller/&\raisebox{0pt}{\lstinline/ timeout(int,boolean)/}\\ 
\lstinline/Poller/&\raisebox{0pt}{\lstinline/ timeout(int,boolean)/}\\ 
\lstinline/CGIEnvironment/&\raisebox{0pt}{\lstinline/ expandCGIScript()/}\\ 
\lstinline/FarmWarDeployer/&\raisebox{0pt}{\lstinline/ booleancopy(Filefrom,Fileto)/}\\ 
\lstinline/StoreConfig/&\raisebox{0pt}{\lstinline/ booleanstore(Context)/}\\ 
\lstinline/ContextConfig/&\raisebox{0pt}{\lstinline/ processAnnotationsFile(File)/}\\ 
\lstinline/JDBCRealm/&\raisebox{0pt}{\lstinline/ startInternal()/}\\ 
\lstinline/AprSocketWrapper/&\raisebox{0pt}{\lstinline/ populateLocalPort()/}\\ 
\lstinline/AprSocketWrapper/&\raisebox{0pt}{\lstinline/ populateLocalPort()/}\\ 
\lstinline/DeltaManager/&\raisebox{0pt}{\lstinline/ messageReceived(SessionMessage,Member)/}\\ 
\lstinline/ReplicatedCont/&\raisebox{0pt}{\lstinline/ startInternal()/}\\ 
\lstinline/Registry/&\raisebox{0pt}{\lstinline/ loadDescriptors(Strng,ClassLoader)/}\\ 
\lstinline/Response/&\raisebox{0pt}{\lstinline/ sendRedirect(int)/}\\ 
\lstinline/AbstractReplicatedMap</&\raisebox{0pt}{\lstinline/ Vput()/}\\ 
\lstinline/AbstractReplicatedMap</&\raisebox{0pt}{\lstinline/ Vput()/}\\ 
\lstinline/StoreConfig/&\raisebox{0pt}{\lstinline/ booleanstore(Context)/}\\ 
\lstinline/NioReplicationTask/&\raisebox{0pt}{\lstinline/ sendAck()/}\\ 
\lstinline/TwoPhaseCommitInterceptor/&\raisebox{0pt}{\lstinline/ heartbeat()/}\\ 
\lstinline/DeltaManager/&\raisebox{0pt}{\lstinline/ messageReceived(SessionMessage)/}\\ 
\lstinline/DeltaManager/&\raisebox{0pt}{\lstinline/ messageReceived(SessionMessage)/}\\ 
\lstinline/WebappClassLoaderBase/&\raisebox{0pt}{\lstinline/ findClassInternal(String)/}\\ 
\lstinline/NamingContext/&\raisebox{0pt}{\lstinline/ Objectlookup(Name,boolean)/}\\ 
\lstinline/ThreadLocalLeakPreventionListener,ContainerListener/&\raisebox{0pt}{\lstinline/ containerEvent(ContainerEvent)/}\\ 
\lstinline/RewriteValve/&\raisebox{0pt}{\lstinline/ parse(BufferedReaderreader)/}\\ 
\lstinline/DeltaManager/&\raisebox{0pt}{\lstinline/ messageReceived(SessionMessage)/}\\ 
\lstinline/StandardHost/&\raisebox{0pt}{\lstinline/ startInternal()/}\\ 
\lstinline/WebappLoader/&\raisebox{0pt}{\lstinline/ booleanbuildClassPath(StringBuilder,ClassLoader)/}\\ 
\lstinline/GlobalResourcesLifecycleListener/&\raisebox{0pt}{\lstinline/ createMBeans(String)/}\\ 
\lstinline/GlobalResourcesLifecycleListener/&\raisebox{0pt}{\lstinline/ createMBeans(String)/}\\ 
\lstinline/ExtendedAccessLogValve/&\raisebox{0pt}{\lstinline/ AccessLogElementgetServletRequestElement(String)/}\\ 
\lstinline/NamingContextListener,ContainerListener,PropertyChangeListener/&\raisebox{0pt}{\lstinline/ removeResourceLink(String)/}\\ 
\lstinline/JAASRealm/&\raisebox{0pt}{\lstinline/ Principalauthenticate(String,CallbackHandler)/}\\ 
\lstinline/JAASRealm/&\raisebox{0pt}{\lstinline/ Principalauthenticate(String,CallbackHandler)/}\\ 
\lstinline/NioReplicationTask/&\raisebox{0pt}{\lstinline/ sendAck(Selection)/}\\ 
\lstinline/NioSender/&\raisebox{0pt}{\lstinline/ disconnect()/}\\ 
\lstinline/JNDIRealm/&\raisebox{0pt}{\lstinline/ startInternal()/}\\ 
\lstinline/JNDIRealm/&\raisebox{0pt}{\lstinline/ startInternal()/}\\ 
\lstinline/AprLifecycleListener/&\raisebox{0pt}{\lstinline/ initializeSSL())/}\\ 
\lstinline/ReplicatedMap/&\raisebox{0pt}{\lstinline/ memberDisappeared(Member)/}\\ 
\lstinline/AbstractReplicatedMap</&\raisebox{0pt}{\lstinline/ Vput()/}\\ 
\lstinline/ReplicatedMap/&\raisebox{0pt}{\lstinline/ memberDisappeared(Member)/}\\ 
\lstinline/MultiCastSender/&\raisebox{0pt}{\lstinline/ intsend(String)/}\\ 
\lstinline/TcpSender/&\raisebox{0pt}{\lstinline/ intsend(String)/}\\ 
\lstinline/NonBlockingCoordinator/&\raisebox{0pt}{\lstinline/ fireInterceptorEvent(InterceptorEvent)/}\\ 
\lstinline/NioReplicationTask/&\raisebox{0pt}{\lstinline/ sendAck(SelectionKey)/}\\ 
\lstinline/BaseModelMBean/&\raisebox{0pt}{\lstinline/ setAttribute(Attribute)/}\\ 
\lstinline/BaseModelMBean/&\raisebox{0pt}{\lstinline/ setAttribute(Attribute)/}\\ 
\lstinline/BaseModelMBean/&\raisebox{0pt}{\lstinline/ setAttribute(Attribute)/}\\ 
\lstinline/FormAuthenticator/&\raisebox{0pt}{\lstinline/ forwardToErrorPage(Request)/}\\ 
\lstinline/FormAuthenticator/&\raisebox{0pt}{\lstinline/ forwardToErrorPage(Request)/}\\ 
\lstinline/ThreadLocalLeakPreventionListener,ContainerListener/&\raisebox{0pt}{\lstinline/ containerEvent(ContainerEvent)/}\\ 
\lstinline/DeltaManager/&\raisebox{0pt}{\lstinline/ messageReceived(SessionMessage,Member)/}\\ 
\lstinline/WebappClassLoaderBase/&\raisebox{0pt}{\lstinline/ findClassInternal(String)/}\\ 
\lstinline/Http11Processorlog.warn(sm.getString("byteBufferUtils.cleaner"),e)/&\raisebox{0pt}{\lstinline/ sslReHandShake()/}\\ 
\lstinline/Http11Processor/&\raisebox{0pt}{\lstinline/ sslReHandShake()/}\\ 
\lstinline/Tool/&\raisebox{0pt}{\lstinline/ usage()/}\\ 
\lstinline/BlockPoller/&\raisebox{0pt}{\lstinline/ run()/}\\ 
\lstinline/ContextConfig/&\raisebox{0pt}{\lstinline/ processAnnotationsFile(File,boolean)/}\\ 
\lstinline/ContextConfig/&\raisebox{0pt}{\lstinline/ processAnnotationsFile(File,boolean)/}\\ 
\lstinline/ContextConfig/&\raisebox{0pt}{\lstinline/ processAnnotationsFile((File,boolean)/}\\ 
\lstinline/WebXmlParser/&\raisebox{0pt}{\lstinline/ booleanparseWebXml(InputSource)/}\\ 
\lstinline/WebXmlParser/&\raisebox{0pt}{\lstinline/ booleanparseWebXml(InputSource)/}\\ 
\lstinline/WebXmlParser/&\raisebox{0pt}{\lstinline/ booleanparseWebXml(InputSource)/}\\ 
\lstinline/SSLUtilBase/&\raisebox{0pt}{\lstinline/ KeyStoregetStore(String)/}\\ 
\lstinline/SSLUtilBase/&\raisebox{0pt}{\lstinline/ KeyStoregetStore(String)/}\\ 
\lstinline/JspReader/&\raisebox{0pt}{\lstinline/ JspReader(JspCompilationContext)/}\\ 
\lstinline/BioReplicationTask/&\raisebox{0pt}{\lstinline/ sendAck(byte[])/}\\ 
\lstinline/JspCompilationContext/&\raisebox{0pt}{\lstinline/ CompilercreateCompiler(String)/}\\ 
\lstinline/JspCompilationContext/&\raisebox{0pt}{\lstinline/ CompilercreateCompiler(String)/}\\ 
\lstinline/PersistentManagerBase/&\raisebox{0pt}{\lstinline/ startInternal())/}\\ 
\lstinline/PasswdUserDatabase/&\raisebox{0pt}{\lstinline/ init()/}\\ 
\lstinline/AuthenticatorBase/&\raisebox{0pt}{\lstinline/ JaspicStategetJaspicState(AuthConfigProvider)/}\\ 
\lstinline/LifecycleMBeanBase/&\raisebox{0pt}{\lstinline/ unregister(ObjectNameon)/}\\ 
\lstinline/LifecycleMBeanBase/&\raisebox{0pt}{\lstinline/ unregister(ObjectNameon)/}\\ 
\lstinline/HostConfigListener/&\raisebox{0pt}{\lstinline/ checkUndeploy()/}\\ 
\lstinline/NamingContextListener/&\raisebox{0pt}{\lstinline/ removeResourceLink(String)/}\\ 
\lstinline/@SuppressWarnings("deprecation")StandardWrapper/&\raisebox{0pt}{\lstinline/ unload()/}\\ 
\lstinline/JreMemoryLeakPreventionListenerListener/&\raisebox{0pt}{\lstinline/ lifecycleEvent(LifecycleEvent)/}\\ 
\lstinline/JreMemoryLeakPreventionListenerListener/&\raisebox{0pt}{\lstinline/ lifecycleEvent(LifecycleEvent)/}\\ 
\lstinline/SslRmiServerBindSocketFactory/&\raisebox{0pt}{\lstinline/ SslRmiServerBindSocketFactory(String[])/}\\ 
\lstinline/WebappClassLoaderBase/&\raisebox{0pt}{\lstinline/ findClassInternal(String)/}\\ 
\lstinline/WebappClassLoaderBase/&\raisebox{0pt}{\lstinline/ findClassInternal(String)/}\\ 
\lstinline/PersistentManagerBase/&\raisebox{0pt}{\lstinline/ startInternal())/}\\ 
\lstinline/PersistentManagerBase/&\raisebox{0pt}{\lstinline/ startInternal())/}\\ 
\lstinline/PersistentManagerBase/&\raisebox{0pt}{\lstinline/ startInternal())/}\\ 
\lstinline/PersistentManagerBase/&\raisebox{0pt}{\lstinline/ startInternal())/}\\ 
\lstinline/StandardServer/&\raisebox{0pt}{\lstinline/ storeContext(Context)}\ 
\lstinline/StandardServer/&\raisebox{0pt}{\lstinline/ storeContext(Context)}\ 
\lstinline/StandardServer/&\raisebox{0pt}{\lstinline/ storeContext(Context)}\ 
\lstinline/MBeanDumper/&\raisebox{0pt}{\lstinline/ StringdumpBeans(MBeanServer))/}\\ 
\lstinline/HostConfigListener/&\raisebox{0pt}{\lstinline/ checkUndeploy()/}\\ 
\lstinline/SessionIdGeneratorBase/&\raisebox{0pt}{\lstinline/ SecureRandomcreateSecureRandom()/}\\ 
\lstinline/SessionIdGeneratorBase/&\raisebox{0pt}{\lstinline/ SecureRandomcreateSecureRandom()/}\\ 
\lstinline/JSSESupport/&\raisebox{0pt}{\lstinline/ getPeerCertificateChain()/}\\ 
\lstinline/AbstractReplicatedMap/&\raisebox{0pt}{\lstinline/ Vput(/}\\ 
\lstinline/SSLValv/&\raisebox{0pt}{\lstinline/ invoke(Requeste)/}\\ 
\lstinline/SSLValv/&\raisebox{0pt}{\lstinline/ invoke(Request)/}\\ 
\lstinline/PojoEndpointBase/&\raisebox{0pt}{\lstinline/ onError(Session)/}\\ 
\lstinline/StoreConfigLifecycleListenerListener/&\raisebox{0pt}{\lstinline/ createMBean(Server)/}\\ 
\lstinline/WebappLoader/&\raisebox{0pt}{\lstinline/ booleanbuildClassPath(StringBuilder)/}\\ 
\lstinline/CompilationUnit/&\raisebox{0pt}{\lstinline/ cgetContents()/}\\ 
\lstinline/JspServlet/&\raisebox{0pt}{\lstinline/ handleMissingResource(HttpServletRequest)/}\\ 
\lstinline/MemoryUserDatabase/&\raisebox{0pt}{\lstinline/ save())/}\\ 
\lstinline/CatalinaShutdownHook/&\raisebox{0pt}{\lstinline/ run()/}\\ 
\lstinline/CatalinaShutdownHook/&\raisebox{0pt}{\lstinline/ run()/}\\ 
\lstinline/CatalinaShutdownHook/&\raisebox{0pt}{\lstinline/ run()/}\\ 
\lstinline/DeltaManager/&\raisebox{0pt}{\lstinline/ messageReceived(SessionMessage,Member)/}\\ 
\lstinline/SimpleTcpClusterr/&\raisebox{0pt}{\lstinline/ memberDisappeared(Member)/}\\ 
\lstinline/RecoveryThread/&\raisebox{0pt}{\lstinline/ run()/}\\ 
\lstinline/NioReceiver/&\raisebox{0pt}{\lstinline/ run()/}\\ 
\lstinline/GzipInterceptor/&\raisebox{0pt}{\lstinline/ messageReceived(ChannelMessage)/}\\ 
\lstinline/GzipInterceptor/&\raisebox{0pt}{\lstinline/ messageReceived(ChannelMessage)/}\\ 
\lstinline/BackupManager/&\raisebox{0pt}{\lstinline/ startInternal())/}\\ 
\lstinline/NioReceiver/&\raisebox{0pt}{\lstinline/ run()/}\\ 
\lstinline/BioReceiver/&\raisebox{0pt}{\lstinline/ listen())/}\\ 
\lstinline/Http11Processor/&\raisebox{0pt}{\lstinline/ sslReHandShake()/}\\ 
\lstinline/Http11Processor/&\raisebox{0pt}{\lstinline/ sslReHandShake()/}\\ 
\lstinline/@SuppressWarnings("deprecation")StandardWrapper/&\raisebox{0pt}{\lstinline/ unload()/}\\ 
\lstinline/ContextConfigListener/&\raisebox{0pt}{\lstinline/ processAnnotationsFile(File)/}\\ 
\lstinline/@SuppressWarnings("deprecation")JspServletWrapper/&\raisebox{0pt}{\lstinline/ destroy()/}\\ 
\lstinline/TagHandlerPool/&\raisebox{0pt}{\lstinline/ doRelease(Taghandler)/}\\ 
\lstinline/StandardContext/&\raisebox{0pt}{\lstinline/ checkUnusualURLPattern(String)/}\\ 
\lstinline/StandardContext/&\raisebox{0pt}{\lstinline/ checkUnusualURLPattern(String)/}\\ 
\lstinline/StandardServer/&\raisebox{0pt}{\lstinline/ storeContext(Context)}\ 
\lstinline/StandardServer/&\raisebox{0pt}{\lstinline/ storeContext(Context)}\ 
\lstinline/ContextConfigListener/&\raisebox{0pt}{\lstinline/ processAnnotationsFile(File)/}\\ 
\lstinline/ContextConfigListener/&\raisebox{0pt}{\lstinline/ processAnnotationsFile(File)/}\\ 
\lstinline/ContextConfigListener/&\raisebox{0pt}{\lstinline/ processAnnotationsFile(File)/}\\ 
\lstinline/ContextConfigListener/&\raisebox{0pt}{\lstinline/ processAnnotationsFile(File)/}\\ 
\lstinline/ContextConfigListener/&\raisebox{0pt}{\lstinline/ processAnnotationsFile(File)/}\\ 
\lstinline/ContextConfigListener/&\raisebox{0pt}{\lstinline/ processAnnotationsFile(File)/}\\ 
\lstinline/WebappLoader/&\raisebox{0pt}{\lstinline/ booleanbuildClassPath(StringBuilder)/}\\ 
\lstinline/BioReplicationTask/&\raisebox{0pt}{\lstinline/ sendAck(byte[])/}\\ 
\lstinline/AuthenticatorBase/&\raisebox{0pt}{\lstinline/ JaspicStategetJaspicState(AuthConfigProvider)/}\\ 
\lstinline/RecoveryThread/&\raisebox{0pt}{\lstinline/ run()/}\\ 
\lstinline/JreMemoryLeakPreventionListenerListener/&\raisebox{0pt}{\lstinline/ lifecycleEvent(LifecycleEvent)/}\\ 
\lstinline/JreMemoryLeakPreventionListenerListener/&\raisebox{0pt}{\lstinline/ lifecycleEvent(LifecycleEvent)/}\\ 
\lstinline/ContextConfigListener/&\raisebox{0pt}{\lstinline/ processAnnotationsFile(File)/}\\ 
\lstinline/ContextConfigListener/&\raisebox{0pt}{\lstinline/ processAnnotationsFile(File)/}\\ 
\lstinline/DataSourceRealm/&\raisebox{0pt}{\lstinline/ getRoles(ConnectiondbConnection,String)/}\\ 
\lstinline/FarmWarDeployer/&\raisebox{0pt}{\lstinline/ booleancopy(File,File)/}\\ 
\lstinline/JDBCRealm/&\raisebox{0pt}{\lstinline/ startInternal())/}\\ 
\lstinline/JAASRealm/&\raisebox{0pt}{\lstinline/ authenticate(String)/}\\ 
\lstinline/UserDatabaseRealm/&\raisebox{0pt}{\lstinline/ startInternal())/}\\ 
\lstinline/HeartbeatThread/&\raisebox{0pt}{\lstinline/ run()/}\\ 
\lstinline/DeltaManager/&\raisebox{0pt}{\lstinline/ messageReceived(SessionMessage,Member)/}\\ 
\lstinline/JreMemoryLeakPreventionListenerListener/&\raisebox{0pt}{\lstinline/ lifecycleEvent(LifecycleEvent)/}\\ 
\lstinline/NonBlockingCoordinator/&\raisebox{0pt}{\lstinline/ fireInterceptorEvent(InterceptorEvent)/}\\ 
\lstinline/NonBlockingCoordinator/&\raisebox{0pt}{\lstinline/ fireInterceptorEvent(InterceptorEvent)/}\\ 
\lstinline/MbeansDescriptorsIntrospectionSource/&\raisebox{0pt}{\lstinline/ createManagedBean(Registry)/}\\ 
\lstinline/Digester/&\raisebox{0pt}{\lstinline/ createSAXException(String,Exception)/}\\ 
\lstinline/Digester/&\raisebox{0pt}{\lstinline/ createSAXException(String,Exception)/}\\ 
\lstinline/Digester/&\raisebox{0pt}{\lstinline/ createSAXException(String,Exception)/}\\ 
\lstinline/Digester/&\raisebox{0pt}{\lstinline/ createSAXException(String,Exception)/}\\ 
\lstinline/Digester/&\raisebox{0pt}{\lstinline/ createSAXException(String,Exception)/}\\ 
\lstinline/Digester/&\raisebox{0pt}{\lstinline/ createSAXException(String,Exception)/}\\ 
\lstinline/JAASRealm/&\raisebox{0pt}{\lstinline/ authenticate(String,CallbackHandler)/}\\ 
\lstinline/RealmBase/&\raisebox{0pt}{\lstinline/ StringDigest(Stringc,String,String)/}\\ 
\lstinline/Digester/&\raisebox{0pt}{\lstinline/ createSAXException(String,Exception)/}\\ 
\lstinline/Digester/&\raisebox{0pt}{\lstinline/ createSAXException(String,Exception)/}\\ 
\lstinline/StoreConfigLifecycleListenerListener/&\raisebox{0pt}{\lstinline/ createMBean(Server)/}\\ 
\lstinline/MbeansDescriptorsDigesterSource/&\raisebox{0pt}{\lstinline/ execute())/}\\ 
\lstinline/JAASMemoryLoginModule/&\raisebox{0pt}{\lstinline/ load()/}\\ 
\lstinline/StandardContext/&\raisebox{0pt}{\lstinline/ checkUnusualURLPattern(String)/}\\ 
\lstinline/MBeanFactory/&\raisebox{0pt}{\lstinline/ removeContext(String))/}\\ 
\lstinline/HeartbeatListenerListener,ContainerListener/&\raisebox{0pt}{\lstinline/ lifecycleEvent(LifecycleEvent)/}\\ 
\lstinline/HeartbeatListenerListener,ContainerListener/&\raisebox{0pt}{\lstinline/ lifecycleEvent(LifecycleEvent)/}\\ 
\lstinline/BaseModelMBean/&\raisebox{0pt}{\lstinline/ setAttribute(Attribute))}\ 
\lstinline/MultiCastSender/&\raisebox{0pt}{\lstinline/ intsend(String))/}\\ 
\lstinline/PersistentManagerBase/&\raisebox{0pt}{\lstinline/ startInternal())/}\\ 
\lstinline/PersistentManagerBase/&\raisebox{0pt}{\lstinline/ startInternal())/}\\ 
\lstinline/PersistentManagerBase/&\raisebox{0pt}{\lstinline/ startInternal())/}\\ 
\lstinline/PersistentManagerBase/&\raisebox{0pt}{\lstinline/ startInternal())/}\\ 
\lstinline/PersistentManagerBase/&\raisebox{0pt}{\lstinline/ startInternal())/}\\ 
\lstinline/WebappClassLoaderBase/&\raisebox{0pt}{\lstinline/ findClassInternal(String)/}\\ 
\lstinline/CatalinaShutdownHook/&\raisebox{0pt}{\lstinline/ run()/}\\ 
\lstinline/AbstractReplicatedMap/&\raisebox{0pt}{\lstinline/ Vput()/}\\ 
\lstinline/AbstractReplicatedMap/&\raisebox{0pt}{\lstinline/ Vput()/}\\ 
\lstinline/AbstractReplicatedMap/&\raisebox{0pt}{\lstinline/ Vput()/}\\ 
\lstinline/PollerEvent/&\raisebox{0pt}{\lstinline/ run()/}\\ 
\lstinline/PollerEvent/&\raisebox{0pt}{\lstinline/ run()/}\\ 
\lstinline/DeltaRequest/&\raisebox{0pt}{\lstinline/ readExternal(j))/}\\ 
\lstinline/HeartbeatThread/&\raisebox{0pt}{\lstinline/ run()/}\\ 
\lstinline/AbstractReplicatedMap/&\raisebox{0pt}{\lstinline/ Vput()/}\\ 
\lstinline/NonBlockingCoordinator/&\raisebox{0pt}{\lstinline/ fireInterceptorEvent(InterceptorEvent)/}\\ 
\lstinline/NonBlockingCoordinator/&\raisebox{0pt}{\lstinline/ fireInterceptorEvent(InterceptorEvent)/}\\ 
\lstinline/DeltaRequest/&\raisebox{0pt}{\lstinline/ readExternal(java.io.ObjectInput))/}\\ 
\lstinline/DeltaRequest/&\raisebox{0pt}{\lstinline/ readExternal(java.io.ObjectInput))/}\\ 
\lstinline/McastService/&\raisebox{0pt}{\lstinline/ setDomain(byte[])/}\\ 
\lstinline/McastService/&\raisebox{0pt}{\lstinline/ setDomain(byte[])/}\\ 
\lstinline/SimpleTcpCluster/&\raisebox{0pt}{\lstinline/ memberDisappeared(Member)/}\\ 
\lstinline/PingThread/&\raisebox{0pt}{\lstinline/ run()/}\\ 
\lstinline/PingThread/&\raisebox{0pt}{\lstinline/ run()/}\\ 
\lstinline/AbstractReplicatedMap/&\raisebox{0pt}{\lstinline/ Vput()/}\\ 
\lstinline/McastService/&\raisebox{0pt}{\lstinline/ setDomain(byte[])/}\\ 
\lstinline/McastService/&\raisebox{0pt}{\lstinline/ setDomain(byte[])/}\\ 
\lstinline/ReplicationValve/&\raisebox{0pt}{\lstinline/ updateStats(long,long)/}\\ 
\lstinline/NioReceiver/&\raisebox{0pt}{\lstinline/ run()/}\\ 
\lstinline/HeartbeatThread/&\raisebox{0pt}{\lstinline/ run()/}\\ 
\lstinline/HeartbeatThread/&\raisebox{0pt}{\lstinline/ run()/}\\ 
\lstinline/RecoveryThread/&\raisebox{0pt}{\lstinline/ run()/}\\ 
\lstinline/RecoveryThread/&\raisebox{0pt}{\lstinline/ run()/}\\ 
\lstinline/RecoveryThread/&\raisebox{0pt}{\lstinline/ run()/}\\ 
\lstinline/RecoveryThread/&\raisebox{0pt}{\lstinline/ run()/}\\ 
\lstinline/RecoveryThread/&\raisebox{0pt}{\lstinline/ run()/}\\ 
\lstinline/StandardEngine/&\raisebox{0pt}{\lstinline/ logAccess(Request)/}\\ 
\lstinline/AbstractReplicatedMap/&\raisebox{0pt}{\lstinline/ Vput()/}\\ 
\lstinline/AbstractReplicatedMap/&\raisebox{0pt}{\lstinline/ Vput()/}\\ 
\lstinline/NioReceiver/&\raisebox{0pt}{\lstinline/ run()/}\\ 
\lstinline/BioReceiver/&\raisebox{0pt}{\lstinline/ listen())/}\\ 
\lstinline/NamingContextListener/&\raisebox{0pt}{\lstinline/ removeResourceLink(String)/}\\ 
\lstinline/NamingContextListener/&\raisebox{0pt}{\lstinline/ removeResourceLink(String)/}\\ 
\lstinline/NamingContextListener/&\raisebox{0pt}{\lstinline/ removeResourceLink(String)/}\\ 
\lstinline/NamingContextListener/&\raisebox{0pt}{\lstinline/ removeResourceLink(String)/}\\ 
\lstinline/ParallelNioSender/&\raisebox{0pt}{\lstinline/ booleankeepalive()/}\\ 
\lstinline/PojoEndpointBase/&\raisebox{0pt}{\lstinline/ onError(Session,Throwable)/}\\ 
\lstinline/PojoEndpointBase/&\raisebox{0pt}{\lstinline/ onError(Session,Throwable)/}\\ 
\lstinline/ObjectReader/&\raisebox{0pt}{\lstinline/ ObjectReader(Socketsocket)/}\\ 
\lstinline/SimpleTcpCluster/&\raisebox{0pt}{\lstinline/ memberDisappeared(Member)/}\\ 
\lstinline/SimpleTcpCluster/&\raisebox{0pt}{\lstinline/ memberDisappeared(Member)/}\\ 
\lstinline/FarmWarDeployer/&\raisebox{0pt}{\lstinline/ booleancopy(File,File)/}\\ 
\lstinline/ApplicationFilterConfig/&\raisebox{0pt}{\lstinline/ unregisterJMX()/}\\ 
\lstinline/LifecycleMBeanBase/&\raisebox{0pt}{\lstinline/ unregister(ObjectNameon)/}\\ 
\lstinline/LifecycleMBeanBase/&\raisebox{0pt}{\lstinline/ unregister(ObjectNameon)/}\\ 
\lstinline/StandardJarScanner/&\raisebox{0pt}{\lstinline/ scan(JarScanType)/}\\ 
\lstinline/StandardContext/&\raisebox{0pt}{\lstinline/ checkUnusualURLPattern(String)/}\\ 
\lstinline/WebappClassLoaderBase/&\raisebox{0pt}{\lstinline/ findClassInternal(String)/}\\ 
\lstinline/WebappClassLoaderBase/&\raisebox{0pt}{\lstinline/ findClassInternal(String)/}\\ 
\lstinline/HostConfigListener/&\raisebox{0pt}{\lstinline/ checkUndeploy()/}\\ 
\lstinline/HostConfigListener/&\raisebox{0pt}{\lstinline/ checkUndeploy()/}\\ 
\lstinline/HostConfigListener/&\raisebox{0pt}{\lstinline/ checkUndeploy()/}\\ 
\lstinline/ApplicationFilterConfig/&\raisebox{0pt}{\lstinline/ unregisterJMX()/}\\ 
\lstinline/ContextConfigListener/&\raisebox{0pt}{\lstinline/ processAnnotationsFile(File)/}\\ 
\lstinline/ContextConfigListener/&\raisebox{0pt}{\lstinline/ processAnnotationsFile(File)/}\\ 
\lstinline/ExpandWar/&\raisebox{0pt}{\lstinline/ booleandeleteDir(Filedir,booleanlogFailure)/}\\ 
\lstinline/StandardContext/&\raisebox{0pt}{\lstinline/ checkUnusualURLPattern(String)/}\\ 
\lstinline/NamingResourcesImpl/&\raisebox{0pt}{\lstinline/ destroyInternal())/}\\ 
\lstinline/AccessLogValve/&\raisebox{0pt}{\lstinline/ open()/}\\ 
\lstinline/ContextConfigListener/&\raisebox{0pt}{\lstinline/ processAnnotationsFile(File)/}\\ 
\lstinline/FarmWarDeployer/&\raisebox{0pt}{\lstinline/ booleancopy(Filefrom,Fileto)/}\\ 
\lstinline/DeltaSession/&\raisebox{0pt}{\lstinline/ doWriteObject(ObjectOutputstream)/}\\ 
\lstinline/SpnegoAuthenticator/&\raisebox{0pt}{\lstinline/ doAuthenticate(Requestrequest)/}\\ 
\lstinline/CombinedRealm/&\raisebox{0pt}{\lstinline/ PrincipalgetPrincipal(String)/}\\ 
\lstinline/PersistentManagerBase/&\raisebox{0pt}{\lstinline/ startInternal())/}\\ 
\lstinline/StandardContext/&\raisebox{0pt}{\lstinline/ checkUnusualURLPattern(String)/}\\ 
\lstinline/StandardJarScanner/&\raisebox{0pt}{\lstinline/ scan()/}\\ 
\lstinline/Nio2SocketWrapper<Nio2Channel>/&\raisebox{0pt}{\lstinline/ doClientAuth(SSLSupport)/}\\ 
\lstinline/Nio2SocketWrapper<Nio2Channel>/&\raisebox{0pt}{\lstinline/ doClientAuth(SSLSupport)/}\\ 
\lstinline/NioSocketWrapper<NioChannel>/&\raisebox{0pt}{\lstinline/ doClientAuth(SSLSupport)/}\\ 
\lstinline/NioSocketWrapper<NioChannel>/&\raisebox{0pt}{\lstinline/ doClientAuth(SSLSupport)/}\\ 
\lstinline/Nio2SocketWrapper<Nio2Channel>/&\raisebox{0pt}{\lstinline/ doClientAuth(SSLSupport)/}\\ 
\lstinline/Nio2SocketWrapper<Nio2Channel>/&\raisebox{0pt}{\lstinline/ doClientAuth(SSLSupport)/}\\ 
\lstinline/Nio2SocketWrapper<Nio2Channel>/&\raisebox{0pt}{\lstinline/ doClientAuth(SSLSupport)/}\\ 
\lstinline/Nio2SocketWrapper<Nio2Channel>/&\raisebox{0pt}{\lstinline/ doClientAuth(SSLSupport)/}\\ 
\lstinline/Nio2SocketWrapper<Nio2Channel>/&\raisebox{0pt}{\lstinline/ doClientAuth(SSLSupport)/}\\ 
\lstinline/Nio2SocketWrapper<Nio2Channel>/&\raisebox{0pt}{\lstinline/ doClientAuth(SSLSupport)/}\\ 
\lstinline/Nio2SocketWrapper<Nio2Channel>/&\raisebox{0pt}{\lstinline/ doClientAuth(SSLSupport)/}\\ 
\lstinline/Nio2SocketWrapper<Nio2Channel>/&\raisebox{0pt}{\lstinline/ doClientAuth(SSLSupport)/}\\ 
\lstinline/Nio2SocketWrapper<Nio2Channel>/&\raisebox{0pt}{\lstinline/ doClientAuth(SSLSupport)/}\\ 
\lstinline/Nio2SocketWrapper<Nio2Channel>/&\raisebox{0pt}{\lstinline/ doClientAuth(SSLSupport)/}\\ 
\lstinline/SingleSignOn/&\raisebox{0pt}{\lstinline/ expire(SingleSignOnSessionKeykey)/}\\ 
\lstinline/JNDIRealm/&\raisebox{0pt}{\lstinline/ startInternal())/}\\ 
\lstinline/AprSocketWrapper<Long>/&\raisebox{0pt}{\lstinline/ populateLocalPort()/}\\ 
\lstinline/AprSocketWrapper<Long>/&\raisebox{0pt}{\lstinline/ populateLocalPort()/}\\ 
\lstinline/AprSocketWrapper<Long>/&\raisebox{0pt}{\lstinline/ populateLocalPort()/}\\ 
\lstinline/AprSocketWrapper<Long>/&\raisebox{0pt}{\lstinline/ populateLocalPort()/}\\ 
\lstinline/AprSocketWrapper<Long>/&\raisebox{0pt}{\lstinline/ populateLocalPort()/}\\ 
\lstinline/AprSocketWrapper<Long>/&\raisebox{0pt}{\lstinline/ populateLocalPort()/}\\ 
\lstinline/AprSocketWrapper<Long>/&\raisebox{0pt}{\lstinline/ populateLocalPort()/}\\ 
\lstinline/AprSocketWrapper<Long>/&\raisebox{0pt}{\lstinline/ populateLocalPort()/}\\ 
\lstinline/AprSocketWrapper<Long>/&\raisebox{0pt}{\lstinline/ populateLocalPort()/}\\ 
\lstinline/AprSocketWrapper<Long>/&\raisebox{0pt}{\lstinline/ populateLocalPort()/}\\ 
\lstinline/AprSocketWrapper<Long>/&\raisebox{0pt}{\lstinline/ populateLocalPort()/}\\ 
\lstinline/AprSocketWrapper<Long>/&\raisebox{0pt}{\lstinline/ populateLocalPort()/}\\ 
\lstinline/WebappClassLoaderBase/&\raisebox{0pt}{\lstinline/ findClassInternal(String)/}\\ 
\lstinline/CatalinaShutdownHook/&\raisebox{0pt}{\lstinline/ run()/}\\ 
\lstinline/JNDIRealm/&\raisebox{0pt}{\lstinline/ startInternal())/}\\ 
\lstinline/UpgradeServletOutputStream/&\raisebox{0pt}{\lstinline/ onError(Throwablet)/}\\ 
\lstinline/UpgradeServletInputStream/&\raisebox{0pt}{\lstinline/ onError(Throwablet)/}\\ 
\lstinline/WebappClassLoaderBase/&\raisebox{0pt}{\lstinline/ findClassInternal(String)/}\\ 
\lstinline/WsSession/&\raisebox{0pt}{\lstinline/ onClose(CloseReasoncloseReason)/}\\ 
\lstinline/WsSession/&\raisebox{0pt}{\lstinline/ onClose(CloseReasoncloseReason)/}\\ 
\lstinline/DirResourceSet/&\raisebox{0pt}{\lstinline/ initInternal())/}\\ 
\lstinline/MemoryRealm/&\raisebox{0pt}{\lstinline/ DigestergetDigester()/}\\ 
\lstinline/MemoryUserDatabase/&\raisebox{0pt}{\lstinline/ save())/}\\ 
\lstinline/HostConfigListener/&\raisebox{0pt}{\lstinline/ checkUndeploy()/}\\ 
\lstinline/NamingResourcesImpl/&\raisebox{0pt}{\lstinline/ destroyInternal())/}\\ 
\lstinline/NamingResourcesImpl/&\raisebox{0pt}{\lstinline/ destroyInternal())/}\\ 
\lstinline/NamingResourcesImpl/&\raisebox{0pt}{\lstinline/ destroyInternal())/}\\ 
\lstinline/NamingResourcesImpl/&\raisebox{0pt}{\lstinline/ destroyInternal())/}\\ 
\lstinline/NamingResourcesImpl/&\raisebox{0pt}{\lstinline/ destroyInternal())/}\\ 
\lstinline/NamingResourcesImpl/&\raisebox{0pt}{\lstinline/ destroyInternal())/}\\ 
\lstinline/NamingResourcesImpl/&\raisebox{0pt}{\lstinline/ destroyInternal())/}\\ 
\lstinline/NamingResourcesImpl/&\raisebox{0pt}{\lstinline/ destroyInternal())/}\\ 
\lstinline/NamingResourcesImpl/&\raisebox{0pt}{\lstinline/ destroyInternal())/}\\ 
\lstinline/NamingResourcesImpl/&\raisebox{0pt}{\lstinline/ destroyInternal())/}\\ 
\lstinline/NamingResourcesImpl/&\raisebox{0pt}{\lstinline/ destroyInternal())/}\\ 
\lstinline/NamingResourcesImpl/&\raisebox{0pt}{\lstinline/ destroyInternal())/}\\ 
\lstinline/ContainerBackgroundProcessor/&\raisebox{0pt}{\lstinline/ processChildren(Container)/}\\ 
\lstinline/ContainerBackgroundProcessor/&\raisebox{0pt}{\lstinline/ processChildren(Container)/}\\ 
\lstinline/CGIRunner/&\raisebox{0pt}{\lstinline/ sendToLog(BufferedReaderrdr)/}\\ 
\lstinline/StandardJarScanner/&\raisebox{0pt}{\lstinline/ scan()/}\\ 
\lstinline/CGIEnvironment/&\raisebox{0pt}{\lstinline/ expandCGIScript()/}\\ 
\lstinline/WsHttpUpgradeHandler/&\raisebox{0pt}{\lstinline/ destroy()/}\\ 
\lstinline/DataSourceRealm/&\raisebox{0pt}{\lstinline/ getRoles(Connection,String)/}\\ 
\lstinline/HostConfigListener/&\raisebox{0pt}{\lstinline/ checkUndeploy()/}\\ 
\lstinline/HostConfigListener/&\raisebox{0pt}{\lstinline/ checkUndeploy()/}\\ 
\lstinline/HostConfigListener/&\raisebox{0pt}{\lstinline/ checkUndeploy()/}\\ 
\lstinline/JNDIRealm/&\raisebox{0pt}{\lstinline/ startInternal())/}\\ 
\lstinline/CGIRunner/&\raisebox{0pt}{\lstinline/ sendToLog(BufferedReaderrdr)/}\\ 
\lstinline/CGIRunner/&\raisebox{0pt}{\lstinline/ sendToLog(BufferedReaderrdr)/}\\ 
\lstinline/PojoEndpointBase/&\raisebox{0pt}{\lstinline/ onError(Session,Throwable)/}\\ 
\lstinline/WebappClassLoaderBase/&\raisebox{0pt}{\lstinline/ findClassInternal(String)/}\\ 
\lstinline/Poller/&\raisebox{0pt}{\lstinline/ timeout(intkeyCount,booleanhasEvents)/}\\ 
\lstinline/Poller/&\raisebox{0pt}{\lstinline/ timeout(intkeyCount,booleanhasEvents)/}\\ 
\lstinline/StandardService/&\raisebox{0pt}{\lstinline/ stopInternal())/}\\ 
\lstinline/StandardService/&\raisebox{0pt}{\lstinline/ stopInternal())/}\\ 
\lstinline/StandardService/&\raisebox{0pt}{\lstinline/ stopInternal())/}\\ 
\lstinline/ContainerBackgroundProcessor/&\raisebox{0pt}{\lstinline/ processChildren(Container)/}\\ 
\lstinline/ContainerBackgroundProcessor/&\raisebox{0pt}{\lstinline/ processChildren(Container)/}\\ 
\lstinline/StandardContext/&\raisebox{0pt}{\lstinline/ checkUnusualURLPattern(String)/}\\ 
\lstinline/StandardContext/&\raisebox{0pt}{\lstinline/ checkUnusualURLPattern(String)/}\\ 
\lstinline/ContainerBackgroundProcessor/&\raisebox{0pt}{\lstinline/ processChildren(Container)/}\\ 
\lstinline/StandardContext/&\raisebox{0pt}{\lstinline/ checkUnusualURLPattern(String)/}\\ 
\lstinline/AsyncChannelWrapperSecure/&\raisebox{0pt}{\lstinline/ close()/}\\ 
\lstinline/StandardService/&\raisebox{0pt}{\lstinline/ stopInternal())/}\\ 
\lstinline/CGIRunner/&\raisebox{0pt}{\lstinline/ sendToLog(BufferedReaderrdr)/}\\ 
\lstinline/Connector/&\raisebox{0pt}{\lstinline/ resume()/}\\ 
\lstinline/Connector/&\raisebox{0pt}{\lstinline/ resume()/}\\ 
\lstinline/ContextConfigListener/&\raisebox{0pt}{\lstinline/ processAnnotationsFile(File)/}\\ 
\lstinline/MessageDispatchInterceptor/&\raisebox{0pt}{\lstinline/ sendAsyncData()/}\\ 
\lstinline/MessageDispatchInterceptor/&\raisebox{0pt}{\lstinline/ sendAsyncData()/}\\ 
\lstinline/ExtensionValidator/&\raisebox{0pt}{\lstinline/ addFolderList(Stringproperty)/}\\ 
\lstinline/ExtensionValidator/&\raisebox{0pt}{\lstinline/ addFolderList(Stringproperty)/}\\ 
\lstinline/SslRmiServerBindSocketFactory/&\raisebox{0pt}{\lstinline/ serverBindSocketFactory(String)/}\\ 
\lstinline/StandardService/&\raisebox{0pt}{\lstinline/ stopInternal())/}\\ 
\lstinline/StandardContext/&\raisebox{0pt}{\lstinline/ checkUnusualURLPattern(String)/}\\ 
\lstinline/StandardContext/&\raisebox{0pt}{\lstinline/ checkUnusualURLPattern(String)/}\\ 
\lstinline/LifecycleBase/&\raisebox{0pt}{\lstinline/ handleSubClassException(Throwable))/}\\ 
\lstinline/ContextConfigListener/&\raisebox{0pt}{\lstinline/ processAnnotationsFile(File)/}\\ 
\lstinline/AccessLogValve/&\raisebox{0pt}{\lstinline/ open()/}\\ 
\lstinline/JreMemoryLeakPreventionListenerListener/&\raisebox{0pt}{\lstinline/ lifecycleEvent(LifecycleEvent)/}\\ 
\lstinline/StandardContext/&\raisebox{0pt}{\lstinline/ checkUnusualURLPattern(String)/}\\ 
\lstinline/StandardService/&\raisebox{0pt}{\lstinline/ stopInternal())/}\\ 
\lstinline/Request/&\raisebox{0pt}{\lstinline/ PrincipalgetUserPrincipal()/}\\ 
\lstinline/CombinedRealm/&\raisebox{0pt}{\lstinline/ PrincipalgetPrincipal(String)/}\\ 
\lstinline/LockOutRealm/&\raisebox{0pt}{\lstinline/ PrincipalfilterLockedAccounts(String)/}\\ 
\lstinline/RealmBase/&\raisebox{0pt}{\lstinline/ StringDigest(String,String,String)/}\\ 
\lstinline/HostConfigListener/&\raisebox{0pt}{\lstinline/ checkUndeploy()/}\\ 
\lstinline/JDBCRealm/&\raisebox{0pt}{\lstinline/ startInternal())/}\\ 
\lstinline/JNDIRealm/&\raisebox{0pt}{\lstinline/ startInternal())/}\\ 
\lstinline/JDBCRealm/&\raisebox{0pt}{\lstinline/ startInternal())/}\\ 
\lstinline/BioReceiver/&\raisebox{0pt}{\lstinline/ listen())/}\\ 
\lstinline/RmiServerBindSocketFactory/&\raisebox{0pt}{\lstinline/ RmiServerBindSocketFactory(String)/}\\ 
\lstinline/SslRmiServerBindSocketFactory/&\raisebox{0pt}{\lstinline/ SslRmiServerBindSocketFactory(String[])/}\\ 
\lstinline/RmiServerBindSocketFactory/&\raisebox{0pt}{\lstinline/ RmiServerBindSocketFactory(String)/}\\ 
\lstinline/SslRmiServerBindSocketFactory/&\raisebox{0pt}{\lstinline/ SslRmiServerBindSocketFactory(String[])/}\\ 
\lstinline/SessionIdGeneratorBase/&\raisebox{0pt}{\lstinline/ SecureRandomcreateSecureRandom()/}\\ 
\lstinline/NamingContextListener/&\raisebox{0pt}{\lstinline/ removeResourceLink(String)/}\\ 
\lstinline/NamingContextListener/&\raisebox{0pt}{\lstinline/ removeResourceLink(String)/}\\ 
\lstinline/NamingContextListener/&\raisebox{0pt}{\lstinline/ removeResourceLink(String)/}\\ 
\lstinline/NamingContextListener/&\raisebox{0pt}{\lstinline/ removeResourceLink(String)/}\\ 
\lstinline/NamingContextListener/&\raisebox{0pt}{\lstinline/ removeResourceLink(String)/}\\ 
\lstinline/NamingContextListener/&\raisebox{0pt}{\lstinline/ removeResourceLink(String)/}\\ 
\lstinline/NamingContextListener/&\raisebox{0pt}{\lstinline/ removeResourceLink(String)/}\\ 
\lstinline/NamingContextListener/&\raisebox{0pt}{\lstinline/ removeResourceLink(String)/}\\ 
\lstinline/SslRmiServerBindSocketFactory/&\raisebox{0pt}{\lstinline/ sslRmiServerBindSocketFactory()/}\\ 
\lstinline/StaticMembershipInterceptor/&\raisebox{0pt}{\lstinline/ sendShutdown(Member[]members)/}\\ 
\lstinline/StaticMembershipInterceptor/&\raisebox{0pt}{\lstinline/ sendShutdown(Member[]members)/}\\ 
\lstinline/PingThread/&\raisebox{0pt}{\lstinline/ run()/}\\ 
\lstinline/StandardContext/&\raisebox{0pt}{\lstinline/ checkUnusualURLPattern(String)/}\\ 
\lstinline/StandardContext/&\raisebox{0pt}{\lstinline/ checkUnusualURLPattern(String)/}\\ 
\lstinline/BasicDataSource/&\raisebox{0pt}{\lstinline/ ObjectNamepreRegister(MBeanServer)/}\\ 
\lstinline/CatalinaShutdownHook/&\raisebox{0pt}{\lstinline/ run()/}\\ 
\lstinline/StandardService/&\raisebox{0pt}{\lstinline/ stopInternal())/}\\ 
\lstinline/RewriteValv/&\raisebox{0pt}{\lstinline/ parse(BufferedReaderreader))/}\\ 
\lstinline/GlobalResourcesLifecycleListenerListener/&\raisebox{0pt}{\lstinline/ createMBeans(String))/}\\ 
\lstinline/CatalinaProperties/&\raisebox{0pt}{\lstinline/ loadProperties()/}\\ 
\lstinline/StandardContext/&\raisebox{0pt}{\lstinline/ checkUnusualURLPattern(String)/}\\ 
\lstinline/StandardContext/&\raisebox{0pt}{\lstinline/ checkUnusualURLPattern(String)/}\\ 
\lstinline/ContainerBackgroundProcessor/&\raisebox{0pt}{\lstinline/ processChildren(Container)/}\\ 
\lstinline/StandardPipeline/&\raisebox{0pt}{\lstinline/ removeValve(Valvevalve)/}\\ 
\lstinline/StandardPipeline/&\raisebox{0pt}{\lstinline/ removeValve(Valvevalve)/}\\ 
\lstinline/ContainerBackgroundProcessor/&\raisebox{0pt}{\lstinline/ processChildren(Container)/}\\ 
\lstinline/StandardPipeline/&\raisebox{0pt}{\lstinline/ removeValve(Valvevalve)/}\\ 
\lstinline/ContainerBackgroundProcessor/&\raisebox{0pt}{\lstinline/ processChildren(Container)/}\\ 
\lstinline/StandardPipeline/&\raisebox{0pt}{\lstinline/ removeValve(Valvevalve)/}\\ 
\lstinline/StandardContext/&\raisebox{0pt}{\lstinline/ checkUnusualURLPattern(String)/}\\ 
\lstinline/ContainerBackgroundProcessor/&\raisebox{0pt}{\lstinline/ processChildren(Container)/}\\ 
\lstinline/StandardPipeline/&\raisebox{0pt}{\lstinline/ removeValve(Valvevalve)/}\\ 
\lstinline/PersistentManagerBase/&\raisebox{0pt}{\lstinline/ startInternal())/}\\ 
\lstinline/ContainerBackgroundProcessor/&\raisebox{0pt}{\lstinline/ processChildren(Container)/}\\ 
\lstinline/ContainerBackgroundProcessor/&\raisebox{0pt}{\lstinline/ processChildren(Container)/}\\ 
\lstinline/DataSourceRealm/&\raisebox{0pt}{\lstinline/ getRoles(ConnectiondbConnection,String)/}\\ 
\lstinline/StandardContext/&\raisebox{0pt}{\lstinline/ checkUnusualURLPattern(String)/}\\ 
\lstinline/HeartbeatListenerListener,ContainerListener/&\raisebox{0pt}{\lstinline/ lifecycleEvent(LifecycleEvent)/}\\ 
\lstinline/Registry/&\raisebox{0pt}{\lstinline/ loadDescriptors(StringpackageName,ClassLoaderclassLoader)/}\\ 
\lstinline/StandardService/&\raisebox{0pt}{\lstinline/ stopInternal())/}\\ 
\lstinline/GlobalResourcesLifecycleListenerListener/&\raisebox{0pt}{\lstinline/ createMBeans(String,Context))/}\\ 
\lstinline/ReplicationValve/&\raisebox{0pt}{\lstinline/ updateStats(long,long/}\\ 
\lstinline/PersistentManagerBase/&\raisebox{0pt}{\lstinline/ startInternal())/}\\ 
\lstinline/BasicDataSource/&\raisebox{0pt}{\lstinline/ NamepreRegister(MBeanServer,ObjectName)/}\\ 
\lstinline/CGIRunner/&\raisebox{0pt}{\lstinline/ sendToLog(BufferedReaderrdr)/}\\ 
\lstinline/CGIRunner/&\raisebox{0pt}{\lstinline/ sendToLog(BufferedReaderrdr)/}\\ 
\lstinline/SecurityListenerListener/&\raisebox{0pt}{\lstinline/ checkUmask()/}\\ 
\lstinline/AbstractReplicatedMap/&\raisebox{0pt}{\lstinline/ Vput()/}\\ 
\lstinline/JspServlet/&\raisebox{0pt}{\lstinline/ handleMissingResource(HttpServletRequest)/}\\ 
\lstinline/StandardManager/&\raisebox{0pt}{\lstinline/ stopInternal())/}\\ 
\lstinline/StandardManager/&\raisebox{0pt}{\lstinline/ stopInternal())/}\\ 
\lstinline/BackgroundProcessManager/&\raisebox{0pt}{\lstinline/ process()/}\\ 
\lstinline/Request/&\raisebox{0pt}{\lstinline/ getUserPrincipal()/}\\ 
\lstinline/StandardContext/&\raisebox{0pt}{\lstinline/ checkUnusualURLPattern(String)/}\\ 
\lstinline/StandardContext/&\raisebox{0pt}{\lstinline/ checkUnusualURLPattern(String)/}\\ 
\lstinline/@SuppressWarnings("deprecation")JspServletWrapper/&\raisebox{0pt}{\lstinline/ destroy()/}\\ 
\lstinline/HostConfigListener/&\raisebox{0pt}{\lstinline/ checkUndeploy()/}\\ 
\lstinline/AprLifecycleListenerListener/&\raisebox{0pt}{\lstinline/ initializeSSL())/}\\ 
\lstinline/AprLifecycleListenerListener/&\raisebox{0pt}{\lstinline/ initializeSSL())/}\\ 
\lstinline/StandardContext/&\raisebox{0pt}{\lstinline/ checkUnusualURLPattern(String)/}\\ 
\lstinline/AsyncContextImpl/&\raisebox{0pt}{\lstinline/ setErrorState(Throwable)/}\\ 
\lstinline/AsyncContextImpl/&\raisebox{0pt}{\lstinline/ setErrorState(Throwable)/}\\ 
\lstinline/AsyncContextImpl/&\raisebox{0pt}{\lstinline/ setErrorState(Throwable)/}\\ 
\lstinline/AsyncContextImpl/&\raisebox{0pt}{\lstinline/ setErrorState(Throwable)/}\\ 
\lstinline/TagHandlerPool/&\raisebox{0pt}{\lstinline/ doRelease(Taghandler)/}\\ 
\lstinline/CoyoteAdapter/&\raisebox{0pt}{\lstinline/ convertURI(MessageBytesuri,Requestrequest)/}\\ 
\lstinline/AccessLogValve/&\raisebox{0pt}{\lstinline/ open()/}\\ 
\lstinline/MessageDigestCredentialHandler/&\raisebox{0pt}{\lstinline/ setEncoding(StringencodingName)/}\\ 
\lstinline/ContextConfigListener/&\raisebox{0pt}{\lstinline/ processAnnotationsFile(File)/}\\ 
\lstinline/SimpleTcpCluster/&\raisebox{0pt}{\lstinline/ memberDisappeared(Member)/}\\ 
\lstinline/AccessLogValve/&\raisebox{0pt}{\lstinline/ open()/}\\ 
\lstinline/FarmWarDeployer/&\raisebox{0pt}{\lstinline/ booleancopy(Filefrom,Fileto)/}\\ 
\lstinline/SocketProcessor/&\raisebox{0pt}{\lstinline/ doRun()/}\\ 
\lstinline/Http11Processor/&\raisebox{0pt}{\lstinline/ sslReHandShake()/}\\ 
\lstinline/DefaultInstanceManager/&\raisebox{0pt}{\lstinline/ loadProperties()/}\\ 
\lstinline/AntCompiler&\raisebox{0pt}{\lstinline/ generateClass(String[])/}\\ 
\lstinline/AntCompiler/&\raisebox{0pt}{\lstinline/ generateClass(String[])/}\\ 
\lstinline/TcpFailureDetector/&\raisebox{0pt}{\lstinline/ booleanmemberAlive(Member)/}\\ 
\lstinline/StandardContext/&\raisebox{0pt}{\lstinline/ checkUnusualURLPattern(String)/}\\ 
\lstinline/McastService/&\raisebox{0pt}{\lstinline/ setDomain(byte[])/}\\ 
\lstinline/Parameters/&\raisebox{0pt}{\lstinline/ processParameters(int)/}\\ 
\lstinline/CoyoteAdapter/&\raisebox{0pt}{\lstinline/ convertURI(MessageByte)/}\\ 
\lstinline/StandardContext/&\raisebox{0pt}{\lstinline/ checkUnusualURLPattern(String)/}\\ 
\lstinline/AccessLogValve/&\raisebox{0pt}{\lstinline/ open()/}\\ 
\lstinline/AccessLogValve/&\raisebox{0pt}{\lstinline/ open()/}\\ 
\lstinline/Poller/&\raisebox{0pt}{\lstinline/ timeout(int)/}\\ 
\lstinline/Poller/&\raisebox{0pt}{\lstinline/ timeout(int)/}\\ 
\lstinline/FarmWarDeployer/&\raisebox{0pt}{\lstinline/ booleancopy(File,File)/}\\ 
\lstinline/NamingContextListener/&\raisebox{0pt}{\lstinline/ removeResourceLink(String)/}\\ 
\lstinline/NamingContextListener/&\raisebox{0pt}{\lstinline/ removeResourceLink(String)/}\\ 
\lstinline/NamingContextListener/&\raisebox{0pt}{\lstinline/ removeResourceLink(String)/}\\ 
\lstinline/NamingContextListener/&\raisebox{0pt}{\lstinline/ removeResourceLink(String)/}\\ 
\lstinline/NamingContextListener/&\raisebox{0pt}{\lstinline/ removeResourceLink(String)/}\\ 
\lstinline/NioReceiver/&\raisebox{0pt}{\lstinline/ run()/}\\ 
\lstinline/Digester/&\raisebox{0pt}{\lstinline/ createSAXException(String,Exception)/}\\ 
\lstinline/JNDIRealm/&\raisebox{0pt}{\lstinline/ startInternal())/}\\ 
\lstinline/BioReplicationTask/&\raisebox{0pt}{\lstinline/ sendAck(byte[])/}\\ 
\lstinline/ReplicationValve/&\raisebox{0pt}{\lstinline/ updateStats(long)/}\\ 
\lstinline/ContainerBackgroundProcessor/&\raisebox{0pt}{\lstinline/ processChildren(Container)/}\\ 
\lstinline/Registry/&\raisebox{0pt}{\lstinline/ loadDescriptors(String)/}\\ 
\lstinline/Compiler/&\raisebox{0pt}{\lstinline/ removeGeneratedClassFiles()/}\\ 
\lstinline/Compiler/&\raisebox{0pt}{\lstinline/ removeGeneratedClassFiles()/}\\ 
\lstinline/AbstractReplicatedMap/&\raisebox{0pt}{\lstinline/ Vput()/}\\ 
\lstinline/CatalinaShutdownHook/&\raisebox{0pt}{\lstinline/ run()/}\\ 
\lstinline/CatalinaShutdownHook/&\raisebox{0pt}{\lstinline/ run()/}\\ 
\lstinline/AbstractReplicatedMap/&\raisebox{0pt}{\lstinline/ Vput()/}\\ 
\lstinline/AbstractReplicatedMap/&\raisebox{0pt}{\lstinline/ Vput()/}\\ 
\lstinline/ContextConfigListener/&\raisebox{0pt}{\lstinline/ processAnnotationsFile(File)/}\\ 
\lstinline/ContextConfigListener/&\raisebox{0pt}{\lstinline/ processAnnotationsFile(File)/}\\ 
\lstinline/WebappClassLoaderBase/&\raisebox{0pt}{\lstinline/ findClassInternal(String)/}\\ 
\lstinline/NamingContextListener/&\raisebox{0pt}{\lstinline/ removeResourceLink(String)/}\\ 
\lstinline/NamingContextListener/&\raisebox{0pt}{\lstinline/ removeResourceLink(String)/}\\ 
\lstinline/AccessLogValve/&\raisebox{0pt}{\lstinline/ open()/}\\ 
\lstinline/RecoveryThread/&\raisebox{0pt}{\lstinline/ run()/}\\ 
\lstinline/StoreLoader/&\raisebox{0pt}{\lstinline/ load()/}\\ 
\lstinline/StoreLoader/&\raisebox{0pt}{\lstinline/ load()/}\\ 
\lstinline/SslRmiServerBindSocketFactory/&\raisebox{0pt}{\lstinline/ SslRmiServerBindSocketFactory(String[])/}\\ 
\lstinline/SessionIdGeneratorBase/&\raisebox{0pt}{\lstinline/ SecureRandomcreateSecureRandom()/}\\ 
\lstinline/Connector/&\raisebox{0pt}{\lstinline/ resume()/}\\ 
\lstinline/HostConfigListener/&\raisebox{0pt}{\lstinline/ checkUndeploy()/}\\ 
\lstinline/HostConfigListener/&\raisebox{0pt}{\lstinline/ checkUndeploy()/}\\ 
\lstinline/UserConfigListener/&\raisebox{0pt}{\lstinline/ lifecycleEvent(LifecycleEvent)/}\\ 
\lstinline/ContextConfigListener/&\raisebox{0pt}{\lstinline/ processAnnotationsFile(File)/}\\ 
\lstinline/EngineConfigListener/&\raisebox{0pt}{\lstinline/ lifecycleEvent(LifecycleEvent)/}\\ 
\lstinline/DataSourceRealm/&\raisebox{0pt}{\lstinline/ getRoles(ConnectiondbConnection,String)/}\\ 
\lstinline/StandardContext/&\raisebox{0pt}{\lstinline/ checkUnusualURLPattern(String)/}\\ 
\lstinline/StoreRegistry/&\raisebox{0pt}{\lstinline/ StoreDescriptionfindDescription(Stringid)/}\\ 
\lstinline/StandardServer/&\raisebox{0pt}{\lstinline/ storeContext(Context)}\ 
\lstinline/StandardContext/&\raisebox{0pt}{\lstinline/ checkUnusualURLPattern(String)/}\\ 
\lstinline/RealmBase/&\raisebox{0pt}{\lstinline/ StringDigest(String)/}\\ 
\lstinline/RealmBase/&\raisebox{0pt}{\lstinline/ StringDigest(String)/}\\ 
\lstinline/JNDIRealm/&\raisebox{0pt}{\lstinline/ startInternal())/}\\ 
\lstinline/JNDIRealm/&\raisebox{0pt}{\lstinline/ startInternal())/}\\ 
\lstinline/NamingResourcesImpl/&\raisebox{0pt}{\lstinline/ destroyInternal())/}\\ 
\lstinline/NamingResourcesImpl/&\raisebox{0pt}{\lstinline/ destroyInternal())/}\\ 
\lstinline/Bootstrap/&\raisebox{0pt}{\lstinline/ main(Stringargs[])/}\\ 
\lstinline/ReceiverBase/&\raisebox{0pt}{\lstinline/ intbindUdp(DatagramSocket)/}\\ 
\lstinline/NamingResourcesImpl/&\raisebox{0pt}{\lstinline/ destroyInternal())/}\\ 
\lstinline/DataSourceRealm/&\raisebox{0pt}{\lstinline/ getRoles(Connection,String)/}\\ 
\lstinline/Tool/&\raisebox{0pt}{\lstinline/ usage()/}\\ 
\lstinline/Tool/&\raisebox{0pt}{\lstinline/ usage()/}\\ 
\lstinline/Tool/&\raisebox{0pt}{\lstinline/ usage()/}\\ 
\lstinline/NonBlockingCoordinator/&\raisebox{0pt}{\lstinline/ fireInterceptorEvent(InterceptorEvent)/}\\ 
\lstinline/CatalinaShutdownHook/&\raisebox{0pt}{\lstinline/ run()/}\\ 
\lstinline/CatalinaShutdownHook/&\raisebox{0pt}{\lstinline/ run()/}\\ 
\lstinline/CatalinaShutdownHook/&\raisebox{0pt}{\lstinline/ run()/}\\ 
\lstinline/Registry/&\raisebox{0pt}{\lstinline/ loadDescriptors(String)/}\\ 
\lstinline/RewriteValv/&\raisebox{0pt}{\lstinline/ parse(BufferedReader))/}\\ 
\lstinline/StoreLoader/&\raisebox{0pt}{\lstinline/ load()/}\\ 
\lstinline/CatalinaProperties/&\raisebox{0pt}{\lstinline/ loadProperties()/}\\ 
\lstinline/CGIRunner/&\raisebox{0pt}{\lstinline/ sendToLog(BufferedReader)/}\\ 
\lstinline/HeartbeatListenerListener,ContainerListener/&\raisebox{0pt}{\lstinline/ lifecycleEvent(LifecycleEvent)/}\\ 
\lstinline/TcpSender/&\raisebox{0pt}{\lstinline/ intsend(String))/}\\ 
\lstinline/SocketProcessor/&\raisebox{0pt}{\lstinline/ doRun()/}\\ 


\bottomrule
\end{tabular}
\end{center}


\subsection{Cluster classified as Outer Method Logging}

\begin{center}
\captionof{figure}{LMs in the cluster $\id{T}_{\id{OM},1}$}
\begin{tabular}{ll}\toprule
\multicolumn{1}{c}{Class}&\multicolumn{1}{c}{Method}\\\midrule
\lstinline/LifecycleListener/&\raisebox{0pt}{\lstinline/log()/}\\ 
\lstinline/LifecycleListener/&\raisebox{0pt}{\lstinline/log()/}\\ 
\lstinline/LifecycleListener/&\raisebox{0pt}{\lstinline/log()/}\\ 
\lstinline/LifecycleListener/&\raisebox{0pt}{\lstinline/log()/}\\ 
\lstinline/LifecycleListener/&\raisebox{0pt}{\lstinline/log()/}\\ 
\lstinline/LifecycleListener/&\raisebox{0pt}{\lstinline/log()/}\\ 
\lstinline/LifecycleListener/&\raisebox{0pt}{\lstinline/log()/}\\ 
\lstinline/LifecycleListener/&\raisebox{0pt}{\lstinline/log()/}\\ 
\lstinline/LifecycleListener/&\raisebox{0pt}{\lstinline/log()/}\\ 
\lstinline/LifecycleListener/&\raisebox{0pt}{\lstinline/log()/}\\ 
\lstinline/LifecycleListener/&\raisebox{0pt}{\lstinline/log()/}\\ 
\lstinline/LifecycleListener/&\raisebox{0pt}{\lstinline/initializeSSL()throwsException/}\\ 
\lstinline/AccessLogValve/&\raisebox{0pt}{\lstinline/getServletRequestElement(String)/}\\ 
\lstinline/AccessLogValve/&\raisebox{0pt}{\lstinline/getServletRequestElement(String)/}\\ 
\lstinline/AccessLogValve/&\raisebox{0pt}{\lstinline/getServletRequestElement(String)/}\\ 
\lstinline/Digester/&\raisebox{0pt}{\lstinline/SAXExceptioncreateSAXException(String,Exception)/}\\ 
\lstinline/AccessLogValve/&\raisebox{0pt}{\lstinline/getServletRequestElement(String)/}\\ 
\lstinline/AccessLogValve/&\raisebox{0pt}{\lstinline/getServletRequestElement(String)/}\\ 
\lstinline/RequestDumperFilter/&\raisebox{0pt}{\lstinline/doLog(String,String)/}\\ 
\lstinline/WebappClassLoaderBase/&\raisebox{0pt}{\lstinline/findClassInternal(String)/}\\ 
\lstinline/CombinedRealm/&\raisebox{0pt}{\lstinline/getPrincipal(String)/}\\ 
\lstinline/CombinedRealm/&\raisebox{0pt}{\lstinline/getPrincipal(String)/}\\ 
\lstinline/AccessLogValve/&\raisebox{0pt}{\lstinline/getServletRequestElement(String)/}\\ 
\lstinline/PortElement/&\raisebox{0pt}{\lstinline/PortElement(String)/}\\ 
\lstinline/Digester/&\raisebox{0pt}{\lstinline/createSAXException(String,Exception)/}\\ 
\lstinline/Digester/&\raisebox{0pt}{\lstinline/createSAXException(String,Exception)/}\\ 
\lstinline/JAASRealm/&\raisebox{0pt}{\lstinline/authenticate(String,CallbackHandler)/}\\ 
\lstinline/Tool/&\raisebox{0pt}{\lstinline/usage()/}\\ 
\bottomrule
\end{tabular}
\end{center}

\subsection{Cluster classified as Control Flow Logging}

\begin{center}
\captionof{figure}{LMs in the cluster $\id{T}_{\id{CF},1}$}
\begin{tabular}{ll}\toprule
\multicolumn{1}{c}{Class}&\multicolumn{1}{c}{Method}\\\midrule
\lstinline/ Parameters/&\raisebox{0pt}{\lstinline/  lstinlineprocessParameters(byte[],int)/}\\ 
\lstinline/ Parameters/&\raisebox{0pt}{\lstinline/  lstinlineprocessParameters(byte[],int)/}\\ 
\lstinline/ Parameters/&\raisebox{0pt}{\lstinline/  lstinlineprocessParameters(byte[],int)/}\\ 
\lstinline/ Parameters/&\raisebox{0pt}{\lstinline/  lstinlineprocessParameters(byte[],int)/}\\ 
\lstinline/ LegacyCookieProcessor/&\raisebox{0pt}{\lstinline/  lstinlineprocessCookieHeader(byte[],int)/}\\ 
\lstinline/ LegacyCookieProcessor/&\raisebox{0pt}{\lstinline/  lstinlineprocessCookieHeader(byte[],int)/}\\ 
\lstinline/ Cookie/&\raisebox{0pt}{\lstinline/  lstinlinelogInvalidVersion(ByteBuffer)/}\\ 
\lstinline/ Cookie/&\raisebox{0pt}{\lstinline/  lstinlinelogInvalidVersion(ByteBuffer)/}\\ 
\lstinline/ ConnectionSettingsBase<TextendsThrowable>/&\raisebox{0pt}{\lstinline/  lstinlineset(Setting,long)/}\\ 
\lstinline/ Http11Processor/&\raisebox{0pt}{\lstinline/  lstinlinesslReHandShake()/}\\ 
\lstinline/ WebXmlextendsXmlEncodingBase/&\raisebox{0pt}{\lstinline/  orderWebFragments(WebXml)/}\\ 
\lstinline/ WebXmlextendsXmlEncodingBase/&\raisebox{0pt}{\lstinline/  orderWebFragments(WebXml)/}\\ 
\lstinline/ PortElementt/&\raisebox{0pt}{\lstinline/  lstinlinePortElement(String)/}\\ 
\lstinline/WebappLoader\/&\raisebox{0pt}{\lstinline/buildClassPath(String)/}\\ 
\lstinline/JAASMemoryLoginModule/&\raisebox{0pt}{\lstinline/load()/}\\ 
\lstinline/Tool/&\raisebox{0pt}{\lstinline/usage()/}\\ 
\lstinline/NioSelectorPool/&\raisebox{0pt}{\lstinline/getSharedSelector()/}\\ 
\lstinline/CatalinaProperties/&\raisebox{0pt}{\lstinline/loadProperties()/}\\ 
\lstinline/TcpSender/&\raisebox{0pt}{\lstinline/intsend(String)/}\\ 
\lstinline/TcpSender/&\raisebox{0pt}{\lstinline/intsend(String)/}\\ 
\lstinline/Diagnostics/&\raisebox{0pt}{\lstinline/setVerboseGarbageCollection(boolean)/}\\ 
\lstinline/Diagnostics/&\raisebox{0pt}{\lstinline/setVerboseGarbageCollection(boolean)/}\\ 
\lstinline/Diagnostics/&\raisebox{0pt}{\lstinline/setVerboseGarbageCollection(boolean)/}\\ 
\lstinline/Diagnostics/&\raisebox{0pt}{\lstinline/setVerboseGarbageCollection(boolean)/}\\ 
\lstinline/Mapper/&\raisebox{0pt}{\lstinline/find(String)/}\\ 
\lstinline/Mapper/&\raisebox{0pt}{\lstinline/find(String)/}\\ 
\lstinline/Mapper/&\raisebox{0pt}{\lstinline/find(String)/}\\ 
\lstinline/Mapper/&\raisebox{0pt}{\lstinline/find(String)/}\\ 
\lstinline/ContainerBackgroundProcessor/&\raisebox{0pt}{\lstinline/processChildren(Container)/}\\ 
\lstinline/DigesterextendsDefaultHandler2/&\raisebox{0pt}{\lstinline/createSAXException(String)/}\\ 
\lstinline/Registry/&\raisebox{0pt}{\lstinline/loadDescriptors(String)/}\\ 
\lstinline/Mapper/&\raisebox{0pt}{\lstinline/find(String)/}\\ 
\lstinline/ExtendedAccessLogValve/&\raisebox{0pt}{\lstinline/getServletRequestElement(String)/}\\ 
\lstinline/ExtendedAccessLogValve/&\raisebox{0pt}{\lstinline/getServletRequestElement(String)/}\\ 
\lstinline/ExtendedAccessLogValve/&\raisebox{0pt}{\lstinline/getServletRequestElement(String)/}\\ 
\lstinline/TcpSender/&\raisebox{0pt}{\lstinline/intsend(Stringmess)/}\\ 
\lstinline/Registry/&\raisebox{0pt}{\lstinline/loadDescriptors(String)/}\\ 
\lstinline/StoreLoader/&\raisebox{0pt}{\lstinline/load()/}\\ 
\lstinline/CatalinaShutdownHookextendsThread/&\raisebox{0pt}{\lstinline/run()/}\\ 
\lstinline/ExtendedAccessLogValve/&\raisebox{0pt}{\lstinline/getServletRequestElement(String)/}\\ 
\lstinline/ExtendedAccessLogValve/&\raisebox{0pt}{\lstinline/getServletRequestElement(String)/}\\ 
\lstinline/ManagerBaseextendsLifecycleMBeanBase/&\raisebox{0pt}{\lstinline/getCreationTime(String)/}\\ 
\lstinline/ExtendedAccessLogValve/&\raisebox{0pt}{\lstinline/getServletRequestElement(String)/}\\ 
\lstinline/JspCextendsTask/&\raisebox{0pt}{\lstinline/locateUriRoot(File)/}\\ 
\lstinline/ContextConfig/&\raisebox{0pt}{\lstinline/processAnnotationsFile(File))/}\\ 
\lstinline/LoaderSFextendsStoreFactoryBase/&\raisebox{0pt}{\lstinline/store(PrintWriter)/}\\ 
\lstinline/JspCextendsTask/&\raisebox{0pt}{\lstinline/locateUriRoot(File)/}\\ 
\lstinline/SSLHostConfig/&\raisebox{0pt}{\lstinline/adjustRelativePath(String)/}\\ 
\lstinline/JAASMemoryLoginModule/&\raisebox{0pt}{\lstinline/load()/}\\ 
\lstinline/JspCextendsTask/&\raisebox{0pt}{\lstinline/locateUriRoot(File)/}\\ 
\lstinline/StandardEngine/&\raisebox{0pt}{\lstinline/logAccess(Request,long,boolean)/}\\ 
\lstinline/Diagnostics/&\raisebox{0pt}{\lstinline/setVerboseGarbageCollection(boolean)/}\\ 

\bottomrule
\end{tabular}
\end{center}

\subsection{Cluster classified as Exception Try-Block Logging}

\begin{center}
\captionof{figure}{LMs in the cluster $\id{T}_{\id{TB},1}$}
\begin{tabular}{ll}\toprule
\multicolumn{1}{c}{Class}&\multicolumn{1}{c}{Method}\\\midrule
\lstinline/WebappClassLoaderBase/&\raisebox{0pt}{\lstinline/findClassInternal(String)/}\\ 
\lstinline/ReceiverBase/&\raisebox{0pt}{\lstinline/intbindUdp(DatagramSocket)/}\\ 
\lstinline/ReceiverBase/&\raisebox{0pt}{\lstinline/intbindUdp(DatagramSocket)/}\\ 
\lstinline/RecoveryThread/&\raisebox{0pt}{\lstinline/run()/}\\ 
\lstinline/SslRmiServerBindSocketFactory/&\raisebox{0pt}{\lstinline/miServerBindSocketFactory(String[],String[],boolean,String)/}\\ 
\lstinline/SocketProcessor/&\raisebox{0pt}{\lstinline/doRun()/}\\ 
\bottomrule
\end{tabular}
\end{center}

\section{Hibernate}\label{hibernate}

\subsection{Cluster classified as Exception Catch-Block Logging}

\begin{center}
\captionof{figure}{LMs in the cluster $\id{H}_{\id{CB},1}$}
\begin{tabular}{ll}\toprule
\multicolumn{1}{c}{Class}&\multicolumn{1}{c}{Method}\\\midrule
\lstinline/LockTest/&\raisebox{0pt}{\lstinline/call()/}\\
\lstinline/LockTest/&\raisebox{0pt}{\lstinline/call()/}\\
\lstinline/LockTest/&\raisebox{0pt}{\lstinline/call()/}\\
\lstinline/LockTest/&\raisebox{0pt}{\lstinline/call()/}\\
\lstinline/LockTest/&\raisebox{0pt}{\lstinline/call()/}\\
\lstinline/SessionFactoryOptionsStateStandardImpl/&\raisebox{0pt}{\lstinline/SessionFactoryOptionsStateStandardImpl(StandardServiceRegistry)/}\\
\lstinline/AbstractLazyInitializer/&\raisebox{0pt}{\lstinline/permissiveInitialization()/}\\
\lstinline/AbstractPersistentCollection/&\raisebox{0pt}{\lstinline/twithTemporarySessionIfNeeded(LazyInitializationWork<T>)/}\\
\lstinline/Helper/&\raisebox{0pt}{\lstinline/performWork(LazyInitializationWork<T>)/}\\
\lstinline/JdbcConnectionAccessConnectionProviderImplim/&\raisebox{0pt}{\lstinline/releaseConnection(Connection)/}\\
\lstinline/JdbcConnectionAccessProvidedConnectionImpl/&\raisebox{0pt}{\lstinline/releaseConnection(Connection)/}\\
\lstinline/ScriptSourceInputFromFile/&\raisebox{0pt}{\lstinline/release()/}\\
\lstinline/ScriptSourceInputFromUrl/&\raisebox{0pt}{\lstinline/release()/}\\
\lstinline/Helper/&\raisebox{0pt}{\lstinline/performWork(LazyInitializationWork<T>))/}\\
\lstinline/OracleTypesHelper/&\raisebox{0pt}{\lstinline/OracleTypesHelper()/}\\
\lstinline/MultiTenantConnectionProviderInitiator<MultiTenantConnectionProvider>/&\raisebox{0pt}{\lstinline/)initiateService(Mapcon)/}\\
\lstinline/MultiTenantConnectionProviderInitiator<MultiTenantConnectionProvider>/&\raisebox{0pt}{\lstinline/initiateService(Mapcon)/}\\
\lstinline/MultipleSessionCollectionTest/&\raisebox{0pt}{\lstinline/testDeleteCopyToNewOwnerInNewSessionBeforeFlush()/}\\
\lstinline/MultipleSessionCollectionTest/&\raisebox{0pt}{\lstinline/testDeleteCopyToNewOwnerNewCollectionRoleInNewSessionBeforeFlush()/}\\
\lstinline/MultipleSessionCollectionTest/&\raisebox{0pt}{\lstinline/testCopyInitializedCollectionReferenceToNewEntityCollectionRoleAfterGet()/}\\
\lstinline/MultipleSessionCollectionTest/&\raisebox{0pt}{\lstinline/testSaveOrUpdateOwnerWithCollectionInNewSessionAfterFlush()/}\\
\lstinline/MultipleSessionCollectionTest/&\raisebox{0pt}{\lstinline/testCopyPersistentCollectionReferenceAfterFlush()/}\\
\lstinline/MultipleSessionCollectionTest/&\raisebox{0pt}{\lstinline/testSaveOrUpdateOwnerWithCollectionInNewSessionBeforeFlush()/}\\
\lstinline/MultipleSessionCollectionTest/&\raisebox{0pt}{\lstinline/testCopyUninitializedCollectionReferenceAfterGet()/}\\
\lstinline/MultipleSessionCollectionTest/&\raisebox{0pt}{\lstinline/testSaveOrUpdateOwnerWithUninitializedCollectionInNewSession()/}\\
\lstinline/MultipleSessionCollectionTest/&\raisebox{0pt}{\lstinline/testCopyPersistentCollectionReferenceBeforeFlush()/}\\
\lstinline/MultipleSessionCollectionTest/&\raisebox{0pt}{\lstinline/testSaveOrUpdateOwnerWithInitializedCollectionInNewSession()/}\\
\lstinline/MultipleSessionCollectionTest/&\raisebox{0pt}{\lstinline/testCopyInitializedCollectionReferenceAfterGet()/}\\
\lstinline/MultipleHiLoPerTableGenerator/&\raisebox{0pt}{\lstinline/IntegralDataTypeHolderexecute(Connection)/}\\
\lstinline/CascadeTest/&\raisebox{0pt}{\lstinline/cleanupData()/}\\
\lstinline/BeforeCompletionFailureTest/&\raisebox{0pt}{\lstinline/testUniqueConstraintViolationDuringManagedFlush()/}\\
\lstinline/LockTest/&\raisebox{0pt}{\lstinline/testFindWithPessimisticWriteLockTimeoutException()/}\\
\lstinline/IdTableHelper/&\raisebox{0pt}{\lstinline/executeIdTableCreationStatements(List<String>,JdbcServices,JdbcConnectionAccess)/}\\
\lstinline/IdTableHelper/&\raisebox{0pt}{\lstinline/executeIdTableDropStatements(String[],JdbcServices,JdbcConnectionAccess)/}\\
\lstinline/IdTableHelper/&\raisebox{0pt}{\lstinline/executeIdTableDropStatements(String[],JdbcServices,JdbcConnectionAccess)/}\\
\lstinline/IdTableHelper/&\raisebox{0pt}{\lstinline/executeIdTableCreationStatements(List<String>,JdbcServices,JdbcConnectionAccess)/}\\
\lstinline/TemporaryTableDropWork/&\raisebox{0pt}{\lstinline/execute(Connectionconnection)/}\\
\lstinline/LockTest/&\raisebox{0pt}{\lstinline/call()/}\\
\lstinline/LockTest/&\raisebox{0pt}{\lstinline/call()/}\\
\lstinline/LockTest/&\raisebox{0pt}{\lstinline/call()/}\\
\lstinline/LockTest/&\raisebox{0pt}{\lstinline/call()/}\\
\lstinline/AbstractServiceRegistryImpl/&\raisebox{0pt}{\lstinline/applyInjections(Rservice)/}\\
\lstinline/TableStructure/&\raisebox{0pt}{\lstinline/integralDataTypeHolderexecute(Connection)/}\\
\lstinline/JavassistProxyFactory/&\raisebox{0pt}{\lstinline/hibernateProxygetProxy(Serializableid,SharedSession)/}\\
\lstinline/JavassistProxyFactory/&\raisebox{0pt}{\lstinline/hibernateProxydeserializeProxy(Serializable)/}\\
\bottomrule
\end{tabular}
\end{center}



\subsection{Cluster classified as Conditional Logging}

\begin{center}
\captionof{figure}{LMs in the cluster $\id{H}_{\id{K},1}$}
\begin{tabular}{ll}\toprule
\multicolumn{1}{c}{Class}&\multicolumn{1}{c}{Method}\\\midrule
\lstinline/ConfigHelper/&\raisebox{0pt}{\lstinline/getConfigStream(String)/}\\
\lstinline/DriverManagerConnectionProviderImpl/&\raisebox{0pt}{\lstinline/buildCreator(MapconfigurationValues)/}\\
\lstinline/TableStructure/&\raisebox{0pt}{\lstinline/erexecute(Connection)/}\\
\lstinline/BaseEntityManagerFunctionalTestCase/&\raisebox{0pt}{\lstinline/releaseUnclosedEntityManager(EntityManager)/}\\
\lstinline/BaseEntityManagerFunctionalTestCase/&\raisebox{0pt}{\lstinline/releaseUnclosedEntityManager(EntityManager)/}\\
\lstinline/SequenceGenerator/&\raisebox{0pt}{\lstinline/configure(Type,Properties,ServiceRegistry)/}\\
\lstinline/BatchFetchQueue/&\raisebox{0pt}{\lstinline/getCollectionBatch(CollectionPersister,Serializable,int)/}\\
\lstinline/CommandLineArgs/&\raisebox{0pt}{\lstinline/parseCommandLineArgs(String[])/}\\
\lstinline/CommandLineArgs/&\raisebox{0pt}{\lstinline/parseCommandLineArgs(String[])/}\\
\lstinline/CommandLineArgs/&\raisebox{0pt}{\lstinline/parseCommandLineArgs(String[])/}\\
\lstinline/CommandLineArgs/&\raisebox{0pt}{\lstinline/parseCommandLineArgs(String[])/}\\
\lstinline/SQLFunctionsTest/&\raisebox{0pt}{\lstinline/testSqlFunctionAsAlias()/}\\
\lstinline/SQLFunctionsInterSystemsTest/&\raisebox{0pt}{\lstinline/testSqlFunctionAsAlias()/}\\ 
\lstinline/IntegratorImpl/&\raisebox{0pt}{\lstinline/integrate(SessionFactory)/}\\
\lstinline/CacheableHbmXmlTest/&\raisebox{0pt}{\lstinline/deleteBinFile(/}\\
\lstinline/CacheableFileXmlSource/&\raisebox{0pt}{\lstinline/writeSerFile(Serializable,File,File)/}\\
\lstinline/SessionImpl/&\raisebox{0pt}{\lstinline/shouldCloseJdbcCoordinatorOnClose(boolean)/}\\
\lstinline/InFlightMetadataCollectorImpl/&\raisebox{0pt}{\lstinline/addFetchProfile(FetchProfile)/}\\
\lstinline/SqlExceptionHelper/&\raisebox{0pt}{\lstinline/logExceptions(SQLExceptionsql,String)/}\\
\lstinline/PooledConnections/&\raisebox{0pt}{\lstinline/close(/}\\
\lstinline/CacheableHbmXmlTest/&\raisebox{0pt}{\lstinline/createBinFile()/}\\
\lstinline/BeforeCompletionFailureTest/&\raisebox{0pt}{\lstinline/testUniqueConstraintViolationDuringManagedFlush()/}\\ 
\bottomrule
\end{tabular}
\end{center}

\subsection{Cluster classified as Outer Method Logging}

\begin{center}
\captionof{figure}{LMs in the cluster $\id{H}_{\id{OM},1}$}
\begin{tabular}{ll}\toprule
\multicolumn{1}{c}{Class}&\multicolumn{1}{c}{Method}\\\midrule
\lstinline/DynamicFilterTest/&\raisebox{0pt}{\lstinline/ queryWithFilters()/}\\ 
\lstinline/DynamicFilterTest/&\raisebox{0pt}{\lstinline/ queryWithFilters()/}\\ 
\lstinline/DynamicFilterTest/&\raisebox{0pt}{\lstinline/ queryWithFilters()/}\\ 
\lstinline/DynamicFilterTest/&\raisebox{0pt}{\lstinline/ queryWithFilters()/}\\ 
\lstinline/DynamicFilterTest/&\raisebox{0pt}{\lstinline/ queryWithFilters()/}\\ 
\lstinline/DynamicFilterTest/&\raisebox{0pt}{\lstinline/ queryWithFilters()/}\\ 
\lstinline/DynamicFilterTest/&\raisebox{0pt}{\lstinline/ queryWithFilters()/}\\ 
\lstinline/DynamicFilterTest/&\raisebox{0pt}{\lstinline/ queryWithFilters()/}\\ 
\lstinline/DynamicFilterTest/&\raisebox{0pt}{\lstinline/ queryWithFilters()/}\\ 
\lstinline/DynamicFilterTest/&\raisebox{0pt}{\lstinline/ queryWithFilters()/}\\ 
\lstinline/DynamicFilterTest/&\raisebox{0pt}{\lstinline/ queryWithFilters()/}\\ 
\lstinline/DynamicFilterTest/&\raisebox{0pt}{\lstinline/ queryWithFilters()/}\\ 
\lstinline/DynamicFilterTest/&\raisebox{0pt}{\lstinline/ queryWithFilters()/}\\ 
\lstinline/DynamicFilterTest/&\raisebox{0pt}{\lstinline/ queryWithFilters()/}\\ 
\lstinline/StandardWarningHandle/&\raisebox{0pt}{\lstinline/ logWarning(String,String)/}\\ 
\lstinline/StandardWarningHandle/&\raisebox{0pt}{\lstinline/ logWarning(String,String)/}\\ 
\lstinline/ErrorCounter/&\raisebox{0pt}{\lstinline/ reportError(String)/}\\ 
\lstinline/ErrorCounter/&\raisebox{0pt}{\lstinline/ reportError(RecognitionException)/}\\ 
\lstinline/DynamicFilterTest/&\raisebox{0pt}{\lstinline/ getFilters()/}\\ 
\lstinline/DynamicFilterTest/&\raisebox{0pt}{\lstinline/ getFilters()/}\\ 
\lstinline/DynamicFilterTest/&\raisebox{0pt}{\lstinline/ inStyleFilterParameter()/}\\ 
\lstinline/DynamicFilterTest/&\raisebox{0pt}{\lstinline/ oneToManyFilters()/}\\ 
\lstinline/DynamicFilterTest/&\raisebox{0pt}{\lstinline/ oneToManyFilters()/}\\ 
\lstinline/DynamicFilterTest/&\raisebox{0pt}{\lstinline/ filtersWithCustomerReadAndWrite()/}\\ 
\lstinline/DynamicFilterTest/&\raisebox{0pt}{\lstinline/ filtersWithCustomerReadAndWrite()/}\\ 
\lstinline/)NaturalIdOnSingleManyToOneTest/&\raisebox{0pt}{\lstinline/ mappingProperties()/}\\ 
\lstinline/DynamicFilterTest/&\raisebox{0pt}{\lstinline/ criteriaQueryFilters()/}\\ 
\lstinline/DynamicFilterTest/&\raisebox{0pt}{\lstinline/ criteriaQueryFilters()/}\\ 
\lstinline/DynamicFilterTest/&\raisebox{0pt}{\lstinline/ criteriaQueryFilters()/}\\ 
\lstinline/DynamicFilterTest/&\raisebox{0pt}{\lstinline/ hqlFilters()/}\\ 
\lstinline/DynamicFilterTest/&\raisebox{0pt}{\lstinline/ hqlFilters()/}\\ 
\lstinline/DynamicFilterTest/&\raisebox{0pt}{\lstinline/ hqlFilters()/}\\ 
\lstinline/SQLFunctionsTest/&\raisebox{0pt}{\lstinline/ sqlFunctionAsAlias()/}\\ 
\lstinline/SQLFunctionsInterSystemsTest/&\raisebox{0pt}{\lstinline/ testSqlFunctionAsAlias()/}\\ 
\lstinline/SessionImpl/&\raisebox{0pt}{\lstinline/ naturalIdLoadAccess(CriteriaImpl)/}\\ 

\bottomrule
\end{tabular}
\end{center}



\subsection{Cluster classified as Exception Try-Block Logging}

\begin{center}
\captionof{figure}{LMs in the cluster $\id{H}_{\id{TB},1}$}
\begin{tabular}{ll}\toprule
\multicolumn{1}{c}{Class}&\multicolumn{1}{c}{Method}\\\midrule
\lstinline/LockTest/&\raisebox{0pt}{\lstinline/ testContendedPessimisticWriteLockNoWait()/}\\ 
\lstinline/LockTest/&\raisebox{0pt}{\lstinline/ testContendedPessimisticWriteLockTimeout()/}\\ 
\lstinline/LockTest/&\raisebox{0pt}{\lstinline/ testQueryTimeoutEMProps()/}\\ 
\lstinline/LockTest/&\raisebox{0pt}{\lstinline/ testQueryTimeout()/}\\ 
\lstinline/LockTest/&\raisebox{0pt}{\lstinline/ testContendedPessimisticReadLockTimeout()/}\\ 
\lstinline/LockTest/&\raisebox{0pt}{\lstinline/ testLockTimeoutEMProps()/}\\ 
\lstinline/UpgradeLockTest/&\raisebox{0pt}{\lstinline/ UpgradeReadLockToOptimisticForceIncrement()/}\\ 
\lstinline/LockTest/&\raisebox{0pt}{\lstinline/ call()/}\\ 
\lstinline/LockTest/&\raisebox{0pt}{\lstinline/ call()/}\\ 
\lstinline/LockTest/&\raisebox{0pt}{\lstinline/ call()/}\\ 
\lstinline/LockTest/&\raisebox{0pt}{\lstinline/ call()/}\\ 
\lstinline/LockTest/&\raisebox{0pt}{\lstinline/ call()/}\\ 
\lstinline/LockTest/&\raisebox{0pt}{\lstinline/ call()/}\\ 
\lstinline/LockTest/&\raisebox{0pt}{\lstinline/ call()/}\\ 
\lstinline/LockTest/&\raisebox{0pt}{\lstinline/ call()/}\\ 
\lstinline/LockTest/&\raisebox{0pt}{\lstinline/ call()/}\\ 
\lstinline/LockTest/&\raisebox{0pt}{\lstinline/ call()/}\\ 
\lstinline/LockTest/&\raisebox{0pt}{\lstinline/ call()/}\\ 
\lstinline/LockTest/&\raisebox{0pt}{\lstinline/ call()/}\\ 
\lstinline/LockTest/&\raisebox{0pt}{\lstinline/ call()/}\\ 
\lstinline/LockTest/&\raisebox{0pt}{\lstinline/ call()/}\\ 

\bottomrule
\end{tabular}
\end{center}

\section{Camel}\label{camel}

\subsection{Cluster classified as Exception Catch-Block Logging}

\begin{center}
\captionof{figure}{LMs in the cluster $\id{C}_{\id{CB},1}$}
\begin{tabular}{ll}\toprule
\multicolumn{1}{c}{Class}&\multicolumn{1}{c}{Method}\\\midrule
\lstinline/MainSupport/&\raisebox{0pt}{\lstinline/run()/}\\
\lstinline/TempFileManager/&\raisebox{0pt}{\lstinline/cleanUpTempFile()/}\\
\lstinline/DefaultShutdownStrategy/&\raisebox{0pt}{\lstinline/shutdownNow(Consumer)/}\\
\lstinline/DefaultShutdownStrategy/&\raisebox{0pt}{\lstinline/suspendNow(Consumer)/}\\
\lstinline/UnitOfWorkHelper/&\raisebox{0pt}{\lstinline/doneUow(UnitOfWork,Exchange)/}\\
\lstinline/UnitOfWorkHelper/&\raisebox{0pt}{\lstinline/doneUow(UnitOfWorkuow,Exchange)/}\\
\lstinline/ComponentConfigurationTest/&\raisebox{0pt}{\lstinline/createNewDefaultComponentEndpoint()/}\\
\lstinline/ComponentConfigurationTest/&\raisebox{0pt}{\lstinline/configureAnExistingSedaEndpoint()/}\\
\lstinline/ComponentConfigurationTest/&\raisebox{0pt}{\lstinline/createNewSedaUriEndpoint()/}\\
\lstinline/ComponentConfigurationTest/&\raisebox{0pt}{\lstinline/configureAnExistingDefaultEndpoint()/}\\
\lstinline/ComponentConfigurationTest/&\raisebox{0pt}{\lstinline/createNewSedaUriEndpoint()/}\\
\lstinline/ComponentConfigurationTest/&\raisebox{0pt}{\lstinline/createNewDefaultComponentEndpoint()/}\\
\lstinline/DefaultAsyncProcessorAwaitManager/&\raisebox{0pt}{\lstinline/doStop()/}\\
\lstinline/DefaultManagementAgent/&\raisebox{0pt}{\lstinline/run()/}\\
\lstinline/WireTapProcessor/&\raisebox{0pt}{\lstinline/Exchangecall()/}\\
\lstinline/DefaultExecutorServiceManager/&\raisebox{0pt}{\lstinline/doShutdown()/}\\
\lstinline/DefaultManagementLifecycleStrategy/&\raisebox{0pt}{\lstinline/onRoutesAdd(Collection<Route>)/}\\
\lstinline/DefaultManagementLifecycleStrategy/&\raisebox{0pt}{\lstinline/onRoutesAdd(Collection<Route>)/}\\
\lstinline/ProcessorDefinitionHelper/&\raisebox{0pt}{\lstinline/run()/}\\
\lstinline/EndpointHelper/&\raisebox{0pt}{\lstinline/pollEndpoint(long)/}\\
\lstinline/DefaultManagementLifecycleStrategy/&\raisebox{0pt}{\lstinline/onErrorHandlerRemove(RouteContext,Processo,ErrorHandlerFactory)/}\\
\lstinline/DefaultManagementLifecycleStrategy/&\raisebox{0pt}{\lstinline/onErrorHandlerAdd(RouteContext,Processo,ErrorHandlerFactory)/}\\
\lstinline/DefaultManagementLifecycleStrategy/&\raisebox{0pt}{\lstinline/onServiceAdd(CamelContext,Service,Route)/}\\
\lstinline/DefaultManagementLifecycleStrategy/&\raisebox{0pt}{\lstinline/onServiceRemove(CamelContext,Service,Route)/}\\
\lstinline/ProducerCache/&\raisebox{0pt}{\lstinline/doInProducer(Endpoint,Exchange,ExchangePattern,ProducerCallback)/}\\
\lstinline/LRUCache<K,V>/&\raisebox{0pt}{\lstinline/onRemoval(Kkey,Vvalue,RemovalCausecause)/}\\
\lstinline/DefaultManagementAgent/&\raisebox{0pt}{\lstinline/doStop()/}\\
\lstinline/CamelContextTrackerRegistry/&\raisebox{0pt}{\lstinline/contextCreated(CamelContext)/}\\
\lstinline/DefaultTraceEventHandler/&\raisebox{0pt}{\lstinline/traceExchange(ProcessorDefinition,Processor,TraceInterceptor,Exchange)/}\\
\lstinline/TempFileManager/&\raisebox{0pt}{\lstinline/onDone(Exchange)/}\\
\lstinline/DefaultManagementLifecycleStrategy/&\raisebox{0pt}{\lstinline/onThreadPoolAdd(CamelContext)/}\\
\lstinline/DefaultManagementAgent/&\raisebox{0pt}{\lstinline/createMBeanServer()/}\\
\lstinline/AggregateProcessor/&\raisebox{0pt}{\lstinline/doForceCompletionOnStop()/}\\
\lstinline/DefaultDebugger/&\raisebox{0pt}{\lstinline/onBeforeProcess(Exchange,Processor,ProcessorDefinition,Breakpoint)/}\\
\lstinline/DefaultDebugger/&\raisebox{0pt}{\lstinline/onAfterProcess(Exchange,Processor,ProcessorDefinition,long,Breakpoint)/}\\
\lstinline/DefaultDebugger/&\raisebox{0pt}{\lstinline/onEvent(Exchange,EventObject,Breakpoint)/}\\
\lstinline/EventHelper/&\raisebox{0pt}{\lstinline/doNotifyEvent(EventNotifier,EventObject)/}\\
\lstinline/RoutingSlip/&\raisebox{0pt}{\lstinline/done(boolean)/}\\
\lstinline/RoutingSlip/&\raisebox{0pt}{\lstinline/doInAsyncProducer(Producer,AsyncProcessor,Exchange,ExchangePattern,AsyncCallback)/}\\
\lstinline/RoutePolicyAdvice/&\raisebox{0pt}{\lstinline/Objectbefore(Exchange)/}\\
\lstinline/RoutePolicyAdvice/&\raisebox{0pt}{\lstinline/after(Exchange,Object)/}\\
\lstinline/DefaultUnitOfWork/&\raisebox{0pt}{\lstinline/DefaultUnitOfWork(Exchange,Logger)/}\\
\lstinline/HangupInterceptor/&\raisebox{0pt}{\lstinline/run()/}\\
\lstinline/DefaultManagementLifecycleStrategy/&\raisebox{0pt}{\lstinline/onThreadPoolRemove(CamelContext,ThreadPoolExecutor)/}\\
\lstinline/DefaultManagementLifecycleStrategy/&\raisebox{0pt}{\lstinline/onRoutesRemove(Collection)/}\\
\lstinline/NIOConverter/&\raisebox{0pt}{\lstinline/ByteBuffertoByteBuffer(String,Exchange)/}\\
\lstinline/DefaultManagementLifecycleStrategy/&\raisebox{0pt}{\lstinline/onComponentAdd(String,Component)/}\\
\lstinline/DefaultManagementLifecycleStrategy/&\raisebox{0pt}{\lstinline/onComponentRemove(String,Component)/}\\
\lstinline/DefaultManagementLifecycleStrategy/&\raisebox{0pt}{\lstinline/onEndpointRemove(Endpoint)/}\\
\lstinline/ScheduledPollConsumer/&\raisebox{0pt}{\lstinline/doRun()/}\\
\lstinline/FlexibleAggregationStrategy/&\raisebox{0pt}{\lstinline/safeInsertIntoCollection(Exchange,Collection,EtoInsert)/}\\
\lstinline/MockEndpoint/&\raisebox{0pt}{\lstinline/assertIsNotSatisfied(long)/}\\
\lstinline/MockEndpoint/&\raisebox{0pt}{\lstinline/assertIsNotSatisfied()/}\\
\lstinline/DavidSiefertTest/&\raisebox{0pt}{\lstinline/testHeaderPredicateFails()/}\\
\lstinline/XmlConverter/&\raisebox{0pt}{\lstinline/createDocumentBuilderFactory()/}\\
\lstinline/XmlConverter/&\raisebox{0pt}{\lstinline/createSAXParserFactory()/}\\
\lstinline/ReduceStacksNeededDuringRoutingTest/&\raisebox{0pt}{\lstinline/process(Exchange)/}\\
\lstinline/MyProducer/&\raisebox{0pt}{\lstinline/booleanprocess(Exchange,AsyncCallback=)/}\\
\lstinline/DefaultManagementLifecycleStrategy/&\raisebox{0pt}{\lstinline/onEndpointAdd(Endpointendpoint)/}\\
\lstinline/DefaultManagementLifecycleStrategy/&\raisebox{0pt}{\lstinline/onContextStop(CamelContextcontext)/}\\
\lstinline/XmlConverter/&\raisebox{0pt}{\lstinline/setupFeatures(DocumentBuilderFactory)/}\\
\lstinline/TempFileManager/&\raisebox{0pt}{\lstinline/onDone(Exchange)/}\\
\lstinline/ManagedRoute/&\raisebox{0pt}{\lstinline/updateRouteFromXml(String)/}\\
\lstinline/ManagedCamelContext/&\raisebox{0pt}{\lstinline/addOrUpdateRoutesFromXml(String,boolean)/}\\
\lstinline/Instance/&\raisebox{0pt}{\lstinline/manageCamelContext(Container,CamelContext)/}\\
\lstinline/XmlConverter/&\raisebox{0pt}{\lstinline/createTransformerFactory()/}\\
\lstinline/XmlConverter/&\raisebox{0pt}{\lstinline/createSAXParserFactory()/}\\
\lstinline/ManagementStrategyFactory/&\raisebox{0pt}{\lstinline/ManagementStrategycreate(CamelContext,boolean)/}\\
\lstinline/XmlConverter/&\raisebox{0pt}{\lstinline/configureSaxonTransformerFactory(TransformerFactory)/}\\
\lstinline/UnitOfWorkHelper/&\raisebox{0pt}{\lstinline/doneSynchronizations(Exchange,List<Synchronization>,Logger)/}\\
\lstinline/RedeliveryErrorHandler/&\raisebox{0pt}{\lstinline/onExceptionOccurred(Exchange,RedeliveryData)/}\\
\lstinline/DefaultUnitOfWork/&\raisebox{0pt}{\lstinline/done(Exchange)/}\\
\lstinline/DefaultConsumerTemplate/&\raisebox{0pt}{\lstinline/doneUoW(Exchange)/}\\
\lstinline/DefaultPackageScanClassResolver/&\raisebox{0pt}{\lstinline/find(PackageScanFilter,String,ClassLoader,Set<Class<?>>)/}\\
\lstinline/DefaultPackageScanClassResolver/&\raisebox{0pt}{\lstinline/DefaultPackageScanClassResolver()/}\\
\lstinline/DefaultUnitOfWork/&\raisebox{0pt}{\lstinline/done(Exchange)/}\\
\lstinline/UnitOfWorkHelper/&\raisebox{0pt}{\lstinline/beforeRouteSynchronizations(Route,Exchange,List<Synchronization>,Logger)/}\\
\lstinline/UnitOfWorkHelper/&\raisebox{0pt}{\lstinline/afterRouteSynchronizations(Route,Exchange,List<Synchronization>,Logger)/}\\
\lstinline/TimerConsumer/&\raisebox{0pt}{\lstinline/run()/}\\
\lstinline/DefaultExecutorServiceManager/&\raisebox{0pt}{\lstinline/doShutdown(ExecutorService,long,boolean)/}\\
\lstinline/DefaultTimeoutMap/&\raisebox{0pt}{\lstinline/run()/}\\
\lstinline/DefaultTimeoutMap/&\raisebox{0pt}{\lstinline/purge()/}\\
\lstinline/ReduceStacksNeededDuringRoutingTest/&\raisebox{0pt}{\lstinline/process(Exchange)/}\\
\lstinline/MyProducer/&\raisebox{0pt}{\lstinline/process(Exchange,AsyncCall)/}\\
\lstinline/XmlConverter/&\raisebox{0pt}{\lstinline/toSAXSourceFromStream(StreamSource,Exchange)/}\\
\lstinline/DefaultPackageScanClassResolver/&\raisebox{0pt}{\lstinline/doLoadJarClassEntries(InputStream,String)/}\\
\lstinline/XmlConverter/&\raisebox{0pt}{\lstinline/createDocumentBuilderFactory()/}\\
\lstinline/DefaultManagementAgent/&\raisebox{0pt}{\lstinline/createMBeanServer()/}\\
\lstinline/ManagedManagementStrategy/&\raisebox{0pt}{\lstinline/isManaged(Object,Object)/}\\
\lstinline/XPathBuilder/&\raisebox{0pt}{\lstinline/logNamespaces(Exchange)/}\\
\bottomrule
\end{tabular}
\end{center}


\subsection{Cluster classified as Conditional Logging}

\begin{center}
\captionof{figure}{LMs in the cluster $\id{T}_{\id{K},1}$}
\begin{tabular}{ll}\toprule
\multicolumn{1}{c}{Class}&\multicolumn{1}{c}{Method}\\\midrule
\lstinline/MainSupport/&\raisebox{0pt}{\lstinline/waitUntilCompleted()/}\\
\lstinline/BaseTypeConverterRegistry/&\raisebox{0pt}{\lstinline/doStop()/}\\
\lstinline/AdviceWithTasks/&\raisebox{0pt}{\lstinline/task()/}\\
\lstinline/AdviceWithTasks/&\raisebox{0pt}{\lstinline/task()/}\\
\lstinline/AdviceWithTasks/&\raisebox{0pt}{\lstinline/task()/}\\
\lstinline/AdviceWithTasks/&\raisebox{0pt}{\lstinline/task()/}\\
\lstinline/MockEndpoint/&\raisebox{0pt}{\lstinline/assertIsSatisfied(long)/}\\
\lstinline/TestSupportJmxCleanup/&\raisebox{0pt}{\lstinline/traceMBeans(String)/}\\
\lstinline/XPathSplitChoicePerformanceTest/&\raisebox{0pt}{\lstinline/process(Exchange)/}\\
\lstinline/TokenPairIteratorSplitChoicePerformanceTest/&\raisebox{0pt}{\lstinline/process(Exchange)/}\\
\lstinline/XPathSplitChoicePerformanceTest/&\raisebox{0pt}{\lstinline/process(Exchange)/}\\
\lstinline/TokenPairIteratorSplitChoicePerformanceTest/&\raisebox{0pt}{\lstinline/process(Exchange)/}\\
\lstinline/XPathSplitChoicePerformanceTest/&\raisebox{0pt}{\lstinline/process(Exchange)/}\\
\lstinline/TokenPairIteratorSplitChoicePerformanceTest/&\raisebox{0pt}{\lstinline/process(Exchange)/}\\
\lstinline/XPathSplitChoicePerformanceTest/&\raisebox{0pt}{\lstinline/process(Exchange)/}\\
\lstinline/TokenPairIteratorSplitChoicePerformanceTest/&\raisebox{0pt}{\lstinline/process(Exchange)/}\\
\lstinline/ThroughputLogger,IdAware/&\raisebox{0pt}{\lstinline/booleanprocess(Exchange,AsyncCallback)/}\\
\lstinline/TwoSchedulerConcurrentTasksOneRouteTest/&\raisebox{0pt}{\lstinline/process(Exchange)/}\\
\lstinline/TwoSchedulerConcurrentTasksOneRouteTest/&\raisebox{0pt}{\lstinline/process(Exchange)/}\\
\lstinline/UnitOfWorkTest/&\raisebox{0pt}{\lstinline/process(Exchange)/}\\
\lstinline/UnitOfWorkSynchronizationAdapterTest/&\raisebox{0pt}{\lstinline/process(Exchange)/}\\
\lstinline/DefaultTimeoutMapTest/&\raisebox{0pt}{\lstinline/testDefaultTimeoutMapPurge()/}\\
\lstinline/DefaultTimeoutMapTest/&\raisebox{0pt}{\lstinline/testDefaultTimeoutMapStopStart()/}\\
\lstinline/DefaultTimeoutMapTest/&\raisebox{0pt}{\lstinline/testExecutor()/}\\
\lstinline/DefaultManagementAgent/&\raisebox{0pt}{\lstinline/doStop()/}\\
\lstinline/RoutingSlip,Traceable,IdAware/&\raisebox{0pt}{\lstinline/EndpointresolveEndpoint(RoutingSlipIterator,Exchange)/}\\
\lstinline/GenericFileOnCompletion/&\raisebox{0pt}{\lstinline/processStrategyRollback()/}\\
\lstinline/CamelLogProcessor,IdAware/&\raisebox{0pt}{\lstinline/process(Exchange,String)/}\\
\lstinline/CamelLogger/&\raisebox{0pt}{\lstinline/process(Exchange,String)/}\\
\lstinline/CamelLogger/&\raisebox{0pt}{\lstinline/process(Exchange,Throwable)/}\\
\lstinline/CamelLogger/&\raisebox{0pt}{\lstinline/booleanprocess(Exchange,AsyncCallback)/}\\
\lstinline/CamelLogger/&\raisebox{0pt}{\lstinline/process(Exchange,String)/}\\
\lstinline/CamelLogger/&\raisebox{0pt}{\lstinline/process(Exchange,Throwable)/}\\
\lstinline/CamelLogger/&\raisebox{0pt}{\lstinline/booleanprocess(Exchange,AsyncCallback)/}\\
\lstinline/CamelLogger/&\raisebox{0pt}{\lstinline/process(Exchange,String)/}\\
\lstinline/CamelLogger/&\raisebox{0pt}{\lstinline/process(Exchange,Throwable)/}\\
\lstinline/CamelLogger/&\raisebox{0pt}{\lstinline/booleanprocess(Exchange,AsyncCallback)/}\\
\lstinline/CamelLogProcessor,IdAware/&\raisebox{0pt}{\lstinline/process(Exchange,Throwable)/}\\
\lstinline/CamelLogProcessor,IdAware/&\raisebox{0pt}{\lstinline/booleanprocess(Exchange,AsyncCallback)/}\\
\lstinline/CamelLogger/&\raisebox{0pt}{\lstinline/log(String,Throwable)/}\\
\lstinline/CamelLogger/&\raisebox{0pt}{\lstinline/log(String)/}\\
\lstinline/CamelLogger/&\raisebox{0pt}{\lstinline/log(String,Throwable)/}\\
\lstinline/CamelLogger/&\raisebox{0pt}{\lstinline/log(String)/}\\
\lstinline/CamelLogger/&\raisebox{0pt}{\lstinline/log(String)/}\\
\lstinline/CamelLogger/&\raisebox{0pt}{\lstinline/log(String,Throwable)/}\\
\lstinline/MockEndpoint/&\raisebox{0pt}{\lstinline/doAssertIsSatisfied(long)/}\\
\lstinline/XMLTokenIterator/&\raisebox{0pt}{\lstinline/XMLTokenIterator(String/}\\
\lstinline/DefaultManagementAgent/&\raisebox{0pt}{\lstinline/registerMBeanWithServer(Object/}\\
\lstinline/DefaultShutdownStrategy/&\raisebox{0pt}{\lstinline/prepareShutdown(Service)/}\\
\lstinline/AnnotationTypeConverterLoader/&\raisebox{0pt}{\lstinline/loadConverterMethods(TypeConverterRegistry)/}\\
\lstinline/XPathBuilder/&\raisebox{0pt}{\lstinline/XPathFactorycreateDefaultXPathFactory()/}\\
\lstinline/ProducerCache/&\raisebox{0pt}{\lstinline/<T>TdoInProducer(Endpoint))/}\\
\lstinline/ValidationBean/&\raisebox{0pt}{\lstinline/someMethod(String)n/}\\
\lstinline/DefaultShutdownStrategy/&\raisebox{0pt}{\lstinline/booleandoShutdown(CamelContext)/}\\
\lstinline/AggregateProcessor/&\raisebox{0pt}{\lstinline/doStart()/}\\
\lstinline/XPathBuilder/&\raisebox{0pt}{\lstinline/XPathFactorycreateDefaultXPathFactory()/}\\
\lstinline/SSLContextClientParameters/&\raisebox{0pt}{\lstinline/configureSSLContext(SSLContext)/}\\
\lstinline/DefaultStreamCachingStrategy/&\raisebox{0pt}{\lstinline/doStart()/}\\
\lstinline/ConfigurationHelperTest/&\raisebox{0pt}{\lstinline/logConfigurationField(EndpointConfiguration)/}\\
\lstinline/LogComponent/&\raisebox{0pt}{\lstinline/EndpointcreateEndpoint(String)/}\\
\lstinline/MarkerFileExclusiveReadLockStrategy/&\raisebox{0pt}{\lstinline/prepareOnStartup()/}\\
\lstinline/AggregateProcessor/&\raisebox{0pt}{\lstinline/doForceCompletionOnStop()/}\\
\lstinline/AggregateProcessor/&\raisebox{0pt}{\lstinline/doStart()/}\\
\lstinline/AggregateProcessor/&\raisebox{0pt}{\lstinline/intforceCompletionOfAllGroups()/}\\
\lstinline/XPathBuilder/&\raisebox{0pt}{\lstinline/XPathExpressioncreateXPathExpression()}\\
\lstinline/DefaultExecutorServiceManager/&\raisebox{0pt}{\lstinline/doShutdown()/}\\
\lstinline/DefaultComponent/&\raisebox{0pt}{\lstinline/StringpreProcessUri(Stringuri)/}\\
\lstinline/CamelContextHelper/&\raisebox{0pt}{\lstinline/ComponentlookupPropertiesComponent(CamelContext)/}\\
\lstinline/RedeliveryOnExceptionBlockedDelayTest/&\raisebox{0pt}{\lstinline/process(Exchange)/}\\
\lstinline/RedeliveryErrorHandlerNonBlockedRedeliveryHeaderTest/&\raisebox{0pt}{\lstinline/process(Exchange)/}\\
\lstinline/RedeliveryErrorHandlerBlockedDelayTest/&\raisebox{0pt}{\lstinline/process(Exchange)/}\\
\lstinline/RedeliveryErrorHandlerNonBlockedDelayTest/&\raisebox{0pt}{\lstinline/process(Exchange)/}\\
\lstinline/AsyncEndpointRedeliveryErrorHandlerNonBlockedDelay2Test/&\raisebox{0pt}{\lstinline/process(Exchange)/}\\
\lstinline/AsyncEndpointRedeliveryErrorHandlerNonBlockedDelayTest/&\raisebox{0pt}{\lstinline/process(Exchange)/}\\
\lstinline/AggregateProcessor/&\raisebox{0pt}{\lstinline/doStart()/}\\
\lstinline/FileIdempotentRepository/&\raisebox{0pt}{\lstinline/appendToStore(StringId)/}\\
\lstinline/)FileEndpointe<File>/&\raisebox{0pt}{\lstinline/FileConsumercreateConsumer(Processorprocessor)/}\\
\lstinline/IOHelper/&\raisebox{0pt}{\lstinline/force(FileOutputStreamo,String,Logger)/}\\
\lstinline/IOHelper/&\raisebox{0pt}{\lstinline/close(Closeable,String,Logger)/}\\
\lstinline/IOHelper/&\raisebox{0pt}{\lstinline/force(FileChannel,String,Logger)/}\\
\lstinline/IOHelper/&\raisebox{0pt}{\lstinline/close(Writer)/}\\
\lstinline/DefaultPackageScanClassResolver/&\raisebox{0pt}{\lstinline/addIfMatching(PackageScanFilter)/}\\
\lstinline/ValidatorEndpointClearCachedSchemaTest/&\raisebox{0pt}{\lstinline/clearCachedSchema()/}\\
\lstinline/ActiveMQUuidGenerator/&\raisebox{0pt}{\lstinline/StringsanitizeHostName(StringhostName)/}\\
\lstinline/MyLoggingSentEventNotifer/&\raisebox{0pt}{\lstinline/notify(EventObjectevent)/}\\
\lstinline/FileEndpoint/&\raisebox{0pt}{\lstinline/FileConsumercreateConsumer(Processorprocessor)/}\\
\lstinline/ThroughputLogger,IdAware/&\raisebox{0pt}{\lstinline/doStart()/}\\
\lstinline/ShutdownTask/&\raisebox{0pt}{\lstinline/run()/}\\
\lstinline/ShutdownTask/&\raisebox{0pt}{\lstinline/run()/}\\
\lstinline/ShutdownTask/&\raisebox{0pt}{\lstinline/run()/}\\
\lstinline/UriEndpointComponent/&\raisebox{0pt}{\lstinline/doWith(Field)/}\\
\lstinline/Activator/&\raisebox{0pt}{\lstinline/extenderCapabilityWired(Bundlebundle)/}\\
\lstinline/DefaultCamelBeanPostProcessor/&\raisebox{0pt}{\lstinline/setterPropertyInjection(Method)/}\\
\lstinline/DefaultCamelBeanPostProcessor/&\raisebox{0pt}{\lstinline/setterInjection(Method)/}\\
\lstinline/DefaultCamelBeanPostProcessor/&\raisebox{0pt}{\lstinline/setterBeanInjection(Method,String,Object,String)/}\\
\lstinline/MarkerFileExclusiveReadLockStrategy/&\raisebox{0pt}{\lstinline/deleteLockFiles()/}\\
\lstinline/DynamicRouterConvertBodyToIssueTest/&\raisebox{0pt}{\lstinline/Stringslip(String))/}\\
\lstinline/UnitOfWorkTest/&\raisebox{0pt}{\lstinline/process(Exchange)/}\\
\lstinline/UnitOfWorkSynchronizationAdapterTest/&\raisebox{0pt}{\lstinline/process(Exchange)/}\\

\bottomrule
\end{tabular}
\end{center}


\subsection{Cluster classified as Outer Method Logging}

\begin{center}
\captionof{figure}{LMs in the cluster $\id{C}_{\id{OM},1}$}
\begin{tabular}{ll}\toprule
\multicolumn{1}{c}{Class}&\multicolumn{1}{c}{Method}\\\midrule
\lstinline/ManagedRouteAddRemoveTest/&\raisebox{0pt}{\lstinline/testRouteAddRemoteRouteWithTo()/}\\
\lstinline/ManagedRouteAddRemoveTest/&\raisebox{0pt}{\lstinline/testRouteAddRemoteRouteWithTo()/}\\
\lstinline/ManagedRouteAddRemoveTest/&\raisebox{0pt}{\lstinline/testRouteAddRemoteRouteWithTo()/}\\
\lstinline/ManagedProducerRouteAddRemoveRegisterAlwaysTest/&\raisebox{0pt}{\lstinline/testRouteAddRemoteRouteWithRecipientList()/}\\
\lstinline/ManagedProducerRouteAddRemoveRegisterAlwaysTest/&\raisebox{0pt}{\lstinline/testRouteAddRemoteRouteWithRecipientList()/}\\
\lstinline/ManagedProducerRouteAddRemoveRegisterAlwaysTest/&\raisebox{0pt}{\lstinline/testRouteAddRemoteRouteWithRecipientList()/}\\
\lstinline/ManagedRouteAddRemoveTest/&\raisebox{0pt}{\lstinline/testRouteAddRemoteRouteWithRoutingSlip()/}\\
\lstinline/ManagedRouteAddRemoveTest/&\raisebox{0pt}{\lstinline/testRouteAddRemoteRouteWithRoutingSlip()/}\\
\lstinline/ManagedRouteAddRemoveTest/&\raisebox{0pt}{\lstinline/testRouteAddRemoteRouteWithRecipientList()/}\\
\lstinline/ManagedRouteAddRemoveTest/&\raisebox{0pt}{\lstinline/testRouteAddRemoteRouteWithRecipientList()/}\\
\lstinline/ManagedRouteAddRemoveTest/&\raisebox{0pt}{\lstinline/testRouteAddRemoteRouteWithRoutingSlip()/}\\
\lstinline/ManagedRouteAddRemoveTest/&\raisebox{0pt}{\lstinline/testRouteAddRemoteRouteWithRecipientList()/}\\
\lstinline/ManagedRouteStopTest/&\raisebox{0pt}{\lstinline/testStopRoute()/}\\
\lstinline/ManagedRouteAddRemoveTest/&\raisebox{0pt}{\lstinline/testRouteAddRemoteRouteWithRecipientListAndContextScopedOnCompletion()/}\\
\lstinline/ManagedRouteAddRemoveTest/&\raisebox{0pt}{\lstinline/testRouteAddRemoteRouteWithRecipientListAndContextScopedOnCompletion()/}\\
\lstinline/ManagedRouteAddRemoveTest/&\raisebox{0pt}{\lstinline/testRouteAddRemoteRouteWithRecipientListAndContextScopedOnCompletion()/}\\
\lstinline/ManagedRouteAddRemoveTest/&\raisebox{0pt}{\lstinline/testRouteAddRemoteRouteWithRecipientListAndContextScopedOnCompletion()/}\\
\lstinline/ManagedRouteAddRemoveTest/&\raisebox{0pt}{\lstinline/testRouteAddRemoteRouteWithRecipientListAndRouteScopedOnCompletion()/}\\
\lstinline/ManagedRouteAddRemoveTest/&\raisebox{0pt}{\lstinline/testRouteAddRemoteRouteWithRecipientListAndRouteScopedOnCompletion()/}\\
\lstinline/ManagedRouteAddRemoveTest/&\raisebox{0pt}{\lstinline/testRouteAddRemoteRouteWithRecipientListAndRouteScopedOnCompletion()/}\\
\lstinline/ManagedRouteAddRemoveTest/&\raisebox{0pt}{\lstinline/testRouteAddRemoteRouteWithRecipientListAndRouteScopedOnCompletion()/}\\
\lstinline/ManagedRouteAddRemoveTest/&\raisebox{0pt}{\lstinline/testRouteAddRemoteRouteWithRecipientListAndContextScopedOnException()/}\\
\lstinline/ManagedRouteAddRemoveTest/&\raisebox{0pt}{\lstinline/testRouteAddRemoteRouteWithRecipientListAndContextScopedOnException()/}\\
\lstinline/ManagedRouteAddRemoveTest/&\raisebox{0pt}{\lstinline/testRouteAddRemoteRouteWithRecipientListAndContextScopedOnException()/}\\
\lstinline/ManagedRouteAddRemoveTest/&\raisebox{0pt}{\lstinline/testRouteAddRemoteRouteWithRecipientListAndContextScopedOnException()/}\\
\lstinline/ManagedRouteAddRemoveTest/&\raisebox{0pt}{\lstinline/testRouteAddRemoteRouteWithRecipientListAndRouteScopedOnException()/}\\
\lstinline/ManagedRouteAddRemoveTest/&\raisebox{0pt}{\lstinline/testRouteAddRemoteRouteWithRecipientListAndRouteScopedOnException()/}\\
\lstinline/ManagedRouteAddRemoveTest/&\raisebox{0pt}{\lstinline/testRouteAddRemoteRouteWithRecipientListAndRouteScopedOnException()/}\\
\lstinline/ManagedRouteAddRemoveTest/&\raisebox{0pt}{\lstinline/testRouteAddRemoteRouteWithRecipientListAndRouteScopedOnException()/}\\
\lstinline/BacklogDebuggerTest/&\raisebox{0pt}{\lstinline/logDebugger()/}\\
\lstinline/BacklogDebuggerTest/&\raisebox{0pt}{\lstinline/logDebuggerSuspendOnlyOneAtBreakpoint()/}\\
\lstinline/ManagedFromRestPlaceholderTest/&\raisebox{0pt}{\lstinline/testFromRestModelPlaceholder()/}\\
\lstinline/ManagedFromRestPlaceholderTest/&\raisebox{0pt}{\lstinline/testFromRestModelPlaceholder()/}\\
\lstinline/ManagedFromRestGetTest/&\raisebox{0pt}{\lstinline/testFromRestModel()/}\\
\lstinline/ManagedFromRestGetTest/&\raisebox{0pt}{\lstinline/testFromRestModel()/}\\
\lstinline/RuntimepointRegistryTest/&\raisebox{0pt}{\lstinline/testRuntimepointRegistry()/}\\
\lstinline/ManagedResourceTest/&\raisebox{0pt}{\lstinline/ManagedResource()/}\\
\lstinline/ManagedResourceTest/&\raisebox{0pt}{\lstinline/ManagedResource()/}\\
\lstinline/ManagedRouteDumpStatsAsXmlAndResetWithCustomDomainTest/&\raisebox{0pt}{\lstinline/testPerformanceCounterStats()/}\\
\lstinline/ManagedRouteDumpStatsAsXmlAndResetWithCustomDomainTest/&\raisebox{0pt}{\lstinline/testPerformanceCounterStats()/}\\
\lstinline/BacklogTracerTest/&\raisebox{0pt}{\lstinline/testBacklogTracerEventMessageAsXml()/}\\
\lstinline/RouteDefinition/&\raisebox{0pt}{\lstinline/routeDefinitionadviceWith(ModelCamelContextc,RouteBuilder)/}\\
\lstinline/ManagedFromRestGetEmbeddedRouteTest/&\raisebox{0pt}{\lstinline/testFromRestModel()/}\\
\lstinline/ManagedFromRestGetEmbeddedRouteTest/&\raisebox{0pt}{\lstinline/testFromRestModel()/}\\
\lstinline/BacklogTracerTest/&\raisebox{0pt}{\lstinline/testBacklogTracerEventMessageDumpAllAsXml()/}\\
\lstinline/ManagedBrowsablepointAsXmlTest/&\raisebox{0pt}{\lstinline/testBrowseablepointAsXml()/}\\
\lstinline/ManagedBrowsablepointAsXmlTest/&\raisebox{0pt}{\lstinline/testBrowseablepointAsXml()/}\\
\lstinline/ManagedRouteAddSecondRouteTest/&\raisebox{0pt}{\lstinline/testRouteAddSecondRoute()/}\\
\lstinline/ManagedRouteAddSecondRouteTest/&\raisebox{0pt}{\lstinline/testRouteAddSecondRoute()/}\\
\lstinline/ManagedRouteAddSecondRouteNotRegisterNewRoutesTest/&\raisebox{0pt}{\lstinline/testRouteAddSecondRoute()/}\\
\lstinline/ManagedRouteAddSecondRouteNotRegisterNewRoutesTest/&\raisebox{0pt}{\lstinline/testRouteAddSecondRoute()/}\\
\lstinline/ManagedCamelContextDumpRoutesCoverageAsXml/&\raisebox{0pt}{\lstinline/testRouteCoverageStats()/}\\
\lstinline/ManagedBrowsablepointAsXmlTest/&\raisebox{0pt}{\lstinline/testBrowseablepointAsXmlAllIncludeBody()/}\\
\lstinline/ManagedBrowsablepointAsXmlTest/&\raisebox{0pt}{\lstinline/testBrowseablepointAsXmlAll()/}\\
\lstinline/ManagedRouteDumpStatsAsXmlCustomDomainTest/&\raisebox{0pt}{\lstinline/testPerformanceCounterStats()/}\\
\lstinline/ManagedRouteDumpStatsAsXmlTest/&\raisebox{0pt}{\lstinline/testPerformanceCounterStats()/}\\
\lstinline/ManagedCamelContextDumpRoutesAsXmlTest/&\raisebox{0pt}{\lstinline/testDumpAsXml()/}\\
\lstinline/ManagedRouteDumpRouteAsXmlTest/&\raisebox{0pt}{\lstinline/testDumpAsXml()/}\\
\lstinline/ManagedCamelContextDumpStatsAsXmlTest/&\raisebox{0pt}{\lstinline/testPerformanceCounterStats()/}\\
\lstinline/ManagedRouteDumpRouteAsXmlPlaceholderTest/&\raisebox{0pt}{\lstinline/testDumpAsXml()/}\\
\lstinline/ManagedBrowsablepointAsXmlTest/&\raisebox{0pt}{\lstinline/testBrowseablepointAsXmlRange()/}\\
\lstinline/ManagedBrowsablepointAsXmlTest/&\raisebox{0pt}{\lstinline/testBrowseablepointAsXmlRangeIncludeBody()/}\\
\lstinline/DumpModelAsXmlTransformRouteLanguageTest/&\raisebox{0pt}{\lstinline/testDumpModelAsXml()/}\\
\lstinline/DumpModelAsXmlSplitBodyRouteTest/&\raisebox{0pt}{\lstinline/testDumpModelAsXml()/}\\
\lstinline/DumpModelAsXmlTransformRouteConstantTest/&\raisebox{0pt}{\lstinline/testDumpModelAsXml()/}\\
\lstinline/DumpModelAsXmlTransformRouteTest/&\raisebox{0pt}{\lstinline/testDumpModelAsXml()/}\\
\lstinline/ProxyReturnFutureListTest/&\raisebox{0pt}{\lstinline/testFutureListCallTwoTimes()/}\\
\lstinline/ProxyReturnFutureListTest/&\raisebox{0pt}{\lstinline/testFutureListCallTwoTimes()/}\\
\lstinline/ProxyReturnFutureListTest/&\raisebox{0pt}{\lstinline/testFutureListCallTwoTimes()/}\\
\lstinline/ProxyReturnFutureListTest/&\raisebox{0pt}{\lstinline/testFutureListCallTwoTimes()/}\\
\lstinline/DefaultCamelContextSuspendResumeRouteTest/&\raisebox{0pt}{\lstinline/testSuspendResume()/}\\
\lstinline/DefaultCamelContextSuspendResumeRouteTest/&\raisebox{0pt}{\lstinline/testSuspendResume()/}\\
\lstinline/TwoRouteSuspendResumeTest/&\raisebox{0pt}{\lstinline/testSuspendResume()/}\\
\lstinline/TwoRouteSuspendResumeTest/&\raisebox{0pt}{\lstinline/testSuspendResume()/}\\
\lstinline/RouteSedaSuspendResumeTest/&\raisebox{0pt}{\lstinline/testSuspendResume()/}\\
\lstinline/RouteSedaSuspendResumeTest/&\raisebox{0pt}{\lstinline/testSuspendResume()/}\\
\lstinline/RouteSedaStopStartTest/&\raisebox{0pt}{\lstinline/testStopStart()/}\\
\lstinline/RouteSedaStopStartTest/&\raisebox{0pt}{\lstinline/testStopStart()/}\\
\lstinline/ComponentConfigurationTest/&\raisebox{0pt}{\lstinline/CreateNewDefaultComponentEndpoint()/}\\
\lstinline/BacklogDebuggerTest/&\raisebox{0pt}{\lstinline/logDebuggerUpdateBodyAndHeader()/}\\
\lstinline/BacklogDebuggerTest/&\raisebox{0pt}{\lstinline/logDebuggerUpdateBodyAndHeaderType()/}\\
\lstinline/BacklogDebuggerTest/&\raisebox{0pt}{\lstinline/logDebuggerRemoveBodyAndHeader()/}\\
\lstinline/BacklogDebuggerTest/&\raisebox{0pt}{\lstinline/logDebuggerConditional()/}\\
\lstinline/ManagedRouteStopAndStartCleanupTest/&\raisebox{0pt}{\lstinline/testStopAndStartRoute()/}\\
\lstinline/ManagedRouteStopAndStartCleanupTest/&\raisebox{0pt}{\lstinline/testStopAndStartRoute()/}\\
\lstinline/ManagedRouteStopAndStartCleanupTest/&\raisebox{0pt}{\lstinline/testStopAndStartRoute()/}\\
\lstinline/ManagedRouteStopAndStartCleanupTest/&\raisebox{0pt}{\lstinline/testStopAndStartRoute()/}\\
\lstinline/ConfigurationHelperTest/&\raisebox{0pt}{\lstinline/logConfigurationObject(EndpointConfiguration)/}\\
\lstinline/ConfigurationHelperTest/&\raisebox{0pt}{\lstinline/logConfigurationObject(EndpointConfiguration)/}\\
\lstinline/ConfigurationHelperTest/&\raisebox{0pt}{\lstinline/logConfigurationObject(EndpointConfiguration)/}\\
\lstinline/ConfigurationHelperTest/&\raisebox{0pt}{\lstinline/logConfigurationObject(EndpointConfiguration)/}\\
\lstinline/TokenPairIteratorSplitChoicePerformanceTest/&\raisebox{0pt}{\lstinline/createDataFile(Logger,int)/}\\
\lstinline/TokenPairIteratorSplitChoicePerformanceTest/&\raisebox{0pt}{\lstinline/createDataFile(Logger,int)/}\\
\lstinline/XPathSplitChoicePerformanceTest/&\raisebox{0pt}{\lstinline/createDataFile(Logger,int)/}\\
\lstinline/XPathSplitChoicePerformanceTest/&\raisebox{0pt}{\lstinline/createDataFile(Logger,int)/}\\
\lstinline/TypeConverterRegistryStatsPerformanceTest/&\raisebox{0pt}{\lstinline/testTransform()/}\\
\lstinline/TypeConverterRegistryStatsPerformanceTest/&\raisebox{0pt}{\lstinline/testTransform()/}\\
\lstinline/TypeConverterRegistryStatsPerformanceTest/&\raisebox{0pt}{\lstinline/testTransform()/}\\
\lstinline/TypeConverterRegistryStatsPerformanceTest/&\raisebox{0pt}{\lstinline/testTransform()/}\\
\lstinline/TypeConverterRegistryStatsPerformanceTest/&\raisebox{0pt}{\lstinline/testTransform()/}\\
\lstinline/ResequencerBatchOrderTest/&\raisebox{0pt}{\lstinline/testIteration(inti)/}\\
\lstinline/ResequencerBatchOrderTest/&\raisebox{0pt}{\lstinline/testIteration(inti)/}\\
\lstinline/TypeConverterConcurrencyIssueTest/&\raisebox{0pt}{\lstinline/testTypeConverter()/}\\
\lstinline/XPathRouteConcurrentBigTest/&\raisebox{0pt}{\lstinline/doSendMessages(int)/}\\
\lstinline/XPathRouteConcurrentBigTest/&\raisebox{0pt}{\lstinline/doSendMessages(int)/}\\
\lstinline/XPathRouteConcurrentBigTest/&\raisebox{0pt}{\lstinline/doSendMessages(int)/}\\
\lstinline/AggregateSimpleExpressionIssueTest/&\raisebox{0pt}{\lstinline/xxxtestAggregateSimpleExpression()/}\\
\lstinline/AggregateSimpleExpressionIssueTest/&\raisebox{0pt}{\lstinline/xxxtestAggregateSimpleExpression()/}\\
\lstinline/AggregateSimpleExpressionIssueTest/&\raisebox{0pt}{\lstinline/xxxtestAggregateSimpleExpression()/}\\
\lstinline/AggregateSimpleExpressionIssueTest/&\raisebox{0pt}{\lstinline/xxxtestAggregateSimpleExpression()/}\\
\lstinline/AggregateSimpleExpressionIssueTest/&\raisebox{0pt}{\lstinline/xxxtestAggregateSimpleExpression()/}\\
\lstinline/TypeConverterRegistryPerformanceTest/&\raisebox{0pt}{\lstinline/disbledtestPerformance()/}\\
\lstinline/ResequencerEngineTest/&\raisebox{0pt}{\lstinline/testRandom()/}\\
\lstinline/ResequencerEngineTest/&\raisebox{0pt}{\lstinline/testRandom()/}\\
\lstinline/ResequencerEngineTest/&\raisebox{0pt}{\lstinline/testRandom()/}\\
\lstinline/StAX2SAXSourceTest/&\raisebox{0pt}{\lstinline/testDefaultPrefixInRootElementWithCopyTransformer()/}\\
\lstinline/FileConcurrentWriteAppendSameFileTest/&\raisebox{0pt}{\lstinline/testConcurrentAppend()/}\\
\lstinline/XPathBuilder/&\raisebox{0pt}{\lstinline/logDiscoveredNamespaces(NodeListnamespaces)/}\\
\lstinline/TestEndpoint/&\raisebox{0pt}{\lstinline/doStart()/}\\
\lstinline/EventNotifierExchangeCompletedTest/&\raisebox{0pt}{\lstinline/testExchangeCompleted()/}\\
\lstinline/AggregateProcessor/&\raisebox{0pt}{\lstinline/restoreTimeoutMapFromAggregationRepository()/}\\
\lstinline/FileConsumerRestartNotLeakThreadTest/&\raisebox{0pt}{\lstinline/testLeak()/}\\
\lstinline/ThroughputLogger/&\raisebox{0pt}{\lstinline/createGroupIntervalLogMessage()/}\\
\lstinline/UnitOfWorkSynchronizationAdapterTestt/&\raisebox{0pt}{\lstinline/process(Exchange)/}\\
\lstinline/ThrottlerTest/&\raisebox{0pt}{\lstinline/assertThrottlerTiming(longelapsedTimeMs,intthrottle,intintervalMs,int)/}\\
\lstinline/DumpModelAsXmlSplitNestedChoiceRouteTest/&\raisebox{0pt}{\lstinline/testDumpModelAsXml()/}\\
\lstinline/ManagedBrowsablepointAsXmlFileTest/&\raisebox{0pt}{\lstinline/testBrowseablepointAsXmlAllIncludeBody()/}\\
\lstinline/ProxyReturnFutureExceptionTest/&\raisebox{0pt}{\lstinline/testFutureEchoException()/}\\
\lstinline/ProxyReturnFutureExceptionTest/&\raisebox{0pt}{\lstinline/testFutureEchoException()/}\\
\lstinline/ProxyReturnFutureListTest/&\raisebox{0pt}{\lstinline/testFutureList()/}\\
\lstinline/ProxyReturnFutureListTest/&\raisebox{0pt}{\lstinline/testFutureList()/}\\
\lstinline/AsyncProcessorAwaitManagerTest/&\raisebox{0pt}{\lstinline/process(Exchange)/}\\
\lstinline/AsyncProcessorAwaitManagerInterruptTest/&\raisebox{0pt}{\lstinline/process(Exchange)/}\\
\lstinline/AsyncProcessorAwaitManagerTest/&\raisebox{0pt}{\lstinline/process(Exchange)/}\\
\lstinline/ComponentConfigurationTest/&\raisebox{0pt}{\lstinline/IntrospectSedaEndpointParameters()/}\\
\lstinline/RedeliveryPolicyPerExceptionTest/&\raisebox{0pt}{\lstinline/testUsingCustomExceptionHandlerAndOneRedelivery()/}\\
\lstinline/RedeliveryPolicyPerExceptionTest/&\raisebox{0pt}{\lstinline/testUsingCustomExceptionHandlerWithNoRedeliveries()/}\\
\lstinline/ProxyReturnFutureTest/&\raisebox{0pt}{\lstinline/testFutureEchoCallTwoTimes()/}\\
\lstinline/ProxyReturnFutureTest/&\raisebox{0pt}{\lstinline/testFutureEchoCallTwoTimes()/}\\
\lstinline/ProxyReturnFutureTest/&\raisebox{0pt}{\lstinline/testFutureEchoCallTwoTimes()/}\\
\lstinline/ProxyReturnFutureTest/&\raisebox{0pt}{\lstinline/testFutureEchoCallTwoTimes()/}\\
\lstinline/ProxyReturnFutureTest/&\raisebox{0pt}{\lstinline/testFutureEcho()/}\\
\lstinline/ProxyReturnFutureTest/&\raisebox{0pt}{\lstinline/testFutureEcho()/}\\
\lstinline/AsyncEndpointEventNotifierTest/&\raisebox{0pt}{\lstinline/testAsyncEndpointEventNotifier()/}\\
\lstinline/XmlParseTest/&\raisebox{0pt}{\lstinline/assertChildTo(Stringmessage,ProcessorDefinition<?>route,String)/}\\
\lstinline/XmlParseTest/&\raisebox{0pt}{\lstinline/assertTo(Stringmessage,ProcessorDefinition<?>processor,String)/}\\
\lstinline/GenerateXmlTest/&\raisebox{0pt}{\lstinline/dump(RouteContainercontext)/}\\
\lstinline/GenerateXmFromCamelContextTest/&\raisebox{0pt}{\lstinline/dump(Objectobject)/}\\
\lstinline/PropertyEditorTypeConverterIssueTest/&\raisebox{0pt}{\lstinline/testPropertyEditorTypeConverter()/}\\
\lstinline/ResequencerEngineTest/&\raisebox{0pt}{\lstinline/testReverse(intcapacity)/}\\
\lstinline/SplitterParallelBigFileTest/&\raisebox{0pt}{\lstinline/xxxtestSplitParallelBigFile()/}\\
\lstinline/BeanInfoTest/&\raisebox{0pt}{\lstinline/assertMethodPattern(BeanInfo)/}\\
\lstinline/UnitOfWorkTest/&\raisebox{0pt}{\lstinline/process(Exchange)/}\\
\lstinline/FileConsumePollEnrichFileTest/&\raisebox{0pt}{\lstinline/testPollEnrich()/}\\
\lstinline/FileConsumePollEnrichFileTest/&\raisebox{0pt}{\lstinline/testPollEnrich()/}\\
\lstinline/PollEnrichFileCustomAggregationStrategyTest/&\raisebox{0pt}{\lstinline/PollEnrichCustomAggregationStrategyBody()/}\\
\lstinline/PollEnrichFileCustomAggregationStrategyTest/&\raisebox{0pt}{\lstinline/PollEnrichCustomAggregationStrategyBody()/}\\
\lstinline/PollEnrichFileDefaultAggregationStrategyTest/&\raisebox{0pt}{\lstinline/PollEnrichDefaultAggregationStrategyBody()/}\\
\lstinline/PollEnrichFileDefaultAggregationStrategyTest/&\raisebox{0pt}{\lstinline/PollEnrichDefaultAggregationStrategyBody()/}\\
\lstinline/EndpointCompletionTest/&\raisebox{0pt}{\lstinline/assertParameterJsonSchema(MBeanServer)/}\\
\lstinline/EndpointCompletionTest/&\raisebox{0pt}{\lstinline/assertParameterJsonSchema(MBeanServer)/}\\
\lstinline/EndpointCompletionTest/&\raisebox{0pt}{\lstinline/assertCompletion(MBeanServermbean)/}\\
\lstinline/XPathSplitChoicePerformanceTest/&\raisebox{0pt}{\lstinline/xxTestXPatPerformanceRoute()/}\\
\lstinline/TokenPairIteratorSplitChoicePerformanceTest/&\raisebox{0pt}{\lstinline/xxxtestTokenPairPerformanceRoute()/}\\
\lstinline/TokenPairIteratorSplitChoicePerformanceTest/&\raisebox{0pt}{\lstinline/xxxtestTokenPairPerformanceRoute()/}\\
\lstinline/TokenPairIteratorSplitChoicePerformanceTest/&\raisebox{0pt}{\lstinline/xxxtestTokenPairPerformanceRoute()/}\\
\lstinline/TokenPairIteratorSplitChoicePerformanceTest/&\raisebox{0pt}{\lstinline/xxxtestTokenPairPerformanceRoute()/}\\
\lstinline/TokenPairIteratorSplitChoicePerformanceTest/&\raisebox{0pt}{\lstinline/xxxtestTokenPairPerformanceRoute()/}\\
\lstinline/XPathSplitChoicePerformanceTest/&\raisebox{0pt}{\lstinline/xxTestXPatPerformanceRoute()/}\\
\lstinline/XPathSplitChoicePerformanceTest/&\raisebox{0pt}{\lstinline/xxTestXPatPerformanceRoute()/}\\
\lstinline/XPathSplitChoicePerformanceTest/&\raisebox{0pt}{\lstinline/xxTestXPatPerformanceRoute()/}\\
\lstinline/XPathSplitChoicePerformanceTest/&\raisebox{0pt}{\lstinline/xxTestXPatPerformanceRoute()/}\\
\lstinline/ConfigurationHelperTest/&\raisebox{0pt}{\lstinline/logConfigurationField(EndpointConfig,Fieldfield)/}\\
\lstinline/RedeliveryErrorHandlerNonBlockedRedeliveryHeaderTest/&\raisebox{0pt}{\lstinline/process(Exchange)/}\\
\lstinline/RedeliveryErrorHandlerNonBlockedRedeliveryHeaderTest/&\raisebox{0pt}{\lstinline/process(Exchange)/}\\
\lstinline/RedeliveryOnExceptionBlockedDelayTest/&\raisebox{0pt}{\lstinline/process(Exchange)/}\\
\lstinline/RedeliveryOnExceptionBlockedDelayTest/&\raisebox{0pt}{\lstinline/process(Exchange)/}\\
\lstinline/RedeliveryErrorHandlerBlockedDelayTest/&\raisebox{0pt}{\lstinline/process(Exchange)/}\\
\lstinline/RedeliveryErrorHandlerBlockedDelayTest/&\raisebox{0pt}{\lstinline/process(Exchange)/}\\
\lstinline/RedeliveryErrorHandlerNonBlockedDelayTest/&\raisebox{0pt}{\lstinline/process(Exchange)/}\\
\lstinline/RedeliveryErrorHandlerNonBlockedDelayTest/&\raisebox{0pt}{\lstinline/process(Exchange)/}\\
\lstinline/AsyncEndpointRedeliveryErrorHandlerNonBlockedDelay2Test/&\raisebox{0pt}{\lstinline/process(Exchange)/}\\
\lstinline/AsyncEndpointRedeliveryErrorHandlerNonBlockedDelay2Test/&\raisebox{0pt}{\lstinline/process(Exchange)/}\\
\lstinline/AsyncEndpointRedeliveryErrorHandlerNonBlockedDelayTest/&\raisebox{0pt}{\lstinline/process(Exchange)/}\\
\lstinline/AsyncEndpointRedeliveryErrorHandlerNonBlockedDelayTest/&\raisebox{0pt}{\lstinline/process(Exchange)/}\\
\lstinline/MyAsyncProducer/&\raisebox{0pt}{\lstinline/Objectcall()/}\\
\lstinline/MyAsyncProducer/&\raisebox{0pt}{\lstinline/Objectcall()/}\\
\lstinline/MyAsyncProducer/&\raisebox{0pt}{\lstinline/booleanprocess(Exchange,AsyncCallback)/}\\
\lstinline/RoutePerformanceTest/&\raisebox{0pt}{\lstinline/testPerformance()/}\\
\lstinline/RoutePerformanceCountTest/&\raisebox{0pt}{\lstinline/testSendMessages()/}\\
\lstinline/RouteDirectSuspendResumeTest/&\raisebox{0pt}{\lstinline/testSuspendResume()/}\\
\lstinline/RouteDirectSuspendResumeTest/&\raisebox{0pt}{\lstinline/testSuspendResume()/}\\
\lstinline/DefaultCamelContextSuspendResumeRouteStartupOrderTest/&\raisebox{0pt}{\lstinline/testSuspendResume()/}\\
\lstinline/DefaultCamelContextSuspendResumeRouteStartupOrderTest/&\raisebox{0pt}{\lstinline/testSuspendResume()/}\\
\lstinline/Activator/&\raisebox{0pt}{\lstinline/start(BundleContext)/}\\
\lstinline/Activator/&\raisebox{0pt}{\lstinline/start(BundleContext)/}\\
\lstinline/MySlowFileProcessor/&\raisebox{0pt}{\lstinline/process(Exchange)/}\\
\lstinline/MySlowFileProcessor/&\raisebox{0pt}{\lstinline/process(Exchange)/}\\
\lstinline/DynamicRouterConvertBodyToIssueTest/&\raisebox{0pt}{\lstinline/process(Exchange)/}\\
\lstinline/FileIdempotentRepository/&\raisebox{0pt}{\lstinline/trunkStore()/}\\
\lstinline/ActiveMQUuidGeneratorTest/&\raisebox{0pt}{\lstinline/testPerformance()/}\\
\lstinline/JavaUuidGeneratorTest/&\raisebox{0pt}{\lstinline/testPerformance()/}\\
\lstinline/SimpleUuidGeneratorTest/&\raisebox{0pt}{\lstinline/testPerformance()/}\\
\lstinline/ActiveMQUuidGeneratorTest/&\raisebox{0pt}{\lstinline/testPerformance()/}\\
\lstinline/ActiveMQUuidGeneratorTest/&\raisebox{0pt}{\lstinline/testPerformance()/}\\
\lstinline/SimpleUuidGeneratorTest/&\raisebox{0pt}{\lstinline/testPerformance()/}\\
\lstinline/SimpleUuidGeneratorTest/&\raisebox{0pt}{\lstinline/testPerformance()/}\\
\lstinline/JavaUuidGeneratorTest/&\raisebox{0pt}{\lstinline/testPerformance()/}\\
\lstinline/JavaUuidGeneratorTest/&\raisebox{0pt}{\lstinline/testPerformance()/}\\
\lstinline/BeanPerformanceTest/&\raisebox{0pt}{\lstinline/testBeanPerformance()/}\\
\lstinline/BeanOgnlPerformanceTest/&\raisebox{0pt}{\lstinline/testBeanOgnlPerformance()/}\\
\lstinline/BeanPerformanceTest/&\raisebox{0pt}{\lstinline/testBeanPerformance()/}\\
\lstinline/BeanVsProcessorPerformanceTest/&\raisebox{0pt}{\lstinline/testProcessor()/}\\
\lstinline/BeanVsProcessorPerformanceTest/&\raisebox{0pt}{\lstinline/testBean()/}\\
\lstinline/PrimitiveTypeConverterIssueTest/&\raisebox{0pt}{\lstinline/testPrimitiveTypeConverter()/}\\
\lstinline/AsyncMDCTest/&\raisebox{0pt}{\lstinline/testThreeMessagesMDC()/}\\
\lstinline/AsyncMDCTest/&\raisebox{0pt}{\lstinline/testThreeMessagesMDC()/}\\
\lstinline/AsyncMDCTest/&\raisebox{0pt}{\lstinline/testThreeMessagesMDC()/}\\
\lstinline/RedeliveryDeadLetterErrorHandlerNoRedeliveryOnShutdownTest/&\raisebox{0pt}{\lstinline/testRedeliveryErrorHandlerNoRedeliveryOnShutdown()/}\\
\lstinline/RedeliveryDeadLetterErrorHandlerNoRedeliveryOnShutdownTest/&\raisebox{0pt}{\lstinline/testRedeliveryErrorHandlerNoRedeliveryOnShutdown()/}\\
\lstinline/SSLContextParameters/&\raisebox{0pt}{\lstinline/createSSLContext(CamelContext)/}\\
\lstinline/DefaultShutdownStrategy/&\raisebox{0pt}{\lstinline/doShutdown(CamelContext,long,TimeUnit,boolean,boolean,boolean)/}\\
\lstinline/ManagedBrowsablepointAsXmlTest/&\raisebox{0pt}{\lstinline/testBrowseablepointAsXmlIncludeBody()/}\\
\lstinline/ManagedBrowsablepointAsXmlTest/&\raisebox{0pt}{\lstinline/testBrowseablepointAsXmlIncludeBody()/}\\
\lstinline/ManagedBrowsablepointAsXmlTest/&\raisebox{0pt}{\lstinline/testBrowseablepointAsXmlIncludeBody()/}\\
\lstinline/ManagedBrowsablepointAsXmlTest/&\raisebox{0pt}{\lstinline/testBrowseablepointAsXmlIncludeBody()/}\\
\lstinline/ManagedBrowsablepointAsXmlTest/&\raisebox{0pt}{\lstinline/testBrowseablepointAsXmlIncludeBody()/}\\
\lstinline/ManagedBrowsablepointAsXmlTest/&\raisebox{0pt}{\lstinline/testBrowseablepointAsXmlIncludeBody()/}\\
\lstinline/ManagedBrowsablepointAsXmlTest/&\raisebox{0pt}{\lstinline/testBrowseablepointAsXmlIncludeBody()/}\\
\lstinline/UnitOfWorkTest/&\raisebox{0pt}{\lstinline/testFail()/}\\
\lstinline/UnitOfWorkTest/&\raisebox{0pt}{\lstinline/testException()/}\\
\lstinline/UnitOfWorkTest/&\raisebox{0pt}{\lstinline/testSuccess()/}\\
\lstinline/MDCOnCompletionTest/&\raisebox{0pt}{\lstinline/process(Exchange)/}\\
\lstinline/MyOnCompletion/&\raisebox{0pt}{\lstinline/onDone(Exchange)/}\\
\lstinline/MDCOnCompletionOnCompletionTest/&\raisebox{0pt}{\lstinline/process(Exchange)/}\\
\lstinline/MDCOnCompletionTest/&\raisebox{0pt}{\lstinline/process(Exchange)/}\\
\lstinline/MockEndpoint/&\raisebox{0pt}{\lstinline/assertIsSatisfied(long)/}\\
\lstinline/DataSetEndpoint/&\raisebox{0pt}{\lstinline/doStart()/}\\
\lstinline/DefaultManagementStrategy/&\raisebox{0pt}{\lstinline/setLoadStatisticsEnabled(boolean)/}\\
\lstinline/DefaultManagementStrategy/&\raisebox{0pt}{\lstinline/onlyManageProcessorWithCustomId(boolean)/}\\
\lstinline/DefaultManagementStrategy/&\raisebox{0pt}{\lstinline/setStatisticsLevel(ManagementStatisticsLevel)/}\\
\lstinline/LimitedPollingConsumerPollStrategy/&\raisebox{0pt}{\lstinline/onSuspend(Consumer,Endpoint)/}\\
\lstinline/Activator/&\raisebox{0pt}{\lstinline/stop(BundleContext)/}\\
\lstinline/Activator/&\raisebox{0pt}{\lstinline/stop(BundleContext)/}\\
\lstinline/MyBean/&\raisebox{0pt}{\lstinline/invoke(List<String>strList)/}\\
\lstinline/XPathSplitChoicePerformanceTest/&\raisebox{0pt}{\lstinline/process(Exchange)/}\\
\lstinline/TokenPairIteratorSplitChoicePerformanceTest/&\raisebox{0pt}{\lstinline/process(Exchange)/}\\
\lstinline/SameRouteAndContextScopedErrorHandlerIssueTest/&\raisebox{0pt}{\lstinline/process(Exchange)/}\\
\lstinline/OnExceptionContinuedIssueTest/&\raisebox{0pt}{\lstinline/process(Exchange)/}\\
\lstinline/OnExceptionContinuedIssueTest/&\raisebox{0pt}{\lstinline/process(Exchange)/}\\
\lstinline/ManagedRedeliverTest/&\raisebox{0pt}{\lstinline/process(Exchange)/}\\
\lstinline/ManagedRedeliverRouteOnlyTest/&\raisebox{0pt}{\lstinline/process(Exchange)/}\\
\lstinline/OnExceptionContinuedIssueTest/&\raisebox{0pt}{\lstinline/process(Exchange)/}\\
\lstinline/OnExceptionContinuedIssueTest/&\raisebox{0pt}{\lstinline/process(Exchange)/}\\
\lstinline/TracingWithDelayTest/&\raisebox{0pt}{\lstinline/process(Exchange)/}\\
\lstinline/SetHeaderInDoCatchIssueTest/&\raisebox{0pt}{\lstinline/process(Exchange)/}\\
\lstinline/SetHeaderInDoCatchIssueTest/&\raisebox{0pt}{\lstinline/process(Exchange)/}\\
\lstinline/LoggingEventNotifier/&\raisebox{0pt}{\lstinline/notify(event)/}\\
\lstinline/FileRenameFileOnCommitIssueTest/&\raisebox{0pt}{\lstinline/process(Exchange)/}\\
\lstinline/ContainerWideInterceptor/&\raisebox{0pt}{\lstinline/process(Exchange)/}\\
\lstinline/XsltErrorListener/&\raisebox{0pt}{\lstinline/fatalError(TransformerException)/}\\
\lstinline/XsltErrorListener/&\raisebox{0pt}{\lstinline/error(TransformerException)/}\\
\lstinline/XsltErrorListener/&\raisebox{0pt}{\lstinline/warning(TransformerException)/}\\
\lstinline/XmlErrorListener/&\raisebox{0pt}{\lstinline/fatalError(TransformerException)/}\\
\lstinline/XmlErrorListener/&\raisebox{0pt}{\lstinline/error(TransformerException)/}\\
\lstinline/XmlErrorListener/&\raisebox{0pt}{\lstinline/warning(TransformerException)/}\\
\lstinline/MainLifecycleStrategy/&\raisebox{0pt}{\lstinline/onContextStop(CamelContext)/}\\
\lstinline/MyBean/&\raisebox{0pt}{\lstinline/read(String)/}\\
\lstinline/MyBean/&\raisebox{0pt}{\lstinline/foo(String)/}\\
\lstinline/MyBean/&\raisebox{0pt}{\lstinline/foo(String)/}\\
\lstinline/MyOtherBean/&\raisebox{0pt}{\lstinline/foo(String)/}\\
\lstinline/MyBean/&\raisebox{0pt}{\lstinline/intread(String)/}\\
\lstinline/ManagedManagementStrategyext/&\raisebox{0pt}{\lstinline/doStart()/}\\
\lstinline/DefaultManagementStrategy/&\raisebox{0pt}{\lstinline/doStart()/}\\
\lstinline/UriEndpointConfiguration/&\raisebox{0pt}{\lstinline/warnMissingUriParamOnProperty(String)/}\\
\lstinline/DefaultTypeConverter/&\raisebox{0pt}{\lstinline/doStart()/}\\
\lstinline/DefaultExecutorServiceManagerTest/&\raisebox{0pt}{\lstinline/run()/}\\
\lstinline/DefaultExecutorServiceManagerTest/&\raisebox{0pt}{\lstinline/run()/}\\
\lstinline/HangupInterceptor/&\raisebox{0pt}{\lstinline/run()/}\\
\lstinline/MyBean/&\raisebox{0pt}{\lstinline/myMethod(Stringfoo,intbar,Stringx)/}\\
\lstinline/MyBean/&\raisebox{0pt}{\lstinline/myMethod(HeadersMap,Object)/}\\
\lstinline/MyBean/&\raisebox{0pt}{\lstinline/myMethod(PropertiesMap,HeadersMap)/}\\
\lstinline/MyBean/&\raisebox{0pt}{\lstinline/read(String,String)/}\\
\lstinline/MyBean/&\raisebox{0pt}{\lstinline/read(String,String)/}\\
\lstinline/MyBean/&\raisebox{0pt}{\lstinline/Stringread(String,String)/}\\
\lstinline/MyBean/&\raisebox{0pt}{\lstinline/myMethod(List<User>,Object)/}\\
\lstinline/MyBean/&\raisebox{0pt}{\lstinline/myMethod(List<User>,Object)/}\\
\lstinline/MyRepo/&\raisebox{0pt}{\lstinline/confirm(CamelContext,String)/}\\
\lstinline/FileIdempotentRepositoryReadLockStrategy/&\raisebox{0pt}{\lstinline/prepareOnStartup(GenericFileOperations<File>)/}\\
\lstinline/DefaultExecutorServiceManager/&\raisebox{0pt}{\lstinline/set(ThreadPoolProfile)/}\\
\lstinline/MyPredicate/&\raisebox{0pt}{\lstinline/booleanmatches(Exchange)/}\\
\lstinline/XmlTestSupport/&\raisebox{0pt}{\lstinline/RouteContainerassertParseAsJaxb(String)/}\\
\lstinline/XmlTestSupport/&\raisebox{0pt}{\lstinline/RestContainerassertParseRestAsJaxb(String)/}\\
\lstinline/JndiTest/&\raisebox{0pt}{\lstinline/testLookupOfTypedObject()/}\\
\lstinline/DirectRequestReplyAndSedaInOnlyTest/&\raisebox{0pt}{\lstinline/testInOut()/}\\
\lstinline/AdviceWithTasks/&\raisebox{0pt}{\lstinline/task()/}\\
\lstinline/AdviceWithTasks/&\raisebox{0pt}{\lstinline/task()/}\\
\lstinline/DumpModelAsXmlRouteExpressionTest/&\raisebox{0pt}{\lstinline/testDumpModelAsXmlXPath()/}\\
\lstinline/DumpModelAsXmlRouteExpressionTest/&\raisebox{0pt}{\lstinline/testDumpModelAsXmlHeader()/}\\
\lstinline/DumpModelAsXmlRoutePredicateTest/&\raisebox{0pt}{\lstinline/testDumpModelAsXml()/}\\
\lstinline/DumpModelAsXmlRoutePredicateTest/&\raisebox{0pt}{\lstinline/testDumpModelAsXmlHeader()/}\\
\lstinline/DumpModelAsXmlRoutePredicateTest/&\raisebox{0pt}{\lstinline/testDumpModelAsXmlXPath()/}\\
\lstinline/DumpModelAsXmlRoutePredicateTest/&\raisebox{0pt}{\lstinline/testDumpModelAsXmlBean()/}\\
\lstinline/DumpModelAsXmlRouteExpressionTest/&\raisebox{0pt}{\lstinline/testDumpModelAsXml()/}\\
\lstinline/ComponentDiscoveryTest/&\raisebox{0pt}{\lstinline/ComponentDocumentation()/}\\
\lstinline/DumpModelAsXmlChoiceFilterRouteTest/&\raisebox{0pt}{\lstinline/testDumpModelAsXml()/}\\
\lstinline/DumpModelAsXmlAggregateRouteTest/&\raisebox{0pt}{\lstinline/testDumpModelAsXml()/}\\
\lstinline/DumpModelAsXmlRouteExpressionTest/&\raisebox{0pt}{\lstinline/testDumpModelAsXmlBean()/}\\
\lstinline/DumpModelAsXmlChoiceFilterRouteTest/&\raisebox{0pt}{\lstinline/testDumpModelAsXmAl()/}\\
\lstinline/DumpModelAsXmlPlaceholdersTest/&\raisebox{0pt}{\lstinline/testDumpModelAsXml()/}\\
\lstinline/EipDocumentationTest/&\raisebox{0pt}{\lstinline/testFailOverDocumentation()/}\\
\lstinline/EipDocumentationTest/&\raisebox{0pt}{\lstinline/testSimpleDocumentation()/}\\
\lstinline/StringDataFormatConfigurationAndDocumentationTest/&\raisebox{0pt}{\lstinline/DataFormatJsonSchema()/}\\
\lstinline/SimpleLanguageConfigurationAndDocumentationTest/&\raisebox{0pt}{\lstinline/LanguageJsonSchema()/}\\
\lstinline/EipDocumentationTest/&\raisebox{0pt}{\lstinline/testDocumentation()/}\\
\lstinline/EipDocumentationTest/&\raisebox{0pt}{\lstinline/testSplitDocumentation()/}\\
\lstinline/GenerateXmFromCamelContextTest/&\raisebox{0pt}{\lstinline/testCreateRouteFromCamelContext()/}\\
\lstinline/LogBodyWithNewLineTest/&\raisebox{0pt}{\lstinline/testNoSkip()/}\\
\lstinline/LogBodyWithNewLineTest/&\raisebox{0pt}{\lstinline/testSkip()/}\\
\lstinline/StreamResequencerTest/&\raisebox{0pt}{\lstinline/booleanuseJmx()/}\\

\bottomrule
\end{tabular}
\end{center}

\subsection{Cluster classified as Control Flow Logging}

\begin{center}
\captionof{figure}{LMs in the cluster $\id{C}_{\id{CF},1}$}
\begin{tabular}{ll}\toprule
\multicolumn{1}{c}{Class}&\multicolumn{1}{c}{Method}\\\midrule
\lstinline/CamelLogger/&\raisebox{0pt}{\lstinline/ log(Logger,Marker,String,Throwable)/}\\
\lstinline/CamelLogger/&\raisebox{0pt}{\lstinline/ log(Logger,Marker,String)/}\\
\lstinline/CamelLogger/&\raisebox{0pt}{\lstinline/ log(Logger,String)/}\\
\lstinline/CamelLogger/&\raisebox{0pt}{\lstinline/ log(Logger,Marker,String)/}\\
\lstinline/CamelLogger/&\raisebox{0pt}{\lstinline/ log(Logger,String,Throwable)/}\\
\lstinline/CamelLogger/&\raisebox{0pt}{\lstinline/ log(Logger,String,Throwable)/}\\
\lstinline/CamelLogger/&\raisebox{0pt}{\lstinline/ log(Logger,Marker,String,Throwable)/}\\
\lstinline/CamelLogger/&\raisebox{0pt}{\lstinline/ log(Logger,Marker,String,Throwable)/}\\
\lstinline/CamelLogger/&\raisebox{0pt}{\lstinline/ log(Logger,Marker,String)/}\\
\lstinline/CamelLogger/&\raisebox{0pt}{\lstinline/ log(Logger,String,Throwable)/}\\
\lstinline/CamelLogger/&\raisebox{0pt}{\lstinline/ log(Logger,String)/}\\
\lstinline/CamelLogger/&\raisebox{0pt}{\lstinline/ log(Logger,String)/}\\
\lstinline/CamelLogger/&\raisebox{0pt}{\lstinline/ log(String,Throwable)/}\\
\lstinline/CamelLogger/&\raisebox{0pt}{\lstinline/ booleanprocess(exchange,AsyncCallback)/}\\
\lstinline/CamelLogger/&\raisebox{0pt}{\lstinline/ process(exchange,String)/}\\
\lstinline/CamelLogger/&\raisebox{0pt}{\lstinline/ process(exchange,Throwable)/}\\
\lstinline/CamelLogger/&\raisebox{0pt}{\lstinline/ log(String)/}\\
\lstinline/ApiMethodParser/&\raisebox{0pt}{\lstinline/ intcompare(ApiMethodModel)/}\\
\lstinline/RecoverTask/&\raisebox{0pt}{\lstinline/ run()/}\\
\lstinline/BeanInfo/&\raisebox{0pt}{\lstinline/ createParameterUnmarshalExpressionForAnnotation()/}\\
\lstinline/DefaultExecutorServiceManager/&\raisebox{0pt}{\lstinline/ awaitTermination(ExecutorService)/}\\
\lstinline/DefaultExecutorServiceManager/&\raisebox{0pt}{\lstinline/ doShutdown(ExecutorService,long)/}\\

\bottomrule
\end{tabular}
\end{center}

\subsection{Cluster classified as Exception Try-Block Logging}

\begin{center}
\captionof{figure}{LMs in the cluster $\id{C}_{\id{TB},1}$}
\begin{tabular}{ll}\toprule
\multicolumn{1}{c}{Class}&\multicolumn{1}{c}{Method}\\\midrule
\lstinline/DirectVmProducerBlockingTest/&\raisebox{0pt}{\lstinline/run()/}\\
\lstinline/DirectProducerBlockingTest/&\raisebox{0pt}{\lstinline/run()/}\\
\lstinline/DefaultManagementAgent/&\raisebox{0pt}{\lstinline/run()/}\\
\lstinline/ShutDown/&\raisebox{0pt}{\lstinline/run()/}\\
\lstinline/ShutDown/&\raisebox{0pt}{\lstinline/run()/}\\
\lstinline/DefaultAsyncProcessorAwaitManager/&\raisebox{0pt}{\lstinline/interrupt(Exchange)/}\\
\lstinline/ShutdownTask/&\raisebox{0pt}{\lstinline/run()/}\\

\bottomrule
\end{tabular}
\end{center}

\section{Solr}\label{solr}

\subsection{Cluster classified as Exception Catch-Block Logging}

\begin{center}
\captionof{figure}{LMs in the cluster $\id{S}_{\id{CB},1}$}
\begin{tabular}{ll}\toprule
\multicolumn{1}{c}{Class}&\multicolumn{1}{c}{Method}\\\midrule
\lstinline/ ChangedSchemaMergeTest/&\raisebox{0pt}{\lstinline/   optimizeDiffSchemas()/}\\ 
\lstinline/ HealthcheckTool/&\raisebox{0pt}{\lstinline/   runCloudTool(CloudSolrClient)/}\\ 
\lstinline/ HealthcheckTool/&\raisebox{0pt}{\lstinline/   runCloudTool(CloudSolrClient)/}\\ 
\lstinline/ SolrDispatchFilter/&\raisebox{0pt}{\lstinline/   authenticateRequest())/}\\ 
\lstinline/ TestStressUserVersions/&\raisebox{0pt}{\lstinline/   run()/}\\ 
\lstinline/ TestStressUserVersions/&\raisebox{0pt}{\lstinline/   run()/}\\ 
\lstinline/ BootstrapStatusRunnable/&\raisebox{0pt}{\lstinline/  sendBootstrapCommand()/}\\ 
\lstinline/ BootstrapStatusRunnable/&\raisebox{0pt}{\lstinline/   sendBootstrapCommand()/}\\ 
\lstinline/ BootstrapStatusRunnable/&\raisebox{0pt}{\lstinline/   sendBootstrapCommand()/}\\ 
\lstinline/ BootstrapStatusRunnable/&\raisebox{0pt}{\lstinline/   sendBootstrapCommand()/}\\ 
\lstinline/ SplitOp/&\raisebox{0pt}{\lstinline/   execute(CoreAdminHandler.CallInfoit)/}\\ 
\lstinline/ FileFetcher/&\raisebox{0pt}{\lstinline/   cleanup()/}\\ 
\lstinline/ FileFetcher/&\raisebox{0pt}{\lstinline/   cleanup()/}\\ 
\lstinline/ TestStressReorder/&\raisebox{0pt}{\lstinline/   run()/}\\ 
\lstinline/ MergeIndexesOp/&\raisebox{0pt}{\lstinline/   execute(CoreAdminHandler.CallInfoit)/}\\ 
\lstinline/ LocalFsFileStream/&\raisebox{0pt}{\lstinline/   write(out)/}\\ 
\lstinline/ ZkContainer/&\raisebox{0pt}{\lstinline/   publishCoresAsDown(List<SolrCore>)/}\\ 
\lstinline/ ZkContainer/&\raisebox{0pt}{\lstinline/   publishCoresAsDown(List<SolrCore>)/}\\ 
\lstinline/ ZkContainer/&\raisebox{0pt}{\lstinline/   publishCoresAsDown(List<SolrCore>)/}\\ 
\lstinline/ SyncShardRequest/&\raisebox{0pt}{\lstinline/   handleUpdates(ShardResponse)/}\\ 
\lstinline/ SyncShardRequest/&\raisebox{0pt}{\lstinline/   handleUpdates(ShardResponse)/}\\ 
\lstinline/ CoreContainer/&\raisebox{0pt}{\lstinline/   swap(String,String)/}\\ 
\lstinline/ CoreContainer/&\raisebox{0pt}{\lstinline/   swap(String,String)/}\\ 
\lstinline/ MoreLikeThisHandler/&\raisebox{0pt}{\lstinline/   handleRequestBody(SolrQueryRequestr,SolrQueryResponser)/}\\ 
\lstinline/ LocalFsFileStream/&\raisebox{0pt}{\lstinline/   write(out)/}\\ 
\lstinline/ LocalFsFileStream/&\raisebox{0pt}{\lstinline/   write(out)/}\\ 
\lstinline/ ManagedIndexSchema/&\raisebox{0pt}{\lstinline/   newFieldType(StringtypeName,StringclassName)/}\\ 
\lstinline/ ZkController/&\raisebox{0pt}{\lstinline/   publishNodeAsDown(String)/}\\ 
\lstinline/ BootstrapStatusRunnable/&\raisebox{0pt}{\lstinline/   sendBootstrapCommand()/}\\ 
\lstinline/ BootstrapStatusRunnable/&\raisebox{0pt}{\lstinline/   sendBootstrapCommand()/}\\ 
\lstinline/ ManagedIndexSchemaFactoryFactory/&\raisebox{0pt}{\lstinline/   zkUgradeToManagedSchema()/}\\ 
\lstinline/ HttpSolrCall/&\raisebox{0pt}{\lstinline/   writeResponse(SolrQueryResponse,QueryResponseWriter,Method)/}\\ 
\lstinline/ ZkController/&\raisebox{0pt}{\lstinline/   publishNodeAsDown(String)/}\\ 
\lstinline/ SystemIdResolver2/&\raisebox{0pt}{\lstinline/   solveEntity(Stringname,StringId,String)/}\\ 
\lstinline/ ShardSplitTest/&\raisebox{0pt}{\lstinline/   logDebugHelp(int[]))/}\\ 
\lstinline/ PKIAuthenticationPlugin/&\raisebox{0pt}{\lstinline/   PublicKeygetRemotePublicKey(String)/}\\ 
\lstinline/ RestoreCore/&\raisebox{0pt}{\lstinline/   booleandoRestore()/}\\ 
\lstinline/ ZKPrinter/&\raisebox{0pt}{\lstinline/   booleanprintZnode(J)/}\\ 
\lstinline/ ZKPrinter/&\raisebox{0pt}{\lstinline/   printZnode(JSONWriter)/}\\ 
\lstinline/ ConfigSetDownloadTool/&\raisebox{0pt}{\lstinline/   runImpl(CommandLine)/}\\ 
\lstinline/ ConfigSetDownloadTool/&\raisebox{0pt}{\lstinline/   runImpl(CommandLine)/}\\ 
\lstinline/ FileFetcher/&\raisebox{0pt}{\lstinline/   cleanup()/}\\ 
\lstinline/ DeleteExpiredDocsRunnable/&\raisebox{0pt}{\lstinline/   run()/}\\ 
\lstinline/ DeleteExpiredDocsRunnable/&\raisebox{0pt}{\lstinline/   run()/}\\ 
\lstinline/ TransformerProvider/&\raisebox{0pt}{\lstinline/   getTemplates()/}\\ 
\lstinline/ URLClassifyProcessor/&\raisebox{0pt}{\lstinline/   processAdd(AddUpdateCommand)/}\\ 
\lstinline/ Req/&\raisebox{0pt}{\lstinline/   trackRequestResult(HttpResponser,boolean)/}\\ 
\lstinline/ Req/&\raisebox{0pt}{\lstinline/   trackRequestResult(HttpResponser,boolean)/}\\ 
\lstinline/ ZkController/&\raisebox{0pt}{\lstinline/   publishNodeAsDown(String)/}\\ 
\lstinline/ ZkController/&\raisebox{0pt}{\lstinline/   publishNodeAsDown(String)/}\\ 
\lstinline/ AddSchemaFieldsUpdateProcessor/&\raisebox{0pt}{\lstinline/   processAdd(AddUpdateCommand)/}\\ 
\lstinline/ LukeRequestHandler/&\raisebox{0pt}{\lstinline/   longgetFileLength(Directory,String)/}\\ 
\lstinline/ LeaderInitiatedRecoveryThread/&\raisebox{0pt}{\lstinline/   publishDownState()/}\\ 
\lstinline/ TestReloadDeadlock/&\raisebox{0pt}{\lstinline/   run()/}\\ 
\lstinline/ RetryNode/&\raisebox{0pt}{\lstinline/   checkRetry()/}\\ 
\lstinline/ DirectoryFileStream/&\raisebox{0pt}{\lstinline/   write(out)/}\\ 
\lstinline/ DirectoryFileStream/&\raisebox{0pt}{\lstinline/   write(out)/}\\ 
\lstinline/ RecentUpdates/&\raisebox{0pt}{\lstinline/   update()/}\\ 
\lstinline/ RecentUpdates/&\raisebox{0pt}{\lstinline/   update()/}\\ 
\lstinline/ FileFetcher/&\raisebox{0pt}{\lstinline/   cleanup()/}\\ 
\lstinline/ FileFetcher/&\raisebox{0pt}{\lstinline/   cleanup()/}\\ 
\lstinline/ FileFloatSource/&\raisebox{0pt}{\lstinline/   getFloats(FileFloatSource,IndexReader)/}\\ 
\lstinline/ TestStressReorder/&\raisebox{0pt}{\lstinline/   run()/}\\ 
\lstinline/ ManagedIndexSchema/&\raisebox{0pt}{\lstinline/   newFieldType(String,String)/}\\ 
\lstinline/ SolrConfigHandler/&\raisebox{0pt}{\lstinline/   waitForAllReplicasState()/}\\ 
\lstinline/ ExactStatsCache/&\raisebox{0pt}{\lstinline/   returnLocalStats(ResponseBuild)/}\\ 
\lstinline/ FileBasedSpellChecker/&\raisebox{0pt}{\lstinline/   loadExternalFileDictionary(SolrCore)/}\\ 
\lstinline/ ZkController/&\raisebox{0pt}{\lstinline/   publishNodeAsDown(String)/}\\ 
\lstinline/ ZkController/&\raisebox{0pt}{\lstinline/   publishNodeAsDown(String)/}\\ 
\lstinline/ ZkController/&\raisebox{0pt}{\lstinline/   publishNodeAsDown(String)/}\\ 
\lstinline/ ManagedIndexSchema/&\raisebox{0pt}{\lstinline/  enewFieldType(String,String,Map<String,?>)/}\\ 
\lstinline/ ZkController/&\raisebox{0pt}{\lstinline/   publishNodeAsDown(String)/}\\ 
\lstinline/ ZkController/&\raisebox{0pt}{\lstinline/   publishNodeAsDown(String)/}\\ 
\lstinline/ UpdateLogSynchronisation/&\raisebox{0pt}{\lstinline/   run()/}\\ 
\lstinline/ GetDirThread/&\raisebox{0pt}{\lstinline/   run()/}\\ 
\lstinline/ RecentUpdates/&\raisebox{0pt}{\lstinline/   update()/}\\ 
\lstinline/ RecentUpdates/&\raisebox{0pt}{\lstinline/   update()/}\\ 
\lstinline/ RecentUpdates/&\raisebox{0pt}{\lstinline/   update()/}\\ 
\lstinline/ RecentUpdates/&\raisebox{0pt}{\lstinline/   update()/}\\ 
\lstinline/ Runner/&\raisebox{0pt}{\lstinline/   resetTaskWithException(OverseerMessageHandler)/}\\ 
\lstinline/ Runner/&\raisebox{0pt}{\lstinline/   resetTaskWithException(OverseerMessageHandler)/}\\ 
\lstinline/ ZkCpTool/&\raisebox{0pt}{\lstinline/   runImpl(CommandLine)/}\\ 
\lstinline/ ZkCpTool/&\raisebox{0pt}{\lstinline/   runImpl(CommandLine)/}\\ 
\lstinline/ ZkMvTool/&\raisebox{0pt}{\lstinline/   runImpl(CommandLine)/}\\ 
\lstinline/ ZkMvTool/&\raisebox{0pt}{\lstinline/   runImpl(CommandLine)/}\\ 
\lstinline/ SpellCheckerListenerr/&\raisebox{0pt}{\lstinline/   buildSpellIndex(SolrIndexSearch)/}\\ 
\lstinline/ SpellCheckerListenerr/&\raisebox{0pt}{\lstinline/   buildSpellIndex(SolrIndexSearch)/}\\ 
\lstinline/ LocalFsFileStreamext/&\raisebox{0pt}{\lstinline/   write(out)/}\\ 
\lstinline/ RegexpBoostProcessor/&\raisebox{0pt}{\lstinline/  initBoostEntries(InputStream)/}\\ 
\lstinline/ Command/&\raisebox{0pt}{\lstinline/   handleCommands(List<CommandOperation>)/}\\ 
\lstinline/ Command/&\raisebox{0pt}{\lstinline/   handleCommands(List<CommandOperation>)/}\\ 
\lstinline/ Worker/&\raisebox{0pt}{\lstinline/   run()/}\\ 
\lstinline/ Worker/&\raisebox{0pt}{\lstinline/   run()/}\\ 
\lstinline/ SearchHandler/&\raisebox{0pt}{\lstinline/   handleRequestBody()/}\\ 
\lstinline/ RecoveryStrategy/&\raisebox{0pt}{\lstinline/   replay(SolrCore)/}\\ 
\lstinline/ RecoveryStrategy/&\raisebox{0pt}{\lstinline/   replay(SolrCore)/}\\ 
\lstinline/ CommitVersionInfo/&\raisebox{0pt}{\lstinline/   CommitVersionInfobuild(IndexCommit)/}\\ 
\lstinline/ CommitVersionInfo/&\raisebox{0pt}{\lstinline/   CommitVersionInfobuild(IndexCommit)/}\\ 
\lstinline/ ZkIndexSchemaReader/&\raisebox{0pt}{\lstinline/   command()/}\\ 
\lstinline/ ZkIndexSchemaReader/&\raisebox{0pt}{\lstinline/   command()/}\\ 
\lstinline/ OverseerElectionContext/&\raisebox{0pt}{\lstinline/   runLeaderProcess(boolean)/}\\ 
\lstinline/ RetryNode/&\raisebox{0pt}{\lstinline/   booleancheckRetry()/}\\ 
\lstinline/ IncRefThread/&\raisebox{0pt}{\lstinline/   run()/}\\ 
\lstinline/ CoreContainer/&\raisebox{0pt}{\lstinline/   swap(String,String)/}\\ 
\lstinline/ BootstrapCallable/&\raisebox{0pt}{\lstinline/   Booleancall()/}\\ 
\lstinline/ MultiThreadedOCPTest/&\raisebox{0pt}{\lstinline/   testLongAndShortRunningParallelApiCalls()/}\\ 
\lstinline/ ShardSplitTest/&\raisebox{0pt}{\lstinline/   logDebugHelp(int[]))/}\\ 
\lstinline/ HttpSolrCall/&\raisebox{0pt}{\lstinline/   writeResponse()/}\\ 
\lstinline/ SolrDynamicMBean/&\raisebox{0pt}{\lstinline/   getAttributes(String[])/}\\ 
\lstinline/ SolrDynamicMBean/&\raisebox{0pt}{\lstinline/   getAttributes(String[])/}\\ 
\lstinline/ SystemInfoHandler/&\raisebox{0pt}{\lstinline/   getJvmInfo()/}\\ 
\lstinline/ RunExampleExecutor/&\raisebox{0pt}{\lstinline/   close()/}\\ 
\lstinline/ RunExampleExecutor/&\raisebox{0pt}{\lstinline/   close()/}\\ 
\lstinline/ HdfsTestUtil/&\raisebox{0pt}{\lstinline/   teardownClass(MiniDFSClusterdfsCluster)/}\\ 
\lstinline/ ManagedResource/&\raisebox{0pt}{\lstinline/   doPut(BaseSolrResource)/}\\ 
\lstinline/ Suggester/&\raisebox{0pt}{\lstinline/   SgetSuggestions(SpellingOptions)/}\\ 
\lstinline/ SolrSuggester/&\raisebox{0pt}{\lstinline/   getSuggestions(SuggesterOptions)/}\\ 
\lstinline/ ClientThread/&\raisebox{0pt}{\lstinline/   run()/}\\ 
\lstinline/ MultiThreadedOCPTest/&\raisebox{0pt}{\lstinline/   testLongAndShortRunningParallelApiCalls()/}\\ 
\lstinline/ Diagnostics/&\raisebox{0pt}{\lstinline/   logThreadDumps(Stringmessage)/}\\ 
\lstinline/ SolrZkServer/&\raisebox{0pt}{\lstinline/   start()/}\\ 
\lstinline/ ZkController/&\raisebox{0pt}{\lstinline/   publishNodeAsDown(String)/}\\ 
\lstinline/ ConfigSetsAPIThread/&\raisebox{0pt}{\lstinline/   run()/}\\ 
\lstinline/ ZkController/&\raisebox{0pt}{\lstinline/   publishNodeAsDown(String)/}\\ 
\lstinline/ ZkController/&\raisebox{0pt}{\lstinline/   publishNodeAsDown(String)/}\\ 
\lstinline/ ZkController/&\raisebox{0pt}{\lstinline/   publishNodeAsDown(String)/}\\ 
\lstinline/ HdfsTransactionLog/&\raisebox{0pt}{\lstinline/   closeOutput()/}\\ 
\lstinline/ BaseCdcrDistributedZkTest/&\raisebox{0pt}{\lstinline/   waitForReplicationToComplete(String)/}\\ 
\lstinline/ BootstrapStatusRunnable/&\raisebox{0pt}{\lstinline/   sendBootstrapCommand()/}\\ 
\lstinline/ RestManager/&\raisebox{0pt}{\lstinline/   attachManagedResource(ManagedResourcere)/}\\ 
\lstinline/ UpdateLog/&\raisebox{0pt}{\lstinline/   seedBucketsWithHighestVersion(SolrIndex)/}\\ 
\lstinline/ MMapDirectoryFactory/&\raisebox{0pt}{\lstinline/   Directorycreate(String)/}\\ 
\lstinline/ FileFetcher/&\raisebox{0pt}{\lstinline/   cleanup()/}\\ 
\lstinline/ FileFetcher/&\raisebox{0pt}{\lstinline/   cleanup()/}\\ 
\lstinline/ FileFetcher/&\raisebox{0pt}{\lstinline/   cleanup()/}\\ 
\lstinline/ FileFetcher/&\raisebox{0pt}{\lstinline/   cleanup()/}\\ 
\lstinline/ FileFetcher/&\raisebox{0pt}{\lstinline/   cleanup()/}\\ 
\lstinline/ CoreContainer/&\raisebox{0pt}{\lstinline/   swap(String,String)/}\\ 
\lstinline/ CoreContainer/&\raisebox{0pt}{\lstinline/   swap(String,String)/}\\ 
\lstinline/ CdcrReplicatorState/&\raisebox{0pt}{\lstinline/   shutdown()/}\\ 
\lstinline/ UpdateLogSynchronisation/&\raisebox{0pt}{\lstinline/   run()/}\\ 
\lstinline/ DefaultSolrCoreState/&\raisebox{0pt}{\lstinline/   run()/}\\ 
\lstinline/ HttpPartitionTest/&\raisebox{0pt}{\lstinline/   waitToSeeReplicasActive(String)/}\\ 
\lstinline/ SolrDynamicMBean/&\raisebox{0pt}{\lstinline/   getAttributes(String[])/}\\ 
\lstinline/ SolrDynamicMBean/&\raisebox{0pt}{\lstinline/   getAttributes(String[])/}\\ 
\lstinline/ RuleBasedAuthorizationPlugin/&\raisebox{0pt}{\lstinline/   init(Map<String,Object>)/}\\ 
\lstinline/ Config/&\raisebox{0pt}{\lstinline/   parseLuceneVersionString(StringmatchVersion)/}\\ 
\lstinline/ CommitVersionInfo/&\raisebox{0pt}{\lstinline/   fobuild(IndexCommit)/}\\ 
\lstinline/ CommitVersionInfo/&\raisebox{0pt}{\lstinline/   fobuild(IndexCommit)/}\\ 
\lstinline/ Resolver/&\raisebox{0pt}{\lstinline/   resolve(Objecto,JavaBinCodeccodec)/}\\ 
\lstinline/ RestoreCore/&\raisebox{0pt}{\lstinline/   booleandoRestore()/}\\ 
\lstinline/ ZkIndexSchemaReader/&\raisebox{0pt}{\lstinline/   command()/}\\ 
\lstinline/ RuntimeLib/&\raisebox{0pt}{\lstinline/   verify()/}\\ 
\lstinline/ BootstrapStatusRunnable/&\raisebox{0pt}{\lstinline/   sendBootstrapCommand()/}\\ 
\lstinline/ BootstrapStatusRunnable/&\raisebox{0pt}{\lstinline/   sendBootstrapCommand()/}\\ 
\lstinline/ Suggester/&\raisebox{0pt}{\lstinline/   getSuggestions(SpellingOptionsoptions)/}\\ 
\lstinline/ SolrSuggester/&\raisebox{0pt}{\lstinline/   getSuggestions(SuggesterOptionsoptions)/}\\ 
\lstinline/ ClusterStateUpdater/&\raisebox{0pt}{\lstinline/  amILeader()/}\\ 
\lstinline/ ClusterStateUpdater/&\raisebox{0pt}{\lstinline/   amILeader()/}\\ 
\lstinline/ PKIAuthenticationPlugin/&\raisebox{0pt}{\lstinline/   getRemotePublicKey(String)/}\\ 
\lstinline/ ShardSplitTest/&\raisebox{0pt}{\lstinline/   logDebugHelp(int[]))/}\\ 
\lstinline/ StatsUtil/&\raisebox{0pt}{\lstinline/   termFromString(Stringdata)/}\\ 
\lstinline/ RunExampleExecutor/&\raisebox{0pt}{\lstinline/   close()/}\\ 
\lstinline/ SolrIndexWriter/&\raisebox{0pt}{\lstinline/   rollback()/}\\ 
\lstinline/ SolrIndexWriter/&\raisebox{0pt}{\lstinline/   rollback()/}\\ 
\lstinline/ ManagedResource/&\raisebox{0pt}{\lstinline/   doPut()/}\\ 


\bottomrule
\end{tabular}
\end{center}


\subsection{Cluster classified as Outer Method Logging}

\begin{center}
\captionof{figure}{LMs in the cluster $\id{S}_{\id{OM},1}$}
\begin{tabular}{ll}\toprule
\multicolumn{1}{c}{Class}&\multicolumn{1}{c}{Method}\\\midrule
\lstinline/TestDistribDocBasedVersion/&\raisebox{0pt}{\lstinline/vdeleteFail(String)/}\\
\lstinline/TestDistribDocBasedVersion/&\raisebox{0pt}{\lstinline/vdeleteFail(String)/}\\
\lstinline/ManagedIndexSchema/&\raisebox{0pt}{\lstinline/new()/}\\
\lstinline/ManagedIndexSchema/&\raisebox{0pt}{\lstinline/new()/}\\
\lstinline/TestMiniSolrCloudClusterSSL/&\raisebox{0pt}{\lstinline/before()/}\\
\lstinline/BackupCmdHttpPartitionTest/&\raisebox{0pt}{\lstinline/call(ClusterState)/}\\
\lstinline/BackupCmdHttpPartitionTest/&\raisebox{0pt}{\lstinline/call(ClusterState)/}\\
\lstinline/BackupCmdHttpPartitionTest/&\raisebox{0pt}{\lstinline/call(ClusterState)/}\\
\lstinline/ActionThrottle/&\raisebox{0pt}{\lstinline/minimumWaitBetweenActions()/}\\
\lstinline/OverriddenZkACLAndCredentialsProvidersTest/&\raisebox{0pt}{\lstinline/setUp()/}\\
\lstinline/OverriddenZkACLAndCredentialsProvidersTest/&\raisebox{0pt}{\lstinline/setUp()/}\\
\lstinline/OverriddenZkACLAndCredentialsProvidersTest/&\raisebox{0pt}{\lstinline/setUp()/}\\
\lstinline/ClusterStateUpdater/&\raisebox{0pt}{\lstinline/LeaderStatusamILeader()/}\\
\lstinline/Overseer/&\raisebox{0pt}{\lstinline/createOverseerNode(SolrZkClientzkClient)/}\\
\lstinline/Overseer/&\raisebox{0pt}{\lstinline/createOverseerNode(SolrZkClientzkClient)/}\\
\lstinline/AutoCommitTest/&\raisebox{0pt}{\lstinline/verbose(Object)/}\\
\lstinline/TestWriterPerf/&\raisebox{0pt}{\lstinline/doPerf()/}\\
\lstinline/SolrXmlInZkTest/&\raisebox{0pt}{\lstinline/setUpZkAndDiskXml(boolean)/}\\
\lstinline/SolrXmlInZkTest/&\raisebox{0pt}{\lstinline/setUpZkAndDiskXml(boolean)/}\\
\lstinline/LeaderFailureAfterFreshStartTest/&\raisebox{0pt}{\lstinline/waitForNewLeader()/}\\
\lstinline/BufferStateWatcher/&\raisebox{0pt}{\lstinline/process(WatchedEvent)/}\\
\lstinline/LeaderElectionIntegrationTest/&\raisebox{0pt}{\lstinline/setUp()/}\\
\lstinline/LeaderElectionIntegrationTest/&\raisebox{0pt}{\lstinline/setUp()/}\\
\lstinline/Diagnostics/&\raisebox{0pt}{\lstinline/logThreadDumps(String)/}\\
\lstinline/BufferStateWatcher/&\raisebox{0pt}{\lstinline/process(WatchedEvent)/}\\
\lstinline/ZkCLITest/&\raisebox{0pt}{\lstinline/setUp()/}\\
\lstinline/ZkCLITest/&\raisebox{0pt}{\lstinline/setUp()/}\\
\lstinline/ZkCLITest/&\raisebox{0pt}{\lstinline/setUp()/}\\
\lstinline/TestIndexingPerformance/&\raisebox{0pt}{\lstinline/testIndexingPerf()/}\\
\lstinline/TestIndexingPerformance/&\raisebox{0pt}{\lstinline/testIndexingPerf()/}\\
\lstinline/HdfsDirectory/&\raisebox{0pt}{\lstinline/close()/}\\
\lstinline/VMParamsZkACLAndCredentialsProvidersTest/&\raisebox{0pt}{\lstinline/setUp()/}\\
\lstinline/VMParamsZkACLAndCredentialsProvidersTest/&\raisebox{0pt}{\lstinline/setUp()/}\\
\lstinline/VMParamsZkACLAndCredentialsProvidersTest/&\raisebox{0pt}{\lstinline/setUp()/}\\
\lstinline/FileStorageIO/&\raisebox{0pt}{\lstinline/configure(SolrResourceLoader)/}\\
\lstinline/ZooKeeperStorageIO/&\raisebox{0pt}{\lstinline/delete(String)/}\\
\lstinline/JsonStoragStorage/&\raisebox{0pt}{\lstinline/store(String)/}\\
\lstinline/ManagedResourceStorage/&\raisebox{0pt}{\lstinline/Objectload(StringresourceId)/}\\
\lstinline/JaspellLookupFactory/&\raisebox{0pt}{\lstinline/Lookupcreate(NamedList)/}\\
\lstinline/LeaderElectionTest/&\raisebox{0pt}{\lstinline/ParallelElection()/}\\
\lstinline/SolrSnapshotMetaDataManager/&\raisebox{0pt}{\lstinline/release(String)/}\\
\lstinline/SolrSnapshotMetaDataManager/&\raisebox{0pt}{\lstinline/release(String)/}\\
\lstinline/TestJmxIntegration/&\raisebox{0pt}{\lstinline/JmxUpdate()/}\\
\lstinline/ZkIndexSchemaReader/&\raisebox{0pt}{\lstinline/command()/}\\
\lstinline/ZkIndexSchemaReader/&\raisebox{0pt}{\lstinline/command()/}\\
\lstinline/QueryElevationComponentTest/&\raisebox{0pt}{\lstinline/writeFile(File)/}\\
\lstinline/HttpPartitionTest/&\raisebox{0pt}{\lstinline/waitToSeeReplicasActive(String)/}\\
\lstinline/HttpPartitionTest/&\raisebox{0pt}{\lstinline/waitToSeeReplicasActive(String)/}\\
\lstinline/HttpPartitionTest/&\raisebox{0pt}{\lstinline/waitToSeeReplicasActive(String)/}\\
\lstinline/SolrIndexSearcher/&\raisebox{0pt}{\lstinline/getStoredHighlightFieldNames()/}\\
\lstinline/OutOfBoxZkACLAndCredentialsProvidersTest/&\raisebox{0pt}{\lstinline/assertOpenACLUnsafeAllover()/}\\
\lstinline/OutOfBoxZkACLAndCredentialsProvidersTest/&\raisebox{0pt}{\lstinline/assertOpenACLUnsafeAllover()/}\\
\lstinline/OutOfBoxZkACLAndCredentialsProvidersTest/&\raisebox{0pt}{\lstinline/assertOpenACLUnsafeAllover()/}\\
\lstinline/BasicAuthIntegrationTeste/&\raisebox{0pt}{\lstinline/setBasicAuthHeader(String)/}\\
\lstinline/RequestSyncShardOp/&\raisebox{0pt}{\lstinline/execute(CallInfoit)/}\\
\lstinline/TriLevelCompositeIdRoutingTest/&\raisebox{0pt}{\lstinline/doTriLevelHashingTestWithBitMask()/}\\
\lstinline/TriLevelCompositeIdRoutingTest/&\raisebox{0pt}{\lstinline/doTriLevelHashingTestWithBitMask()/}\\
\lstinline/CreateShardCmd/&\raisebox{0pt}{\lstinline/call(ClusterState)/}\\
\lstinline/CreateShardCmd/&\raisebox{0pt}{\lstinline/call(ClusterState)/}\\
\lstinline/LogReplayer/&\raisebox{0pt}{\lstinline/run()/}\\
\lstinline/MemClassLoader/&\raisebox{0pt}{\lstinline/loadFromRuntimeLibs(String)/}\\
\lstinline/BlobHandler/&\raisebox{0pt}{\lstinline/indexMap()/}\\
\lstinline/BlobHandler/&\raisebox{0pt}{\lstinline/indexMap()/}\\
\lstinline/MockAuthenticationPlugin/&\raisebox{0pt}{\lstinline/doAuthenticate(ServletRequest)/}\\
\lstinline/MockAuthenticationPlugin/&\raisebox{0pt}{\lstinline/doAuthenticate(ServletRequest)/}\\
\lstinline/MockAuthenticationPlugin/&\raisebox{0pt}{\lstinline/doAuthenticate(ServletRequest)/}\\
\lstinline/ShardLeaderElectionContext/&\raisebox{0pt}{\lstinline/rejoinLeaderElection(SolrCorecore)/}\\
\lstinline/AddBlockUpdateTest/&\raisebox{0pt}{\lstinline/ExceptionThrown()/}\\
\lstinline/CollectionsHandler/&\raisebox{0pt}{\lstinline/waitForActiveCollection(String)/}\\
\lstinline/CollectionsHandler/&\raisebox{0pt}{\lstinline/waitForActiveCollection(String)/}\\
\lstinline/LeaderInitiatedRecoveryOnShardRestartTest/&\raisebox{0pt}{\lstinline/RestartWithAllInLIR()/}\\
\lstinline/RestoreCmdHttpPartitionTest/&\raisebox{0pt}{\lstinline/call(ClusterState)/}\\
\lstinline/RestoreCmdHttpPartitionTest/&\raisebox{0pt}{\lstinline/call(ClusterState)/}\\
\lstinline/BlockDirectory/&\raisebox{0pt}{\lstinline/closeOnShutdown()/}\\
\lstinline/DistribJoinFromCollectionTeste/&\raisebox{0pt}{\lstinline/Classshutdown()/}\\
\lstinline/DistribJoinFromCollectionTeste/&\raisebox{0pt}{\lstinline/Classshutdown()/}\\
\lstinline/SolrSuggester/&\raisebox{0pt}{\lstinline/SuggesterResultgetSuggestions(SuggesterOptions)/}\\
\lstinline/SolrSuggester/&\raisebox{0pt}{\lstinline/SuggesterResultgetSuggestions(SuggesterOptions)/}\\
\lstinline/SolrSuggester/&\raisebox{0pt}{\lstinline/SuggesterResultgetSuggestions(SuggesterOptions)/}\\
\lstinline/SolrSuggester/&\raisebox{0pt}{\lstinline/SuggesterResultgetSuggestions(SuggesterOptions)/}\\
\lstinline/RulesTeste/&\raisebox{0pt}{\lstinline/ModifyColl()/}\\
\lstinline/LeaderInitiatedRecoveryOnCommitTest/&\raisebox{0pt}{\lstinline/oneShardTest()/}\\
\lstinline/LeaderInitiatedRecoveryOnCommitTest/&\raisebox{0pt}{\lstinline/oneShardTest()/}\\
\lstinline/LeaderInitiatedRecoveryOnCommitTest/&\raisebox{0pt}{\lstinline/oneShardTest()/}\\
\lstinline/LeaderInitiatedRecoveryOnCommitTest/&\raisebox{0pt}{\lstinline/oneShardTest()/}\\
\lstinline/LeaderInitiatedRecoveryOnCommitTest/&\raisebox{0pt}{\lstinline/oneShardTest()/}\\
\lstinline/LeaderInitiatedRecoveryOnCommitTest/&\raisebox{0pt}{\lstinline/oneShardTest()/}\\
\lstinline/LeaderInitiatedRecoveryOnCommitTest/&\raisebox{0pt}{\lstinline/oneShardTest()/}\\
\lstinline/LeaderInitiatedRecoveryOnCommitTest/&\raisebox{0pt}{\lstinline/oneShardTest()/}\\
\lstinline/LeaderInitiatedRecoveryOnCommitTest/&\raisebox{0pt}{\lstinline/oneShardTest()/}\\
\lstinline/LeaderInitiatedRecoveryOnCommitTest/&\raisebox{0pt}{\lstinline/oneShardTest()/}\\
\lstinline/SplitOp/&\raisebox{0pt}{\lstinline/execute(CoreAdminHandler.CallInfoit)/}\\
\lstinline/OneIndexer/&\raisebox{0pt}{\lstinline/run()/}\\
\lstinline/OneIndexer/&\raisebox{0pt}{\lstinline/run()/}\\
\lstinline/OverseerRolesTest/&\raisebox{0pt}{\lstinline/setOverseerRole(CloudSolr)/}\\
\lstinline/OverseerRolesTest/&\raisebox{0pt}{\lstinline/setOverseerRole(CloudSolr)/}\\
\lstinline/OverseerRolesTest/&\raisebox{0pt}{\lstinline/setOverseerRole(CloudSolr)/}\\
\lstinline/OverseerRolesTest/&\raisebox{0pt}{\lstinline/setOverseerRole(CloudSolr)/}\\
\lstinline/OverseerRolesTest/&\raisebox{0pt}{\lstinline/setOverseerRole(CloudSolr)/}\\
\lstinline/OverseerRolesTest/&\raisebox{0pt}{\lstinline/setOverseerRole(CloudSolr)/}\\
\lstinline/OverseerRolesTest/&\raisebox{0pt}{\lstinline/setOverseerRole(CloudSolr)/}\\
\lstinline/OverseerRolesTest/&\raisebox{0pt}{\lstinline/setOverseerRole(CloudSolr)/}\\
\lstinline/OverseerRolesTest/&\raisebox{0pt}{\lstinline/setOverseerRole(CloudSolr)/}\\
\lstinline/ProcessStateWatcher/&\raisebox{0pt}{\lstinline/process(WatchedEventevent)/}\\
\lstinline/ZooKeeperStorageIO/&\raisebox{0pt}{\lstinline/delete(String)/}\\
\lstinline/TestManagedSchema/&\raisebox{0pt}{\lstinline/testPersistUniqueKey()/}\\
\lstinline/TestManagedSchema/&\raisebox{0pt}{\lstinline/testPersistUniqueKey()/}\\
\lstinline/StartupLoggingUtils/&\raisebox{0pt}{\lstinline/logNotSupported(String)/}\\
\lstinline/OverseerCollectionSetProcessorTest/&\raisebox{0pt}{\lstinline/testTemplate(Integer)/}\\
\lstinline/TestSolrCloudWithKerberosAl/&\raisebox{0pt}{\lstinline/setupMiniKdc()/}\\
\lstinline/DirectoryFactory/&\raisebox{0pt}{\lstinline/cleanupOldIndexDirectories(String)/}\\
\lstinline/ManagedResource/&\raisebox{0pt}{\lstinline/doPut(BaseSolrResource)/}\\
\lstinline/ManagedResource/&\raisebox{0pt}{\lstinline/doPut(BaseSolrResource)/}\\
\lstinline/TestDistribIDF/&\raisebox{0pt}{\lstinline/addDocsRandomly()=/}\\
\lstinline/SolrCores/&\raisebox{0pt}{\lstinline/waitForLoadingCoreToFinish(String)/}\\
\lstinline/SaslZkACLProviderTest/&\raisebox{0pt}{\lstinline/setUp()/}\\
\lstinline/SaslZkACLProviderTest/&\raisebox{0pt}{\lstinline/setUp()/}\\
\lstinline/SaslZkACLProviderTest/&\raisebox{0pt}{\lstinline/setUp()/}\\
\lstinline/OverseerElectionContext/&\raisebox{0pt}{\lstinline/runLeaderProcess()/}\\ 
\lstinline/ShardSplitTest/&\raisebox{0pt}{\lstinline/logDebugHelp(int[])/}\\
\lstinline/ShardSplitTest/&\raisebox{0pt}{\lstinline/logDebugHelp(int[])/}\\
\lstinline/ShardSplitTest/&\raisebox{0pt}{\lstinline/logDebugHelp(int[])/}\\
\lstinline/ShardSplitTest/&\raisebox{0pt}{\lstinline/logDebugHelp(int[])/}\\
\lstinline/ShardSplitTest/&\raisebox{0pt}{\lstinline/logDebugHelp(int[])/}\\
\lstinline/TestCloudPivotFacet/&\raisebox{0pt}{\lstinline/assertPivotCountsAreCorrect()/}\\
\lstinline/TestCloudPivotFacet/&\raisebox{0pt}{\lstinline/assertPivotCountsAreCorrect()/}\\
\lstinline/TestCloudPivotFacet/&\raisebox{0pt}{\lstinline/assertPivotCountsAreCorrect()/}\\
\lstinline/SchemaManager/&\raisebox{0pt}{\lstinline/ManagedIndexSchemagetFreshManagedSchema(SolrCore)/}\\
\lstinline/SolrConfig/&\raisebox{0pt}{\lstinline/HttpCachingConfig(SolrConfigconf)/}\\
\lstinline/ManagedWordSetResourc/&\raisebox{0pt}{\lstinline/applyUpdatesToManagedData(Object)/}\\
\lstinline/ManagedWordSetResourc/&\raisebox{0pt}{\lstinline/applyUpdatesToManagedData(Object)/}\\
\lstinline/ManagedWordSetResourc/&\raisebox{0pt}{\lstinline/applyUpdatesToManagedData(Object)/}\\
\lstinline/ClusterStateUpdateTest/&\raisebox{0pt}{\lstinline/setUp()/}\\
\lstinline/ClusterStateUpdateTest/&\raisebox{0pt}{\lstinline/setUp()/}\\
\lstinline/BasicDistributedZkTest/&\raisebox{0pt}{\lstinline/testMultipleCollections()/}\\
\lstinline/BasicDistributedZkTest/&\raisebox{0pt}{\lstinline/testMultipleCollections()/}\\
\lstinline/BasicDistributedZkTest/&\raisebox{0pt}{\lstinline/testMultipleCollections()/}\\
\lstinline/BasicDistributedZkTest/&\raisebox{0pt}{\lstinline/testMultipleCollections()/}\\
\lstinline/AbstractSpatialPrefixTree/&\raisebox{0pt}{\lstinline/TnewSpatialStrategy(String)/}\\
\lstinline/IndexSchema/&\raisebox{0pt}{\lstinline/new()/}\\
\lstinline/IndexSchema/&\raisebox{0pt}{\lstinline/new()/}\\
\lstinline/IndexSchema/&\raisebox{0pt}{\lstinline/new()/}\\
\lstinline/IndexSchema/&\raisebox{0pt}{\lstinline/new()/}\\
\lstinline/IndexSchema/&\raisebox{0pt}{\lstinline/new()/}\\
\lstinline/IndexSchema/&\raisebox{0pt}{\lstinline/new()/}\\
\lstinline/IndexSchema/&\raisebox{0pt}{\lstinline/new()/}\\
\lstinline/IndexSchema/&\raisebox{0pt}{\lstinline/new()/}\\
\lstinline/IndexSchema/&\raisebox{0pt}{\lstinline/new()/}\\
\lstinline/IndexSchema/&\raisebox{0pt}{\lstinline/new()/}\\
\lstinline/IndexSchema/&\raisebox{0pt}{\lstinline/new()/}\\
\lstinline/IndexSchema/&\raisebox{0pt}{\lstinline/new()/}\\
\lstinline/IndexSchema/&\raisebox{0pt}{\lstinline/new()/}\\
\lstinline/IndexSchema/&\raisebox{0pt}{\lstinline/new()/}\\
\lstinline/IndexSchema/&\raisebox{0pt}{\lstinline/new()/}\\
\lstinline/IndexSchema/&\raisebox{0pt}{\lstinline/new()/}\\
\lstinline/IndexSchema/&\raisebox{0pt}{\lstinline/new()/}\\
\lstinline/SolrConfigHandler/new()/&\raisebox{0pt}{\lstinline/wait()/}\\
\lstinline/SolrConfigHandler/new()/&\raisebox{0pt}{\lstinline/wait()/}\\
\lstinline/ResourceSharingTestComponent/&\raisebox{0pt}{\lstinline/inform(SolrCorecore)/}\\
\lstinline/CoreContainer/&\raisebox{0pt}{\lstinline/swap(Stringn0,Stringn1)/}\\
\lstinline/CoreContainer/&\raisebox{0pt}{\lstinline/swap(Stringn0,Stringn1)/}\\
\lstinline/CoreContainer/&\raisebox{0pt}{\lstinline/swap(Stringn0,Stringn1)/}\\
\lstinline/BootstrapCallable/&\raisebox{0pt}{\lstinline/call()/}\\
\lstinline/SolrIndexWriter/&\raisebox{0pt}{\lstinline/rollback()/}\\
\lstinline/MockAuthenticationPlugin/&\raisebox{0pt}{\lstinline/doAuthenticate(ServletRequest)/}\\
\lstinline/TestReplicationHandler/&\raisebox{0pt}{\lstinline/RateLimitedReplication()/}\\
\lstinline/TestReplicationHandler/&\raisebox{0pt}{\lstinline/RateLimitedReplication()/}\\
\lstinline/TestTolerantUpdateProcessorRandomCloude/&\raisebox{0pt}{\lstinline/createMiniSolrCloudCluster()/}\\
\lstinline/RegisterCoreAsync/&\raisebox{0pt}{\lstinline/Objectcall()/}\\
\lstinline/ZkController/&\raisebox{0pt}{\lstinline/publishNodeAsDown(String)/}\\
\lstinline/ZkController/&\raisebox{0pt}{\lstinline/publishNodeAsDown(String)/}\\
\lstinline/FileFetcher/&\raisebox{0pt}{\lstinline/cleanup()/}\\
\lstinline/FileFetcher/&\raisebox{0pt}{\lstinline/cleanup()/}\\
\lstinline/FileFetcher/&\raisebox{0pt}{\lstinline/cleanup()/}\\
\lstinline/Runner/&\raisebox{0pt}{\lstinline/resetTaskWithException(String)/}\\
\lstinline/KerberosPlugin/&\raisebox{0pt}{\lstinline/doAuthenticate(ServletRequest)/}\\
\lstinline/RestManagerManagedResourc/&\raisebox{0pt}{\lstinline/applyUpdatesToManagedData(Object)/}\\
\lstinline/ShardRoutingTest/&\raisebox{0pt}{\lstinline/doAtomicUpdate()/}\\
\lstinline/ShardRoutingTest/&\raisebox{0pt}{\lstinline/doAtomicUpdate()/}\\
\lstinline/ShardRoutingTest/&\raisebox{0pt}{\lstinline/doAtomicUpdate()/}\\
\lstinline/TestSolrConfigHandler/&\raisebox{0pt}{\lstinline/MapgetRespMap(String)/}\\
\lstinline/DirectUpdateHandler2/&\raisebox{0pt}{\lstinline/rollback(RollbackUpdateCommand)/}\\
\lstinline/DirectUpdateHandler2/&\raisebox{0pt}{\lstinline/rollback(RollbackUpdateCommand)/}\\
\lstinline/DirectUpdateHandler2/&\raisebox{0pt}{\lstinline/rollback(RollbackUpdateCommand)/}\\
\lstinline/DirectUpdateHandler2/&\raisebox{0pt}{\lstinline/rollback(RollbackUpdateCommand)/}\\
\lstinline/HdfsUpdateLog/&\raisebox{0pt}{\lstinline/clearLog(SolrCore,Plugino)/}\\
\lstinline/OverseerNodePrioritizer/&\raisebox{0pt}{\lstinline/prioritizeOverseerNodes(Stringoverse)/}\\
\lstinline/OverseerNodePrioritizer/&\raisebox{0pt}{\lstinline/prioritizeOverseerNodes(Stringoverse)/}\\
\lstinline/SolrCloudExampleTest/&\raisebox{0pt}{\lstinline/doTestConfigUpdate(Stringt))/}\\
\lstinline/HighlighterConfigTest/&\raisebox{0pt}{\lstinline/testConfig()/}\\
\lstinline/TestAuthorizationFramework/&\raisebox{0pt}{\lstinline/authorizationFrameworkTest()/}\\
\lstinline/TestAuthorizationFramework/&\raisebox{0pt}{\lstinline/authorizationFrameworkTest()/}\\
\lstinline/SolrZkServerProps/&\raisebox{0pt}{\lstinline/parseProperties()/}\\
\lstinline/SnapShooter/&\raisebox{0pt}{\lstinline/deleteNamedSnapshot(ReplicationHandler)/}\\
\lstinline/SnapShooter/&\raisebox{0pt}{\lstinline/deleteNamedSnapshot(ReplicationHandler)/}\\
\lstinline/TransformerProvider/&\raisebox{0pt}{\lstinline/TemplatesgetTemplates()/}\\
\lstinline/FSHDFSUtils/&\raisebox{0pt}{\lstinline/booleansFileClosed(DistributedFileSystemd)/}\\
\lstinline/RecoveryStrategy/&\raisebox{0pt}{\lstinline/Future<RecoveryInfo>replay(SolrCore)/}\\
\lstinline/RecoveryStrategy/&\raisebox{0pt}{\lstinline/Future<RecoveryInfo>replay(SolrCore)/}\\
\lstinline/RecoveryStrategy/&\raisebox{0pt}{\lstinline/Future<RecoveryInfo>replay(SolrCore)/}\\
\lstinline/UnInvertedField/&\raisebox{0pt}{\lstinline/UnInvertedField(Stringfield)/}\\
\lstinline/MockAuthorizationPlugin/&\raisebox{0pt}{\lstinline/AuthorizationResponseauthorize(AuthorizationContext)/}\\
\lstinline/MigrateCmdHttpPartitionTest/&\raisebox{0pt}{\lstinline/migrateKey(ClusterState)/}\\
\lstinline/MigrateCmdHttpPartitionTest/&\raisebox{0pt}{\lstinline/migrateKey(ClusterState)/}\\
\lstinline/MigrateCmdHttpPartitionTest/&\raisebox{0pt}{\lstinline/migrateKey(ClusterState)/}\\
\lstinline/MigrateCmdHttpPartitionTest/&\raisebox{0pt}{\lstinline/migrateKey(ClusterState)/}\\
\lstinline/MigrateCmdHttpPartitionTest/&\raisebox{0pt}{\lstinline/migrateKey(ClusterState)/}\\
\lstinline/MigrateCmdHttpPartitionTest/&\raisebox{0pt}{\lstinline/migrateKey(ClusterState)/}\\
\lstinline/MigrateCmdHttpPartitionTest/&\raisebox{0pt}{\lstinline/migrateKey(ClusterState)/}\\
\lstinline/MigrateCmdHttpPartitionTest/&\raisebox{0pt}{\lstinline/migrateKey(ClusterState)/}\\
\lstinline/MigrateCmdHttpPartitionTest/&\raisebox{0pt}{\lstinline/migrateKey(ClusterState)/}\\
\lstinline/MigrateCmdHttpPartitionTest/&\raisebox{0pt}{\lstinline/migrateKey(ClusterState)/}\\
\lstinline/MigrateCmdHttpPartitionTest/&\raisebox{0pt}{\lstinline/migrateKey(ClusterState)/}\\
\lstinline/MigrateCmdHttpPartitionTest/&\raisebox{0pt}{\lstinline/migrateKey(ClusterState)/}\\
\lstinline/MigrateCmdHttpPartitionTest/&\raisebox{0pt}{\lstinline/migrateKey(ClusterState)/}\\
\lstinline/MigrateCmdHttpPartitionTest/&\raisebox{0pt}{\lstinline/migrateKey(ClusterState)/}\\
\lstinline/LazyPluginHolder/&\raisebox{0pt}{\lstinline/createInst()/}\\
\lstinline/InvokeOp/&\raisebox{0pt}{\lstinline/Map<String,Object>invokeAClass(SolrQueryRequestreq,Stringc)/}\\
\lstinline/RollingRestartTest/&\raisebox{0pt}{\lstinline/waitUntilOverseerDesignateIsLeader()/}\\
\lstinline/RollingRestartTest/&\raisebox{0pt}{\lstinline/waitUntilOverseerDesignateIsLeader()/}\\
\lstinline/ManagedEndpoint/&\raisebox{0pt}{\lstinline/ObjectparseJsonFromRequestBody()/}\\
\lstinline/LazyPluginHolder/&\raisebox{0pt}{\lstinline/createInst()/}\\
\lstinline/LoggingInfoStream/&\raisebox{0pt}{\lstinline/message(Stringcomponent,String)/}\\
\lstinline/FullThrottleStoppableIndexingThread/&\raisebox{0pt}{\lstinline/run()/}\\
\lstinline/TestDistributedStatsComponentCardinality/&\raisebox{0pt}{\lstinline/buildIndex()/}\\
\lstinline/SynonymManager/&\raisebox{0pt}{\lstinline/doDeleteChild(BaseSolrResourcepoint,StringchildId)/}\\
\lstinline/SynonymManager/&\raisebox{0pt}{\lstinline/doDeleteChild(BaseSolrResourcepoint,StringchildId)/}\\
\lstinline/Runner/&\raisebox{0pt}{\lstinline/resetTaskWithException()/}\\
\lstinline/Runner/&\raisebox{0pt}{\lstinline/resetTaskWithException()/}\\
\lstinline/ReplicaMutator/&\raisebox{0pt}{\lstinline/DocCollectioncheckAndCompleteShardSplit()/}\\
\lstinline/ReplicaMutator/&\raisebox{0pt}{\lstinline/DocCollectioncheckAndCompleteShardSplit()/}\\
\lstinline/FileStorageIO/&\raisebox{0pt}{\lstinline/configure()/}\\
\lstinline/TestManagedSchemaAPIe/&\raisebox{0pt}{\lstinline/addStringField(String)/}\\
\lstinline/XSLTResponseWriter/&\raisebox{0pt}{\lstinline/init(NamedListn)/}\\
\lstinline/Suggester/&\raisebox{0pt}{\lstinline/SpellingResultgetSuggestions(SpellingOptionsoptions)/}\\
\lstinline/Suggester/&\raisebox{0pt}{\lstinline/SpellingResultgetSuggestions(SpellingOptionsoptions)/}\\
\lstinline/OneQuery/&\raisebox{0pt}{\lstinline/run()/}\\
\lstinline/OneQuery/&\raisebox{0pt}{\lstinline/run()/}\\
\lstinline/CollectionMutator/&\raisebox{0pt}{\lstinline/checkKeyExistence(ZkNodeProp)/}\\
\lstinline/BaseCdcrDistributedZkTest/&\raisebox{0pt}{\lstinline/waitForReplicationToComplete(String)/}\\
\lstinline/BaseCdcrDistributedZkTest/&\raisebox{0pt}{\lstinline/waitForReplicationToComplete(String)/}\\
\lstinline/ManagedEndpoint/&\raisebox{0pt}{\lstinline/ObjectparseJsonFromRequestBody()/}\\
\lstinline/RestManagerManagedResourc/&\raisebox{0pt}{\lstinline/applyUpdatesToManagedData(Objectupdates)/}\\
\lstinline/RestManager/&\raisebox{0pt}{\lstinline/attachManagedResource(ManagedResource)/}\\
\lstinline/RestManager/&\raisebox{0pt}{\lstinline/attachManagedResource(ManagedResource)/}\\
\lstinline/ProcessStateWatcher/&\raisebox{0pt}{\lstinline/process(WatchedEventevent)/}\\
\lstinline/LeaderInitiatedRecoveryThread/&\raisebox{0pt}{\lstinline/booleanpublishDownState(String)/}\\
\lstinline/HdfsDirectoryFactory/&\raisebox{0pt}{\lstinline/initKerberos()/}\\
\lstinline/FileFloatSource/&\raisebox{0pt}{\lstinline/getFloats()/}\\
\lstinline/FileFloatSource/&\raisebox{0pt}{\lstinline/getFloats()/}\\
\lstinline/FileFloatSource/&\raisebox{0pt}{\lstinline/getFloats()/}\\
\lstinline/UpdateLog/&\raisebox{0pt}{\lstinline/highestVersion(SolrIndex)/}\\
\lstinline/LogReplayer/&\raisebox{0pt}{\lstinline/run()/}\\
\lstinline/SolrDispatchFilter/&\raisebox{0pt}{\lstinline/authenticateRequest(ServletRequest)/}\\
\lstinline/SolrDispatchFilter/&\raisebox{0pt}{\lstinline/authenticateRequest(ServletRequest)/}\\
\lstinline/SolrDispatchFilter/&\raisebox{0pt}{\lstinline/authenticateRequest(ServletRequest)/}\\
\lstinline/SolrDispatchFilter/&\raisebox{0pt}{\lstinline/authenticateRequest(ServletRequest)/}\\
\lstinline/SyncShardRequest/&\raisebox{0pt}{\lstinline/booleanhandleUpdates(ShardResponsesrsp)/}\\
\lstinline/SyncShardRequest/&\raisebox{0pt}{\lstinline/booleanhandleUpdates(ShardResponsesrsp)/}\\
\lstinline/SuggesterListener)/&\raisebox{0pt}{\lstinline/buildSuggesterIndex(SolrIndexSearch)/}\\
\lstinline/SuggesterListener)/&\raisebox{0pt}{\lstinline/buildSuggesterIndex(SolrIndexSearch)/}\\
\lstinline/SuggesterListener)/&\raisebox{0pt}{\lstinline/buildSuggesterIndex(SolrIndexSearch)/}\\
\lstinline/SuggesterListener)/&\raisebox{0pt}{\lstinline/buildSuggesterIndex(SolrIndexSearch)/}\\
\lstinline/DirectSolrSpellChecker/&\raisebox{0pt}{\lstinline/Stringinit(NamedList)/}\\
\lstinline/SliceMutator/&\raisebox{0pt}{\lstinline/ZkWriteCommandremoveRoutingRule(ClusterState)/}\\
\lstinline/SliceMutator/&\raisebox{0pt}{\lstinline/ZkWriteCommandremoveRoutingRule(ClusterState)/}\\
\lstinline/SliceMutator/&\raisebox{0pt}{\lstinline/ZkWriteCommandremoveRoutingRule(ClusterState)/}\\
\lstinline/TestConfigReload/&\raisebox{0pt}{\lstinline/checkConfReload()/}\\
\lstinline/TestConfigReload/&\raisebox{0pt}{\lstinline/checkConfReload()/}\\
\lstinline/CreateCollectionCmd/&\raisebox{0pt}{\lstinline/StringgetConfigName(String)/}\\
\lstinline/ErrorLoggingConcurrentUpdateSolrClient/&\raisebox{0pt}{\lstinline/handleError(Throwable)/}\\
\lstinline/Worker/&\raisebox{0pt}{\lstinline/run()/}\\
\lstinline/JsonStorage/&\raisebox{0pt}{\lstinline/store(String)/}\\
\lstinline/TestRandomFlRTGCloude/&\raisebox{0pt}{\lstinline/addRandomDocument(int)/}\\
\lstinline/TestRandomFlRTGCloude/&\raisebox{0pt}{\lstinline/addRandomDocument(int)/}\\
\lstinline/OverseerTest/&\raisebox{0pt}{\lstinline/RemovalOfLastReplica()/}\\
\lstinline/OverseerTest/&\raisebox{0pt}{\lstinline/RemovalOfLastReplica()/}\\
\lstinline/OverseerTest/&\raisebox{0pt}{\lstinline/RemovalOfLastReplica()/}\\
\lstinline/OverseerTest/&\raisebox{0pt}{\lstinline/RemovalOfLastReplica()/}\\
\lstinline/OverseerTest/&\raisebox{0pt}{\lstinline/RemovalOfLastReplica()/}\\
\lstinline/OverseerTest/&\raisebox{0pt}{\lstinline/RemovalOfLastReplica()/}\\
\lstinline/OverseerTest/&\raisebox{0pt}{\lstinline/RemovalOfLastReplica()/}\\
\lstinline/OverseerTest/&\raisebox{0pt}{\lstinline/RemovalOfLastReplica()/}\\
\lstinline/OverseerTest/&\raisebox{0pt}{\lstinline/RemovalOfLastReplica()/}\\
\lstinline/OverseerTest/&\raisebox{0pt}{\lstinline/RemovalOfLastReplica()/}\\
\lstinline/OverseerTest/&\raisebox{0pt}{\lstinline/RemovalOfLastReplica()/}\\
\lstinline/RegisterCoreAsync/&\raisebox{0pt}{\lstinline/Objectcall()/}\\
\lstinline/TestLogWatcher/&\raisebox{0pt}{\lstinline/Log4jWatcher()/}\\
\lstinline/SolrXmlConfig/&\raisebox{0pt}{\lstinline/PropertiesloadProperties(Config)/}\\
\lstinline/OverseerCollectionMessageHandler/&\raisebox{0pt}{processResponse(NamedList)/}\\
\lstinline/OverseerCollectionMessageHandler/&\raisebox{0pt}{\lstinline/processResponse(NamedList)/}\\
\lstinline/OverseerCollectionMessageHandler/&\raisebox{0pt}{\lstinline/processResponse(NamedList)/}\\
\lstinline/DirectUpdateHandlerTest/&\raisebox{0pt}{\lstinline/PrepareCommit()/}\\
\lstinline/DirectUpdateHandlerTest/&\raisebox{0pt}{\lstinline/PrepareCommit()/}\\
\lstinline/DirectUpdateHandlerTest/&\raisebox{0pt}{\lstinline/PrepareCommit()/}\\
\lstinline/BackupRestoreUtils/&\raisebox{0pt}{\lstinline/intindexDocs(SolrClient)/}\\ 
\lstinline/ErrorReportingConcurrentUpdateSolrClient/&\raisebox{0pt}{\lstinline/handleError(Throwable)/}\\
\lstinline/ClusterStateUpdater/&\raisebox{0pt}{\lstinline/LeaderStatusamILeader()/}\\
\lstinline/ReplicationDistributedZkTest/&\raisebox{0pt}{\lstinline/testBatchBoundaries()/}\\
\lstinline/ReplicationDistributedZkTest/&\raisebox{0pt}{\lstinline/testBatchBoundaries()/}\\
\lstinline/ReplicationDistributedZkTest/&\raisebox{0pt}{\lstinline/testBatchBoundaries()/}\\
\lstinline/ReplicationDistributedZkTest/&\raisebox{0pt}{\lstinline/testBatchBoundaries()/}\\
\lstinline/ReplicationDistributedZkTest/&\raisebox{0pt}{\lstinline/testBatchBoundaries()/}\\
\lstinline/ReplicationDistributedZkTest/&\raisebox{0pt}{\lstinline/testBatchBoundaries()/}\\
\lstinline/ReplicationDistributedZkTest/&\raisebox{0pt}{\lstinline/testBatchBoundaries()/}\\
\lstinline/ReplicationDistributedZkTest/&\raisebox{0pt}{\lstinline/testBatchBoundaries()/}\\
\lstinline/ReplicationDistributedZkTest/&\raisebox{0pt}{\lstinline/testBatchBoundaries()/}\\
\lstinline/ReplicationDistributedZkTest/&\raisebox{0pt}{\lstinline/testBatchBoundaries()/}\\
\lstinline/ReplicationDistributedZkTest/&\raisebox{0pt}{\lstinline/testBatchBoundaries()/}\\
\lstinline/ReplicationDistributedZkTest/&\raisebox{0pt}{\lstinline/testBatchBoundaries()/}\\
\lstinline/ReplicationDistributedZkTest/&\raisebox{0pt}{\lstinline/testBatchBoundaries()/}\\
\lstinline/ReplicationDistributedZkTest/&\raisebox{0pt}{\lstinline/testBatchBoundaries()/}\\
\lstinline/ReplicationDistributedZkTest/&\raisebox{0pt}{\lstinline/testBatchBoundaries()/}\\
\lstinline/ReplicationDistributedZkTest/&\raisebox{0pt}{\lstinline/testBatchBoundaries()/}\\
\lstinline/ReplicationDistributedZkTest/&\raisebox{0pt}{\lstinline/testBatchBoundaries()/}\\
\lstinline/ReplicationDistributedZkTest/&\raisebox{0pt}{\lstinline/testBatchBoundaries()/}\\
\lstinline/ReplicationDistributedZkTest/&\raisebox{0pt}{\lstinline/testBatchBoundaries()/}\\
\lstinline/ReplicationDistributedZkTest/&\raisebox{0pt}{\lstinline/testBatchBoundaries()/}\\
\lstinline/ReplicationDistributedZkTest/&\raisebox{0pt}{\lstinline/testBatchBoundaries()/}\\
\lstinline/ReplicationDistributedZkTest/&\raisebox{0pt}{\lstinline/testBatchBoundaries()/}\\
\lstinline/ReplicationDistributedZkTest/&\raisebox{0pt}{\lstinline/testBatchBoundaries()/}\\
\lstinline/ReplicationDistributedZkTest/&\raisebox{0pt}{\lstinline/testBatchBoundaries()/}\\
\lstinline/ReplicationDistributedZkTest/&\raisebox{0pt}{\lstinline/testBatchBoundaries()/}\\
\lstinline/ReplicationDistributedZkTest/&\raisebox{0pt}{\lstinline/testBatchBoundaries()/}\\
\lstinline/ReplicationDistributedZkTest/&\raisebox{0pt}{\lstinline/testBatchBoundaries()/}\\
\lstinline/ReplicationDistributedZkTest/&\raisebox{0pt}{\lstinline/testBatchBoundaries()/}\\
\lstinline/OpenCloseCoreStressTest/&\raisebox{0pt}{\lstinline/checkResults(HttpSolrClient)/}\\
\lstinline/OpenCloseCoreStressTest/&\raisebox{0pt}{\lstinline/checkResults(HttpSolrClient)/}\\
\lstinline/OpenCloseCoreStressTest/&\raisebox{0pt}{\lstinline/checkResults(HttpSolrClient)/}\\
\lstinline/OpenCloseCoreStressTest/&\raisebox{0pt}{\lstinline/checkResults(HttpSolrClient)/}\\
\lstinline/SolrCore/&\raisebox{0pt}{\lstinline/checkStale()/}\\
\lstinline/IndexFingerprint/&\raisebox{0pt}{\lstinline/getFingerprint(long)/}\\
\lstinline/DocExpirationUpdateProcessorFactory/&\raisebox{0pt}{\lstinline/amInChargeOfPeriodicDeletes()/}\\
\lstinline/DocExpirationUpdateProcessorFactory/&\raisebox{0pt}{\lstinline/amInChargeOfPeriodicDeletes()/}\\
\lstinline/ForceLeaderTest/&\raisebox{0pt}{\lstinline/bringBackOldLeaderAndSendDoc(String,int)/}\\
\lstinline/ForceLeaderTest/&\raisebox{0pt}{\lstinline/bringBackOldLeaderAndSendDoc(String,int)/}\\
\lstinline/ForceLeaderTest/&\raisebox{0pt}{\lstinline/bringBackOldLeaderAndSendDoc(String,int)/}\\
\lstinline/ForceLeaderTest/&\raisebox{0pt}{\lstinline/bringBackOldLeaderAndSendDoc(String,int)/}\\
\lstinline/ForceLeaderTest/&\raisebox{0pt}{\lstinline/bringBackOldLeaderAndSendDoc(String,int)/}\\
\lstinline/ForceLeaderTest/&\raisebox{0pt}{\lstinline/bringBackOldLeaderAndSendDoc(String,int)/}\\
\lstinline/ForceLeaderTest/&\raisebox{0pt}{\lstinline/bringBackOldLeaderAndSendDoc(String,int)/}\\
\lstinline/ForceLeaderTest/&\raisebox{0pt}{\lstinline/bringBackOldLeaderAndSendDoc(String,int)/}\\
\lstinline/ForceLeaderTest/&\raisebox{0pt}{\lstinline/bringBackOldLeaderAndSendDoc(String,int)/}\\
\lstinline/ForceLeaderTest/&\raisebox{0pt}{\lstinline/bringBackOldLeaderAndSendDoc(String,int)/}\\
\lstinline/ForceLeaderTest/&\raisebox{0pt}{\lstinline/bringBackOldLeaderAndSendDoc(String,int)/}\\
\lstinline/ForceLeaderTest/&\raisebox{0pt}{\lstinline/bringBackOldLeaderAndSendDoc(String,int)/}\\
\lstinline/ForceLeaderTest/&\raisebox{0pt}{\lstinline/bringBackOldLeaderAndSendDoc(String,int)/}\\
\lstinline/TestStressUserVersions/&\raisebox{0pt}{\lstinline/run()/}\\
\lstinline/TestStressReorder/&\raisebox{0pt}{\lstinline/run()/}\\
\lstinline/ReplicationFactorTest/&\raisebox{0pt}{\lstinline/()/}\\
\lstinline/ReplicationFactorTest/&\raisebox{0pt}{\lstinline/()/}\\
\lstinline/ReplicationFactorTest/&\raisebox{0pt}{\lstinline/()/}\\
\lstinline/ReplicationFactorTest/&\raisebox{0pt}{\lstinline/()/}\\
\lstinline/SolrSchemaRestApi/&\raisebox{0pt}{\lstinline/RestletcreateInboundRoot()/}\\
\lstinline/SolrSchemaRestApi/&\raisebox{0pt}{\lstinline/RestletcreateInboundRoot()/}\\
\lstinline/SolrCoreCheckLockOnStartupTest/&\raisebox{0pt}{\lstinline/NativeLockErrorOnStartup()/}\\
\lstinline/CommitTracker/&\raisebox{0pt}{\lstinline/commitTracker(oolean)/}\\
\lstinline/QuerySenderListener/&\raisebox{0pt}{\lstinline/newSearcher(SolrIndex)/}\\
\lstinline/QuerySenderListener/&\raisebox{0pt}{\lstinline/newSearcher(SolrIndex)/}\\

\bottomrule
\end{tabular}
\end{center}

\subsection{Cluster classified as Control Flow Logging}

\begin{center}
\captionof{figure}{LMs in the cluster $\id{S}_{\id{CF},1}$}
\begin{tabular}{ll}\toprule
\multicolumn{1}{c}{Class}&\multicolumn{1}{c}{Method}\\\midrule
\lstinline/OpenExchangeRates/&\raisebox{0pt}{\lstinline/ OpenExchangeRates(InputStream}/}\\ 
\lstinline/SolrIndexSplitter/&\raisebox{0pt}{\lstinline/ split(LeafReaderContext)}/}\\
\lstinline/SolrIndexSplitter/&\raisebox{0pt}{\lstinline/ split(LeafReaderContext)}/}\\
\lstinline/SolrIndexSplitter/&\raisebox{0pt}{\lstinline/ split(LeafReaderContext)/}\\ 
\lstinline/SingleThreadedJsonLoader/&\raisebox{0pt}{\lstinline/ processUpdate(Reader)/}\\ 
\lstinline/SingleThreadedJsonLoader/&\raisebox{0pt}{\lstinline/ processUpdate(Reader)/}\\
\lstinline/AtomicUpdateDocumentMerger/&\raisebox{0pt}{\lstinline/ merge(SolrInputDocument)/}\\ 
\lstinline/Config/&\raisebox{0pt}{\stinline/parseLuceneVersionString(String)/}\\ 
\lstinline/XMLLoader/&\raisebox{0pt}{\stinline/SolrInputDocumentreadDoc(XMLStreamReader)/}\\ 
\lstinline/OverseerTest/&\raisebox{0pt}{\stinline/removalOfLastReplica()/}\\ 
\lstinline/RequestHandlers/&\raisebox{0pt}{\stinline/applyInitParams(SolrConfigc,PluginInfo)/}\\ 
\lstinline/SolrDynamicMBean/&\raisebox{0pt}{\stinline/get(String[])/}\\ 
\lstinline/SolrDynamicMBean/&\raisebox{0pt}{\stinline/Aget(String[])/}\\ 
\bottomrule
\end{tabular}
\end{center}

\subsection{Cluster classified as Exception Try-Block Logging}

\begin{center}
\captionof{figure}{LMs in the cluster $\id{S}_{\id{TB},1}$}
\begin{tabular}{ll}\toprule
\multicolumn{1}{c}{Class}&\multicolumn{1}{c}{Method}\\\midrule
\lstinline/FileFetcher/&\raisebox{0pt}{\lstinline/cleanup()/}\\
\lstinline/FileFetcher/&\raisebox{0pt}{\lstinline/cleanup()/}\\
\lstinline/FileFetcher/&\raisebox{0pt}{\lstinline/cleanup()/}\\
\lstinline/FileFetcher/&\raisebox{0pt}{\lstinline/cleanup()/}\\
\lstinline/FileFetcher/&\raisebox{0pt}{\lstinline/cleanup()/}\\
\lstinline/OverseerTest/&\raisebox{0pt}{\lstinline/RemovalOfLastReplica()/}\\
\lstinline/RecoveryStrategy/&\raisebox{0pt}{\lstinline/replay(SolrCore)/}\\
\lstinline/RecoveryStrategy/&\raisebox{0pt}{\lstinline/replay(SolrCore)/}\\
\lstinline/RecoveryStrategy/&\raisebox{0pt}{\lstinline/replay(SolrCore)/}\\
\lstinline/ShardSplitTest/&\raisebox{0pt}{\lstinline/logDebugHelp(QueryResponser,long)/}\\
\lstinline/ShardSplitTest/&\raisebox{0pt}{\lstinline/logDebugHelp(QueryResponser,long)/}\\
\lstinline/ShardSplitTest/&\raisebox{0pt}{\lstinline/logDebugHelp(QueryResponser,long)/}\\
\lstinline/TestMiniSolrCloudCluster/&\raisebox{0pt}{\lstinline/createSearchDelete()/}\\
\lstinline/TestMiniSolrCloudCluster/&\raisebox{0pt}{\lstinline/createSearchDelete()/}\\
\lstinline/TestMiniSolrCloudCluster/&\raisebox{0pt}{\lstinline/createSearchDelete()/}\\
\lstinline/TestMiniSolrCloudCluster/&\raisebox{0pt}{\lstinline/createSearchDelete()/}\\
\lstinline/TestMiniSolrCloudCluster/&\raisebox{0pt}{\lstinline/createSearchDelete()/}\\
\lstinline/TestMiniSolrCloudCluster/&\raisebox{0pt}{\lstinline/createSearchDelete()/}\\
\lstinline/ForceLeaderTest/&\raisebox{0pt}{\lstinline/bringBackOldLeaderAndSendDoc(String)/}\\
\lstinline/ForceLeaderTest/&\raisebox{0pt}{\lstinline/bringBackOldLeaderAndSendDoc(String)/}\\
\lstinline/ForceLeaderTest/&\raisebox{0pt}{\lstinline/bringBackOldLeaderAndSendDoc(String)/}\\
\lstinline/ForceLeaderTest/&\raisebox{0pt}{\lstinline/bringBackOldLeaderAndSendDoc(String)/}\\
\lstinline/ForceLeaderTest/&\raisebox{0pt}{\lstinline/bringBackOldLeaderAndSendDoc(String)/}\\
\lstinline/ForceLeaderTest/&\raisebox{0pt}{\lstinline/bringBackOldLeaderAndSendDoc(String)/}\\
\lstinline/ForceLeaderTest/&\raisebox{0pt}{\lstinline/bringBackOldLeaderAndSendDoc(String)/}\\
\lstinline/DirectUpdateHandler2/&\raisebox{0pt}{\lstinline/rollback(RollbackUpdatecommand))/}\\
\lstinline/DirectUpdateHandler2/&\raisebox{0pt}{\lstinline/rollback(RollbackUpdatecommand))/}\\
\lstinline/DirectUpdateHandler2/&\raisebox{0pt}{\lstinline/rollback(RollbackUpdatecommand))/}\\
\lstinline/DirectUpdateHandler2/&\raisebox{0pt}{\lstinline/rollback(RollbackUpdatecommand))/}\\
\lstinline/SolrCore/&\raisebox{0pt}{\lstinline/checkStale(String,int)/}\\
\lstinline/SolrCore/&\raisebox{0pt}{\lstinline/checkStale(String,int)/}\\
\lstinline/SolrDynamicMBean/&\raisebox{0pt}{\lstinline/AttributeListgetAttributes(String[]attributes)/}\\
\lstinline/DeleteTool/&\raisebox{0pt}{\lstinline/deleteCollection(CloudSolrClient,CommandLine)/}\\
\lstinline/SyncStrategy/&\raisebox{0pt}{\lstinline/requestRecoveries()/}\\
\lstinline/RecoveryStrategy/&\raisebox{0pt}{\lstinline/replay(SolrCore)/}\\
\lstinline/RecoveryStrategy/&\raisebox{0pt}{\lstinline/replay(SolrCore)/}\\
\lstinline/RecoveryStrategy/&\raisebox{0pt}{\lstinline/replay(SolrCore)/}\\
\lstinline/RecoveryStrategy/&\raisebox{0pt}{\lstinline/replay(SolrCore)/}\\
\lstinline/SolrCore/&\raisebox{0pt}{\lstinline/checkStale(String,int)/}\\
\lstinline/LeaderStateWatcher/&\raisebox{0pt}{\lstinline/process(WatchedEventevent)/}\\
\lstinline/SyncShardRequest/&\raisebox{0pt}{\lstinline/handleUpdates(ShardResponsesrsp)/}\\
\lstinline/SolrZkServer/&\raisebox{0pt}{\lstinline/start()/}\\
\lstinline/StartupLoggingUtils/&\raisebox{0pt}{\lstinline/logNotSupported(Stringmsg)/}\\
\lstinline/MigrateCmd/&\raisebox{0pt}{\lstinline/migrateKey(ClusterStatecluster)/}\\
\lstinline/FSHDFSUtils/&\raisebox{0pt}{\lstinline/booleanisFileClosed(Method,Path)/}\\
\lstinline/CryptoKeys/&\raisebox{0pt}{\lstinline/Stringverify(String,Byte)/}\\
\lstinline/LeaderInitiatedRecoveryThrea/&\raisebox{0pt}{\lstinline/publishDownState(String,boolean))/}\\
\lstinline/ManagedIndexSchema/&\raisebox{0pt}{\lstinline/newFieldType(String))/}\\
\lstinline/ProcessStateWatcher/&\raisebox{0pt}{\lstinline/process(WatchedEventevent)/}\\
\lstinline/BufferStateWatcher/&\raisebox{0pt}{\lstinline/process(WatchedEventevent)/}\\
\lstinline/BootstrapCallable/&\raisebox{0pt}{\lstinline/Booleancall()/}\\
\lstinline/RuntimeLib/&\raisebox{0pt}{\lstinline/verify()/}\\
\lstinline/ShardSplitTest/&\raisebox{0pt}{\lstinline/logDebugHelp(QueryResponser,long)/}\\
\lstinline/ClusterStateUpdater,Closeable/&\raisebox{0pt}{\lstinline/LeaderStatusamILeader()/}\\
\lstinline/ForceLeaderTest/&\raisebox{0pt}{\lstinline/bringBackOldLeaderAndSendDoc(String)/}\\
\lstinline/ManagedIndexSchema/&\raisebox{0pt}{\lstinline/newFieldType(String))/}\\
\lstinline/ElectionWatcher/&\raisebox{0pt}{\lstinline/process(WatchedEventevent)/}\\
\lstinline/SpellCheckerListener/&\raisebox{0pt}{\lstinline/buildSpellIndex(sendBootstrapCommand)/}\\
\lstinline/SpellCheckerListener/&\raisebox{0pt}{\lstinline/buildSpellIndex(sendBootstrapCommand)/}\\
\lstinline/OverseerRestarter/&\raisebox{0pt}{\lstinline/run()/}\\
\lstinline/OverseerRestarter/&\raisebox{0pt}{\lstinline/run()/}\\
\lstinline/OverseerRestarter/&\raisebox{0pt}{\lstinline/run()/}\\
\lstinline/OverseerRestarter/&\raisebox{0pt}{\lstinline/run()/}\\
\lstinline/StatusRunnable/&\raisebox{0pt}{\lstinline/sendBootstrapCommand())/}\\
\lstinline/StatusRunnable/&\raisebox{0pt}{\lstinline/sendBootstrapCommand())/}\\
\lstinline/RunExampleExecutor/&\raisebox{0pt}{\lstinline/close())/}\\
\lstinline/RunExampleExecutor/&\raisebox{0pt}{\lstinline/close())/}\\
\lstinline/SyncStrategy/&\raisebox{0pt}{\lstinline/requestRecoveries()/}\\
\lstinline/FileFetcher/&\raisebox{0pt}{\lstinline/cleanup()/}\\
\lstinline/FileFetcher/&\raisebox{0pt}{\lstinline/cleanup()/}\\
\lstinline/RequestReplicationTracker/&\raisebox{0pt}{\lstinline/doLocalCommit(CommitUpdateCommand))/}\\
\lstinline/CoreContainer/&\raisebox{0pt}{\lstinline/swap(Stringn0,Stringn1)/}\\
\lstinline/UpdateThread/&\raisebox{0pt}{\lstinline/run()/}\\
\lstinline/ZkController/&\raisebox{0pt}{\lstinline/publishNodeAsDown(StringnodeName)/}\\
\lstinline/RestoreCore/&\raisebox{0pt}{\lstinline/booleandoRestore()/}\\
\lstinline/LeaderFailureAfterFreshStartTest/&\raisebox{0pt}{\lstinline/waitForNewLeader(CloudSolrClient,String)/}\\
\lstinline/LeaderFailureAfterFreshStartTest/&\raisebox{0pt}{\lstinline/waitForNewLeader(CloudSolrClient,String)/}\\
\lstinline/LeaderFailureAfterFreshStartTest/&\raisebox{0pt}{\lstinline/waitForNewLeader(CloudSolrClient,String)/}\\
\lstinline/PeerSyncReplicationTest/&\raisebox{0pt}{\lstinline/()/}\\
\lstinline/PeerSyncReplicationTest/&\raisebox{0pt}{\lstinline/()/}\\
\lstinline/PeerSyncReplicationTest/&\raisebox{0pt}{\lstinline/()/}\\
\lstinline/OverseerRestarter/&\raisebox{0pt}{\lstinline/run()/}\\
\lstinline/ForceLeaderTest/&\raisebox{0pt}{\lstinline/bringBackOldLeaderAndSendDoc(String)/}\\
\lstinline/SnapShooter/&\raisebox{0pt}{\lstinline/deleteNamedSnapshot(ReplicationHandle)/}\\
\lstinline/LeaderElectionTest/&\raisebox{0pt}{\lstinline/ParallelElection()/}\\
\lstinline/LeaderElectionTest/&\raisebox{0pt}{\lstinline/ParallelElection()/}\\
\lstinline/FileFetcher/&\raisebox{0pt}{\lstinline/cleanup()/}\\
\lstinline/FileFetcher/&\raisebox{0pt}{\lstinline/cleanup()/}\\
\lstinline/CdcrReplicator/&\raisebox{0pt}{\lstinline/handleException(Exceptione)/}\\
\lstinline/CoreContainer/&\raisebox{0pt}{\lstinline/swap(Stringn0,Stringn1)/}\\
\lstinline/DirectUpdateHandler2/&\raisebox{0pt}{\lstinline/rollback(RollbackUpdatecommand))/}\\
\lstinline/DirectUpdateHandler2/&\raisebox{0pt}{\lstinline/rollback(RollbackUpdatecommand))/}\\
\lstinline/DeleteExpiredDocsRunnable/&\raisebox{0pt}{\lstinline/run()/}\\
\lstinline/DeleteExpiredDocsRunnable/&\raisebox{0pt}{\lstinline/run()/}\\
\lstinline/ManagedIndexSchema/&\raisebox{0pt}{\lstinline/newFieldType(String))/}\\
\lstinline/StatusRunnable/&\raisebox{0pt}{\lstinline/sendBootstrapCommand())/}\\
\lstinline/OpenCloseCoreStressTest/&\raisebox{0pt}{\lstinline/checkResults(HttpSolrClient))/}\\
\lstinline/OverseerConfigSetMessageHandler/&\raisebox{0pt}{\lstinline/createConfigSet(ZkNodeProps))/}\\
\lstinline/DefaultSolrCoreState/&\raisebox{0pt}{\lstinline/run()/}\\
\lstinline/RecoveryStrategy/&\raisebox{0pt}{\lstinline/replay(SolrCore)/}\\
\lstinline/StatusRunnable/&\raisebox{0pt}{\lstinline/sendBootstrapCommand())/}\\
\lstinline/StatusRunnable/&\raisebox{0pt}{\lstinline/sendBootstrapCommand())/}\\
\lstinline/SolrIndexSplitterTest/&\raisebox{0pt}{\lstinline/SplitAlternately()/}\\
\lstinline/SSLTestCloudPivotFacet/&\raisebox{0pt}{\lstinline/assertPivotCountsAreCorrect(SolrParams)()/}\\
\lstinline/TestJmxMonitoredMap/&\raisebox{0pt}{\lstinline/beforesetUp()/}\\

\bottomrule
\end{tabular}
\end{center}


\end{document}

