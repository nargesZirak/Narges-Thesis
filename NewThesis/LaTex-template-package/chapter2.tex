\chapter{Motivational Scenario}  \label{ch2}

Printing messages to the console or to a log file is an integral part of software development and can be used to test, debug, and understand what is happening inside an application. In the Java programming language, \name{print} statements are commonly used to print something on console. However, the availability of tools, frameworks, and APIs for logging that offers more powerful and advanced Java logging features, flexibility, and improvement in the logging quality suggests that using \name{print} statements is not sufficient to perform logging practices in real-world applications.
%?
% Java Logging API

The logging frameworks offer many more facilities that are not provided by the \name{print} statements. For example, most logging frameworks (e.g., \name{ Log4J}, \name{SLF4j}, \name{java.util.logging}) use different verbosity levels to control the types of information needed to be logged. That is, by logging at a particular verbosity level, logs at that level and higher levels will be recorded whereas the logs at lower levels will be discarded. Each of these verbosity levels can be used for different applications during software development. For example, the \name{debug} log level messages can be used in a test environment, while the \name{error} log level messages can be used in a production environment. This feature not only produces fewer log messages for each level, but also improves the performance of an application. Also, most logging frameworks allow the production of formatted log messages, which makes it easier for a developer to monitor the behavior of a system. In addition, when a developer is working on a server side application, the only way to know what is happening inside the server is by monitoring the log files. Although logging is a valuable practice for software development and maintenance, it imposes extra time and energy on developers to write, test, and run the code, while affecting the application performance. As latency and speed are major concerns for most software systems, it is necessary for a developer to understand and learn logging practices in great detail in order to perform them in an efficient manner.

To illustrate the inherent challenges of effectively performing logging practices in software systems, one may consider a scenario in which a developer is asked to log an event-based mechanism of a text editor tool written in the Java programming language. In this scenario, the developer is trying to log a Java class of the system (Figure~\ref{ch2-ex}) using the \name{Apache Log4j} framework. She knows that components of this application register with the \code{EditBus} class to receive messages that reflect changes in the application's state, and that the \code{EditBus} class maintains a list of components that have requested to receive messages. That is, when a message is sent using this class, all registered components receive it in turn. Furthermore, any classes that subscribe to the \code{EditBus} and implement the \code{EBComponent} interface define the method \code{EBComponent.handleMessage(EBMessage)} to handle a message sent on by the \code{EditBus}. To perform this logging task, the developer might ask herself several fundamental questions, mostly related to where and what to log.
% Example is a class, not a method. Is it ok???


\begin{figure}[p]
\def\baselinestretch{1}
\begin{lstlisting}
public class EditBus {
    private static ArrayList components = new ArrayList();
    private static EBComponent[] copyComponents;

    private EditBus() {
    }

    public static void addToBus(EBComponent comp) {
        synchronized(components) {
            components.add(comp);
            copyComponents = null;
        }
    }

    public static void removeFromBus(EBComponent comp) {
        synchronized(components) {
            components.remove(comp);
            copyComponents = null;
        }
    }

    public static EBComponent[] getComponents() {
        synchronized(components) {
            if(copyComponents == null) {
                EBComponent[] arr = new EBComponent[components.size()];
                copyComponents =
                    (EBComponent[])components.toArray(arr);
            }
        }
        return copyComponents;
    }

    public static void send(EBMessage message) {
        EBComponent[] comps = getComponents();
        for(int i = 0; i < comps.length; i++) {
            EBComponent comp = comps[i];
            long start = System.currentTimeMillis();
            comp.handleMessage(message);
            long time = (System.currentTimeMillis() - start);
        }
    }
}
\end{lstlisting}
\caption{The \code{EditBus} class.\label{ch2-ex}}
\end{figure}
%Developer's logging task context on

Her first solution might be to simply log at the start and end of every method. However, she believes that logging at the start and end of the \code{addToBus(EBComponent)}, \code{remove}\-\code{From}\-\code{Bus(EBComponent)}, and \code{getComponents()} methods are useless, and will produce redundant information. She assumes that the more she logs, the more she performs file I/O, which slows down the application. Therefore, she decides to log only important information necessary to debug or troubleshoot potential problems. She proceeds to identify the information needed to be logged and then decides on where to use log statements. She thinks that it is important to log the information related to a message sent to a registered component, including the message content and the transmission time, to find the root causes of potential problems in sending messages. She simply wants to begin by using a log statement at the start of the \code{send()} method (line~2 of Figure~\ref{ch2-ex-logged-m1}) to log the information. However, she realizes that this log statement does not allow her to log all the information she wants, as the \code{time} variable is not initialized at the beginning of this method. Therefore, she proceeds to examine the body of the \code{send()} method line-by-line and uses another log statement after the \code{time} variable is initialized. She aims to log the transmission time in case of potential problems in sending messages. Therefore, she decides to insert the logging call inside an \code{if} statement that logs the value of the variable \code{time}, if it is not within a valid range (shown in lines~9--11 of Figure~\ref{ch2-ex-logged-m2}).
%She aims to log the transmission time in case of potential problems in sending messages.?

\begin{figure}[p]
\def\baselinestretch{1}
\begin{lstlisting}[escapechar=!]
public static void send(EBMessage message){
   !\colorbox{yellowGreen}{//log statement}!
   EBComponent[] comps = getComponents();
   for (int i = 0; i < comps.length; i++) {
       EBComponent comp = comps[i];
       long start = System.currentTimeMillis();
       comp.handleMessage(message);
       long time = (System.currentTimeMillis() - start);
   }
}
\end{lstlisting}
\caption[The developer's initial determination of the usage of logging calls.]{The developer's initial determination of the usage of log statements for the \code{send(EBMessage)} method.\label{ch2-ex-logged-m1}}
\end{figure}

\begin{figure}[p]
\def\baselinestretch{1}
\begin{lstlisting}[escapechar=!]
public static void send(EBMessage message) {
    !\colorbox{yellowGreen}{//log statement}!
    EBComponent[] comps = getComponents();
    for(int i = 0; i < comps.length; i++) {
        EBComponent comp = comps[i];
        long start = System.currentTimeMillis();
        comp.handleMessage(message);
        long time = (System.currentTimeMillis() - start);
        if(time >= 1000000) {
            !\colorbox{yellowGreen}{//log statement}!
        }
    }
}
\end{lstlisting}
\caption[The developer's second determination of the usage of log statements.]{The developer's second determination of the usage of log statements for the \code{send(EBMessage)} method.\label{ch2-ex-logged-m2}}
\end{figure}

\begin{figure}[p]
\def\baselinestretch{1}
\begin{lstlisting}[escapechar=!]
public static void send(EBMessage message){
    try {
        !\colorbox{yellowGreen}{//log statement}!
        EBComponent[] comps = getComponents();
        for(int i = 0; i < comps.length; i++) {
            EBComponent comp = comps[i];
            long start = System.currentTimeMillis();
            comp.handleMessage(message);
            long time = (System.currentTimeMillis() - start);
            if(time >= 1000000) {
                !\colorbox{yellowGreen}{//log statement}!
            }
        }
    } catch(Throwable t) {
       !\colorbox{yellowGreen}{//log statement}!
    }
}
\end{lstlisting}
\caption[The developer's third determination of the usage of log statements.]{The developer's third determination of the usage of log statements for the \code{send(EBMessage)} method.\label{ch2-ex-logged-m3}}
\end{figure}

She also believes that it is important to log an error if any problems occur in sending messages to the components. She decides to use a \code{try}/\code{catch} statement, as it is a common way to handle exceptions in the Java programming language. She creates a \code{try}/\code{catch} block to capture the potential failure in sending messages, and uses a log statement inside the \code{catch} block to log the exception (shown in lines~2--16 of Figure~\ref{ch2-ex-logged-m3}). However, she realizes that using this logging call will not allow her to reach the desired functionality, as it does not reveal to which component the problem is related. Thus, she decides to relocate the \code{try}/\code{catch} block inside the \code{for} statement to log an error in case of a problem in sending messages to any components (shown in lines~5--15 of Figure~\ref{ch2-ex-logged-m4}).


\begin{figure}[p]
\def\baselinestretch{1}
\begin{lstlisting}[escapechar=!]
public static void send(EBMessage message) {
    !\colorbox{yellowGreen}{//log statement}!
    EBComponent[] comps = getComponents();
    for (int i = 0; i < comps.length; i++) {
        try {
            EBComponent comp = comps[i];
            long start = System.currentTimeMillis();
            comp.handleMessage(message);
            long time = (System.currentTimeMillis() - start);
            if(time >= 1000000) {
                !\colorbox{yellowGreen}{//log statement}!
            }
        } catch(Throwable t) {
            !\colorbox{yellowGreen}{//log statement}!
        }
    }
}
\end{lstlisting}
\caption[The developer's fourth determination of the usage of log statements.]{The developer's fourth determination of the usage of log statements for the \code{send(EBMessage)} method.\label{ch2-ex-logged-m4}}
\end{figure}

\begin{figure}[p]
\def\baselinestretch{0.94}
\begin{lstlisting}[escapechar=!]
public class EditBus {
    private static ArrayList components = new ArrayList();
    private static EBComponent[] copyComponents;

    private EditBus() {
    }

    public static void addToBus(EBComponent comp) {
        synchronized(components) {
            components.add(comp);
            copyComponents = null;
        }
    }

    public static void removeFromBus(EBComponent comp) {
        synchronized(components) {
            components.remove(comp);
            copyComponents = null;
        }
    }

    public static EBComponent[] getComponents() {
        synchronized(components) {
            if(copyComponents == null) {
                EBComponent[] arr = new EBComponent[components.size()];
                copyComponents = (EBComponent[])components.toArray(arr);
            }
        }
        return copyComponents;
    }

    public static void send(EBMessage message) {
        !\colorbox{yellowGreen}{//log statement}!
        EBComponent[] comps = getComponents();
        for(int i = 0; i < comps.length; i++) {
            try {
                EBComponent comp = comps[i];
                long start = System.currentTimeMillis();
                comp.handleMessage(message);
                long time = (System.currentTimeMillis() - start);
                if(time >= 1000000) {
                    !\colorbox{yellowGreen}{//log statement}!
                }
            } catch(Throwable t) {
                 !\colorbox{yellowGreen}{//log statement}!
            }
        }
    }
}
\end{lstlisting}

\caption[The developer's final determination of the usage of log statements.]{The developer's final determination of the usage of log statements for the \code{EditBus} class.\label{ch2-ex-logged}}
\end{figure}

%which information or what?
Figure~\ref{ch2-ex-logged} shows the developer's final determination of the usage of log statements to perform the logging task of the \code{EditBus} class. By making appropriate decisions about where to use log statements, the developer is in a good position to proceed to write the logging messages by examining the remaining conceptually complex questions. What specific information should I log? How should I choose the log message format? Which information goes to which level of logging? If the developer had reached this point more easily and quickly, she would have had more time and energy to make decisions about the remaining issues and could have completed the logging practice in a timely and appropriate manner.

%\RW{This is followed by a summary, so it is rather redundant.}
%To sum up, this scenario required the developer to make numerous decisions and then act on them. Her attention was split between detailed and the high-level decisions. In addition, time constraints and the performance of tedious tasks can cause the developer to make bad decisions. However, having a vision of where developers usually use logging calls in similar situations could guide her to make informed decisions about where to use logging calls, and she could then perform the logging task in a faster and less error-prone manner.


\section{Summary}  \label{ch2-summary}

This motivational scenario highlights the problems a developer may encounter in performing a logging task. The core problem she faces in this scenario is the difficulty in understanding where to use log statements that enable her to log the desired information. However, having an understanding of how developers usually log in similar situations might assist her to make informed decisions about where to use log statements more effectively, and so she could pay more attention to the remaining, conceptually complex issues to complete the logging task.

% where to use log statements more effectively
%comma???
%in a faster and less error-prone manner


