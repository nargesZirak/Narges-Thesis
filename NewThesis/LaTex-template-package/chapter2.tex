\chapter{Motivational Scenario}  \label{ch2}
\begin{itemize} [leftmargin=.01in]
\item Logging is a systematic way of recording the software runtime information
\item A typical logging call is composed of a log function and its parameters including a text message and verbosity level
\begin{itemize} [leftmargin=.3in]
\item A log text message consists of static text to describe the logged event
and some optional variables related to the event
\item The verbosity level is intended to classify the
severity of the logged event (Fatal, error, warn, info, and debug)
\end{itemize}
\item Consider a developer is given the task of logging the Java class in Figure~\ref{ch2-ex1}
%\item She decides to use log4j framework for logging
\begin{figure}[H]
\def\baselinestretch{1}
\begin{lstlisting}
public class EditBus {
  private static ArrayList components=new ArrayList();
  private static EBComponent[] copyComponents;
  private EditBus(){
  
  public static void addToBus(  EBComponent comp){
    synchronized (components) {
      components.add(comp);
      copyComponents=null;
    }
  }
  public static void removeFromBus(  EBComponent comp){
    synchronized (components) {
      components.remove(comp);
      copyComponents=null;
    }
  }
  public static EBComponent[] getComponents(){
    synchronized (components) {
      if (copyComponents == null) {
        copyComponents=(EBComponent[])components.toArray(new EBComponent[components.size()]);
      }
      return copyComponents;
    }
  }
  public static void send(  EBMessage message){
    EBComponent[] comps=getComponents();
    for (int i=0; i < comps.length; i++) {
        EBComponent comp=comps[i];
        if (Debug.EB_TIMER) {
          long start=System.currentTimeMillis();
          comp.handleMessage(message);
          long time=(System.currentTimeMillis() - start);
        }
        else  comps[i].handleMessage(message);
    }
  }
}
\end{lstlisting}
\caption{A Java class without the usage of logging\label{ch2-ex1}}
\end{figure}

\item She has to make several decisions about
\begin{itemize} [leftmargin=.3in]
\item what events need to be logged?
\item where to use logging calls?
\item how to decide on the log message and verbosity level of each logging call?
\end{itemize}
\item It is recommended to simply log at the start and end of every method
\item However, for example, logging at the start and end of the method \vars{addToBus} is useless, producing redundant information
\item She needs more information to perform logging appropriately
\item Having a characterization  of how usually developers use logging calls in similar situations would assist her in making decisions
\item For example, knowing that developers use logging calls inside of if statements to log a potential error when a variable contains an incorrect value, she adds an if statement to log an error when the value of the variable \vars{time} is \vars{null} (shown in Lines~36-38 of Figure~\ref{ch2-ex2})
\item For example, knowing that developers use logging calls inside catch blocks to record an exception, she creates a try/catch block to capture the potential failure in sending messages and uses a logging call in the catch block (Lines~41-43 of Figure~\ref{ch2-ex2})

\begin{figure}[H]
\def\baselinestretch{1}
\begin{lstlisting}
public class EditBus {
 private static ArrayList components=new ArrayList();
  private static EBComponent[] copyComponents;
  private EditBus(){
  }
  public static void addToBus(  EBComponent comp){
    synchronized (components) {
      components.add(comp);
      copyComponents=null;
    }
  }
  public static void removeFromBus(  EBComponent comp){
    synchronized (components) {
      components.remove(comp);
      copyComponents=null;
    }
  }
  public static EBComponent[] getComponents(){
    synchronized (components) {
      if (copyComponents == null) {
        copyComponents=(EBComponent[])components.toArray(new EBComponent[components.size()]);
      }
      return copyComponents;
    }
  }
  public static void send(  EBMessage message){
    Log.log(Log.DEBUG,EditBus.class,message.toString());
    EBComponent[] comps=getComponents();
    for (int i=0; i < comps.length; i++) {
      try {
        EBComponent comp=comps[i];
        if (Debug.EB_TIMER) {
          long start=System.currentTimeMillis();
          comp.handleMessage(message);
          long time=(System.currentTimeMillis() - start);
          if (time != 0) {
             Log.log(Log.DEBUG,EditBus.class,comp + ": " + time+ " ms");
          }
        }
        else  comps[i].handleMessage(message);
      }catch (Throwable t) {
         Log.log(Log.ERROR,EditBus.class,"Exception" + " while sending message on EditBus:");
      }
    }
  }
}
\end{lstlisting}
\caption{A Java class after the usage of logging calls\label{ch2-ex2}}
\end{figure}

\item Using a concise characterization she would be able to make informed decisions about where to use logging calls more easily and quickly
\item With taking appropriate decisions about where to use logging calls, she can spend more time and energy to write the context of log functions% and even other development and maintenance tasks
\end{itemize}