\documentclass{article}
\usepackage{enumitem}
\usepackage{pgf}
\usepackage{tikz}
\usepackage{tikz-cd}
\usetikzlibrary{arrows, matrix}
\usepackage{amssymb}% http://ctan.org/pkg/amssymb
\usepackage{pifont}% http://ctan.org/pkg/pifont
\newcommand{\cmark}{\ding{51}}%
\newcommand{\xmark}{\ding{55}}%
\newcommand{\pmark}{\ding{59}}%
\newcommand{\bld}{\textbf}
\newcommand{\itt}{\textit}
\newcommand{\tsc}{\textsc}
\newcommand{\vars}{\textit}
\newcommand{\cons}{\textrm}
\usepackage{textcomp}
\newcommand{\nothing}{{``nothing''}}

\long\def\TM#1{{\sc[ #1]}}

\begin{document}



\begin{itemize} [leftmargin=.1in]
\item \bld{Problem} 
\begin{itemize}
\item Determining the detailed structural similarities and differences between source code fragments is a complex task 
\item It can be applied to solve several source code analysis problems
\item As a specific application, our focus is on the study of where logging is used in the source code 
\item logging is a pervasive practice and has various applications in software development and maintenance
\item It is a challenging task for developers to understand how to use logging calls in the source code 
\item In this research, we would like to understand  where developers log in practice, in a detailed way
\end{itemize}
\item \bld{A proposed Solution to the Problem} 
\begin{itemize}
\item Develop an automated approach to detect the detailed structural similarities and differences of Java classes that use logging calls
\item Our solution would: 
\begin{itemize} [leftmargin=.1in]
\item classify logged Java methods into groups using a measure of similarity such that entities in each group has maximum similarity with each other and minimum similarity to other ones 
\item construct a structural generalization of each group that represent the detailed structural similarities and differences of all logged Java methods in the group
\end{itemize}
\end{itemize}

\item \bld{Required Background Information}  
\begin{itemize}
\item Abstract Syntax Tree (AST), which is the basic structure we will use for describing software source code
%\item how ASTs are realized in the Eclipse integrated development environment, the industrial tool we will build atop
\item First-order anti-unification, which is a technique used to construct the structural generalizations% and its limitations to solve our problem context
\item Higher-order anti-unification modulo theories (HOAUMT), which is used to address the limitations of first-order anti-unification 
%that can be applied on an extended form of the AST structure to address our problem
\item The Jigsaw framework, an existing tool for a subset of HOAUMT, which we extend to determine potential correspondences between logged Java methods
\item Agglomerative hierarchical clustering, which is a clustering algorithm used for classifying logged Java methods into groups using a measure of similarity
\end{itemize}

\end{itemize}
\end{document} 